% Figure: Hypovolemia Pathophysiology in ME/CFS
% Comprehensive cascade showing mechanisms, feedback loops, and consequences
% Color coding: GREEN=HIGH certainty, YELLOW=MEDIUM certainty, RED=LOW/research gap

\begin{figure}[htbp]
\centering
\begin{tikzpicture}[
    node distance=2.0cm,
    scale=0.53, every node/.style={scale=0.53},
    % HIGH certainty (replicated studies) - GREEN
    high/.style={draw=green!70!black, fill=green!15, very thick, rounded corners, text width=3.2cm, align=center, minimum height=1cm},
    high-wide/.style={draw=green!70!black, fill=green!15, very thick, rounded corners, text width=4.5cm, align=center, minimum height=1cm},
    % MEDIUM certainty (single studies, awaiting replication) - YELLOW
    medium/.style={draw=yellow!70!black, fill=yellow!20, very thick, rounded corners, text width=3.2cm, align=center, minimum height=1cm},
    medium-wide/.style={draw=yellow!70!black, fill=yellow!20, very thick, rounded corners, text width=4.5cm, align=center, minimum height=1cm},
    % LOW certainty / research gap - RED
    low/.style={draw=red!70!black, fill=red!15, very thick, rounded corners, text width=3.2cm, align=center, minimum height=1cm},
    % Process/condition nodes
    process/.style={draw=blue!60!black, fill=blue!10, very thick, rounded corners, text width=3cm, align=center, minimum height=0.9cm},
    % Outcome nodes
    outcome/.style={draw=purple!60!black, fill=purple!15, very thick, rounded corners, text width=2.8cm, align=center, minimum height=0.9cm},
    % Central convergence
    central/.style={draw=red!50!black, fill=red!25, ultra thick, rounded corners, text width=4.5cm, align=center, minimum height=1.2cm, drop shadow},
    % Arrow styles
    arrow/.style={-latex, very thick, black!70, line width=1.2pt},
    feedback/.style={-latex, thick, orange!80!black, dashed, line width=1.1pt},
    cascade/.style={-latex, thick, purple!70!black, line width=1pt},
    % Chapter reference style
    chapref/.style={font=\scriptsize\sffamily, gray!60!black},
    % Note style
    note/.style={font=\scriptsize\itshape, text width=2cm, align=center},
]

% ============================================================
% TITLE
% ============================================================
\node[font=\large\bfseries, red!70!black] at (0, 12) {Hypovolemia Pathophysiology in ME/CFS};

% ============================================================
% UPSTREAM TRIGGER (TOP)
% ============================================================
\node[low, text width=5cm] (trigger) at (0, 10) {
    \textbf{Viral Infection /}\\
    \textbf{Immune Activation}\\[2pt]
    {\scriptsize (Triggering mechanism unclear)}
};
\node[chapref, right=0.3cm of trigger] {Ch5, Ch7};

% ============================================================
% THREE PARALLEL MECHANISMS (MIDDLE LEVEL)
% ============================================================

% --- MECHANISM 1: Autonomic/RAAS (LEFT) ---
\begin{scope}[xshift=-5.5cm]
    \node[low, text width=3.5cm] (autonomic) at (0, 7.5) {
        \textbf{Autonomic}\\
        \textbf{Dysregulation}\\[2pt]
        {\scriptsize Central cause unknown}
    };
    \node[chapref, left=0.2cm of autonomic] {Ch8, Ch10};

    \node[high, text width=3.5cm] (raas) at (0, 5) {
        \textbf{RAAS Suppression}\\[2pt]
        Renin $\downarrow$\\
        Aldosterone \textbf{$-$34\%}\\
        {\scriptsize (Miwa 2017, Raj 2005)}
    };
    \node[chapref, left=0.2cm of raas] {Ch10};

    \node[high, text width=3.5cm] (sodium) at (0, 2.5) {
        \textbf{Sodium/Water}\\
        \textbf{Retention Failure}\\[2pt]
        {\scriptsize Impaired volume}\\
        {\scriptsize defense}
    };

    \draw[arrow] (autonomic) -- (raas);
    \draw[arrow] (raas) -- (sodium);
\end{scope}

% --- MECHANISM 2: Natriuretic Peptides (CENTER) ---
\begin{scope}[xshift=0cm]
    \node[medium, text width=3.5cm] (bnp-trigger) at (0, 7.5) {
        \textbf{Cardiac Stress}\\
        \textbf{Signaling}\\[2pt]
        {\scriptsize Paradoxical response}
    };

    \node[medium, text width=3.5cm] (bnp) at (0, 5) {
        \textbf{BNP Dysregulation}\\[2pt]
        BNP $\uparrow$ despite\\
        low cardiac volume\\
        {\scriptsize (Inappropriate natriuresis)}
    };
    \node[chapref, right=0.2cm of bnp] {Ch10};

    \node[medium, text width=3.5cm] (diuresis) at (0, 2.5) {
        \textbf{Natriuresis /}\\
        \textbf{Diuresis}\\[2pt]
        {\scriptsize Excessive sodium}\\
        {\scriptsize \& water loss}
    };

    \draw[arrow] (bnp-trigger) -- (bnp);
    \draw[arrow] (bnp) -- (diuresis);
\end{scope}

% --- MECHANISM 3: Endothelial/Capillary Leak (RIGHT) ---
\begin{scope}[xshift=5.5cm]
    \node[high, text width=3.8cm] (endothelial) at (0, 7.5) {
        \textbf{Endothelial}\\
        \textbf{Dysfunction}\\[2pt]
        FMD: \textbf{5.1\%} vs 8.2\%\\
        {\scriptsize (Scherbakov 2020)}
    };
    \node[chapref, right=0.2cm of endothelial] {Ch7, Ch12};

    \node[high, text width=3.8cm] (mediators) at (0, 5) {
        \textbf{Inflammatory}\\
        \textbf{Mediators}\\[2pt]
        Bradykinin $\uparrow$\\
        Histamine $\uparrow$, TNF-$\alpha$ $\uparrow$
    };
    \node[chapref, right=0.2cm of mediators] {Ch7};

    \node[medium, text width=3.8cm] (leak) at (0, 2.5) {
        \textbf{Capillary Leak}\\[2pt]
        {\scriptsize Plasma extravasation}\\
        {\scriptsize (Quantification pending)}
    };

    \draw[arrow] (endothelial) -- (mediators);
    \draw[arrow] (mediators) -- (leak);
\end{scope}

% Arrows from trigger to mechanisms
\draw[arrow] (trigger.south) -- ++(0,-0.5) -| (autonomic.north);
\draw[arrow] (trigger.south) -- (bnp-trigger.north);
\draw[arrow] (trigger.south) -- ++(0,-0.5) -| (endothelial.north);

% ============================================================
% CONVERGENCE: PLASMA VOLUME REDUCTION
% ============================================================
\node[central] (hypovolemia) at (0, 0) {
    \textbf{PLASMA VOLUME}\\
    \textbf{REDUCTION}\\[3pt]
    \textbf{$-$10\% to $-$21\%}\\[2pt]
    {\scriptsize (Raj 2005, van Campen 2018)}
};
\node[chapref, right=0.3cm of hypovolemia] {Ch10, Ch12};

% Arrows converging to hypovolemia
\draw[arrow] (sodium.south) -- ++(0,-0.3) -| ([xshift=-1.5cm]hypovolemia.north);
\draw[arrow] (diuresis.south) -- (hypovolemia.north);
\draw[arrow] (leak.south) -- ++(0,-0.3) -| ([xshift=1.5cm]hypovolemia.north);

% ============================================================
% CARDIAC CONSEQUENCES (LOWER)
% ============================================================
\node[high, text width=3.8cm] (preload) at (-4.5, -2.8) {
    \textbf{Reduced Cardiac}\\
    \textbf{Preload}\\[2pt]
    {\scriptsize Venous return $\downarrow$}
};

\node[high, text width=3.8cm] (stroke) at (0, -2.8) {
    \textbf{Reduced Stroke}\\
    \textbf{Volume}\\[2pt]
    {\scriptsize Frank-Starling}\\
    {\scriptsize mechanism}
};

\node[high, text width=3.8cm] (cardiac) at (4.5, -2.8) {
    \textbf{Reduced Cardiac}\\
    \textbf{Output}\\[2pt]
    {\scriptsize (Reduced preload $\to$}\\
    {\scriptsize reduced CO)}
};
\node[chapref, right=0.2cm of cardiac] {Ch10};

\draw[cascade] (hypovolemia) -- (preload);
\draw[cascade] (hypovolemia) -- (stroke);
\draw[cascade] (hypovolemia) -- (cardiac);
\draw[cascade] (preload) -- (stroke);
\draw[cascade] (stroke) -- (cardiac);

% ============================================================
% CLINICAL MANIFESTATIONS (BOTTOM)
% ============================================================
\node[outcome] (oi) at (-5.5, -5.5) {
    \textbf{Orthostatic}\\
    \textbf{Intolerance}\\[2pt]
    {\scriptsize POTS, OH}
};
\node[chapref, below=0.1cm of oi] {Ch10};

\node[outcome] (exercise) at (-1.8, -5.5) {
    \textbf{Exercise}\\
    \textbf{Intolerance}\\[2pt]
    {\scriptsize VO$_2$max $\downarrow$}
};
\node[chapref, below=0.1cm of exercise] {Ch12};

\node[outcome] (fatigue) at (1.8, -5.5) {
    \textbf{Fatigue}\\[2pt]
    {\scriptsize Tissue hypoxia}
};
\node[chapref, below=0.1cm of fatigue] {Ch5};

\node[outcome] (pem) at (5.5, -5.5) {
    \textbf{PEM}\\[2pt]
    {\scriptsize Delayed crash}\\
    {\scriptsize after exertion}
};
\node[chapref, below=0.1cm of pem] {Ch5, Ch6};

\draw[cascade] (cardiac) -- ++(0,-0.8) -| (oi);
\draw[cascade] (cardiac) -- ++(0,-0.8) -| (exercise);
\draw[cascade] (cardiac) -- ++(0,-0.8) -| (fatigue);
\draw[cascade] (cardiac) -- ++(0,-0.8) -| (pem);

% ============================================================
% FEEDBACK LOOPS (CURVED ARROWS)
% ============================================================

% LOOP 1: Hypovolemia → Sympathetic activation → Catecholamine insufficiency → RAAS failure
\node[process, text width=2.8cm] (sympathetic) at (-9, 3) {
    \textbf{Sympathetic}\\
    \textbf{Activation}\\[2pt]
    {\scriptsize Compensatory}
};
\node[low, text width=2.8cm] (insufficiency) at (-9, 0.5) {
    \textbf{Catecholamine}\\
    \textbf{Insufficiency}\\[2pt]
    {\scriptsize (Strahler 2013)}
};
\draw[feedback, bend right=20] (hypovolemia.west) to node[note, left, xshift=-0.2cm] {Loop 1} (sympathetic.south);
\draw[feedback] (sympathetic) -- (insufficiency);
\draw[feedback, bend right=25] (insufficiency.north) to (raas.west);

% LOOP 2: Hypovolemia → Reduced CBF → Impaired autonomic centers
\node[high, text width=2.8cm] (cbf) at (9, 5) {
    \textbf{Reduced}\\
    \textbf{Cerebral Blood}\\
    \textbf{Flow}\\[2pt]
    {\scriptsize (van Campen 2020)}
};
\node[chapref, right=0.1cm of cbf] {Ch8};

\node[low, text width=2.8cm] (braincenters) at (9, 2.5) {
    \textbf{Impaired}\\
    \textbf{Autonomic}\\
    \textbf{Centers}\\[2pt]
    {\scriptsize Brainstem}
};
\draw[feedback, bend left=15] (hypovolemia.east) to node[note, right, xshift=0.2cm] {Loop 2} (cbf.south);
\draw[feedback] (cbf) -- (braincenters);
\draw[feedback, bend left=20] (braincenters.north) to (autonomic.east);

% LOOP 3: Orthostatic stress → Mast cell activation → Capillary leak
\node[medium, text width=2.8cm] (mast) at (9, -2) {
    \textbf{Mast Cell}\\
    \textbf{Activation}\\[2pt]
    {\scriptsize Histamine release}
};
\node[chapref, right=0.1cm of mast] {Ch7};
\draw[feedback, bend left=20] (oi.east) to node[note, below right, yshift=-0.3cm] {Loop 3} (mast.south);
\draw[feedback, bend left=35] (mast.north) to (leak.east);

% LOOP 4: Sustained immune activation → Cytokines → Endothelial dysfunction
\node[high, text width=2.8cm] (cytokines) at (-9, 7) {
    \textbf{Sustained}\\
    \textbf{Cytokine}\\
    \textbf{Elevation}\\[2pt]
    {\scriptsize IL-1, IL-6, TNF}
};
\node[chapref, left=0.1cm of cytokines] {Ch7};
\draw[feedback, bend right=15] (trigger.west) to node[note, left, xshift=-0.2cm] {Loop 4} (cytokines.east);
\draw[feedback, bend right=30] (cytokines.south) to (endothelial.west);

% ============================================================
% LEGEND
% ============================================================
\node[draw=black!50, fill=white, rounded corners, inner sep=8pt,
      text width=11cm, align=left, font=\small] at (0, -8.5) {
    \textbf{Certainty Legend:}\\[4pt]
    \tikz{\node[high, minimum height=0.5cm, minimum width=1.2cm, text width=1cm, font=\scriptsize] {HIGH};}
    Replicated across multiple studies (RAAS suppression, plasma volume, endothelial dysfunction)\\[2pt]
    \tikz{\node[medium, minimum height=0.5cm, minimum width=1.2cm, text width=1cm, font=\scriptsize] {MED};}
    Single studies or awaiting replication (BNP mechanism, capillary leak quantification)\\[2pt]
    \tikz{\node[low, minimum height=0.5cm, minimum width=1.2cm, text width=1cm, font=\scriptsize] {LOW};}
    Research gap / mechanism unclear (central autonomic cause, triggering mechanism)\\[6pt]
    \textbf{Key References:} Ch5 (Clinical), Ch6 (Energy), Ch7 (Immune), Ch8 (Neurological), Ch10 (Cardiovascular), Ch12 (Exercise)
};

\end{tikzpicture}
{\caption{\parbox{0.9\linewidth}{Hypovolemia mechanisms in ME/CFS. Three pathways (autonomic dysregulation, natriuretic peptide excess, endothelial dysfunction) converge to reduce plasma volume (10--21\%) and cardiac output. Four feedback loops perpetuate dysfunction. Colors: green = high certainty, yellow = medium, red = low/gap.}}}
\label{fig:hypovolemia-mechanisms}
\end{figure}
