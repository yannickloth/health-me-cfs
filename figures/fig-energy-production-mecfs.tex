% Figure: Impaired Energy Production in ME/CFS
% Multiple dysfunction points reduce ATP and cause system-wide failures

\begin{figure}[htbp]
\centering
\begin{tikzpicture}[
    node distance=2.5cm,
    scale=0.85, every node/.style={scale=0.85},
    % Styles
    normal/.style={draw=green!70!black, fill=green!10, very thick, rounded corners, text width=3.2cm, align=center, minimum height=1cm},
    impaired/.style={draw=red!70!black, fill=red!15, very thick, rounded corners, text width=3.2cm, align=center, minimum height=1cm},
    severe/.style={draw=red!50!black, fill=red!25, ultra thick, rounded corners, text width=3.2cm, align=center, minimum height=1.1cm, drop shadow},
    pathological/.style={draw=red!50!black, fill=red!20, very thick, rounded corners, text width=3.5cm, align=center, minimum height=1cm},
    arrow/.style={-latex, very thick, green!70!black, line width=1.2pt},
    impaired-arrow/.style={-latex, very thick, red!70!black, line width=1.2pt},
    cascade-arrow/.style={-latex, thick, orange!80!black, dashed, line width=1.1pt},
    note/.style={font=\small\itshape, text width=2.2cm, align=left, red!60!black},
]

% Title
\node[font=\large\bfseries, red!70!black] at (0, 9) {ME/CFS: Impaired Energy Production};

% LEFT SIDE: Impaired pathway
\begin{scope}[xshift=-3cm]
    % Glucose (preserved)
    \node[normal] (glucose) at (0, 7) {\textbf{Glucose}\\[2pt] Input preserved};

    % Glycolysis (relatively preserved)
    \node[normal] (glycolysis) at (0, 5.4) {\textbf{Glycolysis}\\[2pt] Relatively preserved};
    \draw[arrow] (glucose) -- (glycolysis);

    % Pyruvate - IMPAIRED
    \node[impaired] (pyruvate) at (0, 3.5) {\textbf{Pyruvate Handling}\\[2pt] {\color{red!80!black}IMPAIRED}};
    \draw[impaired-arrow] (glycolysis) -- (pyruvate);
    \node[note, left=0.15cm of pyruvate, anchor=east] {
        \textbullet~Early\\~~~lactate\\~~~shift
    };

    % Krebs - IMPAIRED
    \node[impaired] (krebs) at (0, 1.6) {\textbf{Krebs Cycle}\\[2pt] {\color{red!80!black}Aconitase damaged}};
    \draw[impaired-arrow] (pyruvate) -- (krebs);
    \node[note, left=0.15cm of krebs, anchor=east] {
        \textbullet~ROS\\~~~damage
    };

    % ETC - IMPAIRED
    \node[impaired] (etc) at (0, -0.3) {\textbf{ETC}\\[2pt] {\color{red!80!black}Complex I/III leak}};
    \draw[impaired-arrow] (krebs) -- (etc);
    \node[note, left=0.15cm of etc, anchor=east] {
        \textbullet~5--10\%\\~~~electron\\~~~leak
    };

    % ATP - REDUCED
    \node[severe] (atp) at (0, -2.3) {\textbf{REDUCED ATP}\\[3pt] \textit{Energy crisis}\\[2pt] 50--70\% normal};
    \draw[impaired-arrow] (etc) -- (atp);
\end{scope}

% RIGHT SIDE: System failures from ATP deficit
\begin{scope}[xshift=3.5cm]
    % Central deficit node
    \node[pathological, minimum width=5.5cm, text width=4cm,] (deficit) at (0, 4) {\textbf{ATP DEFICIT}\\Cellular energy crisis};

    % System failures
    \node[pathological] (muscle) at (-2.2, 1.5) {\textbf{Muscle}\\Weakness\\PEM\\Lactate};

    \node[pathological] (brain) at (2.2, 1.5) {\textbf{Brain}\\Fog\\Memory\\Processing};

    \node[pathological] (immune) at (-2.2, -1) {\textbf{Immune}\\T-cell failure\\NK impaired};

    \node[pathological] (cardio) at (2.2, -1) {\textbf{Cardiac}\\Ion pumps\\Arrhythmia};

    \node[pathological] (autonomic) at (0, -3.2) {\textbf{Autonomic}\\Catecholamines\\POTS};

    % Cascade arrows
    \draw[cascade-arrow] (deficit) -- (muscle);
    \draw[cascade-arrow] (deficit) -- (brain);
    \draw[cascade-arrow] (deficit) -- (immune);
    \draw[cascade-arrow] (deficit) -- (cardio);
    \draw[cascade-arrow] (deficit) -- (autonomic);
\end{scope}

% Arrow connecting left to right
\draw[cascade-arrow, line width=2pt] (-1.2, -3) -- (-0.8, -3) -- (-0.8, 3.8) -- (0.8, 3.8);

% Key point box
\node[draw=red!70!black, fill=red!5, rounded corners, text width=12cm, align=left, font=\small, inner sep=8pt] at (0, -5.5) {
\textbf{Multiple dysfunction points:}\\[4pt]
\textbullet~Early lactate shift: Pyruvate cannot efficiently enter mitochondria\\
\textbullet~Krebs cycle impairment: Aconitase inactivation by oxidative stress\\
\textbullet~ETC dysfunction: 5--10\% electron leak (vs. $<$2\% normal)\\[4pt]
\textbf{Result:} 50--70\% ATP reduction causes system-wide failures. This is not ``tiredness'' but an active cellular energy crisis affecting every ATP-dependent process.
};

\end{tikzpicture}
\caption{ME/CFS energy production dysfunction and systemic consequences.}
\label{fig:energy-cascade-mecfs}
\end{figure}
