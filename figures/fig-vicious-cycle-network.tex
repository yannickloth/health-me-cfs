% Figure: Vicious Cycle Network in ME/CFS
% Shows coupling between major pathophysiological cycles

\begin{figure}[htbp]
\centering
\begin{tikzpicture}[
    scale=0.9, every node/.style={scale=0.9},
    % Cycle node styles (colored ellipses)
    cycle node/.style={
        ellipse, draw=#1!60!black, fill=#1!15, very thick,
        minimum width=3.2cm, minimum height=2cm,
        font=\small\bfseries, align=center,
        inner sep=4pt
    },
    % Coupling arrow style
    coupling/.style={
        <->, >=latex, thick, red!70!black,
        shorten >=3pt, shorten <=3pt
    },
    % Coupling label style
    coupling label/.style={
        font=\tiny, fill=white, inner sep=2pt,
        text=red!60!black, align=center
    },
    % Self-reinforcing loop style
    self loop/.style={
        ->, >=latex, thick, #1!50!black,
        looseness=4, in=120, out=60
    },
    % Treatment intervention point
    intervention/.style={
        star, star points=5, star point ratio=2.2,
        draw=orange!80!black, fill=yellow!50,
        minimum size=0.5cm, inner sep=0pt
    }
]

% Title
\node[font=\large\bfseries] at (0, 6.5) {Vicious Cycle Network in ME/CFS};
\node[font=\small\itshape, text width=10cm, align=center] at (0, 5.8) {Bidirectional coupling prevents resolution of individual cycles};

% === CYCLE NODES (Pentagon Layout) ===
% Center: Mitochondrial Dysfunction
\node[cycle node=blue] (mito) at (0, 3) {Mitochondrial\\Dysfunction};

% Top-left: Immune Activation
\node[cycle node=green] (immune) at (-4.5, 1.5) {Immune\\Activation};

% Top-right: Autonomic Dysfunction
\node[cycle node=purple] (autonomic) at (4.5, 1.5) {Autonomic\\Dysfunction};

% Bottom-left: Endocrine Dysfunction
\node[cycle node=teal] (endocrine) at (-3, -2.5) {Endocrine\\Dysfunction};

% Bottom-right: Neuroinflammation
\node[cycle node=orange] (neuro) at (3, -2.5) {Neuroinflammation};

% === SELF-REINFORCING LOOPS ===
\draw[self loop=blue] (mito) to (mito);
\draw[self loop=green] (immune) to (immune);
\draw[self loop=purple] (autonomic) to (autonomic);
\draw[self loop=teal] (endocrine) to (endocrine);
\draw[self loop=orange] (neuro) to (neuro);

% === COUPLING ARROWS WITH LABELS ===

% Mitochondrial <-> Immune: ROS/Cytokines
\draw[coupling] (mito.west) -- (immune.east)
    node[coupling label, midway, above, yshift=2pt] {ROS/Cytokines};

% Mitochondrial <-> Autonomic: Hypoxia/Perfusion
\draw[coupling] (mito.east) -- (autonomic.west)
    node[coupling label, midway, above, yshift=2pt] {Hypoxia/Perfusion};

% Immune <-> Autonomic: via top path
\draw[coupling, bend left=25] (immune.north east) to
    node[coupling label, above, yshift=2pt] {BBB crossing/Autoantibodies} (autonomic.north west);

% Neuroinflammation <-> Endocrine: HPA axis
\draw[coupling] (neuro.west) -- (endocrine.east)
    node[coupling label, midway, below, yshift=-2pt] {HPA axis};

% Endocrine <-> Mitochondrial: Hormonal regulation
\draw[coupling] (endocrine.north) -- (mito.south west)
    node[coupling label, midway, left, xshift=-2pt] {Hormonal\\regulation};

% Additional couplings for network completeness
% Autonomic <-> Neuroinflammation
\draw[coupling] (autonomic.south) -- (neuro.north east)
    node[coupling label, midway, right, xshift=2pt] {Vagal tone/\\Inflammation};

% Immune <-> Endocrine
\draw[coupling] (immune.south) -- (endocrine.north west)
    node[coupling label, midway, left, xshift=-2pt] {Cortisol/\\Cytokines};

% Immune <-> Neuroinflammation
\draw[coupling, bend right=15] (immune.south east) to
    node[coupling label, below, yshift=-5pt] {Microglial activation} (neuro.north west);

% === TREATMENT INTERVENTION POINTS ===
\node[intervention] (treat-mito) at (0.8, 4.2) {};
\node[intervention] (treat-immune) at (-5.5, 2.5) {};
\node[intervention] (treat-autonomic) at (5.5, 2.5) {};
\node[intervention] (treat-endocrine) at (-4, -3.5) {};
\node[intervention] (treat-neuro) at (4, -3.5) {};

% Intervention labels
\node[font=\tiny, align=center, text=orange!70!black] at (0.8, 4.8) {CoQ10\\NAD+};
\node[font=\tiny, align=center, text=orange!70!black] at (-5.5, 3.1) {LDN\\Anti-inflam.};
\node[font=\tiny, align=center, text=orange!70!black] at (5.5, 3.1) {Beta blockers\\Fluids};
\node[font=\tiny, align=center, text=orange!70!black] at (-4, -4.1) {Hormone\\replacement};
\node[font=\tiny, align=center, text=orange!70!black] at (4, -4.1) {Antivirals\\LDN};

% === LEGEND ===
\begin{scope}[shift={(-5.5, -5.5)}]
    \node[font=\small\bfseries] at (5.5, 0) {Legend:};

    % Cycle node example
    \node[ellipse, draw=gray!60!black, fill=gray!15, thick,
          minimum width=1.2cm, minimum height=0.7cm, font=\tiny]
          (leg-cycle) at (2, -0.5) {Cycle};
    \node[font=\tiny, right] at (2.8, -0.5) {Vicious cycle};

    % Coupling arrow
    \draw[coupling] (4.5, -0.5) -- (5.5, -0.5);
    \node[font=\tiny, right] at (5.6, -0.5) {Bidirectional coupling};

    % Self-loop
    \draw[->, >=latex, thick, gray!50!black, looseness=6, in=120, out=60]
          (7.8, -0.5) to (7.8, -0.5);
    \node[font=\tiny, right] at (8.3, -0.5) {Self-reinforcing};

    % Intervention point
    \node[intervention, scale=0.8] at (10, -0.5) {};
    \node[font=\tiny, right] at (10.4, -0.5) {Treatment target};
\end{scope}

\end{tikzpicture}

\caption[Vicious cycle network model of ME/CFS pathophysiology]{%
    \textbf{Network model of vicious cycle coupling in ME/CFS.}
    Five major pathophysiological cycles (colored ellipses) exhibit bidirectional
    reinforcing connections (red arrows). Each cycle contains internal positive feedback
    (self-loops). Yellow stars indicate potential treatment intervention points.
    The network architecture explains why treatment at single nodes often fails:
    coupled cycles re-amplify dysfunction even when one component is partially addressed.
    Effective treatment may require simultaneous intervention at multiple nodes.
}
\label{fig:vicious-cycle-network}
\end{figure}
