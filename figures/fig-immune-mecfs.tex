% Figure: Immune Dysfunction in ME/CFS
% Paradoxical state: chronic inflammation but impaired function

\begin{figure}[htbp]
\centering
\begin{tikzpicture}[scale=1, every node/.style={scale=1},
    % Styles
    normal/.style={draw=green!70!black, fill=green!10, very thick, rounded corners, text width=3cm, align=center, minimum height=1cm},
    impaired/.style={draw=red!70!black, fill=red!15, very thick, rounded corners, text width=3.2cm, align=center, minimum height=1cm},
    severe/.style={draw=red!50!black, fill=red!25, ultra thick, rounded corners, text width=3.2cm, align=center, minimum height=1.1cm, drop shadow},
    pathological/.style={draw=red!50!black, fill=red!20, very thick, rounded corners, text width=2.6cm, align=center, minimum height=0.95cm},
    impaired-arrow/.style={-latex, very thick, red!70!black, line width=1.2pt},
    cycle-arrow/.style={-latex, ultra thick, red!50!black, line width=1.6pt},
    note/.style={font=\small\itshape, text width=2.5cm, align=left, red!60!black},
]

% Title
\node[font=\large\bfseries, red!70!black] at (0, 9.5) {ME/CFS: Immune Dysfunction Cycles};

% TOP: Failed response pathway
\begin{scope}[yshift=5.5cm]
    \node[normal] (pathogen) at (0, 2) {\textbf{Pathogen Exposure}\\[2pt] Virus, bacteria, etc.};

    \node[impaired] (innate) at (0, 0.2) {\textbf{Innate Response}\\[2pt] {\color{red!80!black}NK cells impaired}\\But chronic cytokines};
    \draw[impaired-arrow] (pathogen) -- (innate);
    \node[note, right=0.4cm of innate, anchor=west] {
        \textbullet~Paradox:\\~~~can't kill\\~~~but inflamed
    };

    \node[impaired] (adaptive) at (0, -2) {\textbf{Adaptive Response}\\[2pt] T-cell exhaustion\\Th1/Th2 imbalance};
    \draw[impaired-arrow] (innate) -- (adaptive);

    \node[impaired] (clearance) at (0, -4.2) {\textbf{Poor Clearance}\\[2pt] Persistent pathogens\\Viral reactivation};
    \draw[impaired-arrow] (adaptive) -- (clearance);

    \node[severe] (failed) at (0, -6.5) {\textbf{FAILED RESOLUTION}\\[3pt] \textit{No resolution phase}\\[2pt] Chronic inflammation};
    \draw[impaired-arrow] (clearance) -- (failed);
\end{scope}

% BOTTOM: Two vicious cycles
\begin{scope}[yshift=-5cm]
    \def\radius{2.4}

    % LEFT CYCLE: Chronic Inflammation
    \def\leftx{-4.5}
    \node[font=\small\bfseries, red!40!black] at (\leftx, 3.5) {Cycle 1: Inflammation};

    \node[pathological] (inflam) at (\leftx, \radius + 0.5)
        {\textbf{Chronic}\\  \textbf{Inflammation}\\IL-1, IL-6, TNF-$\alpha$};

    \node[pathological] (ido) at (\leftx + \radius*0.95, 0.5 + \radius*0.31)
        {\textbf{IDO}\\  \textbf{Activation}\\TRP depletion};

    \node[pathological] (atp-immune) at (\leftx + \radius*0.59, 0.5 - \radius*0.81)
        {\textbf{Energy}\\  \textbf{Deficit}\\ATP-limited};

    \node[pathological] (poor) at (\leftx - \radius*0.59, 0.5 - \radius*0.81)
        {\textbf{Poor}\\  \textbf{Control}\\Pathogens persist};

    \node[pathological] (more) at (\leftx - \radius*0.95, 0.5 + \radius*0.31)
        {\textbf{More}\\  \textbf{Activation}\\More cytokines};

    \draw[cycle-arrow, bend left=18] (inflam) to (ido);
    \draw[cycle-arrow, bend left=18] (ido) to (atp-immune);
    \draw[cycle-arrow, bend left=18] (atp-immune) to (poor);
    \draw[cycle-arrow, bend left=18] (poor) to (more);
    \draw[cycle-arrow, bend left=18] (more) to (inflam);

    % RIGHT CYCLE: Exhaustion
    \def\rightx{4.5}
    \node[font=\small\bfseries, red!40!black] at (\rightx, 3.5) {Cycle 2: Exhaustion};

    \node[pathological] (chronic) at (\rightx, \radius + 0.5)
        {\textbf{Chronic}\\  \textbf{Activation}\\Always "on"};

    \node[pathological] (tcell) at (\rightx + \radius*0.95, 0.5 + \radius*0.31)
        {\textbf{T-cell}\\  \textbf{Exhaustion}\\PD-1 up};

    \node[pathological] (nk) at (\rightx + \radius*0.59, 0.5 - \radius*0.81)
        {\textbf{NK Cell}\\  \textbf{Dysfunction}\\Low cytotoxicity};

    \node[pathological] (failclear) at (\rightx - \radius*0.59, 0.5 - \radius*0.81)
        {\textbf{Failed}\\  \textbf{Clearance}\\Viral reactivation};

    \node[pathological] (sustained) at (\rightx - \radius*0.95, 0.5 + \radius*0.31)
        {\textbf{Sustained}\\  \textbf{Signals}\\Danger signals};

    \draw[cycle-arrow, bend left=18] (chronic) to (tcell);
    \draw[cycle-arrow, bend left=18] (tcell) to (nk);
    \draw[cycle-arrow, bend left=18] (nk) to (failclear);
    \draw[cycle-arrow, bend left=18] (failclear) to (sustained);
    \draw[cycle-arrow, bend left=18] (sustained) to (chronic);

    % Connection between cycles
    \draw[cycle-arrow, <->, line width=2pt, red!60!black] (poor) -- (failclear);
    \node[font=\scriptsize, red!60!black, text width=1.8cm, align=center] at (0, -1.8) {Cycles\\reinforce};
\end{scope}

% Key point box
\node[draw=red!70!black, fill=red!5, rounded corners, text width=12cm, align=left, font=\small, inner sep=8pt] at (0, -8.5) {
\textbf{Paradoxical immune state:} Chronically inflamed yet unable to clear pathogens.\\[4pt]
\textbullet~\textbf{Cycle 1 (Inflammation):} Chronic cytokines $\rightarrow$ IDO activation $\rightarrow$ energy deficit $\rightarrow$ poor pathogen control $\rightarrow$ more inflammation\\
\textbullet~\textbf{Cycle 2 (Exhaustion):} Chronic activation $\rightarrow$ T-cell/NK exhaustion $\rightarrow$ failed clearance $\rightarrow$ sustained danger signals\\[4pt]
The two cycles reinforce each other, creating persistent immune dysfunction.
};

\end{tikzpicture}
\caption{ME/CFS immune dysfunction with chronic inflammation and exhaustion cycles.}
\label{fig:immune-mecfs}
\end{figure}
