% Figure: Normal HPA Axis Function
% Appropriate stress response with negative feedback control

\begin{figure}[htbp]
\centering
\begin{tikzpicture}[
    node distance=2.5cm,  % Global minimum vertical spacing
    % Styles
    process/.style={draw=green!70!black, fill=green!10, very thick, rounded corners, text width=4.5cm, align=center, minimum height=1.2cm},
    adaptive/.style={draw=green!70!black, fill=green!20, very thick, rounded corners, text width=4.5cm, align=center, minimum height=1.2cm},
    output/.style={draw=green!50!black, fill=green!30, ultra thick, rounded corners, text width=4.5cm, align=center, minimum height=1.3cm, drop shadow},
    arrow/.style={-latex, very thick, green!70!black, line width=1.2pt},
    feedback/.style={-latex, thick, blue!70!black, dashed, line width=1.1pt},
    note/.style={font=\small\itshape, text width=3.5cm, align=left, green!40!black},
]

% Title
\node[font=\large\bfseries, green!70!black] (title) at (0, 0) {Normal HPA Axis Function};

% Stressor
\node[process, below=3cm of title] (stressor) {\textbf{Stressor}\\[2pt] Physical or psychological};

% Hypothalamus
\node[process, below=of stressor] (hypothalamus) {\textbf{Hypothalamus}\\[2pt] CRH release};
\draw[arrow] (stressor) -- (hypothalamus);

% Pituitary
\node[adaptive, below=of hypothalamus] (pituitary) {\textbf{Pituitary}\\[2pt] ACTH release};
\draw[arrow] (hypothalamus) -- (pituitary);

% Adrenal
\node[adaptive, below=of pituitary] (adrenal) {\textbf{Adrenal Cortex}\\[2pt] Cortisol release};
\draw[arrow] (pituitary) -- (adrenal);
\node[note, left=1.5cm of adrenal, anchor=east] {
    \textbullet~Cortisol rhythm:\\~~~High AM\\~~~Low PM
};

% Cortisol effects
\node[adaptive, below=of adrenal] (cortisol) {\textbf{Cortisol Effects}\\[2pt] Mobilize energy\\Anti-inflammatory};
\draw[arrow] (adrenal) -- (cortisol);

% Negative feedback (right side)
\node[process, text width=3.8cm, right=5.5cm of pituitary] (feedback-box) {\textbf{Negative Feedback}\\[2pt] Cortisol inhibits\\CRH \& ACTH};
\draw[feedback, bend left=25] (cortisol.east) to (feedback-box.south);
\draw[feedback] (feedback-box) -- (hypothalamus);
\draw[feedback] (feedback-box) -- (pituitary);

% Recovery
\node[output, below=of cortisol] (recovery) {\textbf{STRESS RESOLUTION}\\[3pt] \textit{Return to baseline}\\[2pt] Homeostasis restored};
\draw[arrow] (cortisol) -- (recovery);

% Key point box
\node[draw=green!70!black, fill=green!5, rounded corners, text width=10cm, align=left, font=\small, inner sep=8pt, below=2.5cm of recovery] {
\textbf{Key characteristics:}\\[4pt]
\textbullet~Stressor activates hypothalamus $\rightarrow$ pituitary $\rightarrow$ adrenal cascade\\
\textbullet~Cortisol mobilizes energy and suppresses inflammation\\
\textbullet~Negative feedback prevents over-activation\\
\textbullet~System returns to baseline after stress resolves
};

\end{tikzpicture}
\caption{Normal HPA axis stress response with negative feedback control.}
\label{fig:hpa-axis-normal}
\end{figure}
