% Figure: HPA Axis Dysregulation in ME/CFS
% Blunted response, excessive feedback, systemic consequences

\begin{figure}[htbp]
\centering
\begin{tikzpicture}[
    node distance=2.5cm,
    scale=0.85, every node/.style={scale=0.85},
    % Styles
    normal/.style={draw=green!70!black, fill=green!10, very thick, rounded corners, text width=3cm, align=center, minimum height=1cm},
    impaired/.style={draw=red!70!black, fill=red!15, very thick, rounded corners, text width=3.2cm, align=center, minimum height=1cm},
    severe/.style={draw=red!50!black, fill=red!25, ultra thick, rounded corners, text width=3.2cm, align=center, minimum height=1.1cm, drop shadow},
    pathological/.style={draw=red!50!black, fill=red!20, very thick, rounded corners, text width=2.8cm, align=center, minimum height=0.95cm},
    impaired-arrow/.style={-latex, very thick, red!70!black, line width=1.2pt},
    feedback/.style={-latex, thick, red!70!black, dashed, line width=1.1pt},
    cascade-arrow/.style={-latex, thick, orange!80!black, dashed, line width=1.1pt},
    note/.style={font=\small\itshape, text width=2.4cm, align=left, red!60!black},
]

% Title
\node[font=\large\bfseries, red!70!black] at (0, 9.5) {ME/CFS: HPA Axis Dysregulation};

% LEFT SIDE: Impaired pathway
\begin{scope}[xshift=-4.5cm]
    % Stressor
    \node[normal] (stressor) at (0, 7.5) {\textbf{Stressor}\\[2pt] Same intensity};

    % Hypothalamus - blunted
    \node[impaired] (hypothalamus) at (0, 5.8) {\textbf{Hypothalamus}\\[2pt] {\color{red!80!black}Blunted CRH}\\Hypersensitive feedback};
    \draw[impaired-arrow] (stressor) -- (hypothalamus);

    % Pituitary - impaired
    \node[impaired] (pituitary) at (0, 3.9) {\textbf{Pituitary}\\[2pt] {\color{red!80!black}Low ACTH}\\Reduced reserve};
    \draw[impaired-arrow] (hypothalamus) -- (pituitary);

    % Adrenal - atrophy
    \node[impaired] (adrenal) at (0, 2) {\textbf{Adrenal Cortex}\\[2pt] {\color{red!80!black}Atrophy/exhaustion}\\Low output};
    \draw[impaired-arrow] (pituitary) -- (adrenal);
    \node[note, left=0.25cm of adrenal, anchor=east] {
        \textbullet~Flattened\\~~~rhythm\\~~~(low all day)
    };

    % Low cortisol
    \node[severe] (cortisol) at (0, -0.2) {\textbf{LOW CORTISOL}\\[3pt] \textit{Cannot respond to stress}\\[2pt] {\color{red!80!black}Energy \& inflammation issues}};
    \draw[impaired-arrow] (adrenal) -- (cortisol);

    % Excessive feedback
    \node[impaired, text width=2.8cm] (feedback-box) at (3.8, 3.9) {\textbf{Excessive Feedback}\\[2pt] Over-suppression};
    \draw[feedback, bend left=20] (cortisol.east) to (feedback-box.south);
    \draw[feedback] (feedback-box) -- (hypothalamus);
    \draw[feedback] (feedback-box) -- (pituitary);

    % No recovery
    \node[impaired] (no-recovery) at (0, -2.5) {\textbf{No Resolution}\\[2pt] Persistent dysfunction};
    \draw[impaired-arrow] (cortisol) -- (no-recovery);
\end{scope}

% RIGHT SIDE: System failures
\begin{scope}[xshift=4.5cm]
    % Central dysfunction
    \node[pathological, minimum width=3.2cm] (hpa) at (0, 5) {\textbf{HPA AXIS}\\  \textbf{DYSFUNCTION}};

    % Six consequences in 2x3 grid
    \node[pathological] (stress) at (-2, 2.5) {\textbf{Stress}\\  \textbf{Intolerance}\\Can't cope\\Crashes};

    \node[pathological] (inflam) at (2, 2.5) {\textbf{Inflammation}\\  \textbf{Unchecked}\\No cortisol brake\\Cytokines high};

    \node[pathological] (metabolic) at (-2, 0) {\textbf{Metabolic}\\Hypoglycemia\\Poor energy\\mobilization};

    \node[pathological] (autonomic) at (2, 0) {\textbf{Autonomic}\\Orthostatic\\intolerance\\Low volume};

    \node[pathological] (mood) at (-2, -2.5) {\textbf{Mood}\\Depression\\Anxiety\\Brain fog};

    \node[pathological] (immune) at (2, -2.5) {\textbf{Immune}\\Th1/Th2\\imbalance\\Autoimmunity};

    % Cascade arrows
    \draw[cascade-arrow] (hpa) -- (stress);
    \draw[cascade-arrow] (hpa) -- (inflam);
    \draw[cascade-arrow] (hpa) -- (metabolic);
    \draw[cascade-arrow] (hpa) -- (autonomic);
    \draw[cascade-arrow] (hpa) -- (mood);
    \draw[cascade-arrow] (hpa) -- (immune);

    % Feedback from inflammation
    \draw[cascade-arrow, bend left=30, line width=1.5pt] (inflam.north) to (hpa.east);
    \node[font=\scriptsize, red!50!black, text width=1.5cm, align=center] at (3.5, 4) {Inflammation\\worsens HPA};
\end{scope}

% Key point box
\node[draw=red!70!black, fill=red!5, rounded corners, text width=12cm, align=left, font=\small, inner sep=8pt] at (0, -6.5) {
\textbf{Blunted stress response:}\\[4pt]
\textbullet~Hypersensitive negative feedback over-suppresses HPA axis\\
\textbullet~Low cortisol $\rightarrow$ cannot mobilize energy for stress response\\
\textbullet~Unchecked inflammation, metabolic instability, autonomic dysfunction\\
\textbullet~Flattened circadian cortisol rhythm (low throughout day)\\[4pt]
Chronic inflammation feeds back to further worsen HPA function.
};

\end{tikzpicture}
\caption{ME/CFS HPA axis dysregulation with blunted response and systemic consequences.}
\label{fig:hpa-axis-mecfs}
\end{figure}
