% Figure: Oxidative Stress Vicious Cycle in ME/CFS
% Imbalance creates self-perpetuating damage cascade

\begin{figure}[htbp]
\centering
\begin{tikzpicture}[
    node distance=2.5cm,  % Global minimum vertical spacing
    scale=0.85, every node/.style={scale=0.85},
    % Styles
    impaired/.style={draw=red!70!black, fill=red!15, very thick, rounded corners, text width=4.5cm, align=center, minimum height=1cm},
    severe/.style={draw=red!50!black, fill=red!25, ultra thick, rounded corners, text width=7.5cm, align=center, minimum height=1.1cm},
    pathological/.style={draw=red!50!black, fill=red!20, very thick, rounded corners, text width=3.5cm, align=center, minimum height=1cm},
    impaired-arrow/.style={-latex, very thick, red!70!black, line width=1.2pt},
    cycle-arrow/.style={-latex, ultra thick, red!50!black, line width=1.6pt},
    note/.style={font=\small\itshape, text width=2.5cm, align=left, red!60!black},
]

% Title
\node[font=\large\bfseries, red!70!black] (title) at (0, 0) {ME/CFS: Oxidative Stress Vicious Cycle};

% TOP: Imbalance
\begin{scope}[yshift=-3cm]
    % Excessive ROS (left)
    \node[impaired, below left=3cm and 4cm of title] (ros) {\textbf{Excessive ROS}\\[2pt] 5--10\% ETC leak\\Complex I/III damage};

    % Depleted antioxidants (right)
    \node[impaired, below right=3cm and 4cm of title] (antioxidants) {\textbf{Depleted Defense}\\[2pt] GSH $\downarrow$30\%\\SOD/GPx impaired};

    % Imbalance
    \node[severe, below=5.5cm of title] (imbalance) {\textbf{ROS IMBALANCE}\\[3pt] \textit{Chronic oxidative stress}\\[2pt] Production $\gg$ Neutralization};

    \draw[impaired-arrow] (ros) -- (imbalance);
    \draw[impaired-arrow] (antioxidants) -- (imbalance);
\end{scope}

% BOTTOM: Vicious cycle with proper radius
\begin{scope}[yshift=-18cm]
    \def\radius{4.2}  % Increased from 4.0cm for better spacing
    \def\centerx{0}
    \def\centery{0}

    % Node 1: Mitochondrial damage (top)
    \node[pathological] (mito) at (\centerx, \centery + \radius)
        {\textbf{Mitochondrial Damage}\\ETC complexes\\Membrane integrity};

    % Node 2: Increased ROS (right)
    \node[pathological] (ros-up) at (\centerx + \radius*0.95, \centery + \radius*0.31)
        {\textbf{More ROS}\\Increased leakage\\Superoxide surge};

    % Node 3: Lipid peroxidation (bottom right)
    \node[pathological] (lipid) at (\centerx + \radius*0.59, \centery - \radius*0.81)
        {\textbf{Lipid Damage}\\4-HNE toxic\\MDA accumulates};

    % Node 4: Protein damage (bottom left)
    \node[pathological] (protein) at (\centerx - \radius*0.59, \centery - \radius*0.81)
        {\textbf{Protein Damage}\\Carbonylation\\Aconitase loss};

    % Node 5: DNA damage (left)
    \node[pathological] (dna) at (\centerx - \radius*0.95, \centery + \radius*0.31)
        {\textbf{DNA Damage}\\8-OHdG\\mtDNA mutations};

    % Cycle arrows (clockwise)
    \draw[cycle-arrow, bend left=18] (mito) to (ros-up);
    \draw[cycle-arrow, bend left=18] (ros-up) to (lipid);
    \draw[cycle-arrow, bend left=18] (lipid) to (protein);
    \draw[cycle-arrow, bend left=18] (protein) to (dna);
    \draw[cycle-arrow, bend left=18] (dna) to (mito);

    % Central label
    \node[font=\bfseries, red!40!black] at (\centerx, \centery) {VICIOUS};
    \node[font=\bfseries, red!40!black] at (\centerx, \centery - 0.4) {CYCLE};
\end{scope}

% Arrow from imbalance to cycle
\draw[impaired-arrow] (imbalance) -- (mito);

% Key point box
\node[draw=red!70!black, fill=red!5, rounded corners, text width=11cm, align=left, font=\small, inner sep=8pt] at (0, -27) {
\textbf{Self-perpetuating damage cascade:}\\[4pt]
\textbullet~Damaged mitochondria leak more electrons (5--10\% vs. $<$2\%)\\
\textbullet~Excess ROS damages lipids (4-HNE, MDA), proteins, and DNA\\
\textbullet~Damaged components further impair mitochondrial function\\
\textbullet~Depleted antioxidants (GSH $\downarrow$30\%) cannot neutralize ROS\\[4pt]
Breaking this cycle requires both reducing ROS production and restoring antioxidant capacity.
};

\end{tikzpicture}
\caption{ME/CFS oxidative stress vicious cycle with self-perpetuating damage.}
\label{fig:oxidative-stress-mecfs}
\end{figure}
