% TARGET: ch07-immune-dysfunction.tex - after autoantibody discussion
% TASK: 3.1 (Opus) - Write novel mechanistic hypothesis
% STATUS: Placeholder - content to be developed

% Hypothesis framework:
% - Observation: Pediatric recovery despite similar autoantibody prevalence at onset
% - Hypothesis: Developing immune systems actively DELETE aberrant memory cells via
%   peripheral tolerance mechanisms that adults lack
% - Mechanism:
%   - Pubertal immune reorganization includes "quality control" checkpoints
%   - Autoreactive B cells undergo receptor editing or apoptosis
%   - Adults lack these developmental tolerance signals
%   - Result: Autoantibodies persist in adults, decline in children
% - Testable predictions:
%   - Declining autoantibody titers in recovering children vs stable in adults
%   - Gene signatures of active tolerance mechanisms in pediatric samples
%   - Correlation of tolerance markers with recovery
% - Treatment implications:
%   - Re-induce tolerance checkpoints (Treg therapy, low-dose antigen)
%   - Timing of B cell depletion (adolescents vs adults)
%
% Requirements:
% - Use \begin{hypothesis}[Immune Memory Pruning in Development] environment
% - Cross-reference to pediatric chapters, autoantibody data
% - Include appropriate uncertainty language
%
% Target length: 800-1000 words