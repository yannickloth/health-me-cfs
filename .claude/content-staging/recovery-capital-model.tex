% SOURCE: Literature/pediatric-insights/scientific-insights-pediatric-chapters.md
% Section 5.1 - Suggested Addition to Chapter 13 (Integrative Models)
% STATUS: Raw content - requires review and integration

\begin{speculation}[Recovery Capital Model]
\label{spec:recovery-capital}
We propose a conceptual framework of ``Recovery Capital''---the cumulative
biological capacity for recovery that is consumed by severe post-exertional
malaise episodes and regenerated over time. In this model, children possess
high initial Recovery Capital (developmental plasticity, immune renewal,
metabolic flexibility) and regenerate it rapidly, while adults start with
lower capital and regenerate slowly if at all. Each severe crash ``spends''
Recovery Capital through epigenetic changes, accumulated cellular damage,
and immune exhaustion. Once Recovery Capital is depleted below a threshold,
recovery becomes unlikely. This framework explains why strict pacing (capital
preservation) and early intervention (maximizing capital before depletion)
are particularly critical in pediatric patients, and why aggressive early
treatment in adult patients may preserve recovery potential that would
otherwise be lost.
\end{speculation}
