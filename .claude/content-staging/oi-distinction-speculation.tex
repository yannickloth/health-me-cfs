% SOURCE: Literature/pediatric-insights/scientific-insights-pediatric-chapters.md
% Section 5.1 - Suggested Addition to Chapter 10 (Cardiovascular)
% STATUS: Raw content - requires review and integration

\begin{speculation}[Functional vs. Structural OI Distinction]
\label{spec:oi-distinction}
The higher prevalence but better reversibility of orthostatic intolerance in
pediatric ME/CFS (70--90\% prevalence with high response to treatment) compared
to adult disease suggests qualitatively different mechanisms. We speculate that
pediatric OI may primarily represent functional miscalibration of an autonomic
system still undergoing developmental tuning, while adult OI may involve
structural damage (small fiber neuropathy, receptor autoantibody-mediated
damage) accumulated over illness duration. This distinction would explain why
OI treatment in children often produces multi-domain improvement (fatigue,
cognition, wellbeing) while adult responses may be more limited. If correct,
this supports aggressive early OI treatment in both populations to prevent
functional miscalibration from progressing to structural damage.
\end{speculation}
