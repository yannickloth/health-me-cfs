% TARGET: ch14b treatment chapter - pacing section
% TASK: 4.2 (Sonnet) - Write periodized activity protocol
% STATUS: Placeholder - content to be developed

% Content framework:
% - Cross-domain insight: Overtraining syndrome (OTS) similarities but better outcomes
% - Key difference: Athletes recover with structured rest-activity cycles
% - Adaptation for ME/CFS:
%   - Periodization principle: Rhythmic activity patterns with recovery weeks
%   - Biomarker monitoring: HRV tracking
%   - Recovery nutrition: Anti-inflammatory loading around planned activity
%   - Sleep priority: Extension during recovery phases (10+ hours)
% - Example 8-week cycle:
%   - Weeks 1-2: Complete rest (deload)
%   - Weeks 3-4: Minimal sustainable activity only
%   - Weeks 5-6: Slight increase if HRV stable
%   - Weeks 7-8: 80% baseline if tolerating
%   - Repeat
% - Distinction from standard pacing: Structured periodicity vs constant envelope
% - Cautions: Not for severe patients, must monitor for PEM
%
% Requirements:
% - Use \begin{protocol}[Periodized Activity Protocol] environment
% - Cross-reference to HRV-guided section
% - Note this is experimental adaptation requiring monitoring
%
% Target length: 700-900 words