% TARGET: ch08-neuroimmune.tex - after microglial activation discussion
% TASK: 3.3 (Opus) - Write glial maturation window hypothesis
% STATUS: Placeholder - content to be developed

% Hypothesis framework:
% - Observation gap: Document discusses microglial activation (Nakatomi PET) but not
%   developmental context
% - Hypothesis: Adolescent brain undergoes microglial reprogramming that can "reset"
%   neuroinflammatory states
% - Developmental neuroscience background:
%   - Microglia dramatically reprogram during adolescence
%   - Synaptic pruning peak around puberty requires coordinated microglial activity
%   - This pruning may "sweep away" aberrant activation patterns
%   - Process largely complete by early 20s
% - ME/CFS connection:
%   - Chronic microglial activation documented (PET data)
%   - In children, developmental pruning may reset to surveillance state
%   - Adults lack the signal to reset "locked" activation
%   - Result: Persistent neuroinflammation in adults
% - Treatment implications:
%   - CSF-1R inhibitors force microglial turnover (PLX5622, pexidartinib)
%   - Fasting-mimicking diets promote microglial turnover
%   - BDNF promoters may restore plasticity
% - Research direction: PET imaging comparing pediatric vs adult ME/CFS
%
% Requirements:
% - Use \begin{speculation}[Glial Maturation Window] environment
% - Reference Nakatomi 2014 PET study
% - Cross-reference to pediatric advantage
%
% Target length: 900-1100 words