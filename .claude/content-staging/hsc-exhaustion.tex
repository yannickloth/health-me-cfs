% TARGET: ch13-integrative-models.tex - after multi-system models
% TASK: 3.2 (Opus) - Write stem cell exhaustion hypothesis
% STATUS: Placeholder - content to be developed

% Hypothesis framework:
% - Concept: ME/CFS involves premature exhaustion of hematopoietic stem cells
% - Mechanism:
%   - Initial infection triggers massive immune expansion (draws on HSC reserves)
%   - Repeated crashes cause additional waves (further HSC depletion)
%   - Children have larger HSC pools + higher regenerative capacity
%   - Adults show progressive HSC decline (aging + disease)
%   - Eventually HSCs cannot regenerate healthy immune cells -> permanent dysfunction
% - Cross-domain insight: HSC exhaustion is hallmark of aging, accelerated in ME/CFS
% - Biomarker development:
%   - CD34+ circulating progenitors
%   - HSC clonality (diversity = health)
%   - Telomere length in HSC compartment
%   - Single-cell RNA-seq of bone marrow
% - Treatment implications:
%   - HSC-protective agents (reduce draws on reserves)
%   - HSC regeneration (fasting-mimicking diet, growth factors)
%   - Autologous HSC boost (speculative)
% - Connection to Recovery Capital: HSC reserve IS part of Recovery Capital
%
% Requirements:
% - Use \begin{speculation}[Stem Cell Exhaustion Model] environment
% - Link to epigenetic aging, Recovery Capital framework
% - Cite aging biology literature where applicable
%
% Target length: 1000-1200 words