% SOURCE: Literature/pediatric-insights/scientific-insights-pediatric-chapters.md
% Section 5.1 - Suggested Addition to Chapter 5 (Disease Course)
% STATUS: Raw content - requires review and integration

\begin{hypothesis}[Developmental Plasticity Window]
\label{hyp:developmental-window}
The dramatically better prognosis in pediatric ME/CFS (54--94\% improvement)
versus adult disease ($\leq$22\%) suggests that biological factors related to
developmental plasticity fundamentally affect recovery potential. We propose
that this reflects: (1) ongoing epigenetic reprogramming during development
that can override ME/CFS-associated changes, (2) active immune cell turnover
that clears dysfunctional cell populations, and (3) metabolic flexibility that
allows compensation for mitochondrial dysfunction. This plasticity appears to
narrow with age and illness duration, supporting the urgency of early
intervention in pediatric cases and suggesting that aggressive early treatment
in adult patients may preserve recovery potential.
\end{hypothesis}