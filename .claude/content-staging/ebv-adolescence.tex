% TARGET: ch07-immune-dysfunction.tex - in autoantibody or infectious triggers section
% TASK: 6.2 (Sonnet) - Write EBV-adolescence mechanistic connection
% STATUS: Placeholder - content to be developed

% Content framework:
% - Epidemiological observation: EBV mono common trigger in adolescents
% - Immunological timing: EBV infects B cells during pubertal immune maturation
% - Hypothesis: Creates abnormal memory B cell populations producing GPCR autoantibodies
% - Age-dependent outcomes:
%   - Younger children: Ongoing immune development may clear aberrant clones
%   - Adolescents (edge of maturation): Populations may persist
%   - Adults: Without renewal mechanisms, become permanent
% - Treatment implication: B cell depletion (rituximab) might be particularly effective
%   in adolescents/young adults with recent EBV-triggered ME/CFS
% - Research direction: Compare GPCR autoantibody titers by age at onset + EBV status
%
% Requirements:
% - Use hypothesis or speculation environment (~600-800 words)
% - Cross-reference to infectious triggers section, autoantibody data
% - Link to immune memory pruning hypothesis
%
% Target length: 700-900 words