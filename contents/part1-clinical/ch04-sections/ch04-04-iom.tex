\label{sec:iom}

The Institute of Medicine (now National Academy of Medicine) published diagnostic criteria in 2015 following a comprehensive systematic review~\cite{IOM2015}. The IOM report proposed renaming the condition ``Systemic Exertion Intolerance Disease'' (SEID) to emphasize the central role of post-exertional malaise and to move away from the stigmatizing ``chronic fatigue'' label. However, the SEID terminology has seen limited clinical adoption.

\subsection{Required Core Symptoms}

\begin{requirement}[IOM Diagnostic Algorithm]
\label{req:iom}
Diagnosis requires ALL THREE of the following core symptoms to be present:

\paragraph{1. Substantial Reduction or Impairment in Activity Level (MANDATORY)}

A substantial reduction or impairment in the ability to engage in pre-illness levels of occupational, educational, social, or personal activities that:
\begin{itemize}
    \item Persists for \textbf{more than 6 months}
    \item Is accompanied by fatigue (often profound)
    \item Is of \textbf{new or definite onset} (not lifelong)
    \item Is \textbf{not the result of ongoing excessive exertion}
    \item Is \textbf{not substantially alleviated by rest}
\end{itemize}

\paragraph{2. Post-Exertional Malaise (PEM) --- MANDATORY}

Worsening of symptoms following physical, cognitive, or emotional exertion that would not have caused a problem before illness. Characteristics include:
\begin{itemize}
    \item Symptoms typically worsen 12--48 hours after activity
    \item Often leads to relapse lasting days, weeks, or longer
    \item Exertion threshold for triggering symptoms is low
    \item Recovery is prolonged
\end{itemize}

The IOM emphasizes that PEM is \textbf{the hallmark symptom} that distinguishes ME/CFS from other fatiguing conditions.

\paragraph{3. Unrefreshing Sleep (MANDATORY)}

Patients wake feeling unrefreshed regardless of sleep duration. Sleep may be:
\begin{itemize}
    \item Disrupted (frequent awakenings, difficulty initiating sleep)
    \item Prolonged (hypersomnia with no restoration)
    \item Reversed sleep/wake cycle
\end{itemize}

The exhaustion persists despite adequate sleep duration.
\end{requirement}

\subsection{Additional Required Symptoms}

\begin{requirement}[At Least ONE of the Following]
\label{req:iom-additional}

\paragraph{Cognitive Impairment}
Problems with thinking, memory, information processing, or executive function. May include:
\begin{itemize}
    \item Difficulty finding words, storing and retrieving information
    \item Slowed processing speed
    \item Inability to focus or multitask
    \item Problems with short-term memory
\end{itemize}

Cognitive symptoms may worsen with physical or mental exertion, emotional stress, or time pressure.

\paragraph{OR}

\paragraph{Orthostatic Intolerance}
Worsening of symptoms upon assuming or maintaining upright posture. May include:
\begin{itemize}
    \item Lightheadedness, dizziness, fainting
    \item Worsening fatigue or cognitive impairment when upright
    \item Palpitations, nausea
    \item Symptoms improve (but may not resolve) when lying down
\end{itemize}

Objective findings may include abnormal heart rate or blood pressure responses during tilt table testing or standing test.
\end{requirement}

\subsection{Diagnostic Algorithm Structure}

The IOM criteria can be formalized as a logical algorithm:

\begin{equation}
\text{ME/CFS}_{\text{IOM}} = \left\{
\begin{array}{l}
\text{Substantial Activity Reduction} \\
\land \text{ Post-Exertional Malaise} \\
\land \text{ Unrefreshing Sleep} \\
\land \left( \text{Cognitive Impairment} \lor \text{Orthostatic Intolerance} \right) \\
\land \text{ Duration} \geq 6 \text{ months} \\
\land \text{ Exclusions ruled out}
\end{array}
\right\}
\end{equation}

This represents a \textbf{minimal sufficient set}: three universal core features plus at least one of two common manifestations.

\subsection{Exclusions and Comorbidities}

\begin{itemize}
    \item \textbf{Exclusions}: Medical conditions that could fully explain the symptoms must be ruled out through appropriate testing (hypothyroidism, anemia, sleep apnea, etc.)
    \item \textbf{Comorbidities allowed}: Fibromyalgia, irritable bowel syndrome, depression, and anxiety frequently co-occur and do not exclude ME/CFS diagnosis
    \item \textbf{Important distinction}: Comorbid depression is reactive (consequence of severe disability) rather than causative
\end{itemize}

\subsection{Strengths and Limitations}

\begin{observation}[IOM Strengths]
\label{obs:iom-strengths}
The IOM criteria offer several advantages:
\begin{itemize}
    \item \textbf{Simplicity}: Four required features (3 core + 1 of 2 additional) make diagnosis straightforward
    \item \textbf{High sensitivity}: Captures broader range of ME/CFS patients than ICC or Canadian Consensus
    \item \textbf{Evidence-based}: Derived from systematic review identifying most discriminating symptoms
    \item \textbf{PEM emphasis}: Recognizes post-exertional malaise as the pathognomonic feature
    \item \textbf{Clinical practicality}: Feasible in primary care settings without extensive symptom checklists
    \item \textbf{Rapid assessment}: Can be evaluated in a standard office visit
\end{itemize}
\end{observation}

\begin{warning}[IOM Limitations]
\label{warn:iom-limitations}
The simplified structure creates potential issues:
\begin{itemize}
    \item \textbf{Reduced specificity}: More inclusive criteria may capture patients with other conditions (long COVID, post-viral fatigue that will resolve)
    \item \textbf{Heterogeneity}: Broader patient population increases phenotypic variance in research cohorts
    \item \textbf{Cognitive OR orthostatic requirement}: Patients may meet criteria with only one of these domains, potentially missing multi-system nature
    \item \textbf{SEID terminology rejected}: Proposed name change has not gained acceptance in patient or research communities
    \item \textbf{Set-theoretic relationship}: $\text{ICC} \subset \text{Canadian} \subset \text{IOM}$ --- IOM captures the broadest population
\end{itemize}
\end{warning}

\subsection{Clinical and Research Application}

\begin{observation}[When to Use IOM Criteria]
\label{obs:iom-application}
\textbf{For clinical diagnosis}: The IOM criteria are excellent for primary care and general practice:
\begin{itemize}
    \item Simple enough for non-specialists to apply
    \item High sensitivity ensures few false negatives
    \item Enables early diagnosis and intervention
\end{itemize}

\textbf{For research}: The IOM criteria are appropriate when:
\begin{itemize}
    \item Study aims to represent the full ME/CFS population
    \item Recruitment needs to be pragmatic and efficient
    \item Results should generalize to clinical settings
\end{itemize}

\textbf{Not optimal for}: Mechanistic research or treatment trials requiring homogeneous cohorts (use ICC or Canadian Consensus with biomarker stratification instead).
\end{observation}

