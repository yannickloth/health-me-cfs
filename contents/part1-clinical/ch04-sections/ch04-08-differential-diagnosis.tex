\section{Differential Diagnosis}
\label{sec:differential}

ME/CFS is a diagnosis of exclusion, requiring careful evaluation to rule out other conditions that can present with similar symptoms. This section addresses conditions that can mimic ME/CFS, distinguishing features, and the critical distinction between alternative diagnoses and comorbidities.

\subsection{Conditions That Can Mimic ME/CFS}

\subsubsection{Endocrine Disorders}

\begin{requirement}[Must Rule Out Before Diagnosing ME/CFS]
\label{req:endocrine-exclusions}

\paragraph{Hypothyroidism:}
\begin{itemize}
    \item \textbf{Symptoms}: Fatigue, cognitive impairment (``brain fog''), cold intolerance, weight gain, constipation
    \item \textbf{Distinguishing features}: Gradual onset, no post-exertional malaise, responds to thyroid replacement
    \item \textbf{Testing}: TSH, free T4; if TSH elevated and free T4 low, hypothyroidism is confirmed
    \item \textbf{Note}: Subclinical hypothyroidism (mildly elevated TSH with normal T4) is controversial; may contribute to fatigue but is insufficient to explain ME/CFS severity
\end{itemize}

\paragraph{Addison Disease (Primary Adrenal Insufficiency):}
\begin{itemize}
    \item \textbf{Symptoms}: Profound fatigue, orthostatic hypotension, salt craving, hyperpigmentation
    \item \textbf{Distinguishing features}: Progressive worsening, life-threatening if untreated, responds to cortisol replacement
    \item \textbf{Testing}: Morning cortisol, ACTH stimulation test; electrolytes show hyponatremia and hyperkalemia
\end{itemize}

\paragraph{Diabetes Mellitus:}
\begin{itemize}
    \item \textbf{Symptoms}: Fatigue, polyuria, polydipsia, weight loss
    \item \textbf{Testing}: Fasting glucose, HbA1c
\end{itemize}
\end{requirement}

\subsubsection{Sleep Disorders}

\begin{warning}[Obstructive Sleep Apnea Can Fully Mimic ME/CFS]
\label{warn:osa-mimic}

\paragraph{Obstructive Sleep Apnea (OSA):}
OSA is a critical exclusion because it can produce a symptom profile nearly identical to ME/CFS:
\begin{itemize}
    \item \textbf{Symptoms}: Profound fatigue, unrefreshing sleep, cognitive impairment, morning headaches
    \item \textbf{Distinguishing features}:
    \begin{itemize}
        \item Snoring, witnessed apneas (ask bed partner)
        \item Obesity (BMI >30) is common but not required
        \item \textbf{Critical}: OSA patients do NOT have post-exertional malaise with delayed onset
        \item Improvement with CPAP treatment (if true OSA, dramatic improvement within weeks)
    \end{itemize}
    \item \textbf{Testing}: Polysomnography (sleep study); apnea-hypopnea index (AHI) $\geq 5$ events/hour is diagnostic
    \item \textbf{Important}: OSA and ME/CFS can coexist; treating comorbid OSA improves but does not cure ME/CFS
\end{itemize}

\paragraph{Upper Airway Resistance Syndrome (UARS):}
\begin{itemize}
    \item Milder form of sleep-disordered breathing
    \item May have normal AHI but increased respiratory effort-related arousals (RERAs)
    \item Presents with fatigue and unrefreshing sleep similar to ME/CFS
\end{itemize}

\paragraph{Idiopathic Hypersomnia:}
\begin{itemize}
    \item Excessive daytime sleepiness despite adequate sleep duration
    \item No post-exertional malaise
    \item Multiple sleep latency test (MSLT) shows short sleep latency
\end{itemize}
\end{warning}

\subsubsection{Autoimmune and Inflammatory Diseases}

\begin{observation}[Inflammatory Markers Distinguish ME/CFS from Autoimmune Disease]
\label{obs:inflammatory-distinction}
The following autoimmune and inflammatory diseases share symptoms with ME/CFS but can be distinguished by specific biomarkers and clinical features.
\end{observation}

\textbf{Systemic Lupus Erythematosus (SLE):}
\begin{itemize}
    \item \textbf{Symptoms}: Fatigue, arthralgia, cognitive impairment (``lupus fog''), photosensitivity
    \item \textbf{Distinguishing features}: Malar rash, serositis (pleuritis, pericarditis), renal involvement
    \item \textbf{Testing}: ANA positive (high titer, typically $\geq 1:160$), anti-dsDNA, anti-Sm antibodies; low complement (C3, C4); elevated ESR/CRP during flares
    \item \textbf{Key distinction}: SLE has \textit{elevated} inflammatory markers; ME/CFS has \textit{normal or low} ESR/CRP
\end{itemize}

\textbf{Sjögren Syndrome:}
\begin{itemize}
    \item \textbf{Symptoms}: Fatigue, dry eyes (keratoconjunctivitis sicca), dry mouth (xerostomia)
    \item \textbf{Distinguishing features}: Objective evidence of decreased tear/saliva production
    \item \textbf{Testing}: Anti-Ro (SSA), anti-La (SSB) antibodies; Schirmer test, salivary flow rate
\end{itemize}

\textbf{Rheumatoid Arthritis:}
\begin{itemize}
    \item \textbf{Symptoms}: Fatigue, joint pain
    \item \textbf{Distinguishing features}: Joint swelling, morning stiffness $>1$ hour, symmetric small joint involvement
    \item \textbf{Testing}: Rheumatoid factor (RF), anti-CCP antibodies, elevated ESR/CRP
\end{itemize}

\textbf{Multiple Sclerosis (MS):}
\begin{itemize}
    \item \textbf{Symptoms}: Fatigue, cognitive impairment, sensory disturbances
    \item \textbf{Distinguishing features}: Focal neurological deficits (optic neuritis, weakness, sensory loss), relapsing-remitting pattern
    \item \textbf{Testing}: MRI brain and spine (demyelinating lesions disseminated in space and time), CSF oligoclonal bands
\end{itemize}

\subsubsection{Hematologic Disorders}

\begin{requirement}[Anemia Workup]
\label{req:anemia-exclusion}
Hematologic disorders must be ruled out in the differential diagnosis of ME/CFS.
\end{requirement}

\textbf{Iron Deficiency Anemia:}
\begin{itemize}
    \item \textbf{Symptoms}: Fatigue, dyspnea on exertion, pica (ice chewing)
    \item \textbf{Testing}: CBC shows microcytic anemia (low MCV); ferritin low (<30 ng/mL)
    \item \textbf{Note}: Iron deficiency \textit{without anemia} (normal hemoglobin, low ferritin) can cause fatigue and restless legs syndrome; should be treated but is insufficient to explain ME/CFS severity
\end{itemize}

\textbf{Vitamin B12 Deficiency:}
\begin{itemize}
    \item \textbf{Symptoms}: Fatigue, cognitive impairment, peripheral neuropathy, macrocytic anemia
    \item \textbf{Testing}: B12 level <200 pg/mL; methylmalonic acid (MMA) elevated if tissue deficiency
\end{itemize}

\subsubsection{Infectious Diseases}

\begin{observation}[Post-Infectious vs.\ Chronic Active Infection]
\label{obs:infection-distinction}
Distinguishing ME/CFS from active infections and post-infectious fatigue is essential for proper diagnosis and management.
\end{observation}

\textbf{Chronic Active Infections (Must Rule Out):}
\begin{itemize}
    \item \textbf{HIV/AIDS}: Check HIV antibody/antigen test
    \item \textbf{Hepatitis B/C}: Check HBsAg, anti-HCV
    \item \textbf{Tuberculosis}: In endemic areas or high-risk patients, check tuberculin skin test or interferon-gamma release assay
    \item \textbf{Lyme disease}: In endemic areas with appropriate exposure history, check Lyme serology (ELISA, Western blot)
\end{itemize}

\textbf{Post-Infectious Fatigue vs.\ ME/CFS:}
Many acute infections (influenza, mononucleosis, COVID-19) are followed by transient fatigue lasting weeks to months. Distinguish from ME/CFS by:
\begin{itemize}
    \item \textbf{Duration}: Post-infectious fatigue typically improves by 3--6 months; ME/CFS persists $>6$ months without improvement
    \item \textbf{Post-exertional malaise}: True PEM with delayed onset and prolonged recovery is specific to ME/CFS
    \item \textbf{Trajectory}: Post-infectious fatigue shows gradual improvement; ME/CFS shows plateau or worsening
\end{itemize}

This distinction is critical in the first 6 months post-infection, as aggressive pacing during this period may prevent transition to established ME/CFS.

\subsubsection{Malignancy}

\begin{warning}[Occult Malignancy]
\label{warn:malignancy-screening}
Cancer-related fatigue can mimic ME/CFS, particularly in early stages without obvious tumor burden:
\begin{itemize}
    \item \textbf{Red flags}: Unintentional weight loss, night sweats, fever, lymphadenopathy, age >50 with new-onset fatigue
    \item \textbf{Screening}: Age-appropriate cancer screening (colonoscopy, mammography); if red flags present, consider CT chest/abdomen/pelvis
    \item \textbf{Laboratory clues}: Anemia, elevated ESR, abnormal WBC count
\end{itemize}
\end{warning}

\subsubsection{Psychiatric Disorders}

\begin{observation}[Depression vs.\ ME/CFS: Critical Distinctions]
\label{obs:depression-distinction}

Major depression can cause fatigue and cognitive impairment, but several features distinguish it from ME/CFS:

\paragraph{Post-Exertional Malaise (Pathognomonic for ME/CFS):}
\begin{itemize}
    \item \textbf{ME/CFS}: Physical or cognitive exertion triggers delayed (12--72 hours) symptom worsening lasting days to weeks
    \item \textbf{Depression}: Activity may be difficult due to lack of motivation, but exertion does NOT trigger delayed physiological crashes
    \item \textbf{Key question}: ``If you push through and do an activity you enjoy, do you crash afterward?''
    \begin{itemize}
        \item ME/CFS: Yes, even desired activities trigger PEM
        \item Depression: Enjoyable activities may temporarily improve mood
    \end{itemize}
\end{itemize}

\paragraph{Anhedonia (Pathognomonic for Depression):}
\begin{itemize}
    \item \textbf{Depression}: Loss of interest or pleasure in previously enjoyed activities (anhedonia is a core feature)
    \item \textbf{ME/CFS}: Patients \textit{want} to do activities but are physically unable; they retain interest but lack capacity
\end{itemize}

\paragraph{Effort vs.\ Performance:}
\begin{itemize}
    \item \textbf{Depression}: Reduced effort (``I don't feel like doing this''), but if motivation can be mustered, performance is intact
    \item \textbf{ME/CFS}: Normal or increased effort with reduced performance (``I'm trying as hard as I can but my body won't do it'')
\end{itemize}

\paragraph{Objective Biomarkers:}
\begin{itemize}
    \item \textbf{Two-day CPET}: ME/CFS shows failure to reproduce VO$_2$max on Day 2; depression does not
    \item \textbf{Orthostatic intolerance}: Objective POTS/NMH on testing supports ME/CFS
    \item \textbf{Inflammatory markers}: Heng panel, cytokine signatures abnormal in ME/CFS
\end{itemize}

\paragraph{Comorbid Depression in ME/CFS:}
Many ME/CFS patients develop \textbf{reactive depression} (consequence of severe disability, loss of career/social life). This is distinct from primary depression:
\begin{itemize}
    \item Reactive depression: Depression began \textit{after} ME/CFS onset; patient grieves loss of function
    \item Primary depression: Depression preceded fatigue; fatigue is a symptom of depression
\end{itemize}

Treating comorbid depression in ME/CFS is appropriate and may improve quality of life, but antidepressants do not cure ME/CFS.
\end{observation}

\subsection{Comorbid Conditions vs.\ Alternative Diagnoses}

\begin{observation}[The ME/CFS Septad]
\label{obs:mecfs-septad}
Several conditions frequently co-occur with ME/CFS at rates far exceeding chance, suggesting shared pathophysiology:

\begin{enumerate}
    \item \textbf{ME/CFS} (Myalgic Encephalomyelitis/Chronic Fatigue Syndrome)
    \item \textbf{Fibromyalgia}: Widespread pain with tender points (30--70\% of ME/CFS patients)
    \item \textbf{POTS} (Postural Orthostatic Tachycardia Syndrome): (70--90\% of ME/CFS patients)
    \item \textbf{MCAS} (Mast Cell Activation Syndrome): Histamine-mediated symptoms (estimates 10--50\%)
    \item \textbf{hEDS} (Hypermobile Ehlers-Danlos Syndrome): Joint hypermobility (higher in ME/CFS than general population)
    \item \textbf{IBS} (Irritable Bowel Syndrome): Functional GI symptoms (30--50\% of ME/CFS patients)
    \item \textbf{IC} (Interstitial Cystitis): Bladder pain, urinary frequency
\end{enumerate}

\textbf{Clinical Implication:}
These conditions are \textbf{comorbidities}, not alternative diagnoses. Their presence does NOT exclude ME/CFS. In fact, meeting criteria for multiple septad conditions strengthens the ME/CFS diagnosis and suggests common underlying mechanisms (autonomic dysfunction, small fiber neuropathy, immune activation).
\end{observation}

\subsection{When Comorbidities May Be Primary Drivers}
\label{sec:comorbidity-primary}

While ME/CFS and its comorbidities typically coexist, in some patients a ``comorbidity'' may actually be the \textit{primary driver} of symptoms, with ME/CFS-like presentation being downstream consequence rather than the core disease.

\begin{observation}[Comorbidities as Potential Primary Pathology]
\label{obs:comorbidity-primary}
Consider whether an apparent ``comorbidity'' might be the primary driver when:

\textbf{MCAS/Histamine Intolerance as Primary:}
\begin{itemize}
    \item Symptoms fluctuate dramatically with dietary triggers (high-histamine foods)
    \item Marked improvement with H1/H2 blockade disproportionate to typical ME/CFS response
    \item Intestinal symptoms preceded and dominate fatigue
    \item Proposed cascade: MCAS $\rightarrow$ intestinal barrier dysfunction $\rightarrow$ amino acid malabsorption $\rightarrow$ mitochondrial failure $\rightarrow$ ME/CFS phenotype (see Section~\ref{sec:gut-metabolic-cascade})
\end{itemize}

\textbf{POTS as Primary:}
\begin{itemize}
    \item Fatigue and cognitive symptoms are primarily orthostatic (worse upright, better supine)
    \item Dramatic improvement with POTS-specific treatment (compression, fluids, fludrocortisone, ivabradine)
    \item Heart rate criteria met ($\geq$30 bpm increase within 10 minutes of standing)
    \item Consider whether ``ME/CFS'' is actually deconditioning secondary to untreated POTS
\end{itemize}

\textbf{Chronic Viral Reactivation as Primary:}
\begin{itemize}
    \item Documented elevated EBV or HHV-6 titers (especially IgM or PCR positivity)
    \item Symptom flares correlate with viral reactivation markers
    \item Response to antiviral therapy (valacyclovir for EBV, valganciclovir for HHV-6/CMV)
    \item Cimetidine trial produces dramatic improvement (suggests immune enhancement against herpesvirus)
\end{itemize}

\textbf{Craniocervical Instability (CCI) as Primary:}
\begin{itemize}
    \item Symptoms markedly position-dependent
    \item Hypermobility (hEDS) with progressive neurological features
    \item Suboccipital headaches, neck pain, visual disturbances
    \item Upright MRI shows craniocervical abnormalities (see Section~\ref{sec:septad})
\end{itemize}
\end{observation}

\paragraph{Clinical Approach: Test the Primary Driver Hypothesis.}

When a comorbidity might be primary:
\begin{enumerate}
    \item \textbf{Prioritize that condition's workup}: Comprehensive evaluation of the suspected primary driver
    \item \textbf{Targeted treatment trial}: If the comorbidity is primary, treating it should produce disproportionate improvement
    \item \textbf{Assess response}: Dramatic improvement ($>$50\% symptom reduction) with targeted treatment suggests the ``comorbidity'' was actually the primary pathology
    \item \textbf{Re-evaluate if partial response}: Partial improvement suggests the comorbidity contributes but is not the sole driver; ME/CFS diagnosis remains appropriate
\end{enumerate}

\begin{warning}[Avoid Premature Reclassification]
\label{warn:reclassification}
Do NOT reclassify ME/CFS as ``just POTS'' or ``just MCAS'' without:
\begin{itemize}
    \item Demonstrating that treating the suspected primary condition produces substantial, sustained improvement
    \item Confirming that post-exertional malaise resolves (not just fatigue)
    \item Documenting functional recovery, not just symptom reduction
\end{itemize}
Many patients have true comorbid ME/CFS + POTS + MCAS where all contribute. Treating comorbidities improves but does not cure ME/CFS in most cases.
\end{warning}

\paragraph{The Cascading Comorbidity Model.}

Rather than independent conditions, the septad conditions may cascade:

\begin{itemize}
    \item \textbf{hEDS} $\rightarrow$ Vascular laxity $\rightarrow$ \textbf{POTS}
    \item \textbf{MCAS} $\rightarrow$ Intestinal barrier dysfunction $\rightarrow$ Malabsorption $\rightarrow$ \textbf{Mitochondrial dysfunction}
    \item \textbf{Dysautonomia} $\rightarrow$ Vagal dysfunction $\rightarrow$ \textbf{GI dysmotility} $\rightarrow$ \textbf{SIBO}
    \item \textbf{Chronic infection} $\rightarrow$ Immune exhaustion $\rightarrow$ \textbf{Autoimmunity}
    \item \textbf{Small fiber neuropathy} $\rightarrow$ Autonomic neuropathy $\rightarrow$ \textbf{POTS}
\end{itemize}

This cascading model suggests that identifying and treating upstream conditions may interrupt downstream pathology. For detailed discussion of the ``Septad'' framework and its limitations, see Section~\ref{sec:septad}.

\subsection{Diagnostic Algorithm}

\begin{observation}[Decision Tree for ME/CFS Diagnosis]
\label{obs:diagnostic-algorithm}

\begin{enumerate}
    \item \textbf{Step 1: Screen for post-exertional malaise}
    \begin{itemize}
        \item If PEM absent → Consider alternative diagnosis (depression, deconditioning, other fatiguing condition)
        \item If PEM present → Proceed to Step 2
    \end{itemize}

    \item \textbf{Step 2: Rule out exclusions via laboratory testing}
    \begin{itemize}
        \item CBC, CMP, TSH/free T4, ESR/CRP, ANA, vitamin D, B12, sleep study
        \item If positive finding that fully explains symptoms → Treat that condition
        \item If tests normal or findings insufficient to explain severity → Proceed to Step 3
    \end{itemize}

    \item \textbf{Step 3: Assess duration and functional impact}
    \begin{itemize}
        \item Duration $\geq 6$ months? (or $\geq 3$ months in severe pediatric cases)
        \item Substantial functional impairment?
        \item If yes to both → Proceed to Step 4
    \end{itemize}

    \item \textbf{Step 4: Apply diagnostic criteria}
    \begin{itemize}
        \item Use Canadian Consensus, IOM, or ICC criteria
        \item All require: PEM, unrefreshing sleep, multi-system symptoms
        \item If criteria met → Diagnose ME/CFS
    \end{itemize}

    \item \textbf{Step 5: Assess for comorbidities}
    \begin{itemize}
        \item Screen for septad conditions (fibromyalgia, POTS, MCAS, hEDS, IBS)
        \item Document which conditions are present (multi-label classification)
        \item These do not exclude ME/CFS; they inform treatment strategy
    \end{itemize}

    \item \textbf{Step 6: Biological phenotyping (if resources permit)}
    \begin{itemize}
        \item Apply Tier 2 framework: assess autoimmune, mitochondrial, neuroinflammatory, dysautonomia, endothelial domains
        \item Guide treatment stratification
    \end{itemize}

    \item \textbf{Step 7: Risk stratification}
    \begin{itemize}
        \item Apply Tier 3 RED FLAG criteria
        \item If $\geq 2$ RED FLAGS → Emergency intervention protocol
    \end{itemize}
\end{enumerate}
\end{observation}

\subsection{When to Reconsider the Diagnosis}

\begin{warning}[Red Flags Suggesting Alternative Diagnosis]
\label{warn:reconsider-diagnosis}
ME/CFS diagnosis should be reconsidered if:
\begin{itemize}
    \item \textbf{New focal neurological signs}: Weakness, sensory loss, visual changes (suggests MS, tumor, stroke)
    \item \textbf{Fever, night sweats, unintentional weight loss}: Suggests infection, malignancy, autoimmune disease
    \item \textbf{Rapid progression over weeks}: ME/CFS typically progresses over months to years; rapid worsening suggests acute process
    \item \textbf{Lack of PEM}: If re-evaluation reveals no true post-exertional malaise, reconsider alternative diagnoses
    \item \textbf{Complete resolution with psychiatric treatment}: If depression treatment alone fully resolves ``fatigue,'' the diagnosis was likely primary depression, not ME/CFS
\end{itemize}
\end{warning}
