\section{Novel Biology-Informed Diagnostic Framework}
\label{sec:novel-framework}

This section proposes an updated diagnostic framework that synthesizes current pathophysiological understanding with clinical reality. Unlike existing criteria that treat ME/CFS as a single homogeneous entity, this three-tiered approach recognizes disease heterogeneity while maintaining diagnostic precision.

\subsection{Rationale for a New Framework}
\label{subsec:framework-rationale}

\subsubsection{The Logic Chain: From Biology to Diagnosis}

The three-tiered diagnostic framework follows logically from four fundamental observations about ME/CFS:

\paragraph{Observation 1: ME/CFS is a Clinical Syndrome with a Core Pathognomonic Feature}

Post-exertional malaise (PEM) with delayed onset, disproportionate severity, and prolonged recovery distinguishes ME/CFS from all other fatiguing conditions. This symptom:
\begin{itemize}
    \item Cannot be explained by deconditioning (which improves with gradual activity)
    \item Cannot be explained by depression (which may improve somewhat with activity)
    \item Has objective correlates (2-day CPET showing Day 2 deterioration~\cite{lim2020cpet,keller2024cpet})
    \item Reflects underlying cellular energy failure (ATP depletion~\cite{heng2025mecfs,Syed2025})
\end{itemize}

\textbf{Logical consequence}: Tier 1 must retain syndrome-based diagnosis with PEM as mandatory criterion, ensuring we capture the defining pathophysiology while maintaining compatibility with existing frameworks.

\paragraph{Observation 2: ME/CFS Heterogeneity Reflects Multiple Causal Pathways, Not Measurement Error}

The failure of single-target treatments in randomized controlled trials does not mean ``ME/CFS has no biological basis''---it means we are mixing biologically distinct subgroups:

\begin{itemize}
    \item Rituximab (anti-CD20 B cell depletion) failed in large trials~\cite{Fluge2015rituximab_rct} despite initial promise
    \item Daratumumab (anti-CD38 plasma cell depletion) succeeded in 60\% of patients in pilot study~\cite{Fluge2025daratumumab}
    \item \textbf{Interpretation}: Not ``autoimmunity isn't involved,'' but rather ``wrong cell type targeted'' (short-lived B cells vs. long-lived plasma cells) AND ``only a subset has autoimmune-driven disease''
\end{itemize}

The Heng 2025 study~\cite{heng2025mecfs} demonstrated that a 7-biomarker panel spanning three systems (energy, immune, vascular) achieved 91\% diagnostic accuracy. This implies:
\begin{enumerate}
    \item All three systems are coordinately dysfunctional (not independent)
    \item ME/CFS is not five separate diseases but one syndrome with five co-occurring mechanisms
    \item Treatment must address multiple domains simultaneously
\end{enumerate}

\textbf{Logical consequence}: Tier 2 must assess all relevant biological domains and document which are present, rather than forcing patients into exclusive categories. The question is not ``Is this autoimmune OR metabolic ME/CFS?'' but ``Which of the five domains show dysfunction in this patient?''

\paragraph{Observation 3: Treatment Response Depends on Rate-Limiting Steps, Not Just Presence of Pathology}

Consider two patients, both with elevated GPCR autoantibodies and mitochondrial dysfunction:
\begin{itemize}
    \item \textbf{Patient A}: Autoantibodies are driving ongoing inflammation → mitochondria are secondarily impaired → removing autoantibodies allows mitochondrial recovery → daratumumab produces dramatic improvement
    \item \textbf{Patient B}: Initial autoimmune trigger has resolved, but mitochondrial damage is now self-perpetuating (WASF3 accumulation, cristae disruption) → removing residual autoantibodies doesn't help because mitochondria cannot recover → daratumumab fails
\end{itemize}

Both patients are ``autoantibody-positive,'' but only Patient A responds. The difference: which domain is \textbf{rate-limiting} (the bottleneck preventing recovery).

This explains:
\begin{itemize}
    \item Why daratumumab works in 60\% not 100\% of patients~\cite{Fluge2025daratumumab}
    \item Why low baseline NK cell count predicted non-response (suggests irreversible immune exhaustion)
    \item Why biomarker-positive patients don't uniformly respond to biomarker-targeted treatments
\end{itemize}

\textbf{Logical consequence}: Tier 2 must enable multi-target treatment protocols. We cannot predict \textit{a priori} which domain is rate-limiting, so we must:
\begin{enumerate}
    \item Treat all accessible, low-risk domains simultaneously (``quick wins'')
    \item Reassess at 3--6 months to identify which domains responded vs. persisted
    \item Intensify treatment for persistent domains (these are likely rate-limiting)
\end{enumerate}

\paragraph{Observation 4: ME/CFS Has Irreversible Thresholds, Making Timing Critical}

The natural history literature~\cite{Maksoud2020natural} and patient reports converge on temporal patterns:

\begin{itemize}
    \item \textbf{6 months}: If symptoms persist beyond 6 months, spontaneous resolution becomes unlikely (transition from ``post-viral fatigue'' to ``established ME/CFS'')
    \item \textbf{2 years}: Around 2 years, disease transitions from early (hypermetabolic, potentially reversible) to established (hypometabolic, epigenetically locked) state
    \item \textbf{Cumulative crashes}: Repeated PEM episodes cause progressive damage; there may be a threshold (5--10 severe crashes) beyond which recovery capacity is permanently impaired
    \item \textbf{25\% severe}: One-quarter of ME/CFS patients become housebound/bedbound, most starting with mild disease
\end{itemize}

The progression from mild to severe appears \textbf{preventable in many cases} through aggressive pacing, yet existing diagnostic criteria provide no framework for:
\begin{itemize}
    \item Identifying patients at high risk of progression
    \item Defining what ``aggressive pacing'' means operationally
    \item Communicating urgency of intervention before crossing irreversible thresholds
\end{itemize}

\textbf{Logical consequence}: Tier 3 must prospectively assess progression risk and provide actionable intervention protocols. The diagnostic framework must be \textbf{dynamic} (tracking trajectory) not static (labeling current state).

\subsubsection{Why Three Tiers? Why Not Two or Four?}

The three-tiered structure reflects three distinct clinical questions:

\begin{enumerate}
    \item \textbf{Tier 1 (Syndrome)}: Does this patient have ME/CFS? (Yes/No based on universal clinical features)
    \item \textbf{Tier 2 (Biology)}: Which pathophysiological mechanisms are driving this patient's disease? (Multi-label classification across 5 domains)
    \item \textbf{Tier 3 (Trajectory)}: How severe is the disease currently, and what is the risk of irreversible progression? (Severity + prospective risk)
\end{enumerate}

These cannot be collapsed:
\begin{itemize}
    \item Tier 1 alone (current criteria) misses treatment stratification and progression prevention
    \item Tier 2 alone (biology-only) would miss patients without access to biomarkers and wouldn't address progression risk
    \item Tier 3 alone (severity-only) would lack diagnostic specificity and treatment guidance
\end{itemize}

Each tier serves a distinct purpose and requires different information.

\subsubsection{Limitations of Existing Criteria}

Existing diagnostic criteria (Fukuda, Canadian Consensus, ICC, IOM) share important limitations that this framework addresses:

\begin{itemize}
    \item \textbf{Syndrome-based only}: Rely exclusively on symptom constellations without biological stratification, preventing precision medicine
    \item \textbf{Static classification}: Diagnose a point-in-time state without assessing progression risk, missing the 25\% who will become severe
    \item \textbf{Assume homogeneity}: Force heterogeneous patients into single diagnostic category, explaining why single-target trials fail
    \item \textbf{Limited treatment guidance}: Diagnosis doesn't inform which interventions to prioritize, leading to trial-and-error
    \item \textbf{Miss therapeutic windows}: Fail to identify the 6-month and 2-year critical intervention windows
\end{itemize}

\subsubsection{How Recent Advances Enable This Framework}

The proposed three-tiered framework would not have been possible a decade ago. Recent advances now make it feasible:

\begin{itemize}
    \item \textbf{Objective biomarkers}: GPCR autoantibodies~\cite{Loebel2016,Bynke2020}, Heng 7-marker panel~\cite{heng2025mecfs}, 2-day CPET~\cite{lim2020cpet,keller2024cpet} provide biological stratification
    \item \textbf{Mechanistic understanding}: Autoimmunity~\cite{Fluge2025daratumumab}, mitochondrial dysfunction~\cite{wang2023wasf3,Syed2025}, neuroinflammation~\cite{Nakatomi2014neuroinflammation} explain heterogeneity
    \item \textbf{Treatment stratification proof-of-concept}: Daratumumab 60\% response in autoimmune subset~\cite{Fluge2025daratumumab}, immunoadsorption for GPCR autoantibodies~\cite{Stein2024immunoadsorption} demonstrate that biomarker-guided treatment works
    \item \textbf{Natural history data}: Critical intervention windows~\cite{Maksoud2020natural}, progression patterns~\cite{Chu2019}, cumulative damage model validated
\end{itemize}

The proposed framework integrates these advances into clinically actionable diagnostic tiers.

\subsubsection{Summary: The Logical Structure}

\begin{observation}[Framework Logic]
\textbf{Premise 1}: ME/CFS is a clinical syndrome with a pathognomonic feature (PEM) that has objective correlates

$\Rightarrow$ \textbf{Tier 1}: Syndrome-based diagnosis with PEM mandatory

\vspace{0.5em}

\textbf{Premise 2}: ME/CFS heterogeneity reflects multiple co-occurring biological mechanisms; treatment response depends on which mechanism is rate-limiting

$\Rightarrow$ \textbf{Tier 2}: Multi-domain biological phenotyping to enable multi-target treatment

\vspace{0.5em}

\textbf{Premise 3}: ME/CFS has irreversible thresholds (6 months, 2 years, cumulative crashes); progression to severe disease is often preventable

$\Rightarrow$ \textbf{Tier 3}: Severity classification + prospective risk assessment with emergency protocols

\vspace{0.5em}

The three-tiered structure is not arbitrary---it reflects the logical necessity of answering three distinct clinical questions (diagnosis, mechanism, trajectory) that cannot be collapsed without losing critical information.
\end{observation}

\subsection{Tier 1: Clinical Syndrome Criteria}
\label{subsec:tier1}

Tier 1 establishes the diagnosis of ME/CFS based on clinical features. These criteria are universal---all patients must meet Tier 1 to receive the diagnosis.

\subsubsection{Core Diagnostic Features (All Required)}

\begin{requirement}[Post-Exertional Malaise (Mandatory Hallmark)]
\label{req:pem-criterion}
Post-exertional malaise must be present with ALL of the following characteristics:

\begin{itemize}
    \item \textbf{Delayed onset}: Symptom exacerbation occurs 12--72 hours after triggering activity (not immediately)
    \item \textbf{Disproportionate severity}: Minimal exertion produces profound symptom worsening far beyond normal fatigue
    \item \textbf{Multi-domain triggers}: Symptoms triggered by physical exertion \textit{AND} cognitive exertion \textit{AND} emotional exertion
    \item \textbf{Prolonged recovery}: Symptom exacerbation persists $>$24 hours (mild cases) to weeks or months (severe cases)
\end{itemize}

\paragraph{Objective verification (optional but supportive):}
Two-day cardiopulmonary exercise testing showing Day 2 deterioration: workload at ventilatory threshold decreases $\geq$20\% on Day 2 compared to Day 1~\cite{lim2020cpet,keller2024cpet}.
\end{requirement}

\begin{requirement}[Baseline Energy Insufficiency]
\label{req:energy-insufficiency}
Patients must demonstrate chronic energy deficit characterized by:

\begin{itemize}
    \item \textbf{Morning depletion}: Waking already exhausted despite sleep duration
    \item \textbf{Disproportionate activity cost}: Activities of daily living (hygiene, eating, sitting upright) consume excessive energy relative to effort
    \item \textbf{No functional reserve}: Zero capacity to handle unexpected physical, cognitive, or emotional demands
    \item \textbf{Effort-performance disconnect}: Subjective experience of maximal effort producing minimal objective output (a phenomenon that distinguishes ME/CFS from deconditioning or primary depression)
\end{itemize}

The effort-performance disconnect represents a novel diagnostic criterion capturing the lived experience of ME/CFS: patients describe ``giving everything'' to accomplish minimal tasks, feeling as though simple activities require marathon-level exertion while producing negligible results~\cite{strassheim2021experiences,fennell2021elements}.
\end{requirement}

\begin{requirement}[Duration and Exclusion Criteria]
\label{req:duration-exclusion}
\begin{itemize}
    \item \textbf{Duration}: Symptoms must persist $\geq$6 months
    \item \textbf{Rationale}: Six-month persistence indicates transition from post-viral fatigue (which typically resolves) to established ME/CFS with aberrant pathophysiology~\cite{Maksoud2020natural}
    \item \textbf{Exclusions}: Symptoms not better explained by:
    \begin{itemize}
        \item Active medical conditions (untreated hypothyroidism, sleep apnea, anemia, malignancy)
        \item Primary psychiatric disorders (though secondary depression/anxiety are common and do not exclude ME/CFS)
        \item Medication side effects
    \end{itemize}
\end{itemize}

\textbf{Important}: Comorbid conditions that are part of the ME/CFS disease spectrum (POTS, fibromyalgia, MCAS, IBS) do \textit{not} exclude the diagnosis---these represent overlapping pathophysiology rather than alternative explanations.
\end{requirement}

\subsubsection{Supporting Features (≥3 of 5 Required)}

In addition to the three core features, patients must have at least three of the following five supporting features:

\begin{enumerate}
    \item \textbf{Unrefreshing Sleep}
    \begin{itemize}
        \item Sleep that fails to restore energy regardless of duration
        \item Waking feeling as exhausted as when going to bed
        \item Present in 95--100\% of ME/CFS patients~\cite{Jason2010sleepMECFS,Unger2016sleepPrevalence}
    \end{itemize}

    \item \textbf{Cognitive Impairment}
    \begin{itemize}
        \item Processing speed deficits (most robust finding: Hedges' g = -0.82)~\cite{Cvejic2022cognitive}
        \item Attention and working memory impairment
        \item Word-finding difficulties, linguistic reversals
        \item Brain fog that is not attributable to fatigue or depression~\cite{MCAM2024cognitive}
    \end{itemize}

    \item \textbf{Autonomic Dysfunction}
    \begin{itemize}
        \item Orthostatic intolerance: symptoms worsened by upright posture
        \item POTS (heart rate increase $\geq$30 bpm upon standing), orthostatic hypotension, or neurally mediated hypotension
        \item Temperature dysregulation, inappropriate sweating or lack of sweating
        \item Present in 70--90\% of ME/CFS patients~\cite{Newton2007autonomicDysfunction}
    \end{itemize}

    \item \textbf{Pain}
    \begin{itemize}
        \item Myalgia (muscle pain), particularly with post-exertional exacerbation
        \item Arthralgia (joint pain, characteristically migratory without inflammation)
        \item Headaches (migraine or tension-type)~\cite{Ravindran2011headache}
        \item Pain present in $\sim$80\% of patients~\cite{Unger2017pain}
    \end{itemize}

    \item \textbf{Sensory Hypersensitivity}
    \begin{itemize}
        \item Photophobia (light sensitivity requiring sunglasses indoors or dimmed environment)
        \item Phonophobia (sound sensitivity; normal volumes feel uncomfortable)
        \item Chemical sensitivity (fragrances, cleaning products, exhaust)
        \item Touch hypersensitivity or allodynia
        \item Present in 70--90\% of patients~\cite{Jason2013sensory}
    \end{itemize}
\end{enumerate}

\begin{observation}[Tier 1 Summary]
Tier 1 criteria establish ME/CFS as a clinical syndrome with mandatory post-exertional malaise, baseline energy insufficiency, and 6-month duration. Supporting features (sleep, cognition, autonomic, pain, sensory) must be present in sufficient number ($\geq$3 of 5) to confirm the characteristic multi-system presentation. These criteria are compatible with existing frameworks (Canadian Consensus, ICC, IOM) but add explicit recognition of the effort-performance disconnect and specify the 6-month threshold as marking transition to established disease.
\end{observation}

\subsection{Tier 2: Biological Phenotyping (Multi-Domain Assessment)}
\label{subsec:tier2}

Once Tier 1 criteria are met, patients should undergo comprehensive biological phenotyping to identify which pathophysiological domains are involved. This enables targeted treatment and research stratification.

\subsubsection{Rationale: Co-Occurrence Rather Than Predominance}

Critical insight: ME/CFS patients typically have dysfunction in \textit{multiple} biological domains simultaneously. The Heng 2025 study demonstrated that a 7-biomarker panel spanning energy metabolism, immune function, and vascular endothelium achieved 91\% diagnostic accuracy precisely because \textit{all three systems show coordinated dysfunction}~\cite{heng2025mecfs}. This finding validates the multi-lock model (Chapter~\ref{ch:speculative-hypotheses}): ME/CFS persists because multiple self-reinforcing pathophysiological processes operate concurrently.

\begin{hypothesis}[Multi-Domain Co-Occurrence Model]
\label{hyp:multi-domain}
ME/CFS should be understood as a syndrome with five co-occurring, mutually reinforcing biological domains. Most patients have abnormalities in $\geq$3 domains:

\begin{itemize}
    \item Autoimmune features: 30--60\% (GPCR autoantibodies~\cite{Loebel2016,Bynke2020})
    \item Mitochondrial/metabolic dysfunction: 70--95\% (ATP abnormalities~\cite{heng2025mecfs}, lactate elevation~\cite{Lien2019lactate})
    \item Neuroinflammation/central sensitization: 70--90\% (central sensitization 84\%~\cite{Nijs2021sensitization}, sensory sensitivities 70--90\%~\cite{Jason2013sensory})
    \item Dysautonomia: 70--90\% (POTS 25--50\%, broader orthostatic intolerance 70--90\%~\cite{Newton2007autonomicDysfunction})
    \item Endothelial dysfunction: Prevalence unknown (Heng 2025 documented elevation in ME/CFS cohort~\cite{heng2025mecfs})
\end{itemize}

These domains are interdependent:
\begin{itemize}
    \item Autoimmunity (GPCR autoantibodies) $\rightarrow$ Mitochondrial dysfunction ($\beta_2$-adrenergic signaling regulates mitochondrial biogenesis)
    \item Mitochondrial dysfunction (ATP depletion) $\rightarrow$ Neuroinflammation (danger signal release, ionic gradient failure)
    \item Endotheliopathy (impaired vasodilation) $\rightarrow$ Dysautonomia (orthostatic intolerance, cerebral hypoperfusion)
    \item Neuroinflammation (cytokine production) $\rightarrow$ Autoimmunity (B cell activation)
\end{itemize}

Treatment targeting a single domain may fail because untreated domains maintain dysfunction. The multi-domain model predicts that:
\begin{enumerate}
    \item Patients with more domains affected will have worse outcomes
    \item Multi-target interventions will outperform single-target interventions
    \item Treatment response requires both (a) presence of dysfunction in a domain AND (b) that domain being rate-limiting (the bottleneck gating recovery)
\end{enumerate}
\end{hypothesis}

\subsubsection{Domain 1: Autoimmune Features}

\paragraph{Assessment:}
\begin{itemize}
    \item GPCR autoantibodies ($\beta_2$-adrenergic, M3 muscarinic, M4 muscarinic) above age/sex-matched reference ranges~\cite{Loebel2016,Bynke2020}
    \item ANA (any titer; present in 20--30\% ME/CFS vs. 5--10\% healthy controls)
    \item Plasma cell expansion on flow cytometry (CD38$^+$CD138$^+$ if available)
    \item Low NK cell count (<5th percentile) with normal total lymphocytes
\end{itemize}

\paragraph{If Present → Diagnosis:} ``ME/CFS with Autoimmune Component''

\paragraph{Treatment Implications:}
\begin{itemize}
    \item Candidate for immunoadsorption (IgG removal)~\cite{Stein2024immunoadsorption}
    \item Candidate for daratumumab (anti-CD38, depletes plasma cells)~\cite{Fluge2025daratumumab}
    \item Candidate for BC007 (GPCR autoantibody neutralizer)~\cite{Hohberger2021bc007}
    \item Monitor for worsening with immune-stimulating interventions
\end{itemize}

\paragraph{Prevalence:} 30--60\% of ME/CFS patients

\subsubsection{Domain 2: Mitochondrial/Metabolic Dysfunction}

\paragraph{Assessment:}
\begin{itemize}
    \item Heng 7-marker panel (if available): Elevated AMP, ADP (energy depletion arm)~\cite{heng2025mecfs}
    \item Elevated lactate: Resting >2.0 mmol/L or abnormal accumulation during 2-day CPET~\cite{Lien2019lactate}
    \item ATP profile abnormalities (if specialized testing available)
    \item WASF3 elevation on skeletal muscle biopsy (if indicated for severe cases)~\cite{wang2023wasf3}
\end{itemize}

\paragraph{If Present → Diagnosis:} ``ME/CFS with Mitochondrial Dysfunction''

\paragraph{Treatment Implications:}
\begin{itemize}
    \item CoQ10 (ubiquinol 200--400 mg/day)
    \item NAD$^+$ precursors (nicotinamide riboside 1000--2000 mg/day, treatment duration $\geq$10 weeks)
    \item D-ribose, B-complex vitamins, alpha-lipoic acid, PQQ
    \item Strict pacing critical (ATP depletion is cumulative)
    \item Heart rate monitoring (stay below 60\% maximum heart rate during activity)
\end{itemize}

\paragraph{Prevalence:} 70--95\% (virtually all ME/CFS patients show some degree of energy metabolism dysfunction)

\subsubsection{Domain 3: Neuroinflammation/Central Sensitization}

\paragraph{Assessment:}
\begin{itemize}
    \item \textbf{Research settings}: PET evidence of microglial activation~\cite{Nakatomi2014neuroinflammation}, fMRI showing altered temporoparietal junction or salience network connectivity~\cite{walitt2024deep,Shan2020neuroimaging}
    \item \textbf{Clinically accessible}:
    \begin{itemize}
        \item Central sensitization confirmed by quantitative sensory testing: pressure pain thresholds <5th percentile at $\geq$3 standardized sites~\cite{Nijs2021sensitization}
        \item Small fiber neuropathy: skin biopsy showing intraepidermal nerve fiber density <5th percentile~\cite{Oaklander2022SFN}
        \item Severe sensory sensitivities requiring environmental modification (inability to tolerate normal lighting, sound levels, or chemical exposures)
    \end{itemize}
\end{itemize}

\paragraph{If Present → Diagnosis:} ``ME/CFS with Neuroinflammatory Component''

\paragraph{Treatment Implications:}
\begin{itemize}
    \item Low-dose naltrexone (LDN 1.5--4.5 mg at bedtime)
    \item Environmental modification (dimmed lighting, noise reduction, fragrance-free environment)
    \item IVIG (if small fiber neuropathy documented and insurance approves)
    \item Avoid activities that trigger sensory overload (cognitive post-exertional malaise)
\end{itemize}

\paragraph{Prevalence:} 70--90\% (sensory sensitivities 70--90\%, central sensitization 84\%, small fiber neuropathy 30--38\%)

\subsubsection{Domain 4: Dysautonomia}

\paragraph{Assessment:}
\begin{itemize}
    \item \textbf{Gold standard}: Tilt table testing showing POTS (heart rate increase $\geq$30 bpm within 10 minutes), orthostatic hypotension (blood pressure drop $\geq$20/10 mmHg), or neurally mediated hypotension
    \item \textbf{Clinically accessible}: NASA Lean Test (10-minute standing test; positive if heart rate increases $\geq$30 bpm)
    \item Heart rate variability analysis (reduced HRV indicating sympathetic dominance)
    \item QSART/thermoregulatory sweat test (if available)
\end{itemize}

\paragraph{If Present → Diagnosis:} ``ME/CFS with Dysautonomia''

\paragraph{Treatment Implications:}
\begin{itemize}
    \item Volume expansion: 3--10 g sodium + 2--3 L fluids daily
    \item Compression garments (20--30 mmHg waist-high or thigh-high)
    \item Pharmacological:
    \begin{itemize}
        \item Fludrocortisone 0.05--0.2 mg daily
        \item Midodrine 2.5--10 mg three times daily
        \item Ivabradine 2.5--7.5 mg twice daily
        \item Low-dose beta-blockers (propranolol 10--20 mg as needed)
    \end{itemize}
    \item Positional strategies: elevate head of bed, avoid prolonged standing, sit when possible
\end{itemize}

\paragraph{Prevalence:} 70--90\% (POTS 25--50\%, broader orthostatic intolerance 70--90\%)

\subsubsection{Domain 5: Endothelial Dysfunction}

\paragraph{Assessment:}
\begin{itemize}
    \item Heng 7-marker panel (if available): Elevated von Willebrand factor, fibronectin, thrombospondin-1 (endothelial activation arm)~\cite{heng2025mecfs}
    \item Clinical markers of microvascular dysfunction:
    \begin{itemize}
        \item Livedo reticularis (mottled skin discoloration)
        \item Raynaud's phenomenon (cold-induced color changes in fingers/toes)
        \item Delayed capillary refill (>3 seconds)
    \end{itemize}
    \item Cerebral hypoperfusion on SPECT imaging (if available)
\end{itemize}

\paragraph{If Present → Diagnosis:} ``ME/CFS with Endothelial Dysfunction''

\paragraph{Treatment Implications (experimental):}
\begin{itemize}
    \item L-citrulline 3--6 g/day or L-arginine (for nitric oxide production)
    \item Omega-3 fatty acids (EPA/DHA 2--4 g/day)
    \item Low-dose aspirin 81 mg daily (if no contraindications)
    \item Emerging research: anticoagulation trials, fibrinolytic protocols (investigational only)
\end{itemize}

\paragraph{Prevalence:} Unknown (Heng 2025 documented elevation in ME/CFS cohort; prevalence in broader ME/CFS population requires validation)

\subsubsection{Multi-Label Diagnosis Example}

\begin{observation}[Sample Comprehensive Diagnosis]
\textbf{Primary Diagnosis}: Myalgic Encephalomyelitis/Chronic Fatigue Syndrome (ME/CFS)

\textbf{Biological Phenotype (Tier 2 Multi-Domain Assessment):}
\begin{itemize}
    \item[$\checkmark$] Autoimmune component: GPCR autoantibodies $\beta_2$-adrenergic 12.5 U/mL (ref $<$8), M3 muscarinic 9.2 U/mL (ref $<$7)
    \item[$\checkmark$] Mitochondrial dysfunction: Fasting lactate 2.8 mmol/L (ref $<$2.0); 2-day CPET workload at VT decreased 35\% on Day 2
    \item[$\checkmark$] Neuroinflammatory component: Pressure pain thresholds 1.8 kg (ref $>$4.0); photophobia requiring indoor sunglasses; phonophobia limiting social interaction
    \item[$\checkmark$] Dysautonomia: POTS confirmed on tilt table (supine HR 65 bpm $\rightarrow$ standing HR 102 bpm at 8 minutes); HRV SDNN 18 ms (ref $>$50)
    \item[$\times$] Endothelial dysfunction: Not assessed (markers unavailable)
\end{itemize}

\textbf{Severity (Tier 3)}: Moderate (housebound 40\% of time; can perform remote work 20 hours/week with careful pacing)

\textbf{Progression Risk}: HIGH—RED FLAGS present: ratcheting baseline over past 9 months (each crash leaves lower functional floor); recovery time lengthening from 3 days → 10 days for equivalent exertion

\textbf{Treatment Plan}:
\begin{enumerate}
    \item \textbf{Foundation}: Aggressive pacing (50\% rule, heart rate monitoring <105 bpm)
    \item \textbf{Dysautonomia} (quick win, high accessibility): Fludrocortisone 0.1 mg daily, sodium 6 g/day, fluids 2.5 L/day, compression stockings
    \item \textbf{Mitochondrial} (quick win, high accessibility): CoQ10 300 mg, nicotinamide riboside 1000 mg, B-complex
    \item \textbf{Neuroinflammation} (quick win, high accessibility): LDN 3 mg at bedtime, light/sound environmental control
    \item \textbf{Autoimmune} (if accessible): Immunoadsorption or daratumumab candidate; pursue if no improvement after 6 months on above protocol
\end{enumerate}

\textbf{Reassessment}: 3-month follow-up to evaluate response in each domain; adjust treatment based on which domains improve vs. persist
\end{observation}

\subsection{Tier 3: Severity Classification and Progression Risk}
\label{subsec:tier3}

Tier 3 classifies current functional severity and prospectively assesses risk of progression to severe disease. This enables appropriate resource allocation, guides intervention intensity, and identifies patients requiring emergency intervention.

\subsubsection{Functional Severity Scale}

\begin{enumerate}
    \item \textbf{Mild ME/CFS}
    \begin{itemize}
        \item Can work or study at 50--80\% of pre-illness capacity, though with significant difficulty
        \item Post-exertional malaise occurs after moderate exertion
        \item Recovery from PEM takes days to 1--2 weeks
        \item Can perform most activities of daily living independently
        \item Energy envelope is reduced but allows meaningful activity
        \item Appears functional to outside observers (``invisible illness'')
    \end{itemize}

    \item \textbf{Moderate ME/CFS}
    \begin{itemize}
        \item Reduced daily activity to <50\% of pre-illness level
        \item Housebound 50\% or more of the time
        \item Unable to work or study full-time; may work part-time with difficulty
        \item Post-exertional malaise triggered by minimal exertion
        \item Recovery from PEM takes weeks
        \item Requires extended rest periods daily
        \item Significant impairment in social and occupational function
    \end{itemize}

    \item \textbf{Severe ME/CFS}
    \begin{itemize}
        \item Mostly bedbound (>50\% of waking hours)
        \item Can perform only minimal self-care activities (brief washing, feeding)
        \item Post-exertional malaise triggered by activities of daily living
        \item Cognitive impairment prevents reading, sustained conversation
        \item Sensory sensitivities may require dimmed environment, minimal sound
        \item Unable to leave home except for essential medical appointments
        \item Requires assistance with instrumental activities of daily living
    \end{itemize}

    \item \textbf{Very Severe ME/CFS}
    \begin{itemize}
        \item Bedbound continuously
        \item Unable to perform most self-care activities without assistance
        \item Profound sensitivity to light (requiring darkness), sound (requiring silence), touch
        \item May be unable to tolerate speaking or being spoken to
        \item Tube feeding may be required if swallowing is impaired
        \item Requires full-time care assistance
        \item Represents approximately 10\% of severe ME/CFS cases (2--3\% of total ME/CFS population)
    \end{itemize}
\end{enumerate}

\subsubsection{Progression Risk Stratification}

\begin{warning}[HIGH RISK for Progression to Severe Disease]
\label{warn:progression-risk}
Patients meeting $\geq$2 of the following RED FLAG criteria are at immediate risk of transitioning to severe, potentially irreversible disease and require emergency intervention:

\paragraph{RED FLAGS (Immediate Danger):}
\begin{enumerate}
    \item \textbf{Ratcheting baseline}: Each post-exertional crash leaves patient at a lower functional floor; baseline is trending downward over 6--12 months rather than returning to previous level

    \item \textbf{Recovery time lengthening}: PEM recovery now requires >2 weeks (previously required only days to 1 week)

    \item \textbf{Shrinking energy envelope}: Activities that were safely within the energy envelope 6 months ago now trigger post-exertional malaise

    \item \textbf{New sensory sensitivities}: Photophobia, phonophobia, or chemical sensitivities emerging or rapidly worsening

    \item \textbf{Cognitive decline}: Word-finding difficulties, memory impairment, or inability to read/process information worsening (cognitive symptoms are most resistant to recovery)~\cite{Chu2019}

    \item \textbf{Forced overexertion}: Patient cannot stop working or reduce activity due to financial necessity, despite clear evidence of deterioration (structural inability to pace)

    \item \textbf{Weight loss from energy insufficiency}: Eating and food preparation have become too effortful; weight loss indicates severe energy depletion

    \item \textbf{Social withdrawal by necessity}: Cannot tolerate visitors, phone calls, or any social interaction due to symptom exacerbation (not due to depression)
\end{enumerate}

\paragraph{Emergency Action Protocol:}
Patients with HIGH RISK status require immediate intervention to prevent crossing the ``point of no return'' to irreversible severe ME/CFS:

\begin{enumerate}
    \item \textbf{Within 48 hours}:
    \begin{itemize}
        \item Reduce all non-essential activity by 50\%
        \item Implement aggressive horizontal rest (50--75\% of waking hours)
        \item Cancel social commitments, request emergency work accommodation
    \end{itemize}

    \item \textbf{Within 1 week}:
    \begin{itemize}
        \item Physician visit for medical leave documentation
        \item Formal workplace accommodation request (reduced hours 50--75\%, remote work, flexible schedule)
        \item Begin disability application process if accommodations denied or insufficient
    \end{itemize}

    \item \textbf{Within 4--8 weeks}:
    \begin{itemize}
        \item Achieve baseline stabilization: Goal of ZERO post-exertional malaise episodes for 4 continuous weeks
        \item This proves patient is within energy envelope
        \item Accept that functional capacity is very low during this period---this is temporary to prevent permanent severe disease
    \end{itemize}
\end{enumerate}

\paragraph{Rationale:}
Research and patient reports demonstrate that repeated post-exertional malaise episodes cause cumulative physiological damage: mitochondrial dysfunction accumulation~\cite{wang2023wasf3,Syed2025}, endothelial dysfunction~\cite{heng2025mecfs}, neuroinflammation~\cite{Nakatomi2014neuroinflammation}, and immune exhaustion~\cite{iu2024tcell_exhaustion}. There appears to be a threshold (anecdotally 5--10 severe crashes) beyond which recovery capacity is permanently impaired. The goal is to avoid severe crashes entirely, not merely to minimize them.
\end{warning}

\subsubsection{Critical Temporal Windows}

\begin{observation}[The 6-Month Rule and 2-Year Establishment Threshold]
\label{obs:temporal-windows}
Two temporal thresholds mark critical transitions in ME/CFS natural history~\cite{Maksoud2020natural}:

\paragraph{6-Month Persistence Mark:}
If symptoms persist beyond 6 months without improvement, this indicates that normal homeostatic recovery mechanisms have failed and aberrant pathophysiology is becoming established. This marks the transition from ``post-viral fatigue that might spontaneously resolve'' to ``ME/CFS requiring active intervention.''

\paragraph{2-Year Establishment Threshold:}
Around 2 years post-onset, ME/CFS transitions from early disease (hypermetabolic, potentially modifiable) to established disease (hypometabolic, potentially entrenched). This transition involves:
\begin{itemize}
    \item Epigenetic changes altering gene expression patterns
    \item Immune exhaustion (CD8$^+$ T cell exhaustion~\cite{iu2024tcell_exhaustion}, NK cell dysfunction)
    \item Normalization of inflammatory markers despite ongoing dysfunction
    \item Brain structural changes visible on advanced imaging~\cite{Shan2020neuroimaging}
    \item Metabolic state shift from high (inefficient) energy expenditure to low energy production
\end{itemize}

\textbf{Implication}: The first 2 years represent a critical intervention window. Aggressive pacing, early biological phenotyping, and domain-targeted treatment during this period may prevent progression to established severe disease. After 2 years, reversal becomes substantially more difficult (though not impossible).

\textbf{Clinical application}: Patients diagnosed within 6 months of onset should be counseled on the criticality of aggressive pacing to prevent establishment. Patients approaching the 2-year mark should undergo comprehensive Tier 2 phenotyping to guide maximal intervention before the window closes.
\end{observation}

\subsection{Implementation and Clinical Workflow}
\label{subsec:implementation}

\subsubsection{Minimum Diagnostic Workup (All Patients)}

\begin{enumerate}
    \item \textbf{Tier 1 Clinical Assessment}:
    \begin{itemize}
        \item Detailed history: onset pattern, post-exertional malaise characteristics, sleep quality, cognitive symptoms, autonomic symptoms, pain, sensory sensitivities
        \item Physical examination: orthostatic vital signs, neurological examination, tender point assessment
        \item Functional capacity assessment: Bell Disability Scale, SF-36, or equivalent
    \end{itemize}

    \item \textbf{Objective Testing} (if accessible):
    \begin{itemize}
        \item Two-day cardiopulmonary exercise testing (gold standard for PEM documentation)
        \item Tilt table testing or NASA Lean Test (dysautonomia assessment)
    \end{itemize}

    \item \textbf{Basic Laboratory Testing} (rule out exclusions):
    \begin{itemize}
        \item Complete blood count (CBC)
        \item Comprehensive metabolic panel (CMP)
        \item Thyroid-stimulating hormone (TSH), free T4
        \item Ferritin (low ferritin contributes to fatigue and restless legs)
        \item Antinuclear antibody (ANA)
        \item Erythrocyte sedimentation rate (ESR), C-reactive protein (CRP)
        \item Vitamin D, vitamin B12
    \end{itemize}

    \item \textbf{Sleep Study}:
    \begin{itemize}
        \item Polysomnography to rule out obstructive sleep apnea (OSA) or upper airway resistance syndrome (UARS)
        \item OSA can mimic ME/CFS; treatment with CPAP dramatically improves symptoms in true OSA cases
        \item OSA and ME/CFS can coexist; treating comorbid OSA improves but does not cure ME/CFS
    \end{itemize}
\end{enumerate}

\subsubsection{Advanced Phenotyping (Tier 2, If Resources Permit)}

\begin{enumerate}
    \item \textbf{Autoimmune Domain}:
    \begin{itemize}
        \item GPCR autoantibody panel ($\beta_2$-adrenergic, M3 muscarinic, M4 muscarinic)
        \item NK cell count and function assay
        \item Flow cytometry for plasma cell populations (CD38$^+$CD138$^+$)
    \end{itemize}

    \item \textbf{Mitochondrial/Metabolic Domain}:
    \begin{itemize}
        \item Heng 7-biomarker panel (when commercially available): AMP, ADP, VWF, fibronectin, thrombospondin-1, PDGF-BB, TGF-$\beta$3
        \item Fasting lactate
        \item ATP profile (if specialized laboratory available)
    \end{itemize}

    \item \textbf{Neuroinflammation Domain}:
    \begin{itemize}
        \item Quantitative sensory testing (pressure pain thresholds)
        \item Skin biopsy for small fiber neuropathy (intraepidermal nerve fiber density)
    \end{itemize}

    \item \textbf{Dysautonomia Domain}:
    \begin{itemize}
        \item Tilt table testing (if not already performed)
        \item Heart rate variability analysis
        \item QSART or thermoregulatory sweat test (if available)
    \end{itemize}

    \item \textbf{Comorbidity Screening} (Septad components):
    \begin{itemize}
        \item MCAS workup: serum tryptase, 24-hour urine methylhistamine, prostaglandin D$_2$
        \item Hypermobility assessment: Beighton score
        \item If hEDS + progressive neurological symptoms: upright MRI for craniocervical instability screening
        \item Gastrointestinal: gastric emptying study, SIBO breath test (if prominent GI symptoms)
    \end{itemize}
\end{enumerate}

\subsubsection{Treatment Prioritization Based on Phenotype}

\begin{table}[htbp]
\centering
\caption{Treatment prioritization by biological domain}
\label{tab:treatment-prioritization}
\begin{tabular}{p{3cm}p{4cm}p{2cm}p{2cm}p{2cm}}
\toprule
\textbf{Domain} & \textbf{Treatment Options} & \textbf{Risk Level} & \textbf{Access} & \textbf{Priority} \\
\midrule
\textbf{Pacing} & Activity management, heart rate monitoring & None & High & \textbf{FIRST} (always) \\
\addlinespace
\textbf{Dysautonomia} & Salt, fluids, compression, fludrocortisone, midodrine & Low & High & \textbf{SECOND} (quick wins) \\
\addlinespace
\textbf{Mitochondrial} & CoQ10, NR/NMN, B vitamins & Low & High & \textbf{SECOND} (quick wins) \\
\addlinespace
\textbf{Neuroinflam.} & LDN, environmental modification & Low & High & \textbf{SECOND} (quick wins) \\
\addlinespace
\textbf{Autoimmune} & Immunoadsorption, daratumumab, BC007 & Moderate-High & Very Low & \textbf{THIRD} (if accessible) \\
\addlinespace
\textbf{Endothelial} & L-citrulline, omega-3, aspirin & Low & High & \textbf{THIRD} (experimental) \\
\bottomrule
\end{tabular}
\end{table}

\paragraph{Rationale:}
\begin{itemize}
    \item \textbf{Foundation}: Pacing is universal and non-negotiable---prevents cumulative damage regardless of biological phenotype
    \item \textbf{Quick wins}: High-accessibility, low-risk interventions (dysautonomia, mitochondrial, neuroinflammation) initiated simultaneously to address multiple domains
    \item \textbf{Reassessment}: At 3--6 months, evaluate response in each domain; persistent dysfunction despite accessible interventions justifies pursuit of high-intensity/low-accessibility treatments (immunoadsorption, daratumumab)
    \item \textbf{Multi-target approach}: Addresses multiple locks simultaneously, recognizing that single-domain interventions often fail due to reinforcement from untreated domains
\end{itemize}

\subsection{Research Implications and Validation Needs}
\label{subsec:research-implications}

This novel diagnostic framework generates testable predictions that should be validated in prospective studies:

\begin{enumerate}
    \item \textbf{Hypothesis}: Patients with $\geq$4 domains positive will have worse functional outcomes, longer illness duration, and lower treatment response rates than patients with 1--2 domains
    \begin{itemize}
        \item \textbf{Test}: Correlate number of positive domains with SF-36 Physical Function, Bell Disability Scale, work/school capacity, and hospitalization rates
    \end{itemize}

    \item \textbf{Hypothesis}: Multi-target interventions (treating all present domains) will produce superior outcomes compared to single-target interventions
    \begin{itemize}
        \item \textbf{Test}: Randomized controlled trial comparing CoQ10 monotherapy vs. CoQ10 + LDN + fludrocortisone (in patients with mitochondrial + neuroinflammatory + dysautonomia domains positive)
    \end{itemize}

    \item \textbf{Hypothesis}: The RED FLAG progression risk criteria (Tier 3) prospectively identify patients who will develop severe ME/CFS
    \begin{itemize}
        \item \textbf{Test}: Cohort study assessing RED FLAG status at enrollment, then tracking functional severity at 1 year and 2 years; calculate sensitivity/specificity of RED FLAG criteria for predicting progression to severe disease
    \end{itemize}

    \item \textbf{Hypothesis}: Treatment response to domain-specific interventions requires both (a) presence of dysfunction in that domain AND (b) that domain being rate-limiting (the bottleneck)
    \begin{itemize}
        \item \textbf{Test}: Measure all 5 domains → administer domain-specific treatment → identify responders vs. non-responders → retrospectively determine which baseline features predicted response
        \item Example: Daratumumab trial measuring GPCR autoantibodies, lactate, HRV, QST, VWF at baseline, then analyzing which baseline profile predicts 60\% responder group vs. 40\% non-responder group
    \end{itemize}

    \item \textbf{Hypothesis}: The Heng 7-marker panel achieves high diagnostic accuracy because it captures coordinated dysfunction across three systems (energy, immune, vascular), and symptom severity correlates with multi-system burden rather than single-marker elevation
    \begin{itemize}
        \item \textbf{Test}: Network analysis or partial least squares regression to determine if symptoms correlate with individual markers or require multi-marker patterns
    \end{itemize}

    \item \textbf{Hypothesis}: Early intervention (within the first 2 years) prevents establishment of refractory disease
    \begin{itemize}
        \item \textbf{Test}: Compare outcomes of patients receiving comprehensive Tier 2 phenotyping + multi-target treatment within 1 year of onset vs. those diagnosed/treated after 2+ years
        \item Ethical note: This should be observational (registry-based) rather than randomized, as withholding early treatment would be unethical if the hypothesis is correct
    \end{itemize}
\end{enumerate}

\subsection{Comparison to Existing Criteria}
\label{subsec:comparison}

Table~\ref{tab:framework-comparison} compares the novel biology-informed framework to established diagnostic criteria.

\begin{table}[htbp]
\centering
\caption{Comparison of diagnostic frameworks}
\label{tab:framework-comparison}
\begin{tabular}{p{3.5cm}p{2cm}p{2cm}p{2cm}p{2.5cm}}
\toprule
\textbf{Feature} & \textbf{Fukuda (1994)} & \textbf{Canadian (2003)} & \textbf{IOM (2015)} & \textbf{Novel Framework (2026)} \\
\midrule
PEM required & No & Yes & Yes & Yes (detailed criteria) \\
\addlinespace
Duration & 6 months & 6 months & 6 months & 6 months (establishment threshold) \\
\addlinespace
Biological phenotyping & No & No & No & Yes (5 domains) \\
\addlinespace
Progression risk assessment & No & No & No & Yes (RED FLAGS) \\
\addlinespace
Treatment stratification & No & No & No & Yes (domain-targeted) \\
\addlinespace
Temporal windows & No & No & No & Yes (2-year critical window) \\
\addlinespace
Recognizes heterogeneity & No & Partially & No & Yes (multi-label classification) \\
\addlinespace
Objective biomarkers & No & Optional & Optional & Integrated (Tier 2) \\
\addlinespace
Subgroup identification & No & No & No & Yes (co-occurrence model) \\
\bottomrule
\end{tabular}
\end{table}

\begin{observation}[Framework Compatibility]
The novel framework is \textit{compatible} with existing criteria rather than contradictory:
\begin{itemize}
    \item Tier 1 clinical criteria align with Canadian Consensus and IOM requirements
    \item Post-exertional malaise remains the mandatory hallmark (consistent with ICC, Canadian, IOM)
    \item 6-month duration threshold maintained (all modern criteria)
    \item Tier 2 and Tier 3 represent \textit{additions} that do not invalidate previous diagnoses
\end{itemize}

Patients meeting Fukuda, Canadian Consensus, ICC, or IOM criteria will meet Tier 1 of the novel framework. The novel framework adds biological stratification (Tier 2) and risk assessment (Tier 3) that can be applied retroactively to existing cohorts.
\end{observation}

\subsection{Clinical Advantages of the Novel Framework}
\label{subsec:advantages}

\begin{enumerate}
    \item \textbf{Precision medicine}: Biological phenotyping enables targeted treatment rather than trial-and-error

    \item \textbf{Explains treatment heterogeneity}: Response variability attributed to different rate-limiting domains rather than ``treatment doesn't work''

    \item \textbf{Early intervention guidance}: 6-month and 2-year thresholds identify critical windows for aggressive treatment

    \item \textbf{Progression prevention}: RED FLAG criteria enable emergency intervention before irreversible severe disease

    \item \textbf{Research stratification}: Multi-domain classification allows trials to enrich for patients with specific phenotypes (e.g., daratumumab trial selecting autoimmune-domain-positive patients)

    \item \textbf{Acknowledges complexity}: Multi-label classification reflects biological reality (most patients have 3+ domains) rather than forcing heterogeneous patients into single category

    \item \textbf{Actionable at point of care}: Tier 1 (clinical) immediately implementable; Tier 2 (biological) scalable as biomarkers become commercially available; Tier 3 (risk) requires only clinical observation
\end{enumerate}

