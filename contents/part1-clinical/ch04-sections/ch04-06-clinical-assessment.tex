\section{Clinical Assessment}
\label{sec:assessment}

A thorough clinical assessment is essential for ME/CFS diagnosis, both to establish the presence of diagnostic criteria and to rule out alternative explanations for symptoms.

\subsection{History Taking}

\subsubsection{Onset Characterization}

\begin{observation}[Onset Patterns as Diagnostic Clues]
\label{obs:onset-patterns}
The pattern of disease onset provides diagnostic and prognostic information:

\paragraph{Sudden Onset (60--80\% of cases):}
\begin{itemize}
    \item Patient can identify exact date or event when illness began
    \item Most commonly follows acute infection: infectious mononucleosis (EBV), influenza, COVID-19, gastroenteritis
    \item May follow other physiological stressors: surgery, trauma, childbirth, vaccination
    \item Strong temporal association suggests post-infectious mechanism
    \item \textbf{Diagnostic value}: Sudden onset after documented infection strongly supports ME/CFS diagnosis
\end{itemize}

\paragraph{Gradual Onset (20--40\% of cases):}
\begin{itemize}
    \item Symptoms develop over weeks to months without clear precipitant
    \item May follow period of chronic stress, cumulative sleep deprivation, or overwork
    \item Patient cannot identify specific triggering event
    \item \textbf{Diagnostic challenge}: Gradual onset requires more thorough differential diagnosis (autoimmune disease, occult malignancy, endocrine disorders)
\end{itemize}
\end{observation}

\subsubsection{Key Questions for Establishing Diagnosis}

\paragraph{Post-Exertional Malaise Assessment:}
\begin{enumerate}
    \item ``After physical activity, do you feel worse than expected?''
    \item ``Is there a delay between activity and symptom worsening? How long?'' (Typical: 12--72 hours)
    \item ``How long does it take to recover from doing too much?'' (ME/CFS: days to weeks)
    \item ``Can you reliably predict what activities will make you crash?''
    \item ``Do mental tasks (reading, concentration) also trigger symptom flares?'' (Cognitive PEM distinguishes ME/CFS from deconditioning)
\end{enumerate}

\paragraph{Sleep Assessment:}
\begin{enumerate}
    \item ``Do you wake up feeling refreshed?'' (ME/CFS: No, regardless of duration)
    \item ``How many hours do you sleep?'' (Rule out insufficient sleep)
    \item ``Do you snore? Have you been told you stop breathing during sleep?'' (Screen for obstructive sleep apnea)
    \item ``What time do you go to sleep and wake up?'' (Assess circadian rhythm disorders)
\end{enumerate}

\paragraph{Cognitive Dysfunction:}
\begin{enumerate}
    \item ``Do you have trouble finding words or finishing sentences?''
    \item ``Do you lose your train of thought mid-conversation?''
    \item ``Can you read and retain information like you used to?''
    \item ``Do you have difficulty with tasks that require sustained focus?''
    \item ``Are these problems worse after physical or mental exertion?'' (Cognitive PEM)
\end{enumerate}

\paragraph{Orthostatic Symptoms:}
\begin{enumerate}
    \item ``Do you feel dizzy or lightheaded when standing up?''
    \item ``Are showers or baths difficult? Do you need to sit?''
    \item ``Do your symptoms worsen when standing for extended periods?''
    \item ``Do you feel better when lying down?''
\end{enumerate}

\paragraph{Functional Impact Assessment:}
\begin{enumerate}
    \item ``What percentage of your pre-illness activity level can you sustain now?''
    \item ``What activities have you had to give up?'' (Work, social activities, hobbies, childcare)
    \item ``On a scale of 0--100 (Bell Disability Scale), what is your functional capacity?''
    \item ``How many hours per week do you spend horizontal (lying down)?''
\end{enumerate}

\subsection{Physical Examination}

\subsubsection{Orthostatic Vital Signs}

\begin{requirement}[NASA Lean Test or Orthostatic Vital Signs]
\label{req:orthostatic-testing}
Orthostatic intolerance assessment should be performed on all ME/CFS patients:

\paragraph{Protocol:}
\begin{enumerate}
    \item Patient supine for 5 minutes → measure heart rate (HR) and blood pressure (BP)
    \item Patient stands upright → measure HR and BP at 1, 3, 5, and 10 minutes
    \item Record symptoms during test (lightheadedness, nausea, cognitive impairment)
\end{enumerate}

\paragraph{Abnormal Findings:}
\begin{itemize}
    \item \textbf{Postural Orthostatic Tachycardia Syndrome (POTS)}: HR increase $\geq 30$ bpm (or $\geq 40$ bpm in adolescents) within 10 minutes of standing, without orthostatic hypotension
    \item \textbf{Orthostatic Hypotension}: Systolic BP decrease $\geq 20$ mmHg or diastolic BP decrease $\geq 10$ mmHg
    \item \textbf{Neurally Mediated Hypotension (NMH)}: Delayed BP drop (after 5--10 minutes standing)
    \item \textbf{Symptom reproduction}: Patient reports typical symptoms even without meeting BP/HR criteria
\end{itemize}

\paragraph{Interpretation:}
70--90\% of ME/CFS patients demonstrate orthostatic intolerance on objective testing. Absence of objective findings does not exclude ME/CFS, but presence strongly supports the diagnosis and guides treatment (salt, fluids, fludrocortisone, midodrine).
\end{requirement}

\subsubsection{Neurological Examination}

\begin{observation}[Neurological Findings in ME/CFS]
\label{obs:neuro-exam}
The neurological examination in ME/CFS typically shows:

\paragraph{Usually Normal:}
\begin{itemize}
    \item Cranial nerves intact
    \item Motor strength 5/5 (though patients report subjective weakness)
    \item Deep tendon reflexes normal
    \item No pathological reflexes (Babinski negative)
\end{itemize}

\paragraph{Potential Abnormalities:}
\begin{itemize}
    \item \textbf{Cognitive testing}: Impaired serial 7s, word recall, attention tasks
    \item \textbf{Tandem gait or Romberg}: May reveal subtle ataxia or balance impairment
    \item \textbf{Sustained muscle testing}: Rapid fatigability (e.g., handgrip dynamometer shows dramatic decline with repeated testing)
    \item \textbf{Sensory testing}: Hyperalgesia or allodynia in some patients (small fiber neuropathy)
\end{itemize}

\paragraph{Clinical Significance:}
The paucity of objective findings on standard neurological examination \textit{despite severe symptoms} is characteristic of ME/CFS. This discordance (severe functional impairment with normal gross exam) historically led to dismissal of ME/CFS as ``psychosomatic,'' but advanced imaging and functional testing reveal objective abnormalities (cerebral blood flow reduction, autonomic dysfunction, immune activation).
\end{observation}

\subsubsection{Tender Point Assessment}

\begin{observation}[Fibromyalgia Overlap]
\label{obs:fibromyalgia-overlap}
30--70\% of ME/CFS patients meet criteria for fibromyalgia (widespread pain with tender points). Assessment:
\begin{itemize}
    \item Digital palpation of 18 tender point sites with 4 kg pressure
    \item Fibromyalgia: $\geq 11$ of 18 sites tender
    \item Presence of fibromyalgia does not exclude ME/CFS; these are frequently comorbid
    \item Guides pain management strategy
\end{itemize}
\end{observation}

\subsection{Laboratory Testing}

\subsubsection{Mandatory Exclusionary Testing}

\begin{requirement}[Minimum Laboratory Workup]
\label{req:lab-workup}
The following tests are required to rule out alternative diagnoses:

\paragraph{Hematology:}
\begin{itemize}
    \item \textbf{Complete Blood Count (CBC)}: Rule out anemia, leukemia, lymphoma
    \item If anemia present: Iron studies, B12, folate
\end{itemize}

\paragraph{Chemistry:}
\begin{itemize}
    \item \textbf{Comprehensive Metabolic Panel (CMP)}: Rule out renal failure, hepatic dysfunction, electrolyte disorders, diabetes
\end{itemize}

\paragraph{Endocrine:}
\begin{itemize}
    \item \textbf{Thyroid function}: TSH, free T4 (hypothyroidism is a common mimic)
    \item Consider: Morning cortisol, ACTH stimulation test if Addison disease suspected
\end{itemize}

\paragraph{Inflammation:}
\begin{itemize}
    \item \textbf{Erythrocyte Sedimentation Rate (ESR)} or \textbf{C-Reactive Protein (CRP)}: Rule out active inflammatory disease
    \item \textbf{Note}: ESR/CRP are typically \textit{normal} in ME/CFS, distinguishing it from autoimmune diseases
\end{itemize}

\paragraph{Autoimmune Screening:}
\begin{itemize}
    \item \textbf{Antinuclear Antibody (ANA)}: Screen for lupus, Sjögren syndrome, other connective tissue diseases
    \item If ANA positive: Reflex to specific antibodies (anti-dsDNA, anti-Ro, anti-La)
\end{itemize}

\paragraph{Vitamins:}
\begin{itemize}
    \item \textbf{Vitamin D}: Deficiency is extremely common and contributes to fatigue
    \item \textbf{Vitamin B12}: Deficiency causes fatigue and cognitive impairment
\end{itemize}

\paragraph{Sleep Disorders:}
\begin{itemize}
    \item \textbf{Polysomnography}: Rule out obstructive sleep apnea (OSA) or upper airway resistance syndrome (UARS)
    \item OSA can fully mimic ME/CFS; CPAP treatment produces dramatic improvement in true OSA
    \item OSA and ME/CFS can coexist; treating comorbid OSA improves but does not cure ME/CFS
\end{itemize}
\end{requirement}

\begin{observation}[Typical Laboratory Profile in ME/CFS]
\label{obs:lab-profile}
The characteristic laboratory finding in ME/CFS is that \textbf{standard tests are normal}:
\begin{itemize}
    \item CBC: Normal (no anemia, normal WBC count)
    \item CMP: Normal (normal kidney, liver, electrolytes)
    \item TSH: Normal
    \item ESR/CRP: Normal or low-normal (distinguishes from inflammatory autoimmune diseases)
    \item ANA: Usually negative (or low-titer positive without clinical significance)
\end{itemize}

This pattern---severe functional disability with normal routine labs---is diagnostically significant. It distinguishes ME/CFS from conditions that present with obvious laboratory abnormalities.
\end{observation}

\subsubsection{Advanced Biomarker Testing (If Available)}

\begin{observation}[Emerging Biomarkers for Biological Phenotyping]
\label{obs:advanced-biomarkers}
If resources permit, advanced testing can guide treatment:

\paragraph{Autoimmune Domain:}
\begin{itemize}
    \item GPCR autoantibody panel (β₂-adrenergic, M3/M4 muscarinic)
    \item NK cell count and cytotoxicity assay
    \item Flow cytometry for plasma cell populations (CD38⁺CD138⁺)
\end{itemize}

\paragraph{Metabolic Domain:}
\begin{itemize}
    \item Heng 7-biomarker panel (AMP, ADP, VWF, fibronectin, TSP-1, PDGF-BB, TGF-β3) when commercially available
    \item Fasting lactate (elevated suggests mitochondrial dysfunction)
    \item ATP profile (specialized labs only)
\end{itemize}

\paragraph{Objective Functional Testing:}
\begin{itemize}
    \item \textbf{Two-day cardiopulmonary exercise testing (CPET)}: Gold standard for documenting PEM
    \item Day 1 vs.\ Day 2 comparison shows failure to reproduce work capacity
    \item Reduction in VO₂max, ventilatory threshold on Day 2 is diagnostic
\end{itemize}

\paragraph{Autonomic Testing:}
\begin{itemize}
    \item Formal tilt table testing (if orthostatic symptoms prominent)
    \item Heart rate variability analysis
    \item Quantitative sudomotor axon reflex test (QSART)
\end{itemize}

These tests are \textit{not required for diagnosis} but enable Tier 2 biological phenotyping and treatment stratification.
\end{observation}

