\section{International Consensus Criteria (2011)}
\label{sec:icc}

The International Consensus Criteria (ICC), published in 2011 by Carruthers et al.~\cite{Carruthers2011}, represents the most restrictive and biologically-oriented diagnostic framework. The ICC explicitly adopts the term ``myalgic encephalomyelitis'' (ME) to emphasize the neurological and immunological features of the disease, rejecting the broader ``chronic fatigue syndrome'' label as insufficiently specific.

\subsection{Required Criteria}

\begin{requirement}[International Consensus Criteria Structure]
\label{req:icc}
Diagnosis of myalgic encephalomyelitis requires \textbf{post-exertional neuroimmune exhaustion (PENE)} as the mandatory hallmark PLUS manifestations from at least THREE neurological impairment categories PLUS at least ONE manifestation from each of THREE immune/gastro-intestinal/genitourinary, energy metabolism/transport, and cardiovascular/respiratory/thermoregulatory categories.

\paragraph{A. Post-Exertional Neuroimmune Exhaustion (PENE) --- MANDATORY}

\textbf{PENE is the central diagnostic feature and must be present.}

Pathological inability to produce sufficient energy on demand with the following characteristics:
\begin{itemize}
    \item \textbf{Marked, rapid physical and/or cognitive fatigability} in response to exertion
    \item \textbf{Post-exertional symptom exacerbation}: Disproportionate loss of physical and mental stamina, rapid muscular and cognitive fatigability, post-exertional malaise and/or pain, and tendency for other associated symptoms to worsen
    \item \textbf{Post-exertional exhaustion}: May occur immediately after activity or be delayed by hours or days
    \item \textbf{Recovery period is prolonged}: Usually 24 hours or longer
    \item \textbf{Low threshold of physical and mental fatigability}: Results in substantial reduction in pre-illness activity level
\end{itemize}

\paragraph{B. Neurological Impairments (at least THREE required)}

\begin{enumerate}
    \item \textbf{Neurocognitive Impairments}:
    \begin{itemize}
        \item Difficulty processing information (slowed thought, impaired concentration)
        \item Short-term memory loss
        \item Word-finding difficulty, impaired psychomotor function
        \item Perceptual/sensory disturbances (spatial instability, disorientation, inability to focus vision)
        \item Ataxia, muscle weakness, fasciculations
    \end{itemize}

    \item \textbf{Pain}:
    \begin{itemize}
        \item Headaches (new type, pattern, or severity)
        \item Significant pain in muscles, muscle-tendon junctions, joints, abdomen, or chest
        \item Pain can be migratory, generalized or localized, often changing in distribution
    \end{itemize}

    \item \textbf{Sleep Disturbance}:
    \begin{itemize}
        \item Disturbed sleep patterns: insomnia, prolonged sleep (hypersomnia), disturbed sleep/wake cycle
        \item Unrefreshing sleep: Patient awakens feeling exhausted regardless of sleep duration
    \end{itemize}

    \item \textbf{Neurosensory, Perceptual, and Motor Disturbances}:
    \begin{itemize}
        \item Sensory hypersensitivity: photophobia, hyperacusis, heightened sensitivities to odors, taste, touch
        \item Motor disturbances: muscle weakness, twitching, poor coordination, ataxia
    \end{itemize}
\end{enumerate}

\paragraph{C. Immune, Gastro-Intestinal, and Genitourinary Impairments (at least ONE)}

\begin{itemize}
    \item \textbf{Immune}: Tender lymph nodes, recurrent sore throat, recurrent flu-like symptoms, general malaise, new sensitivities to food/medications/chemicals
    \item \textbf{Gastro-intestinal}: Nausea, abdominal pain, bloating, irritable bowel syndrome
    \item \textbf{Genitourinary}: Urinary urgency or frequency, nocturia
\end{itemize}

\paragraph{D. Energy Production/Transportation Impairments (at least ONE)}

\begin{itemize}
    \item \textbf{Cardiovascular}: Inability to tolerate upright posture (orthostatic intolerance, neurally mediated hypotension, postural orthostatic tachycardia syndrome), palpitations, lightheadedness
    \item \textbf{Respiratory}: Dyspnea, labored breathing, air hunger
    \item \textbf{Loss of thermostatic stability}: Subnormal body temperature, marked diurnal fluctuation, sweating episodes, cold extremities, intolerance to heat or cold
    \item \textbf{Intolerance to extremes of temperature}
\end{itemize}
\end{requirement}

\subsection{Phenotype Categories}

The ICC proposes operational phenotype categories to capture disease heterogeneity:

\begin{enumerate}
    \item \textbf{ME with Fibromyalgia}: Patients meeting ME criteria with widespread pain and tenderness
    \item \textbf{ME with Myofascial Pain Syndrome}: Regional pain with trigger points
    \item \textbf{ME with Postural Orthostatic Tachycardia Syndrome (POTS)}: ME with documented autonomic dysfunction
    \item \textbf{ME with Irritable Bowel Syndrome}: ME with prominent gastrointestinal manifestations
    \item \textbf{ME with Multiple Chemical Sensitivity}: ME with sensitivity to environmental chemicals
\end{enumerate}

These categories are \textbf{not mutually exclusive}; patients may meet criteria for multiple phenotypes simultaneously.

\subsection{Duration and Exclusions}

\begin{itemize}
    \item \textbf{Duration}: Symptom persistence for at least \textbf{6 months}
    \item \textbf{Pediatric Exception}: In children and adolescents, 3 months may be sufficient for diagnosis given the urgency of early intervention
    \item \textbf{Exclusions}: Active disease processes that explain most symptoms must be ruled out (e.g., untreated hypothyroidism, obstructive sleep apnea)
    \item \textbf{Comorbidities allowed}: Fibromyalgia, myofascial pain, temporomandibular disorder, irritable bowel syndrome, interstitial cystitis, Raynaud phenomenon, mitral valve prolapse, migraines can coexist with ME
\end{itemize}

\subsection{Strengths and Limitations}

\begin{observation}[ICC Strengths]
\label{obs:icc-strengths}
The ICC framework has several advantages:
\begin{itemize}
    \item \textbf{Biological orientation}: Emphasizes objective neurological and immune manifestations rather than subjective fatigue
    \item \textbf{Post-exertional neuroimmune exhaustion as mandatory}: Recognizes PEM as the pathognomonic feature
    \item \textbf{Multi-system requirement}: Requires manifestations across multiple physiological systems, increasing specificity
    \item \textbf{Phenotype categories}: Acknowledges heterogeneity and common comorbidities
    \item \textbf{Higher specificity}: More restrictive than Canadian Consensus or Fukuda, resulting in more homogeneous research cohorts
\end{itemize}
\end{observation}

\begin{warning}[ICC Limitations]
\label{warn:icc-limitations}
The restrictiveness of ICC creates challenges:
\begin{itemize}
    \item \textbf{Excludes mild cases}: Patients with genuine ME/CFS who do not yet manifest symptoms across all required categories may be missed
    \item \textbf{Clinical impracticality}: Detailed assessment across 8 categories requires extensive clinical time and expertise
    \item \textbf{Reduced sensitivity}: Systematic review found ICC identifies only 60\% of patients meeting Canadian Consensus Criteria~\cite{Brurberg2014}
    \item \textbf{Formal set-theoretic relationship}: $\text{ICC} \subset \text{Canadian Consensus} \subset \text{Fukuda}$ --- ICC is the most restrictive subset
    \item \textbf{Delayed diagnosis risk}: Waiting for full symptom constellation may delay intervention during the critical 6-month window
\end{itemize}
\end{warning}

\subsection{Research and Clinical Application}

The ICC is \textbf{optimal for research} where high specificity and phenotypic homogeneity are priorities, reducing heterogeneity that can obscure treatment signals. However, for \textbf{clinical practice}, the more inclusive Canadian Consensus Criteria or IOM criteria are preferred to avoid missing early-stage or mild cases that would benefit from intervention.

\section{Institute of Medicine Criteria (2015)}
