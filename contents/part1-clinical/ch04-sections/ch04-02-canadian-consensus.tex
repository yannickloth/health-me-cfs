\section{Canadian Consensus Criteria (2003)}
\label{sec:ccc}

The Canadian Consensus Criteria~\cite{Carruthers2003} emerged from a panel of physicians, researchers, and teaching faculty to provide a clinically oriented case definition emphasizing characteristic features.

\subsection{Required Criteria}

\begin{requirement}[Canadian Consensus Criteria Structure]
\label{req:ccc-structure}
Diagnosis requires ALL of the following:

\paragraph{1. Fatigue}
\begin{itemize}
    \item Clinically evaluated, unexplained persistent or relapsing chronic fatigue
    \item New or definite onset (not lifelong)
    \item Not result of ongoing exertion
    \item Not substantially alleviated by rest
    \item Substantial reduction in previous levels of occupational, educational, social, or personal activities
\end{itemize}

\paragraph{2. Post-Exertional Malaise and/or Fatigue (MANDATORY)}
\begin{itemize}
    \item Inappropriate loss of physical and mental stamina
    \item Rapid muscular and cognitive fatigability
    \item Post-exertional malaise and/or fatigue
    \item Tendency for other associated symptoms within the patient's cluster to worsen
    \item \textbf{Recovery period}: Pathologically slow (24 hours or longer)
\end{itemize}

\paragraph{3. Sleep Dysfunction}
\begin{itemize}
    \item Unrefreshing sleep or sleep quantity or rhythm disturbances (hypersomnia, insomnia, reversed sleep-wake cycle)
\end{itemize}

\paragraph{4. Pain (significant degree in at least one location)}
\begin{itemize}
    \item Myalgia: muscle pain, aching, or stiffness
    \item Arthralgia: joint pain (migratory, without joint swelling or redness)
    \item Headaches of new type, pattern, or severity
\end{itemize}

\paragraph{5. Neurological/Cognitive Manifestations ($\geq$2 required)}
\begin{itemize}
    \item Confusion, impaired concentration/short-term memory, disorientation
    \item Difficulty with information processing, categorizing, word retrieval
    \item Perceptual/sensory disturbances (spatial instability, disorientation, inability to focus vision)
    \item Ataxia, muscle weakness, fasciculations
\end{itemize}

\paragraph{6. At Least ONE Symptom from TWO of the Following Categories:}

\begin{itemize}
    \item \textbf{Autonomic Manifestations}: Orthostatic intolerance (neurally mediated hypotension, POTS, delayed postural hypotension), lightheadedness, extreme pallor, nausea and irritable bowel syndrome, urinary frequency/bladder dysfunction, palpitations with or without cardiac arrhythmias, exertional dyspnea

    \item \textbf{Neuroendocrine Manifestations}: Loss of thermostatic stability (subnormal body temperature, marked daily fluctuation), intolerance of extremes of heat and cold, marked weight change, loss of adaptability/worsening symptoms with stress

    \item \textbf{Immune Manifestations}: Tender lymph nodes, recurrent sore throat, recurrent flu-like symptoms, general malaise, new sensitivities to food/medications/chemicals
\end{itemize}

\paragraph{7. Duration}
Illness persists $\geq$6 months. May be preceded by various infections or other triggering events.
\end{requirement}

\subsection{Exclusions}

Exclude active disease processes that explain most major symptoms. Comorbid conditions do not exclude diagnosis if they do not explain the constellation of symptoms.

\subsection{Strengths and Limitations}

\paragraph{Strengths}
\begin{itemize}
    \item Emphasis on post-exertional malaise as mandatory criterion
    \item Comprehensive symptom coverage across multiple systems
    \item Widely adopted in clinical practice
    \item More specific than Fukuda criteria
\end{itemize}

\paragraph{Limitations}
\begin{itemize}
    \item Complex algorithm may be difficult to apply consistently
    \item Selects more severely affected patients (lower sensitivity for mild cases)
    \item Some symptom requirements (e.g., ``significant degree'') lack operational definitions
    \item Not validated against objective biomarkers or treatment response
\end{itemize}

