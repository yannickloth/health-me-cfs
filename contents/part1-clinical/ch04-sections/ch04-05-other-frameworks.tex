\section{Other Diagnostic Frameworks}
\label{sec:other-criteria}

\subsection{Fukuda Criteria (1994)}

The Fukuda criteria, published by the CDC in 1994~\cite{Fukuda1994}, represented the first widely-adopted standardized definition of chronic fatigue syndrome. These criteria dominated ME/CFS research for two decades.

\begin{requirement}[Fukuda Diagnostic Criteria]
\label{req:fukuda}
Diagnosis requires:

\paragraph{1. Clinically Evaluated, Unexplained, Persistent or Relapsing Chronic Fatigue that:}
\begin{itemize}
    \item Is of \textbf{new or definite onset} (not lifelong)
    \item Is \textbf{not the result of ongoing exertion}
    \item Is \textbf{not substantially alleviated by rest}
    \item Results in \textbf{substantial reduction} in previous levels of occupational, educational, social, or personal activities
\end{itemize}

\paragraph{2. Four or More of the Following Symptoms (concurrent for ≥6 months):}
\begin{enumerate}
    \item Impaired memory or concentration
    \item Sore throat
    \item Tender cervical or axillary lymph nodes
    \item Muscle pain
    \item Multi-joint pain without swelling or redness
    \item Headaches of new type, pattern, or severity
    \item Unrefreshing sleep
    \item Post-exertional malaise lasting more than 24 hours
\end{enumerate}
\end{requirement}

\begin{observation}[Historical Significance]
The Fukuda criteria played a crucial role in standardizing ME/CFS research:
\begin{itemize}
    \item First internationally-adopted consensus definition
    \item Enabled comparison across studies and centers
    \item Established 6-month duration threshold
    \item Required objective clinical evaluation
\end{itemize}
\end{observation}

\begin{warning}[Critical Limitations]
\label{warn:fukuda-limitations}
The Fukuda criteria have fundamental flaws that limit their current utility:

\paragraph{Post-Exertional Malaise Not Mandatory:}
The most pathognomonic feature of ME/CFS (PEM) is merely one of eight optional symptoms. This allows diagnosis of patients without the hallmark feature, including those with:
\begin{itemize}
    \item Primary depression
    \item Deconditioning
    \item Other fatiguing conditions without PEM
\end{itemize}

\paragraph{Mathematical Analysis of Heterogeneity:}
The requirement of ``4 or more of 8 symptoms'' yields:
\begin{equation}
\binom{8}{4} + \binom{8}{5} + \binom{8}{6} + \binom{8}{7} + \binom{8}{8} = 70 + 56 + 28 + 8 + 1 = 163 \text{ distinct profiles}
\end{equation}

Two patients can both meet Fukuda criteria while sharing only 2 of 8 symptoms (50\% overlap in the limiting case). This mathematical heterogeneity explains null results in many Fukuda-based trials.

\paragraph{Overinclusion:}
Systematic comparison studies found that Fukuda criteria capture patients who do not meet more restrictive criteria (Canadian Consensus, ICC) and who have:
\begin{itemize}
    \item Less severe functional impairment
    \item Better prognosis
    \item Lower biomarker abnormality rates
    \item Higher rates of primary psychiatric diagnoses
\end{itemize}

\paragraph{Research Impact:}
Many failed clinical trials used Fukuda criteria, likely enrolling heterogeneous populations including patients without true ME/CFS. This contributed to therapeutic nihilism.
\end{warning}

\begin{observation}[Current Status]
Modern research increasingly avoids Fukuda criteria in favor of Canadian Consensus (2003), ICC (2011), or IOM (2015). The Fukuda criteria remain historically important but are now recognized as insufficiently specific for ME/CFS.
\end{observation}

\subsection{Oxford Criteria (1991)}

The Oxford criteria~\cite{Sharpe1991oxford}, published in 1991, represent the \textbf{broadest and most controversial} definition of chronic fatigue syndrome.

\begin{requirement}[Oxford Diagnostic Criteria]
\label{req:oxford}
Diagnosis requires:
\begin{enumerate}
    \item \textbf{Severe disabling fatigue} of at least 6 months' duration that:
    \begin{itemize}
        \item Affects physical and mental functioning
        \item Was present for more than 50\% of the time
    \end{itemize}
    \item \textbf{Other symptoms}, particularly myalgia, mood disturbance, and sleep disturbance, may be present
    \item \textbf{Exclusions}: Defined medical conditions, psychotic disorders, substance abuse, eating disorders
    \item \textbf{Depression and anxiety NOT excluded}
\end{enumerate}
\end{requirement}

\begin{warning}[Fundamental Problems with Oxford Criteria]
\label{warn:oxford-problems}
The Oxford criteria are widely rejected by patients, clinicians, and researchers for the following reasons:

\paragraph{No Requirement for Post-Exertional Malaise:}
The pathognomonic feature of ME/CFS is entirely absent. Patients meeting Oxford criteria may have:
\begin{itemize}
    \item Primary depression with fatigue
    \item Deconditioning from sedentary lifestyle
    \item Idiopathic chronic fatigue (fatigue without clear cause)
\end{itemize}

\paragraph{Allows Primary Psychiatric Diagnoses:}
Unlike all other ME/CFS criteria, Oxford explicitly allows comorbid depression and anxiety \textit{even when these could fully explain the fatigue}. This conflates ME/CFS with depression-related fatigue.

\paragraph{Set-Theoretic Implications:}
Define patient populations by criteria:
\begin{equation}
O \supset F \supset C \supset I
\end{equation}
where $O$ = Oxford, $F$ = Fukuda, $C$ = Canadian Consensus, $I$ = ICC.

The Oxford criteria capture a superset that includes patients with ME/CFS (satisfying more restrictive criteria) plus patients with primary depression, idiopathic fatigue, and deconditioning.

\paragraph{Harm from CBT/GET Trials:}
The most harmful aspect: Oxford criteria were used in the PACE trial~\cite{White2011pace} and other studies promoting cognitive behavioral therapy (CBT) and graded exercise therapy (GET) as treatments for ``CFS.'' However:
\begin{itemize}
    \item Patient surveys show GET causes harm in 50--70\% of ME/CFS patients~\cite{MEAssociation2015survey}
    \item CBT/GET may be appropriate for depression or deconditioning but are contraindicated for true ME/CFS
    \item By enrolling patients without PEM, these trials tested interventions on a population distinct from ME/CFS
\end{itemize}
\end{warning}

\begin{observation}[Research Community Consensus]
\label{obs:oxford-rejection}
The Oxford criteria are now explicitly rejected by:
\begin{itemize}
    \item The NIH (U.S. National Institutes of Health) ME/CFS research guidelines
    \item The CDC (U.S. Centers for Disease Control)
    \item Leading ME/CFS researchers and clinicians
    \item Patient advocacy organizations
\end{itemize}

Studies using Oxford criteria should be interpreted with extreme caution, as they likely include substantial proportions of patients without ME/CFS.
\end{observation}

\subsection{Pediatric Criteria}

ME/CFS in children and adolescents presents diagnostic challenges due to developmental differences in symptom expression, comorbidities, and functional impact.

\begin{requirement}[Pediatric Adaptations]
\label{req:pediatric}
Diagnosis in children should use the same criteria (Canadian Consensus, IOM, or ICC) with the following modifications:

\paragraph{1. Duration:}
\begin{itemize}
    \item Standard: 6 months in adults
    \item Pediatric: May use \textbf{3 months} if symptoms are severe and progression is documented
    \item Rationale: Early diagnosis enables intervention during critical developmental windows
\end{itemize}

\paragraph{2. Activity Reduction:}
\begin{itemize}
    \item Assess relative to \textbf{age-appropriate activities}: school attendance, sports participation, social activities with peers
    \item May manifest as: inability to attend full school day, requiring home tutoring, dropping out of extracurricular activities
    \item Pediatric patients may have better baseline reserves, so functional impairment can be harder to detect
\end{itemize}

\paragraph{3. Post-Exertional Malaise:}
\begin{itemize}
    \item Children may not articulate PEM clearly; ask caregivers about: ``Does your child crash after activities?''
    \item School attendance patterns are diagnostic: can attend Monday but not Tuesday (PEM delay)
    \item May manifest as behavioral changes (irritability, emotional lability) rather than reported exhaustion
\end{itemize}

\paragraph{4. Cognitive Symptoms:}
\begin{itemize}
    \item Assess relative to prior academic performance, not population norms
    \item May manifest as: declining grades, inability to complete homework, processing speed reduction
    \item Distinguish from learning disabilities (which would have been present earlier)
\end{itemize}

\paragraph{5. Orthostatic Intolerance:}
\begin{itemize}
    \item Highly prevalent in pediatric ME/CFS (70--90\%)
    \item May present as: difficulty standing in school assemblies, morning symptom worsening (after overnight recumbency), shower intolerance
    \item Objective testing: NASA Lean Test or tilt table (age-appropriate protocols)
\end{itemize}
\end{requirement}

\begin{warning}[Pediatric Differential Diagnosis]
\label{warn:pediatric-differential}
Additional considerations for children:
\begin{itemize}
    \item \textbf{School avoidance vs. ME/CFS}: Distinguish by presence of PEM (in ME/CFS, even desired activities trigger crashes)
    \item \textbf{Growth and puberty}: Rule out growth-related fatigue, iron deficiency from menstruation
    \item \textbf{Viral triggers}: Infectious mononucleosis is a common ME/CFS trigger in adolescents
    \item \textbf{Comorbidities}: POTS and orthostatic intolerance are especially common in pediatric onset
\end{itemize}
\end{warning}

\begin{observation}[Prognosis and Early Intervention]
\label{obs:pediatric-prognosis}
Pediatric ME/CFS has distinct prognostic features:
\begin{itemize}
    \item \textbf{Better prognosis than adults}: Some studies suggest 50--70\% improvement or recovery rates in adolescents, though methodological issues may inflate these estimates
    \item \textbf{Critical intervention window}: Early aggressive pacing and school accommodation may prevent progression to severe disease
    \item \textbf{Educational impact}: Lost school years during critical developmental periods create long-term consequences
    \item \textbf{Recommendation}: Diagnose at 3 months if symptoms are severe; immediate school accommodations (reduced hours, remote learning, rest breaks) to prevent cumulative PEM damage
\end{itemize}

For detailed pediatric treatment protocols, see Chapter~\ref{ch:pediatric-severe} (severe/housebound cases) and Chapter~\ref{ch:pediatric-ambulatory} (school-attending cases).
\end{observation}

