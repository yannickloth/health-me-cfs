\chapter{Introduction to ME/CFS}
\label{ch:introduction}

Myalgic encephalomyelitis/chronic fatigue syndrome (ME/CFS) is a complex, chronic, multi-system disease characterized by severe and disabling fatigue, post-exertional malaise, unrefreshing sleep, cognitive dysfunction, and autonomic dysregulation. This document provides a comprehensive overview of current understanding, research, and clinical approaches to ME/CFS.

\section{Overview and Terminology}
\label{sec:terminology}

ME/CFS has been recognized as a distinct clinical entity by major health organizations, including the World Health Organization (ICD-11 code 8E49), the Centers for Disease Control and Prevention, and the National Institutes of Health. The condition affects an estimated 0.89\% to 2.5\% of the global population, with significant variation based on diagnostic criteria used.

The terminology surrounding this condition has evolved over time. While ``chronic fatigue syndrome'' became widely used in the late 1980s, many patient advocates and researchers prefer ``myalgic encephalomyelitis'' as it better reflects the neurological and immunological aspects of the disease. This document uses the combined term ME/CFS to acknowledge both naming conventions.

\section{Historical Context}
\label{sec:history}

% Historical development of understanding ME/CFS
% Key outbreaks and case clusters
% Evolution of diagnostic criteria
% Changes in medical understanding

\section{Epidemiology}
\label{sec:epidemiology}

% Prevalence and incidence data
% Geographic distribution
% Demographic patterns (age, sex, ethnicity)
% Risk factors

\section{Disease Impact}
\label{sec:impact}

% Quality of life studies
% Disability assessments
% Economic burden
% Social and psychological impact
