% Onset patterns content for Chapter 5
% This file is \input from ch05-disease-course.tex

The manner in which ME/CFS begins has both diagnostic and prognostic significance. Two primary onset patterns are recognized: acute (typically post-infectious) and gradual \cite{Jason2019onset}. Understanding these patterns helps clinicians recognize the disease earlier and may inform treatment approaches.

\subsubsection{Post-Infectious Onset}

Approximately 64\% of ME/CFS cases begin with an acute infectious illness \cite{Jason2019onset}. The patient experiences what appears to be a typical viral infection---influenza, infectious mononucleosis, respiratory illness, or gastrointestinal infection---but fails to recover. Weeks pass, then months, and the expected return to health never comes.

\paragraph{Common Triggering Infections.}
Documented infectious triggers include:
\begin{itemize}
    \item \textbf{Epstein-Barr virus (EBV)}: The most studied trigger, with 10--12\% of infectious mononucleosis cases progressing to ME/CFS
    \item \textbf{SARS-CoV-2}: COVID-19 has created a new wave of post-infectious ME/CFS (Long COVID with ME/CFS phenotype)
    \item \textbf{Influenza}: Both seasonal and pandemic strains
    \item \textbf{Enteroviruses}: Including coxsackieviruses and echoviruses
    \item \textbf{Ross River virus}: Endemic trigger in Australia
    \item \textbf{Q fever} (\textit{Coxiella burnetii}): Bacterial trigger with well-documented post-infectious fatigue
    \item \textbf{Giardiasis}: Parasitic infection linked to post-infectious ME/CFS in outbreak studies
\end{itemize}

\begin{observation}[Viral Associations: Meta-Analytic Evidence]
\label{obs:viral-meta}
A 2023 systematic review and meta-analysis of 64 studies (n=4,971 ME/CFS patients, n=9,221 controls) examining 18 viral species identified significant associations between ME/CFS and multiple viral infections~\cite{hwang2023viral}. Five viruses demonstrated odds ratios exceeding 2.0: Borna disease virus (OR$\geq$3.47), HHV-7 (OR>2.0), parvovirus B19 (OR>2.0), enterovirus (OR>2.0), and coxsackie B virus (OR>2.0). However, high heterogeneity (>50\%) was observed for EBV and enterovirus associations, suggesting these viral triggers may apply to specific subgroups rather than uniformly across all ME/CFS cases.
\end{observation}

\begin{warning}[Association vs.\ Causation in Viral Triggers]
\label{warn:viral-causation}
While meta-analytic evidence demonstrates statistical associations between viral infections and ME/CFS onset~\cite{hwang2023viral}, these data cannot establish causation. Viral reactivation may represent a consequence of immune dysfunction rather than the initiating cause. Additionally, detection bias may inflate associations, as ME/CFS patients typically undergo more extensive viral testing than matched controls. The observed heterogeneity across studies indicates that viral etiology likely applies to subsets of ME/CFS patients rather than representing a universal mechanism.
\end{warning}

The NIH deep phenotyping study focused specifically on post-infectious ME/CFS, providing detailed characterization of this subgroup \cite{walitt2024deep}.

\paragraph{Temporal Pattern.}
In acute post-infectious onset, the transition from acute infection to chronic illness is often abrupt. Patients can frequently identify the specific day or week when their illness began. The typical pattern:
\begin{enumerate}
    \item Acute infectious illness with standard symptoms (fever, malaise, respiratory or gastrointestinal symptoms)
    \item Expected recovery does not occur after 2--4 weeks
    \item Persistent fatigue, cognitive impairment, and post-exertional malaise emerge
    \item Full ME/CFS symptom complex develops over weeks to months
    \item Stabilization at significantly reduced functional capacity
\end{enumerate}

\paragraph{Pathophysiological Implications.}
Post-infectious onset suggests mechanisms involving:
\begin{itemize}
    \item Persistent viral reservoirs or reactivation of latent viruses
    \item Post-infectious autoimmunity triggered by molecular mimicry
    \item Chronic immune activation and inflammation
    \item Disruption of the gut microbiome
    \item Autonomic nervous system dysregulation
\end{itemize}

Brain imaging studies show distinct abnormalities in post-infectious ME/CFS compared to gradual-onset cases, supporting the notion that different onset patterns may involve different pathophysiological mechanisms.

\paragraph{Prognosis.}
Some studies suggest that post-infectious onset may carry a better prognosis than gradual onset, particularly when the triggering infection can be identified and when illness duration is short before diagnosis. However, this finding is not consistent across all studies, and many post-infectious cases progress to severe, permanent disability.

\subsubsection{Gradual Onset}

Approximately 36\% of ME/CFS cases (range 23--41\% across studies) develop gradually without a clear infectious trigger \cite{Jason2019onset}. Symptoms accumulate over months to years, making it difficult to identify when the illness truly began.

\paragraph{Progressive Symptom Accumulation.}
Gradual-onset ME/CFS typically follows a pattern of:
\begin{enumerate}
    \item Increasing fatigue attributed to stress, overwork, or aging
    \item Sleep disturbances that fail to respond to standard interventions
    \item Cognitive difficulties (brain fog, concentration problems, word-finding difficulties)
    \item Exercise intolerance that progressively worsens
    \item Development of post-exertional malaise, often initially unrecognized
    \item Eventual recognition that something is fundamentally wrong
\end{enumerate}

The insidious nature of gradual onset often delays diagnosis, as patients and clinicians attribute symptoms to other causes. Mean diagnostic delay is longer in gradual-onset cases, which has prognostic significance (see Section~\ref{sec:prognosis}).

\paragraph{Risk Factors.}
Gradual onset has been associated with:
\begin{itemize}
    \item Prior history of multiple infections (cumulative immune burden)
    \item Chronic stress or overwork
    \item Other chronic illnesses
    \item Higher rates of psychiatric comorbidity (though causation is unclear)
    \item Possible environmental exposures
\end{itemize}

The association with psychiatric comorbidity is controversial. It may reflect true biological comorbidity, diagnostic confusion (ME/CFS misdiagnosed as depression), or shared underlying mechanisms affecting both mood and energy regulation.

\paragraph{Diagnostic Challenges.}
Gradual onset creates particular diagnostic challenges:
\begin{itemize}
    \item No clear temporal marker for illness onset
    \item Symptoms may be attributed to depression, anxiety, or somatization
    \item Lack of infectious trigger makes the diagnosis seem less ``legitimate''
    \item Pre-illness functional level may be difficult to establish
    \item Patients may have adapted to declining function without recognizing its significance
\end{itemize}

\subsubsection{Two-Phase Onset Pattern}

A third pattern has been identified in some patients: two-phase onset \cite{Jason2019onset}. This pattern involves:
\begin{enumerate}
    \item Initial sharp deterioration (often post-infectious)
    \item Partial improvement over months
    \item Secondary deterioration to chronic ME/CFS
\end{enumerate}

This pattern may represent failed recovery from post-infectious illness, with initial improvement reflecting resolution of acute infection while underlying ME/CFS pathophysiology continues to develop.

\subsubsection{Clinical Significance of Onset Pattern}

While onset pattern provides useful clinical information, it should not be overemphasized in individual patient management. Both post-infectious and gradual-onset patients develop the same symptom complex and require the same management approaches. The key clinical implications of onset pattern include:

\begin{itemize}
    \item \textbf{Diagnostic confidence}: Post-infectious onset with clear temporal association may increase diagnostic confidence
    \item \textbf{Patient validation}: Understanding that infections can trigger chronic illness helps patients understand their condition
    \item \textbf{Research stratification}: Onset pattern may be important for identifying disease subgroups in research
    \item \textbf{Epidemiological monitoring}: Tracking post-infectious ME/CFS helps quantify the burden of infectious diseases
\end{itemize}
