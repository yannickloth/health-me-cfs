\chapter{Diagnostic Criteria and Clinical Assessment}
\label{ch:diagnostic-criteria}

Multiple diagnostic criteria have been developed for ME/CFS. This chapter reviews major frameworks and their application.

\section{Overview of Diagnostic Approaches}
\label{sec:diagnostic-overview}

% Evolution of criteria
% Comparison between frameworks
% Sensitivity and specificity
% Patient population differences

\section{Canadian Consensus Criteria (2003)}
\label{sec:ccc}

% Detailed breakdown of criteria
% Required symptoms
% Exclusions
% Strengths and limitations

\section{International Consensus Criteria (2011)}
\label{sec:icc}

% Detailed breakdown of criteria
% Focus on post-exertional neuroimmune exhaustion
% Myalgic encephalomyelitis emphasis
% Phenotype categories

\section{Institute of Medicine Criteria (2015)}
\label{sec:iom}

% SEID (Systemic Exertion Intolerance Disease)
% Core symptoms required
% Supporting symptoms
% Diagnostic algorithm

\section{Other Diagnostic Frameworks}
\label{sec:other-criteria}

\subsection{Fukuda Criteria (1994)}
% CDC criteria
% Historical significance
% Limitations

\subsection{Oxford Criteria (1991)}
% Broadest definition
% Controversies
% Why avoided in research

\subsection{Pediatric Criteria}
% Adaptations for children and adolescents
% Age-specific considerations

\section{Clinical Assessment}
\label{sec:assessment}

\subsection{History Taking}
% Key questions
% Onset patterns (sudden vs. gradual)
% Triggering events
% Functional assessment

\subsection{Physical Examination}
% What to look for
% Orthostatic vital signs
% Neurological examination
% Tender points

\subsection{Laboratory Testing}
% Ruling out exclusions
% Supportive findings
% Experimental biomarkers

\section{Differential Diagnosis}
\label{sec:differential}

% Conditions that can mimic ME/CFS
% How to distinguish
% Comorbid conditions vs. alternative diagnoses
