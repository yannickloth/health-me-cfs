\chapter{Core Symptoms}
\label{ch:core-symptoms}

ME/CFS is characterized by several hallmark symptoms that must be present for diagnosis across most diagnostic frameworks. This chapter provides detailed descriptions of each core symptom.

\section{Post-Exertional Malaise (PEM)}
\label{sec:pem}

Post-exertional malaise (PEM), also termed post-exertional symptom exacerbation (PESE) or post-exertional neuroimmune exhaustion (PENE), is considered the hallmark feature of ME/CFS.

\subsection{Definition and Characteristics}

Post-exertional malaise represents an abnormal response to physical, cognitive, or emotional exertion in which even minor activity triggers a cascade of worsening symptoms. Unlike normal fatigue, PEM is characterized by:

\begin{itemize}
    \item \textbf{Delayed onset}: Symptoms typically worsen 12--48 hours after the triggering activity
    \item \textbf{Disproportionate severity}: Minimal exertion produces profound symptom exacerbation
    \item \textbf{Prolonged recovery}: Symptom worsening persists for days to weeks or longer
    \item \textbf{Cumulative effect}: Sequential exertions compound impairment
    \item \textbf{Unpredictable threshold}: The level of activity that triggers PEM varies and may decrease over time
\end{itemize}

\subsubsection{Common Triggers}

PEM can be triggered by various forms of exertion:

\paragraph{Physical Exertion}
\begin{itemize}
    \item Walking, standing, or basic activities of daily living
    \item Exercise or physical therapy
    \item Household tasks
    \item Sexual activity
    \item Medical procedures or examinations
\end{itemize}

\paragraph{Cognitive Exertion}
\begin{itemize}
    \item Reading, writing, or computer work
    \item Conversation or social interaction
    \item Decision-making or problem-solving
    \item Sensory stimulation (light, sound, crowds)
    \item Concentration or sustained attention
\end{itemize}

\paragraph{Emotional Exertion}
\begin{itemize}
    \item Stress or anxiety
    \item Emotional processing
    \item Social demands
    \item Medical appointments or advocacy
\end{itemize}

\subsubsection{Subjective Phenomenology: The Effort-Performance Disconnect}

One of the most psychologically devastating aspects of PEM is the profound disconnect between subjective effort and objective performance. Patients consistently describe an internal experience of maximal exertion that produces minimal external results---a phenomenon that fundamentally challenges their sense of agency and capability~\cite{strassheim2021experiences,fennell2021elements}.

\paragraph{The Experience of Maximal Effort Producing Minimal Output}

Unlike healthy individuals or those with deconditioning, ME/CFS patients report that activities feel intensely demanding internally while producing negligible observable output. A patient attempting to walk across a room may experience the subjective intensity of running a marathon---racing heart, overwhelming fatigue, sense of desperation---while moving slowly and covering minimal distance. This creates a surreal mismatch between internal state and external reality.

This disconnect extends beyond physical tasks:
\begin{itemize}
    \item \textbf{Physical tasks}: Simple actions feel extraordinarily difficult; patients describe ``giving everything'' yet achieving almost nothing
    \item \textbf{Cognitive tasks}: Intense concentration yields minimal comprehension or output
    \item \textbf{Emotional regulation}: Enormous internal effort required to maintain composure or engage socially
\end{itemize}

\paragraph{Psychological Sequelae: Helplessness and Loss of Agency}

The persistent effort-performance disconnect produces profound psychological consequences distinct from primary depression:

\begin{description}
    \item[Learned helplessness] Repeated experiences of maximal effort failing to produce normal results can induce a state resembling learned helplessness---the recognition that one's actions do not reliably produce expected outcomes. This is not a cognitive distortion but an accurate perception of physiological reality.

    \item[Loss of self-efficacy] The inability to generate normal performance despite perceived maximum effort erodes confidence in one's capability. Patients often describe feeling ``weak'' or ``useless,'' not as depression-related negative cognition but as direct experiential feedback.

    \item[Betrayal by one's body] Many patients describe their body as having ``betrayed'' them or become ``enemy territory''---the normal unity between intention and execution has fractured. Motor commands and cognitive efforts no longer reliably produce proportional results.

    \item[Social invalidation] Because the internal experience of extreme exertion is invisible to observers, patients face disbelief from family, friends, employers, and medical professionals. The statement ``you don't look sick'' becomes particularly traumatic when one is experiencing maximum physiological stress.

    \item[Anticipatory anxiety] Knowledge that even minor exertion may trigger severe crashes creates pervasive anxiety around all activities. Patients must constantly calculate risk, leading to hypervigilance and decision paralysis.
\end{description}

\paragraph{Distinction from Primary Depression}

While the phenomenology of PEM may superficially resemble depression, key distinctions exist:

\begin{itemize}
    \item \textbf{Effort expenditure}: Depressed individuals typically experience reduced motivation to initiate effort; ME/CFS patients expend maximum subjective effort but achieve minimal results
    \item \textbf{Activity relationship}: Depression may improve somewhat with activity; ME/CFS worsens predictably with exertion
    \item \textbf{Physiological markers}: PEM produces objective physiological changes (documented via two-day CPET) absent in primary depression
    \item \textbf{Cognitive content}: The helplessness in ME/CFS arises from accurate perception of physiological limitation, not cognitive distortion~\cite{geraghty2019cognitive}
\end{itemize}

Many ME/CFS patients develop secondary depression as a consequence of chronic illness and loss of function, but the core effort-performance disconnect represents a direct physiological phenomenon, not a psychological disorder. The majority (78.1\%) of ME/CFS patients who experience depression develop it \emph{after} disease onset, and 96\% attribute their depression to disease severity and external factors rather than pre-existing psychiatric conditions~\cite{konig2024mental}.

\paragraph{Vulnerability and Existential Threat}

The profound energy deficit creates an acute sense of vulnerability. Patients describe feeling as though they ``wouldn't amount to shit'' in any demanding situation---an accurate assessment of their current physiological capacity, not a self-esteem issue. This recognition of one's fundamental vulnerability in a world that demands productivity and physical capability constitutes an ongoing existential threat.

For patients previously defined by physical capability, intellectual performance, or caregiving roles, the loss of reliable energy production represents a fundamental identity disruption. The inability to protect oneself, care for dependents, or meet basic social obligations creates legitimate existential distress~\cite{fennell2021elements}. Quality of life in ME/CFS is profoundly diminished, with patients scoring lower than those with multiple sclerosis, stroke, cancer, and other serious chronic conditions across nearly all functional domains~\cite{hvidberg2015quality,kingdon2018functional}.

\subsubsection{Severity Spectrum}

PEM severity varies considerably:

\begin{description}
    \item[Mild] Increased symptoms for 1--3 days following moderate exertion; can usually continue limited activities with careful pacing
    \item[Moderate] Severe symptom exacerbation lasting days to weeks following minimal exertion; requires extended rest periods
    \item[Severe] Profound crashes triggered by activities of daily living; largely bedbound; recovery may take weeks to months
    \item[Very severe] Any stimulation (light, sound, conversation) triggers immediate worsening; may be unable to tolerate even basic self-care
\end{description}

\subsubsection{Baseline Energy Insufficiency: Living Below the Survival Threshold}

While PEM represents the acute exacerbation following exertion, many ME/CFS patients describe a more insidious and pervasive problem: chronic baseline energy levels insufficient for basic existence. This creates a fundamentally different experience from episodic illness---it is a continuous state of inadequacy~\cite{strassheim2021experiences}.

\paragraph{The Experience of Perpetual Insufficiency}

Patients describe waking already depleted, as if they have already run a marathon before the day begins. Unlike healthy individuals who start each day with a replenished energy reserve, ME/CFS patients begin from deficit:

\begin{itemize}
    \item \textbf{Morning depletion}: Waking feeling as exhausted as when going to sleep, or worse
    \item \textbf{Minimum activity burden}: Even basic hygiene, eating, or sitting upright feels overwhelming
    \item \textbf{Continuous depletion}: Energy steadily drains throughout the day regardless of activity level
    \item \textbf{No reserve}: Zero capacity to handle unexpected demands
    \item \textbf{Micro-activities as exertion}: Actions that should be automatic (maintaining posture, processing sensory input) require conscious effort and consume limited energy
\end{itemize}

The experience of legs aching simply from sitting at a computer exemplifies this phenomenon. Maintaining posture---a task that should require minimal conscious attention---becomes actively depleting. Muscles fatigue from static contraction, venous pooling worsens due to inadequate muscle pump activity, and the metabolic cost of remaining upright exceeds available cellular ATP production.

\paragraph{Forced Overexertion: When Life Does Not Accommodate Limits}

Unlike research protocols where patients can carefully pace within their limits, real life imposes non-negotiable demands. This creates a situation of continuous forced overexertion:

\begin{description}
    \item[Basic survival needs] Eating, toileting, hygiene cannot be deferred indefinitely. Even these minimal activities may exceed available energy.

    \item[Medical appointments] Navigating healthcare---attending appointments, waiting in waiting rooms, explaining symptoms, completing forms---requires energy patients do not have, creating the paradox of becoming sicker from seeking medical care.

    \item[Caregiving responsibilities] Parents must feed children, pet owners must care for animals, adult children must respond to aging parents' needs. These responsibilities do not pause for energy availability.

    \item[Work and financial survival] Many patients cannot afford to stop working despite severe energy limitations. The choice becomes: exceed limits and worsen disease, or face homelessness and starvation.

    \item[Emergencies] House fires, medical emergencies, natural disasters, family crises demand immediate responses that may require weeks or months of energy expenditure in moments.

    \item[Social obligations] Complete withdrawal results in loss of relationships, but social interaction is energetically costly. Patients must choose between isolation and overexertion.

    \item[Bureaucratic demands] Disability applications, insurance appeals, medical documentation require sustained cognitive effort precisely when cognition is most impaired.
\end{description}

\paragraph{The Impossibility of Perfect Pacing}

While pacing (staying within energy limits to avoid PEM) represents the primary management strategy~\cite{jason2012energy}, perfect pacing is functionally impossible for most patients:

\begin{itemize}
    \item \textbf{Unknown threshold}: The exertion level that will trigger PEM is variable and often unknowable in advance
    \item \textbf{Declining reserves}: The safe activity level may decrease over time, making previously manageable activities dangerous
    \item \textbf{Life is not optional}: Survival needs create forced exertion regardless of consequences
    \item \textbf{Delayed feedback}: PEM onset occurs 12--48 hours after trigger, preventing real-time adjustment
    \item \textbf{Compounding factors}: Stress, infection, hormonal cycles, weather, and other factors unpredictably lower the threshold
    \item \textbf{Cumulative depletion}: Multiple small activities compound, each individually acceptable but collectively triggering crashes
\end{itemize}

This creates a chronic state of being forced to operate beyond one's physiological capacity. Patients are not failing to pace properly---they are trapped in circumstances that structurally require overexertion for survival. Research demonstrates that exceeding energy limits worsens functional outcomes, yet life circumstances often make such overexertion unavoidable~\cite{jason2009energy,brown2011activity}.

\paragraph{The Grinding Exhaustion of Baseline Inadequacy}

The continuous nature of baseline energy insufficiency distinguishes it from acute exhaustion:

\begin{itemize}
    \item \textbf{No recovery window}: There is no point at which energy feels restored; at best, crashes are avoided
    \item \textbf{Perpetual calculation}: Every action requires assessment of energy cost versus necessity
    \item \textbf{Invisible to others}: The constant internal struggle to perform basic tasks is entirely invisible; patients appear to be ``doing nothing'' while experiencing maximum effort to remain upright and conscious
    \item \textbf{Accumulating deficits}: Years of operating below subsistence level compound, potentially worsening disease trajectory
    \item \textbf{Eroded quality of life}: Even when avoiding severe crashes, life becomes reduced to the bare minimum, with no energy for joy, connection, or meaning
\end{itemize}

\paragraph{Psychological Impact of Chronic Insufficiency}

The experience of perpetual energy deficit below survival requirements produces distinct psychological consequences:

\begin{itemize}
    \item \textbf{Perpetual crisis state}: Living constantly at the edge of capacity creates unrelenting stress
    \item \textbf{Inability to plan}: When basic function is uncertain day-to-day, future planning becomes impossible
    \item \textbf{Loss of identity}: Activities that defined one's self become permanently inaccessible
    \item \textbf{Anticipatory dread}: Every upcoming obligation triggers fear about whether one will have sufficient energy
    \item \textbf{Grief without resolution}: Unlike grief over a discrete loss, the loss of capability is ongoing and total
    \item \textbf{Existential exhaustion}: Beyond physical fatigue, the sheer effort of continuing to exist in this state becomes overwhelming
\end{itemize}

This baseline insufficiency, combined with forced overexertion and the acute crashes of PEM, creates a situation of profound and continuous suffering that is difficult for healthy individuals to conceptualize. It is not merely ``being tired''---it is operating every moment at a fundamental energy deficit incompatible with sustainable human function.

\subsection{Physiological Basis}

\subsubsection{Mitochondrial Dysfunction and Energy Depletion}

\begin{observation}[WASF3-Mediated Mitochondrial Dysfunction]
\label{obs:wasf3-mito}
Skeletal muscle biopsies from ME/CFS patients (n=14) demonstrated significantly elevated WASF3 protein levels compared to healthy controls (n=10), with WASF3 overexpression correlating inversely with Complex IV function (r=-0.55, p=0.005)~\cite{wang2023wasf3}. Mechanistic studies revealed that endoplasmic reticulum (ER) stress induces WASF3 protein accumulation at ER-mitochondrial contact sites, where it disrupts respiratory supercomplex assembly and inhibits mitochondrial respiration. Transgenic mice with elevated WASF3 expression recapitulated the human phenotype, exhibiting impaired exercise capacity and reduced oxygen consumption. shRNA-mediated WASF3 knockdown in patient-derived cells restored respiratory capacity, demonstrating reversibility of the dysfunction.
\end{observation}

\begin{hypothesis}[WASF3 as Subset-Specific Mechanism]
\label{hyp:wasf3-subset}
The WASF3-mediated mitochondrial dysfunction mechanism may explain exercise intolerance in a subset of ME/CFS patients, particularly those with post-viral onset~\cite{wang2023wasf3,Syed2025}. The pathway linking viral infection → ER stress → WASF3 elevation → mitochondrial dysfunction → ATP depletion provides a coherent mechanistic framework. However, the prevalence of this mechanism across the broader ME/CFS population remains undetermined, as the initial finding derives from a small cohort (n=14). Independent replication and larger validation studies are needed to establish what proportion of ME/CFS patients exhibit this pathway.
\end{hypothesis}

The WASF3 mechanism aligns with broader evidence of mitochondrial dysfunction in ME/CFS~\cite{Syed2025}. ATP depletion following exertion explains the delayed onset of PEM (cellular energy stores require 24--72 hours to regenerate) and the disproportionate symptom severity (cells cannot meet metabolic demands even for basic function). WASF3 overexpression promotes actin polymerization, driving a metabolic shift toward glycolysis while further suppressing mitochondrial oxidative phosphorylation. This creates a self-reinforcing cycle: reduced ATP generation → increased cellular stress → sustained WASF3 elevation → continued mitochondrial impairment.

\subsection{Measurement and Assessment}

\subsubsection{Objective Measurement via Two-Day Cardiopulmonary Exercise Testing}

\begin{observation}[Two-Day CPET: Objective PEM Measurement]
\label{obs:2day-cpet}
Two-day cardiopulmonary exercise testing (CPET) provides objective evidence for post-exertional malaise through repeated maximal exercise tests separated by 24 hours~\cite{lim2020cpet}. Meta-analysis of five studies (n=98 ME/CFS patients, n=51 controls) demonstrated that ME/CFS patients fail to reproduce Day 1 performance on Day 2, whereas healthy sedentary controls maintain or improve performance. The most sensitive metric, workload at ventilatory threshold (VT), showed significant deterioration in ME/CFS patients (mean change from baseline: -33.0W on Day 2 vs. -10.8W on Day 1, p<0.05) while controls demonstrated improvement. This pattern has been independently replicated in subsequent larger cohorts exceeding 150 patients~\cite{keller2024cpet}, establishing 2-day CPET as the gold standard for objective PEM documentation.
\end{observation}

The physiological mechanisms underlying the Day 2 deterioration include:
\begin{itemize}
    \item \textbf{ATP depletion}: Mitochondrial dysfunction prevents normal energy regeneration within 24 hours~\cite{Syed2025,wang2023wasf3}
    \item \textbf{Immune activation}: Exercise triggers pro-inflammatory cytokine release that persists beyond the immediate post-exercise period
    \item \textbf{Oxidative stress}: Reactive oxygen species accumulate faster than antioxidant systems can neutralize them
    \item \textbf{Anaerobic threshold shift}: Early shift to anaerobic metabolism indicates impaired mitochondrial oxidative capacity
    \item \textbf{Prolonged recovery}: Unlike healthy controls who recover within 48 hours, ME/CFS patients may require 13+ days to return to baseline~\cite{keller2024cpet}
\end{itemize}

\begin{hypothesis}[2-Day CPET as Diagnostic Tool]
\label{hyp:cpet-diagnostic}
Two-day CPET may serve as an objective diagnostic biomarker for ME/CFS, particularly for distinguishing genuine post-exertional malaise from deconditioning or other fatiguing conditions~\cite{lim2020cpet}. The consistent Day 2 deterioration pattern appears specific to ME/CFS, with sedentary controls, fibromyalgia patients, and depression patients not exhibiting this phenotype. However, larger validation studies comparing ME/CFS to comprehensive disease control groups are needed to establish clinical sensitivity, specificity, and diagnostic thresholds before 2-day CPET can be implemented as a standalone diagnostic test.
\end{hypothesis}

\subsubsection{Clinical Assessment Tools}

While 2-day CPET provides objective measurement, it remains research-grade and inaccessible to most clinicians. Patient-reported outcome measures remain essential for clinical practice:

\begin{itemize}
    \item \textbf{DePaul Symptom Questionnaire (DSQ)}: Validated tool specifically measuring PEM frequency and severity
    \item \textbf{Pacing diaries}: Patient tracking of activity-symptom relationships
    \item \textbf{Functional capacity scales}: Bell Disability Scale, SF-36, and ME/CFS-specific measures
    \item \textbf{Activity monitors}: Actigraphy to objectively measure movement patterns (though cannot distinguish voluntary pacing from incapacity)
\end{itemize}

\section{Unrefreshing Sleep}
\label{sec:sleep}

Unrefreshing sleep is a cardinal symptom of ME/CFS, reported by 95--100\% of patients in most cohorts~\cite{Jason2010sleepMECFS,Unger2016sleepPrevalence}. Despite sleeping adequate or even excessive hours, patients wake feeling as exhausted as when they went to bed. This distinguishes ME/CFS sleep dysfunction from simple insomnia, where patients feel better after sleep even if it takes time to fall asleep.

\subsection{Sleep Dysfunction Patterns}

ME/CFS patients experience multiple overlapping sleep disturbances:

\subsubsection{Unrefreshing Sleep Despite Adequate Duration}

The core feature is lack of restoration from sleep:
\begin{itemize}
    \item Patients may sleep 8--12+ hours yet wake completely unrefreshed
    \item Morning exhaustion equal to or worse than evening exhaustion
    \item No correlation between sleep duration and daytime function
    \item Paradox: Some patients feel better with \textit{less} sleep (4--6 hours) than with full nights
\end{itemize}

\subsubsection{Sleep Maintenance Problems}

Beyond non-restorative sleep, many patients experience:
\begin{itemize}
    \item \textbf{Frequent nocturnal awakenings}: Waking 5--20+ times per night
    \item \textbf{Light, fragmented sleep}: Unable to maintain continuous deep sleep
    \item \textbf{Delayed sleep phase}: Inability to fall asleep until 2--4 AM despite exhaustion
    \item \textbf{Reversed circadian rhythm}: Sleeping during day, awake at night (in severe cases)
    \item \textbf{``Tired but wired''}: Physical exhaustion but mental hyperarousal preventing sleep
\end{itemize}

\subsubsection{Sleep Inertia and Hypersomnia}

Some patients experience:
\begin{itemize}
    \item \textbf{Severe sleep inertia}: Taking 2--4 hours to become functional after waking
    \item \textbf{Hypersomnia}: Sleeping 12--16 hours per day, particularly during crashes
    \item \textbf{Inability to wake}: Sleeping through alarms, phone calls, physical touch
    \item \textbf{Nap non-restoration}: Naps fail to provide refreshment (unlike healthy fatigue)
\end{itemize}

\subsection{Polysomnography Findings}

Objective sleep studies in ME/CFS reveal measurable abnormalities:

\subsubsection{Sleep Architecture Disruption}

Studies have documented~\cite{Jackson2012sleepAbnormalities,Reeves2006sleepCharacteristics}:
\begin{itemize}
    \item \textbf{Reduced slow-wave sleep (Stage N3)}: The deepest, most restorative sleep stage is diminished
    \item \textbf{Alpha-delta sleep}: Intrusion of waking alpha waves (8--13 Hz) into delta sleep, preventing deep sleep~\cite{Moldofsky1975alphaDeltaSleep}
    \item \textbf{Increased sleep fragmentation}: More frequent stage transitions and microarousals
    \item \textbf{Reduced sleep efficiency}: Lower percentage of time in bed actually spent asleep
    \item \textbf{REM abnormalities}: Some studies show reduced or disrupted REM sleep
\end{itemize}

The alpha-delta pattern is particularly notable~\cite{Moldofsky1975alphaDeltaSleep}---the brain shows mixed activity suggesting it never fully enters restorative deep sleep, explaining the subjective experience of ``sleeping but not resting.''

\subsubsection{Autonomic Dysfunction During Sleep}

Polysomnography with additional monitoring reveals:
\begin{itemize}
    \item \textbf{Abnormal heart rate variability}: Reduced parasympathetic tone during sleep
    \item \textbf{Elevated heart rate}: Persistent tachycardia even during sleep
    \item \textbf{Blood pressure instability}: Failure of normal nocturnal dipping
    \item \textbf{Temperature dysregulation}: Abnormal core body temperature curves
\end{itemize}

\subsubsection{Limitations of Standard Polysomnography}

Standard sleep studies may appear ``normal'' in ME/CFS because:
\begin{itemize}
    \item Sleep stages are scored by visual inspection of 30-second epochs
    \item Microarousals shorter than 3 seconds are not scored
    \item Alpha-delta intrusion requires specialized analysis
    \item Restorative quality cannot be directly measured
\end{itemize}

Patients often report polysomnography results labeled ``normal sleep'' despite severe subjective non-refreshment, leading to gaslighting. More detailed spectral analysis or multi-night home monitoring may reveal abnormalities missed by single-night laboratory studies.

\subsection{Related Sleep Disorders}

ME/CFS overlaps with and must be distinguished from primary sleep disorders:

\subsubsection{Obstructive Sleep Apnea (OSA)}

Sleep apnea can mimic ME/CFS symptoms:
\begin{itemize}
    \item \textbf{Overlap}: Fatigue, unrefreshing sleep, cognitive dysfunction, morning headaches
    \item \textbf{Prevalence}: Affects 10--30\% of general population~\cite{Peppard2013OSAprevalence,Senaratna2017OSAmeta,Young2002OSAepidemiology}; higher in ME/CFS due to weight gain from inactivity
    \item \textbf{Diagnostic clue}: Witnessed apneas, loud snoring, gasping during sleep
    \item \textbf{Resolution}: CPAP treatment resolves symptoms in true OSA; improves but doesn't cure comorbid OSA in ME/CFS
\end{itemize}

\textbf{Clinical importance}: Some patients misdiagnosed with ME/CFS for years experience dramatic improvement with CPAP, indicating primary OSA was the cause. Polysomnography should be standard workup before diagnosing ME/CFS.

\subsubsection{Upper Airway Resistance Syndrome (UARS)}

A subtler form of sleep-disordered breathing:
\begin{itemize}
    \item Increased upper airway resistance without frank apneas
    \item Causes repeated arousals (respiratory effort-related arousals, RERAs)
    \item May be missed on standard apnea-hypopnea index (AHI)
    \item Requires esophageal pressure monitoring for diagnosis
    \item Responds to CPAP or oral appliances
\end{itemize}

\subsubsection{Restless Legs Syndrome (RLS) and Periodic Limb Movement Disorder (PLMD)}

Common in ME/CFS:
\begin{itemize}
    \item \textbf{RLS}: Uncomfortable sensations in legs requiring movement to relieve, worse at night
    \item \textbf{PLMD}: Involuntary leg jerks during sleep causing microarousals
    \item \textbf{Prevalence}: Higher in ME/CFS than general population
    \item \textbf{Treatment}: Iron supplementation (if ferritin <75 ng/mL~\cite{Allen2018RLSironThreshold}), dopamine agonists, gabapentin
\end{itemize}

\subsubsection{Idiopathic Hypersomnia}

Overlapping features:
\begin{itemize}
    \item Excessive daytime sleepiness despite adequate nighttime sleep
    \item Sleep inertia lasting hours
    \item Non-restorative sleep
    \item Requires Multiple Sleep Latency Test (MSLT) to differentiate from ME/CFS
\end{itemize}

\subsubsection{Circadian Rhythm Disorders}

ME/CFS frequently involves circadian disruption:
\begin{itemize}
    \item \textbf{Delayed Sleep-Wake Phase Disorder}: Cannot fall asleep until 2--6 AM
    \item \textbf{Non-24-Hour Sleep-Wake Disorder}: Sleep time progressively delays each day
    \item \textbf{Irregular Sleep-Wake Rhythm}: Fragmented sleep-wake patterns across 24 hours
    \item May respond to light therapy, melatonin timing, or chronotherapy
\end{itemize}

\subsection{Differential Diagnosis Approach}

When evaluating unrefreshing sleep in suspected ME/CFS:

\begin{enumerate}
    \item \textbf{Rule out primary sleep disorders first}: Polysomnography, MSLT if indicated
    \item \textbf{Assess for comorbid conditions}: OSA + ME/CFS can coexist; treat both
    \item \textbf{Check serum ferritin}: Levels <75 ng/mL may cause RLS/PLMD
    \item \textbf{Evaluate autonomic function}: Tilt table, heart rate variability
    \item \textbf{Trial therapeutic interventions}: Response to CPAP, iron, or circadian treatments provides diagnostic information
\end{enumerate}

The key distinction: Primary sleep disorders improve significantly with appropriate treatment (CPAP, iron, etc.), while ME/CFS sleep dysfunction persists despite these interventions, though comorbid treatment helps partially.

\section{Cognitive Impairment}
\label{sec:cognitive}

Cognitive dysfunction, often described as ``brain fog,'' is a prominent and disabling feature of ME/CFS, affecting 85--95\% of patients~\cite{Cvejic2022cognitive}. Unlike fatigue-related cognitive slowing in healthy individuals, ME/CFS cognitive impairment persists despite rest and worsens substantially following exertion.

\subsection{Domains of Cognitive Dysfunction}
\label{subsec:cognitive-domains}

\paragraph{Processing Speed.}
Processing speed deficits represent the most robust and consistently replicated cognitive finding in ME/CFS. A meta-analysis of 40 studies found large effect sizes for reading speed (Hedges' g = -0.82, p < 0.0001) and moderate-to-large effects for other timed tasks~\cite{Cvejic2022cognitive}. Patients perform 0.5--1.0 standard deviations below healthy controls on processing speed measures, indicating clinically significant impairment. Recent studies using the Stroop task demonstrate that ME/CFS patients show ``significantly longer response times than controls indicating cognitive dysfunction'' with ``global slowing of response times that cannot be overcome by practice''~\cite{Thapaliya2024stroop}.

\paragraph{Attention and Concentration.}
Patients demonstrate reduced attentional capacity on effortful tasks, with impaired sustained attention during demanding cognitive work~\cite{Cvejic2022cognitive,MCAM2024cognitive}. Critically, these deficits persist after controlling for depression and are not explained by psychiatric comorbidity. The constant internal effort required to maintain focus depletes already-limited energy reserves, contributing to cognitive post-exertional malaise.

\paragraph{Memory.}
Memory impairments follow a specific pattern:
\begin{itemize}
    \item \textbf{Visuospatial immediate memory}: Moderate impairment (g = -0.55, p = 0.007), with visual modality more affected than verbal~\cite{Cvejic2022cognitive}
    \item \textbf{Working memory}: Impaired primarily on demanding tasks requiring interference resistance
    \item \textbf{Episodic memory}: Difficulties in storage, retrieval, and recognition processes, though less consistently affected than processing speed
    \item \textbf{Short-term memory}: Variable findings across studies
\end{itemize}

\paragraph{Executive Function.}
Executive functions appear relatively preserved compared to processing speed and memory. Meta-analysis found that ``executive functions seemed little or not affected and instrumental functions appeared constantly preserved''~\cite{Cvejic2022cognitive}. However, some patients demonstrate difficulties with mental flexibility, cognitive inhibition, and information generation, particularly under demanding conditions.

\paragraph{Language and Word-Finding.}
Verbal fluency deficits manifest as word retrieval problems, slowed speech, and linguistic reversals (mixing up word order)~\cite{MCAM2024cognitive}. Patients often describe ``tip of the tongue'' experiences and difficulty with verbal tests of unrelated word association learning and letter fluency. Communication difficulties extend to auditory sequencing problems that impair comprehension of spoken language.

\subsection{Neuropsychological Testing}
\label{subsec:neuropsych-testing}

\paragraph{Objective Test Results.}
The Multi-Site Clinical Assessment of ME/CFS (MCAM) study (n=261 ME/CFS patients vs.\ 165 healthy controls) confirmed deficits in processing speed, attention, working memory, and learning efficiency using standardized neuropsychological batteries~\cite{MCAM2024cognitive}. Between 21--38\% of patients perform below the 1.5 standard deviation cutoff for clinically significant impairment on Stroop tests.

\paragraph{Pattern of Deficits.}
The hierarchy of cognitive impairment from most to least affected is:
\begin{enumerate}
    \item Processing speed (most robust, largest effect sizes)
    \item Attention span and working memory (consistently impaired)
    \item Immediate memory, especially visual (moderate deficits)
    \item Episodic memory (variable across studies)
    \item Executive function (relatively preserved)
\end{enumerate}

This pattern differs from depression (which shows more diffuse cognitive effects) and multiple sclerosis (which shows more widespread deficits including greater executive impairment)~\cite{DeLuca2004comparison,Teodoro2025MECFSvsMS}.

\paragraph{Distinction from Depression.}
Comparative studies demonstrate that ME/CFS cognitive deficits are not attributable to depression. In three-way comparisons of ME/CFS, major depression, and healthy controls, cognitive patterns differ significantly: ME/CFS patients show primary deficits in processing speed and logical memory that persist after controlling for depressive symptoms~\cite{DeLuca2004comparison}. Additionally, cognitive performance in ME/CFS does not correlate with fatigue, pain, or depression levels, indicating independent pathophysiology~\cite{Teodoro2025MECFSvsMS}.

\paragraph{Subjective-Objective Dissociation.}
A notable finding is poor correlation between subjective cognitive complaints and objective test performance. Self-reported cognitive dysfunction correlates more strongly with fatigue (p < 0.001), pain (p < 0.001), and depression (p < 0.001) than with actual measured deficits~\cite{MCAM2024cognitive}. This suggests subjective complaints reflect overall symptom burden rather than specific cognitive impairments. However, strong concordance exists between subjective mental fatigue complaints and objective cognitive decline following exertion, highlighting the importance of assessing cognition in relation to activity.

\subsection{Neuroimaging Findings}
\label{subsec:neuroimaging}

\paragraph{Functional MRI: Increased Activation.}
The most consistent fMRI finding is that ME/CFS patients exhibit ``increased activations and recruited additional brain regions during cognitive tasks''~\cite{Shan2020neuroimaging}. This compensatory activation suggests the brain works harder to achieve equivalent performance. Tasks with increasing complexity produce decreased activation in task-specific regions, indicating failure of normal efficiency mechanisms under cognitive load.

\paragraph{Functional Connectivity Abnormalities.}
High-field (7T) fMRI studies reveal altered connectivity patterns. Abnormal salience network connectivity, particularly involving the right insula, appears across multiple studies---8 of 10 different ME/CFS-specific connections involve a salience network hub~\cite{Shan2020neuroimaging}. Specific findings include:
\begin{itemize}
    \item Stronger connections between salience network and hippocampus
    \item Stronger connections between salience network and brainstem reticular activation system
    \item Reduced dopaminergic hippocampal-nucleus-accumbens connectivity, implying blunted motivation and cognition~\cite{Faro2024connectivity}
    \item Extensive aberrant ponto-cerebellar connections consistent with ME/CFS symptomatology
\end{itemize}

\paragraph{The 2024 NIH Study: Temporoparietal Junction.}
The NIH deep phenotyping study identified decreased activity in the temporoparietal junction (TPJ) during effort-based tasks~\cite{walitt2024deep}. The TPJ is responsible for effort-based decision-making, and its dysfunction ``may cause fatigue by disrupting the way the brain decides how to exert effort.'' While controls showed increased blood oxygen levels in task-relevant regions, ME/CFS patients showed decreased levels in the TPJ, superior parietal lobule, and right temporal gyrus. This finding provides a neural substrate for the effort-performance disconnect described by patients.

\paragraph{Neuroinflammation Studies.}
PET studies using TSPO ligands (markers of microglial activation) have produced conflicting results. Nakatomi et al.\ (2014) found increased binding in cingulate cortex, hippocampus, amygdala, thalamus, midbrain, and pons, suggesting widespread neuroinflammation associated with symptom severity~\cite{Nakatomi2014neuroinflammation}. However, Raijmakers et al.\ (2021) failed to replicate these findings in a similar-sized cohort~\cite{Raijmakers2021neuroinflammation}. Methodological factors and small sample sizes (n=9--14) limit conclusions. The role of neuroinflammation in ME/CFS cognitive dysfunction remains an active area of investigation.

\paragraph{Structural Changes.}
Structural MRI studies have identified:
\begin{itemize}
    \item Reduced gray matter in occipital lobes, right angular gyrus, and left parahippocampal gyrus
    \item Frontal lobe volume reductions correlating with fatigue scores~\cite{Shan2020neuroimaging}
    \item Reduced white matter volume in left occipital lobe and left inferior fronto-occipital fasciculus
    \item Elevated T1w/T2w ratios suggesting increased myelin and/or iron in subcortical structures
\end{itemize}

White matter abnormalities of unknown etiology have been observed in some patients, though not consistently. Importantly, structural changes may not be prominent in early or pediatric cases, suggesting they develop with illness duration.

\paragraph{Brainstem Involvement.}
Multiple neuroimaging modalities (fMRI, PET, MRS) converge on brainstem abnormalities as a consistent finding in ME/CFS~\cite{Shan2020neuroimaging}. FDG-PET demonstrates glucose hypometabolism in the brainstem, supporting a physiological basis for fatigue, unrefreshing sleep, and cognitive symptoms. Impaired connectivity involving the brainstem has been identified in multiple studies and may reflect dysautonomia contributing to cognitive dysfunction through cerebral hypoperfusion.

\section{Autonomic Dysfunction}
\label{sec:autonomic}

Autonomic dysfunction is present in 70--90\% of ME/CFS patients~\cite{Newton2007autonomicDysfunction}, manifesting as orthostatic intolerance, temperature dysregulation, and cardiovascular symptoms. The autonomic nervous system controls involuntary functions including heart rate, blood pressure, digestion, temperature regulation, and bladder control. Dysautonomia in ME/CFS creates a cascade of disabling symptoms often misattributed to anxiety or deconditioning.

\subsection{Orthostatic Intolerance}

Orthostatic intolerance (OI) refers to symptoms triggered or worsened by upright posture. It is one of the most common and disabling features of ME/CFS autonomic dysfunction.

\subsubsection{Clinical Presentation}

Symptoms upon standing or prolonged sitting include:
\begin{itemize}
    \item \textbf{Lightheadedness or dizziness}: Feeling faint, vision graying out
    \item \textbf{Palpitations}: Awareness of rapid or pounding heartbeat
    \item \textbf{Tremulousness}: Shaking, feeling weak or unstable
    \item \textbf{Cognitive impairment}: ``Coat hanger pain'' (neck/shoulder aching from reduced cerebral perfusion)
    \item \textbf{Nausea}: Gastrointestinal symptoms triggered by position change
    \item \textbf{Shortness of breath}: Air hunger despite normal oxygen saturation
    \item \textbf{Fatigue exacerbation}: Profound worsening of exhaustion when upright
\end{itemize}

Patients often develop adaptive behaviors: sitting while showering, lying down frequently, avoiding standing in lines, preferring reclined positions.

\subsubsection{Postural Orthostatic Tachycardia Syndrome (POTS)}

POTS is the most common form of orthostatic intolerance in ME/CFS, affecting 25--50\% of patients~\cite{Newton2008POTSprevalence}.

\paragraph{Diagnostic Criteria.}
\begin{itemize}
    \item \textbf{Heart rate increase}: $\geq$30 bpm within 10 minutes of standing (or $\geq$40 bpm in adolescents)~\cite{Sheldon2015POTScriteria}
    \item \textbf{Absence of orthostatic hypotension}: Blood pressure remains stable or increases
    \item \textbf{Symptom provocation}: OI symptoms occur with the tachycardia
    \item \textbf{Duration}: Symptoms present for $\geq$3 months
    \item \textbf{Exclusions}: No other cause (dehydration, medications, prolonged bed rest alone)
\end{itemize}

\paragraph{Physiological Mechanisms.}
POTS in ME/CFS may involve:
\begin{itemize}
    \item \textbf{Hypovolemia}: Reduced blood volume (measured via Evans blue dye dilution studies)
    \item \textbf{Venous pooling}: Impaired vasoconstriction allows blood to pool in lower extremities
    \item \textbf{Hyperadrenergic state}: Excessive norepinephrine release upon standing
    \item \textbf{Baroreceptor dysfunction}: Impaired blood pressure sensing
    \item \textbf{Autoimmunity}: Antibodies against adrenergic and muscarinic receptors affecting vascular tone
\end{itemize}

\paragraph{Measurement.}
\begin{itemize}
    \item \textbf{NASA Lean Test}: 10-minute standing test measuring heart rate and blood pressure every 2 minutes
    \item \textbf{Tilt table testing}: Gold standard, involves passive upright tilt to 70° for up to 45 minutes
    \item \textbf{Home monitoring}: Patients can document HR/BP changes with home devices
\end{itemize}

\subsubsection{Orthostatic Hypotension (OH)}

Less common than POTS but present in some ME/CFS patients:
\begin{itemize}
    \item \textbf{Definition}: Sustained drop in systolic BP $\geq$20 mmHg or diastolic BP $\geq$10 mmHg within 3 minutes of standing
    \item \textbf{Symptoms}: Severe lightheadedness, syncope, visual blurring, cognitive impairment
    \item \textbf{Mechanism}: Inadequate vasoconstriction response to postural change
    \item \textbf{Treatment}: Different from POTS; requires blood pressure support (fludrocortisone, midodrine)
\end{itemize}

\subsubsection{Neurally Mediated Hypotension (NMH)}

Also called vasovagal syncope or neurocardiogenic syncope:
\begin{itemize}
    \item \textbf{Presentation}: Delayed blood pressure drop and bradycardia after prolonged standing (typically 15--45 minutes)
    \item \textbf{Mechanism}: Paradoxical vagal activation causing vasodilation and heart rate slowing
    \item \textbf{Tilt table pattern}: Initial normal response, then sudden BP/HR drop with near-syncope
    \item \textbf{Overlap}: Can coexist with POTS in same patient
\end{itemize}

\subsubsection{Tilt Table Testing Protocol}

The gold standard for diagnosing orthostatic intolerance:

\begin{enumerate}
    \item \textbf{Preparation}: Patient lies supine on motorized table with footboard support
    \item \textbf{Baseline}: 10--20 minutes supine to establish baseline HR and BP
    \item \textbf{Tilt}: Table tilted to 70° head-up position
    \item \textbf{Monitoring}: Continuous HR, BP, and symptoms recorded for up to 45 minutes
    \item \textbf{Endpoints}: Test terminated if syncope occurs, BP drops dangerously, or maximum duration reached
\end{enumerate}

\paragraph{Interpretation.}
\begin{itemize}
    \item \textbf{POTS pattern}: Sustained HR increase $\geq$30 bpm without BP drop
    \item \textbf{Orthostatic hypotension}: BP drop within 3 minutes
    \item \textbf{NMH pattern}: Delayed sudden BP/HR drop after 15--45 minutes
    \item \textbf{Normal response}: HR increase <30 bpm, stable BP
\end{itemize}

\textbf{Clinical note}: Some ME/CFS patients have severe OI symptoms with ``normal'' tilt table results. This may reflect:
\begin{itemize}
    \item Cerebral hypoperfusion despite maintained BP (impaired cerebral autoregulation)
    \item Small fiber neuropathy not detected by standard autonomic testing
    \item Endothelial dysfunction affecting microvascular perfusion
\end{itemize}

\subsection{Other Autonomic Symptoms}

Beyond orthostatic intolerance, ME/CFS patients experience widespread autonomic dysfunction:

\subsubsection{Temperature Dysregulation}

Impaired thermoregulation manifests as:
\begin{itemize}
    \item \textbf{Subnormal body temperature}: Chronic low-grade hypothermia (96--97°F / 35.5--36°C)
    \item \textbf{Temperature instability}: Fluctuations throughout day without infection
    \item \textbf{Heat intolerance}: Severe symptom exacerbation in warm environments
    \item \textbf{Cold intolerance}: Inability to warm up, cold extremities even in warm rooms
    \item \textbf{Inappropriate sweating}: Night sweats, profuse sweating with minimal exertion
    \item \textbf{Lack of sweating}: Some patients lose ability to sweat (anhidrosis)
\end{itemize}

\subsubsection{Sweating Abnormalities}

Thermoregulatory and sympathetic sweating dysfunction:
\begin{itemize}
    \item \textbf{Hyperhidrosis}: Excessive sweating of hands, feet, or generalized
    \item \textbf{Hypohidrosis/anhidrosis}: Reduced or absent sweating capacity
    \item \textbf{Gustatory sweating}: Sweating triggered by eating (cranial autonomic dysfunction)
    \item \textbf{Night sweats}: Drenching sweats during sleep requiring clothing/bedding changes
\end{itemize}

\subsubsection{Gastrointestinal Symptoms}

Autonomic control of GI function is commonly impaired:
\begin{itemize}
    \item \textbf{Gastroparesis}: Delayed gastric emptying causing early satiety, nausea, bloating
    \item \textbf{Irritable Bowel Syndrome (IBS)}: Diarrhea-predominant, constipation-predominant, or alternating
    \item \textbf{Dysmotility}: Impaired intestinal peristalsis
    \item \textbf{Nausea}: Chronic or episodic, often worse upon standing (orthostatic nausea)
    \item \textbf{Abdominal pain}: Cramping, visceral hypersensitivity
\end{itemize}

\subsubsection{Urinary Dysfunction}

Bladder autonomic control abnormalities include:
\begin{itemize}
    \item \textbf{Urgency and frequency}: Needing to urinate frequently with sudden urgency
    \item \textbf{Nocturia}: Waking multiple times at night to urinate
    \item \textbf{Incomplete emptying}: Sensation of residual urine
    \item \textbf{Interstitial cystitis overlap}: Bladder pain, pressure, frequency
\end{itemize}

\subsubsection{Cardiac Symptoms}

Beyond POTS-related tachycardia:
\begin{itemize}
    \item \textbf{Inappropriate sinus tachycardia}: Resting heart rate >100 bpm without postural trigger
    \item \textbf{Palpitations}: Awareness of heartbeat, skipped beats, forceful beats
    \item \textbf{Chest pain}: Non-cardiac chest pain (microvascular angina, costochondritis)
    \item \textbf{Heart rate variability reduction}: Reduced parasympathetic tone
    \item \textbf{Exercise intolerance}: Exaggerated HR response to minimal exertion
\end{itemize}

\subsubsection{Pupillary Abnormalities}

Autonomic control of pupils may be affected:
\begin{itemize}
    \item \textbf{Light sensitivity (photophobia)}: Inability to tolerate bright lights
    \item \textbf{Impaired pupil constriction}: Sluggish response to light
    \item \textbf{Anisocoria}: Unequal pupil sizes
\end{itemize}

\subsection{Autonomic Testing Battery}

Comprehensive autonomic function assessment may include:

\begin{itemize}
    \item \textbf{Tilt table test}: Orthostatic intolerance assessment
    \item \textbf{Valsalva maneuver}: Tests baroreceptor and cardiovagal function
    \item \textbf{Deep breathing test}: Measures heart rate variability during paced breathing
    \item \textbf{Quantitative sudomotor axon reflex test (QSART)}: Assesses sweating capacity
    \item \textbf{Thermoregulatory sweat test}: Maps sweating across entire body
    \item \textbf{Pupillometry}: Automated pupil response measurement
    \item \textbf{Skin biopsy}: Small fiber neuropathy assessment (intraepidermal nerve fiber density)
\end{itemize}

Many ME/CFS specialty centers lack access to full autonomic testing, making tilt table and basic orthostatic vitals the most commonly used assessments.

\subsection{Clinical Implications}

Autonomic dysfunction in ME/CFS is:
\begin{itemize}
    \item \textbf{Objectively measurable}: Tilt table, HRV, and other tests provide objective abnormalities
    \item \textbf{Highly disabling}: OI can prevent standing long enough to shower or prepare meals
    \item \textbf{Treatable}: Salt, fluids, compression, and medications can significantly improve symptoms
    \item \textbf{Not anxiety}: Patients are often told POTS is anxiety; it is a physiological abnormality
    \item \textbf{Connected to energy metabolism}: Autonomic dysfunction may reflect mitochondrial impairment in autonomic neurons
\end{itemize}

Recognition and treatment of dysautonomia is often the first step in improving ME/CFS functional capacity.

\section{Pain}
\label{sec:pain}

Pain is a prominent symptom in ME/CFS, with approximately 80\% of patients reporting significant pain in the past week~\cite{Unger2017pain}. Pain is included as a diagnostic criterion in multiple case definitions and contributes substantially to disability and reduced quality of life.

\subsection{Types of Pain in ME/CFS}
\label{subsec:pain-types}

\paragraph{Myalgia (Muscle Pain).}
Muscle pain is the most common pain complaint, affecting 72--94\% of ME/CFS patients~\cite{Nijs2012muscle}. The pain is typically widespread rather than localized and characteristically worsens 8--72 hours following physical exertion as part of post-exertional malaise. Patients describe deep, aching pain that differs from delayed-onset muscle soreness in healthy individuals---it occurs following minimal exertion, lasts substantially longer, and is accompanied by other PEM symptoms. The pain reflects underlying skeletal muscle dysfunction including mitochondrial impairment, oxidative stress, reduced heat shock proteins, and impaired muscle contractility~\cite{Jammes2021muscle}.

\paragraph{Arthralgia (Joint Pain).}
Joint pain affects 58--84\% of patients and is included as a criterion in both Fukuda and Canadian Consensus definitions~\cite{Fukuda1994,Carruthers2003}. The pattern is characteristically migratory (moving between joints) and occurs without the swelling, redness, warmth, or deformity seen in inflammatory arthritis. This distinction is clinically important: presence of joint inflammation suggests an alternative diagnosis or comorbid condition requiring separate evaluation.

\paragraph{Headaches.}
Headaches are significantly more common in ME/CFS than the general population: 84\% experience migraine headaches (versus 5\% in healthy controls) and 81\% have tension-type headaches (versus 45\% in controls)~\cite{Ravindran2011headache}. The breakdown includes migraine without aura (60\%), migraine with aura (24\%), tension headaches only (12\%), and no headaches (4\%). ME/CFS patients with migraine demonstrate lower pressure pain thresholds (2.36 kg versus 5.23 kg in controls, p<0.001) and higher fibromyalgia comorbidity (47\% versus 0\%)~\cite{Ravindran2011headache}. Headaches are listed in Fukuda criteria as one of eight minor symptoms.

\paragraph{Neuropathic Pain.}
A subset of ME/CFS patients experience neuropathic pain characterized by burning, tingling, or electric shock sensations. This correlates with the finding that 30--38\% of ME/CFS patients have small fiber neuropathy (SFN) confirmed by skin biopsy demonstrating reduced intraepidermal nerve fiber density~\cite{Oaklander2022SFN}. Of those with confirmed SFN, 93\% have comorbid postural orthostatic tachycardia syndrome (POTS) or other orthostatic intolerance, suggesting shared pathophysiology involving autonomic small fibers~\cite{Devigili2023SFN}.

\subsection{Pain Mechanisms}
\label{subsec:pain-mechanisms}

\paragraph{Central Sensitization.}
Central sensitization---increased excitability of central nervous system pain pathways---is present in 84\% of ME/CFS patients, compared to 95\% of fibromyalgia patients and 0\% of healthy controls~\cite{Nijs2021sensitization}. This is defined by enhanced temporal summation (wind-up) combined with inefficient conditioned pain modulation. Clinical manifestations include:
\begin{itemize}
    \item Generalized hyperalgesia to electrical, mechanical, heat, and chemical stimuli
    \item Affects multiple tissues including skin, muscle, and viscera
    \item Hyperalgesia augmented rather than decreased following exercise or other stressors
    \item Lower pressure pain thresholds: ME/CFS median 222 kPa versus healthy controls 311 kPa (p<0.05)~\cite{Nijs2021sensitization}
\end{itemize}

Central sensitization is driven by neuroinflammation---glial cell activation (microglia and astrocytes) in the spinal cord and brain releasing pro-inflammatory cytokines and chemokines that sustain neural hypersensitivity~\cite{Nijs2017neuroinflammation}.

\paragraph{Small Fiber Neuropathy.}
Small fiber neuropathy provides an objective, biopsy-confirmed mechanism for pain in a substantial subset of patients. Studies find 30--38\% of ME/CFS patients meet diagnostic criteria for SFN~\cite{Oaklander2022SFN}. Small fibers (A-delta and C fibers) mediate pain, temperature sensation, and autonomic function, explaining the overlap between pain and dysautonomia. The etiology of SFN in ME/CFS is not fully established but may involve autoimmune mechanisms, as autoantibodies against small fiber antigens have been identified in some patients.

\paragraph{Peripheral Mechanisms.}
Peripheral contributors to ME/CFS pain include:
\begin{itemize}
    \item \textbf{Elevated blood lactate}: Nearly half of ME/CFS patients have elevated resting lactate levels, correlating with more severe post-exertional malaise~\cite{Lien2019lactate}. Lactate accumulation reflects anaerobic metabolism predominance due to mitochondrial dysfunction.
    \item \textbf{Metabolic dysfunction}: Impaired ATP synthesis leads to toxic metabolite accumulation that activates muscle nociceptors~\cite{Jammes2021muscle}.
    \item \textbf{Impaired proton handling}: Profound intramuscular acidosis develops following minimal exertion.
    \item \textbf{Reduced oxygen delivery}: Endothelial dysfunction and microvascular abnormalities may limit oxygen supply to exercising muscles.
\end{itemize}

\paragraph{Relationship to Post-Exertional Malaise.}
Pain is a core component of PEM. A meta-analysis found small to moderate pain increases following exercise in ME/CFS versus controls (Hedges' d = 0.42, 95\% CI: 0.16--0.67), with delayed pain showing larger effects at 8--72 hours (d = 0.71) than at 0--2 hours (d = 0.32)~\cite{Barhorst2022painPEM}. This delayed, disproportionate pain response parallels the temporal pattern of other PEM symptoms and likely reflects the same underlying metabolic and immune dysfunction. Factor analysis of PEM symptoms identifies a distinct ``musculoskeletal factor'' comprising muscle pain, weakness, and post-exertional fatigue~\cite{Barhorst2022painPEM}.

\subsection{Pain Assessment and Management Considerations}
\label{subsec:pain-management}

\paragraph{Quantitative Sensory Testing.}
Quantitative sensory testing (QST) can objectively document pain hypersensitivity. Commonly used measures include pressure pain thresholds at standard sites (trapezius, forearm, 18 fibromyalgia tender points), thermal thresholds, and temporal summation protocols. QST findings may support disability claims and guide treatment by identifying central versus peripheral contributions.

\paragraph{Overlap with Fibromyalgia.}
ME/CFS and fibromyalgia show substantial clinical overlap: 47.3\% (95\% CI: 45.97--48.63) of ME/CFS diagnoses overlap with fibromyalgia, with 35--75\% of ME/CFS patients meeting fibromyalgia criteria and 20--70\% of fibromyalgia patients meeting ME/CFS criteria~\cite{Pendergrast2016overlap}. Cerebrospinal fluid proteomics are indistinguishable between ME/CFS patients with and without comorbid fibromyalgia, suggesting shared pathophysiology~\cite{Nilsson2023proteomics}. Key clinical distinctions:
\begin{itemize}
    \item Fibromyalgia: Pain predominant, fatigue secondary
    \item ME/CFS: Fatigue and PEM predominant, pain prominent but not defining
    \item Comorbid patients have worse outcomes: greater physical disability, more severe pain, and more pronounced post-exertional symptoms than either condition alone
\end{itemize}

\paragraph{Treatment Implications.}
Pain management in ME/CFS must account for the underlying mechanisms:
\begin{itemize}
    \item Standard analgesics may be insufficient given central sensitization
    \item Interventions targeting neuroinflammation (e.g., low-dose naltrexone) may address central mechanisms
    \item Activity pacing prevents pain exacerbation from PEM
    \item Treatment of underlying small fiber neuropathy (if present) with IVIG has shown benefit in some patients
    \item Medications effective for fibromyalgia pain (duloxetine, pregabalin) may help the subset with overlapping presentations
\end{itemize}

\section{Sensory Sensitivities}
\label{sec:sensory}

Heightened sensitivity to sensory stimuli is a common but often underrecognized feature of ME/CFS, present in 70--90\% of patients~\cite{Jason2013sensory}. These sensitivities can be profoundly disabling and contribute significantly to activity limitation and social isolation.

\subsection{Types of Sensory Sensitivity}
\label{subsec:sensory-types}

\paragraph{Photophobia (Light Sensitivity).}
Light sensitivity affects approximately 70\% of ME/CFS patients~\cite{Jason2013sensory}. Manifestations include:
\begin{itemize}
    \item Inability to tolerate bright lights, including sunlight and fluorescent lighting
    \item Need for sunglasses indoors or dimmed environments
    \item Headaches or symptom exacerbation triggered by light exposure
    \item Difficulty with screens (computers, phones, televisions)
    \item Preference for dark or low-light environments
\end{itemize}

Light sensitivity may reflect autonomic dysfunction affecting pupillary control, central sensitization affecting visual processing, or neuroinflammation in visual pathways.

\paragraph{Phonophobia (Sound Sensitivity).}
Sound sensitivity affects 60--80\% of patients and can be severely disabling~\cite{Jason2013sensory}:
\begin{itemize}
    \item Normal conversation volumes feel uncomfortably loud
    \item Sudden or unexpected sounds cause startle responses and symptom flares
    \item Multiple simultaneous sounds (e.g., conversations in a restaurant) are intolerable
    \item Background noise prevents concentration
    \item Need for quiet environments or noise-canceling headphones
\end{itemize}

In severe cases, patients cannot tolerate any sound and require complete silence, significantly limiting social contact and access to medical care.

\paragraph{Chemical Sensitivity (Multiple Chemical Sensitivity).}
Sensitivity to chemicals and odors affects 40--60\% of ME/CFS patients~\cite{Jason2013sensory}:
\begin{itemize}
    \item Fragrances (perfumes, cleaning products, air fresheners) trigger symptoms
    \item Exhaust fumes and other environmental pollutants cause reactions
    \item New materials (carpets, furniture, paint) provoke symptoms
    \item Symptoms may include headache, cognitive dysfunction, nausea, respiratory symptoms
    \item Overlap with Multiple Chemical Sensitivity (MCS) syndrome
\end{itemize}

\paragraph{Touch and Pressure Sensitivity.}
Tactile hypersensitivity manifests as:
\begin{itemize}
    \item Allodynia---painful response to normally non-painful touch
    \item Clothing tags, seams, or tight clothing feel unbearable
    \item Difficulty tolerating physical examination
    \item Hyperalgesia---exaggerated pain response to mildly painful stimuli
\end{itemize}

This overlaps with the central sensitization mechanisms described in the Pain section.

\paragraph{Temperature Sensitivity.}
Intolerance to temperature extremes affects most patients:
\begin{itemize}
    \item Heat intolerance with symptom exacerbation in warm environments
    \item Cold intolerance with difficulty warming up
    \item Narrow range of comfortable temperatures
    \item Symptoms triggered by temperature changes
\end{itemize}

This reflects autonomic dysfunction affecting thermoregulation (see Section~\ref{sec:autonomic}).

\subsection{Mechanisms of Sensory Sensitivity}
\label{subsec:sensory-mechanisms}

\paragraph{Central Sensitization.}
The same central sensitization mechanisms that produce pain hypersensitivity likely underlie broader sensory sensitivities. Reduced inhibitory control in the central nervous system leads to amplification of all sensory inputs, not just nociceptive signals~\cite{Nijs2017neuroinflammation}.

\paragraph{Neuroinflammation.}
Glial activation and neuroinflammatory processes may directly affect sensory processing pathways, reducing thresholds for activation and impairing habituation to repeated stimuli.

\paragraph{Autonomic Dysfunction.}
Dysautonomia contributes to sensory sensitivity through impaired pupillary control (photophobia), altered blood flow to sensory organs, and dysfunctional sympathetic responses to stimuli.

\paragraph{Energy Depletion.}
Sensory processing requires energy. With baseline energy insufficiency, normal sensory processing may exceed available cellular resources, leading to symptoms from stimulation that healthy individuals filter automatically.

\subsection{Clinical Implications}
\label{subsec:sensory-implications}

\paragraph{Activity Limitation.}
Sensory sensitivities profoundly limit function:
\begin{itemize}
    \item Medical appointments become challenging (bright lights, waiting room noise, chemical smells)
    \item Shopping, restaurants, and public spaces are often intolerable
    \item Work environments may be impossible to tolerate
    \item Social gatherings exceed sensory capacity
\end{itemize}

\paragraph{Assessment Considerations.}
When evaluating ME/CFS patients, clinicians should:
\begin{itemize}
    \item Ask specifically about sensory sensitivities
    \item Modify examination environments (dim lights, reduce noise)
    \item Allow patients to wear sunglasses or earplugs
    \item Avoid fragranced products
    \item Recognize that sensory overload can trigger PEM
\end{itemize}

\paragraph{Management.}
Management focuses on environmental modification:
\begin{itemize}
    \item Sunglasses, tinted lenses, or FL-41 lenses for photophobia
    \item Noise-canceling headphones or earplugs for phonophobia
    \item Fragrance-free environments and products
    \item Loose, soft clothing without tags or seams
    \item Temperature-controlled environments with ability to layer clothing
    \item Gradual, controlled exposure when improvement occurs
\end{itemize}
