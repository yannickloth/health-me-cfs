% FILE: Resources and support organizations — references, support groups, information sources
\chapter{Resources and Support}
\label{app:resources}

This appendix provides a comprehensive guide to ME/CFS patient organizations, online communities, advocacy resources, and prominent patient voices. Unlike the scientific literature cited throughout this document, these resources represent the patient community's perspective and lived experience---an essential complement to clinical and research knowledge.

\begin{observation}[On Patient-Generated Knowledge]
The ME/CFS patient community has developed sophisticated knowledge networks that often outpace formal medical understanding. Forums like Phoenix Rising and Science for ME regularly discuss research papers with clinical depth that would be impressive in any medical setting. Patient advocates have successfully challenged flawed research (notably the PACE trial), influenced government policy, and funded significant research initiatives. This appendix acknowledges that patient expertise is a legitimate and valuable form of knowledge.
\end{observation}

%===============================================================================
\section{International Patient Organizations}
\label{sec:international-orgs}
%===============================================================================

\subsection{Major Global Organizations}

\begin{description}[style=nextline]
\item[Solve ME/CFS Initiative (Solve M.E.)] \hfill \url{https://solvecfs.org/}\\
US-based non-profit serving as a catalyst for research into ME/CFS, Long COVID, and other infection-associated chronic conditions. Operates the You+ME patient registry and biobank, funds research, and conducts policy advocacy. One of the largest and most influential ME/CFS organizations globally.

\item[Open Medicine Foundation (OMF)] \hfill \url{https://www.omf.ngo/}\\
Founded by Linda Tannenbaum, whose son has ME/CFS. Directs the Scientific Advisory Board chaired by Ron Davis, PhD. Operates six international ME/CFS Collaborative Research Centers and funds the End ME/CFS Project. Known for patient-centered research approach and involvement of patients and family members at leadership level.

\item[International Association for CFS/ME (IACFS/ME)] \hfill \url{https://www.iacfsme.org/}\\
International non-profit organization of clinicians, scientists, professionals, patients, and advocates. Publishes the peer-reviewed journal \textit{Fatigue: Biomedicine, Health, and Behavior}, organizes international conferences, and promotes science-based care.

\item[\#MEAction Network] \hfill \url{https://www.meaction.net/}\\
International patient-led advocacy network fighting for health equality. Co-founded by filmmaker Jennifer Brea. Organizes the annual \#MillionsMissing protests, provides advocacy training, and supports local patient groups worldwide. Known for effective use of social media and grassroots organizing.

\item[World ME Alliance] \hfill \url{https://worldmealliance.org/}\\
Global coalition of national ME organizations working to coordinate international advocacy efforts and share resources across countries.
\end{description}

\subsection{European Organizations}

\begin{description}[style=nextline]
\item[European ME Alliance (EMEA)] \hfill \url{https://www.europeanmealliance.org/}\\
Pan-European patient organization representing 18 countries. Founded in 2008 as collaboration of national patient charities. Member of European Patients' Forum (EPF) and European Federation of Neurological Associations (EFNA). Conducts the Pan-European ME Patient Survey (over 11,000 respondents in 2024). Created the European ME Research Group (EMERG) and European ME Clinicians Council.

\item[EUROMENE] \hfill \url{https://www.euromene.eu/}\\
European Network on ME/CFS---a COST (European Cooperation in Science and Technology) supported network of research groups across Europe. Published expert consensus on diagnosis, service provision, and care in Europe (2021).
\end{description}

%===============================================================================
\section{National Patient Organizations by Country}
\label{sec:national-orgs}
%===============================================================================

\subsection{Belgium}
\label{subsec:belgium-orgs}

\begin{remark}[Belgian Organizations]
Belgian organizations are primarily Flemish-based, with limited French-language resources for Wallonia. In Belgium, the condition is typically referred to as ``CVS'' (Chronisch Vermoeidheidssyndroom) rather than ME.
\end{remark}

\begin{description}[style=nextline]
\item[ME-Vereniging vzw] \hfill \url{https://www.me-vereniging.be/}\\
The ME Association (Belgium) raises awareness and strives for recognition of the disease. Organizes support groups in Antwerp, Hasselt, Ypres, and Nieuwrode. Operates an ME help-line. Co-founder of the European ME Alliance.

\item[12ME] \hfill Belgian non-profit drawing attention to ME/CFS seriousness with a positive approach.

\item[CVS contact groep vzw] \hfill \url{http://www.cvs-contactgroep.be/}\\
Aims at CFS and fibromyalgia patients, provides information on legitimate scientific research. Publishes quarterly magazine ``Immune'' and organizes meetings in Flanders.
\end{description}

\subsubsection{RIZIV/INAMI-Recognized Diagnostic Centers}

\paragraph{Historical Context}
Belgium originally established \textbf{five CVS reference centers} around 2002 at university hospitals (UZ Leuven, UZ Gent, UZ Antwerpen, UZ Brussel, and one in Wallonia). These centers operated until \textbf{2012}, when RIZIV abruptly cut funding. In 2014, a new system of ``Multidisciplinary Diagnostic Centers'' replaced them, but with reduced scope. As of 2024, only one center has signed the current convention.

\begin{description}[style=nextline]
\item[UPC KU Leuven -- Multidisciplinair Diagnostisch Centrum ME/CVS] \hfill\\
\textbf{Address:} Leuvensesteenweg 517, 3070 Kortenberg\\
\textbf{Phone:} +32 2 758 05 11 (general); +32 2 758 16 77 (CVS consultation)\\
\textbf{Email:} cvs@upckuleuven.be\\
\textbf{Website:} \url{https://www.upckuleuven.be/nl/zorgaanbod/cvs}\\
\textbf{Consultation hours:} Tuesday 9:00--12:00, Wednesday 9:30--12:00 (by appointment)\\
As of 2024, this is the only center in Belgium with an official RIZIV/INAMI convention for ME/CVS. The center provides multidisciplinary diagnostic assessment and, if ME/CVS is confirmed, develops a care trajectory in collaboration with the patient's GP. Referral must come from a GP who suspects ME/CVS. Convention runs until 2028.
\end{description}

\begin{remark}[CBT-Based Treatment: Critical Context]
The RIZIV/INAMI convention mandates cognitive behavioral therapy (CBT) and graded exercise therapy (GET) as the reimbursed treatments---an approach based primarily on the UK PACE trial, which has been heavily criticized for methodological flaws and subsequently disavowed by NICE guidelines (2021). Large patient surveys consistently report that GET worsens symptoms in the majority of ME/CFS patients, while CBT shows limited benefit for core symptoms. Belgian patient organizations have criticized this policy as outdated and potentially harmful. Patients should be aware that accepting the convention's treatment pathway means committing to CBT/GET-based rehabilitation, which may not align with current international best practices emphasizing pacing and symptom management.

The convention offers reimbursement for up to 17 CBT sessions (individual: 50 min at \texteuro 86.69; group: 90 min at \texteuro 57.80 per participant---2024 rates). Maximum 8 sessions may be in group format. Contact RIZIV: Evi Declercq, +32 2 739 71 97, evi.declercq@riziv-inami.fgov.be.
\end{remark}

\subsubsection{Other Clinical Resources}

\begin{description}[style=nextline]
\item[UZ Leuven] \hfill \url{https://www.uzleuven.be/nl/chronisch-vermoeidheidssyndroom-cvs}\\
University Hospital Leuven provides care for pronounced fatigue lasting more than 6 months that doesn't improve with rest and significantly limits daily activities.

\item[UZA -- Centrum voor Gedragstherapie bij Vermoeidheid] \hfill \url{https://www.uza.be/behandeling/chronisch-vermoeidheids-syndroom-cvs}\\
University Hospital Antwerp's Center for Behavioral Therapy for Fatigue (CGVF) treats patients with fatigue and functional complaints including CVS, chronic fatigue, fibromyalgia, and post-infectious fatigue.

\item[UZ Gent CVS Network] \hfill\\
UZ Gent coordinates an integrated care model for abnormal fatigue in East and West Flanders. Partners include AZ Alma, AZ Groeninge, Jan Yperman Ziekenhuis, AZ Maria Middelares, AZ Sint-Jan, AZ Nikolaas, AZ Delta, and others. The GP serves as care process manager.
\end{description}

\subsubsection{Sleep Medicine (Wallonia)}

\begin{description}[style=nextline]
\item[Centre Multidisciplinaire de Somnologie -- Clinique Saint-Luc Bouge] \hfill\\
\textbf{Address:} Rue Saint-Luc 8, 5004 Bouge (Namur)\\
\textbf{Phone:} +32 81 20 94 61\\
\textbf{Email:} labosommeil@slbo.be\\
\textbf{Website:} \url{https://slbo.be/services/centres-integres-et-pluridisciplinaires/centre-multidisciplinaire-de-somnologie/}\\
Established in 1993, this is a major French-language sleep medicine center in Wallonia. INAMI-accredited since 2002 for CPAP treatment and since 2018 for mandibular advancement devices. The multidisciplinary team includes pneumologists, neurologists, psychiatrists, ORL specialists, and psychologists. Accepts patients from age 15 with any sleep disorder. As of September 2025, waiting times reduced to approximately 6 weeks (compared to 6--8 months at many Belgian hospitals). Chief of Service: Dr Richard Frognier.
\end{description}

\begin{remark}[Sleep Disorders and ME/CFS]
While sleep clinics do not diagnose ME/CFS directly, they are valuable for ruling out primary sleep disorders (sleep apnea, narcolepsy, etc.) that can cause chronic fatigue, and for documenting the non-restorative sleep characteristic of ME/CFS. A sleep study may be part of the differential diagnosis workup.
\end{remark}

\subsection{France}
\label{subsec:france-orgs}

\begin{description}[style=nextline]
\item[ASFC (Association française du Syndrome de Fatigue Chronique)] \hfill \url{https://www.asso-sfc.org/}\\
The only ME/CFS patient association approved by the French Ministry of Health (2015). Located in Lille. Works with a Scientific Board to welcome, inform, and support patients. Operates phone hotline, organizes regular patient meetings throughout France, and annual meetings with expert scientists.

\item[EM Action France] \hfill French website reporting international ME news and research.

\item[Millions Missing France] \hfill \url{https://millionsmissing.fr/}\\
French chapter of the \#MillionsMissing movement.
\end{description}

\begin{remark}[French Recognition]
ME/CFS is not officially recognized by the French Department of Health, leading to under-diagnosis and lack of disability recognition.
\end{remark}

\subsection{Luxembourg}

No dedicated ME/CFS patient organization exists in Luxembourg as of 2025. Patients may connect with:
\begin{itemize}
    \item Belgian or French organizations (French-language)
    \item German organizations (German-language)
    \item EUROMENE network for research connections
    \item European ME Alliance for pan-European advocacy
\end{itemize}

\subsection{Germany}
\label{subsec:germany-orgs}

\begin{description}[style=nextline]
\item[Deutsche Gesellschaft für ME/CFS] \hfill \url{https://www.mecfs.de/}\\
German Society for ME/CFS, founded in 2016, based in Hamburg. Run by volunteers advocating for patient rights and medical/social recognition. Organized the first Parliamentary Expert Discussion on ME/CFS in the Bundestag (March 2020). Successfully lobbied for ME/CFS and Long COVID mention in federal government coalition agreement (2021). Successfully challenged the German ``Tiredness'' guideline's recommendations for GET and CBT.

\item[Fatigatio e.V.] \hfill \url{https://www.fatigatio.de/}\\
Federal Association ME/CFS, founded in 1993, based in Berlin. Over 2,900 members. Operates 15 regional self-help groups across German cities. Organizes annual hybrid ME/CFS conference with national and international experts.
\end{description}

\begin{remark}[German Prevalence]
Germany has seen dramatic growth in ME/CFS cases post-COVID: from an estimated 250,000 pre-pandemic to over 650,000 by end of 2024.
\end{remark}

\subsection{Netherlands}
\label{subsec:netherlands-orgs}

\begin{description}[style=nextline]
\item[ME/cvs Vereniging] \hfill \url{https://www.me-cvs.nl/}\\
Dutch association founded in 2005. The lowercase ``cvs'' deliberately underscores the desire for the medical world to stop using the name ``Chronic Fatigue Syndrome.'' Involved in international partnerships including UK ME/CFS Biobank, Solve ME, Open Medicine Foundation, and Charité Berlin.

\item[ME Vereniging Nederland] \hfill Founded 2011. Membership open only to ME patients. Focus on improving living conditions and reducing social exclusion.

\item[ME/CVS-Stichting Nederland] \hfill Founded 1987. Receives government funding as recognized national ME/CFS patient organization.

\item[Steungroep ME en Arbeidsongeschiktheid] \hfill Founded 1994. Support group focused on employment, education, disability, and benefits issues. Campaigns against exclusion from disability benefits.
\end{description}

The \textbf{Netherlands ME/CFS Cohort and Biobank (NMCB)} consortium is a national collaboration of research institutes, patient organizations, and clinical centers establishing a comprehensive patient cohort and biobank.

\subsection{United Kingdom}
\label{subsec:uk-orgs}

\begin{description}[style=nextline]
\item[ME Association] \hfill \url{https://meassociation.org.uk/}\\
One of the two largest UK ME/CFS charities. Provides information, advocacy, and services. Publishes quarterly magazine \textit{ME Essential}. Funds the UK ME/CFS Biobank. Conducted major patient surveys (2010, 2015) documenting treatment experiences. Hosts local support groups nationwide.

\item[Action for ME] \hfill \url{https://www.actionforme.org.uk/}\\
Founded 1987 as The M.E. Action Campaign. Merged with Association of Young People with ME in 2017. Funds high-quality research including the groundbreaking DecodeME study (largest ME/CFS genetic study ever, 15,000+ UK participants). Offers free support services including holistic healthcare services.

\item[BACME (British Association of Clinicians in ME/CFS)] \hfill \url{https://bacme.info/}\\
Multidisciplinary organization for UK healthcare professionals delivering care to ME/CFS patients.
\end{description}

\subsection{Ireland}
\label{subsec:ireland-orgs}

\begin{description}[style=nextline]
\item[Irish ME/CFS Association] \hfill \url{https://www.irishmecfs.org/}\\
Works to improve the situation for people with ME/CFS in Ireland. Notable advocate: Tom Kindlon (Assistant Chairperson), who has been housebound with severe ME for over 22 years. Known internationally for his extensive analysis and publications on the PACE trial and harms of graded exercise therapy.

\item[Hope for ME \& Fibro Northern Ireland] \hfill Founded 2011 by Joan McParland MBE, inspired by Tom Kindlon's work.
\end{description}

\subsection{Norway}
\label{subsec:norway-orgs}

\begin{description}[style=nextline]
\item[Norges ME Forening] \hfill \url{https://www.me-foreningen.no/}\\
Norwegian ME Association, founded 1987 by Ellen Piro. Represents over 6,000 ME patients. Party-political independent organization. Works to ensure diagnosis based on Canadian Consensus Criteria. Member of European ME Alliance.

\item[ME-Fondet] \hfill \url{https://www.me-fondet.no/}\\
Norwegian non-profit foundation dedicated to funding biomedical ME research. Supporting a promising daratumumab pilot study at Haukeland University Hospital.
\end{description}

\begin{remark}[Norwegian Research Leadership]
Norway has been a leader in ME/CFS research, pioneering innovative treatment approaches including the rituximab trials and subsequent work on autoimmunity.
\end{remark}

\subsection{Denmark}
\label{subsec:denmark-orgs}

\begin{description}[style=nextline]
\item[ME Foreningen (Danish ME Association)] \hfill \url{https://me-foreningen.dk/}\\
National association since 1992. Works to increase knowledge in the Danish healthcare system about ME as a physical/biomedical disease. Achieved unanimous Danish parliament vote to separate ME (ICD-10 G93.3) from Functional Disorders. Counseling available Wednesday/Friday 12--14 at +45 44 95 97 00.
\end{description}

\subsection{Sweden}
\label{subsec:sweden-orgs}

\begin{description}[style=nextline]
\item[Riksföreningen för ME-patienter (RME)] \hfill Swedish national association for ME patients. Member of European ME Alliance.
\end{description}

\subsection{Switzerland}
\label{subsec:switzerland-orgs}

\begin{description}[style=nextline]
\item[ME/CFS Verein Schweiz] \hfill \url{https://www.mecfs.ch/}\\
Self-help organization founded 1993 in Zurich. Offers information platform, networking, and support. Hosts regular group meetings in several Swiss cities.

\item[Schweizerische Gesellschaft für ME \& CFS] \hfill \url{https://sgme.ch/}\\
Swiss Association for ME \& CFS, founded 2019. Fights for recognition and adequate care. Conducts biennial comprehensive surveys on Swiss ME patient situations. First analysis published 2021.
\end{description}

\begin{remark}[Swiss Diagnostic Challenges]
Swiss research shows mean diagnosis time of 6.7 years, average 11.1 different appointments, 2.6 misdiagnoses, and 13.5\% of patients traveling abroad to seek diagnosis. 90.5\% of patients were told at least once that symptoms were psychosomatic.
\end{remark}

\subsection{Spain}
\label{subsec:spain-orgs}

\begin{description}[style=nextline]
\item[CONFESQ] \hfill \url{http://confederacion-fm-sfc.es/}\\
National Coalition of FM, CFS/ME, MCS, and EHS. Established 2004. Based in Jerez de la Frontera.

\item[ONG-PEM (Asociación de Personas con Encefalomielitis Miálgica)] \hfill Founded and run by severely ill patients. Exclusively represents Myalgic Encephalomyelitis.

\item[Associació Catalana d'Afectats SFC/EM] \hfill \url{http://www.acsfcem.org/}\\
Catalonia-based patient association.
\end{description}

Spanish Facebook groups include VIVIR CON SFC/EM and \#MillonesAusentes (Spanish \#MillionsMissing).

\subsection{Italy}
\label{subsec:italy-orgs}

\begin{description}[style=nextline]
\item[CFS/ME Associazione Italiana] \hfill \url{http://www.stanchezzacronica.it/}\\
Founded 1991 by Prof. Umberto Tirelli in Udine---first Italian physician to identify CFS cases. Based at Centro di Riferimento Oncologico, Aviano.

\item[Associazione Malati di CFS ODV] \hfill \url{http://www.associazionecfs.com/}\\
Patient advocacy group founded 2004, based in Pavia. Part of Rare Disease Alliance (Alleanza delle Malattie Rare). Celebrated 20 years in 2024.

\item[CFS/ME Organizzazione di Volontariato] \hfill \url{https://www.cfsme.it/}\\
Veneto-based patient organization.
\end{description}

\textbf{CFS Italia Forum} (\url{http://www.cfsitalia.it/}) provides Italian-language patient community and information exchange.

\subsection{Australia}
\label{subsec:australia-orgs}

\begin{description}[style=nextline]
\item[Emerge Australia] \hfill \url{https://emerge.org.au/}\\
National organization providing services, evidence-based education, advocacy, and research. Free national health and support line: 1800 865 321 (9am--4:30pm Mon--Fri). Offers online patient/carer education, peer support groups, RACGP CPD-approved healthcare professional education. Partners with Solve ME on AusME patient registry and biobank.

\item[ME/CFS Australia] \hfill \url{https://mecfs.org.au/}\\
Peak body for patient-led ME/CFS charities. Focuses on federal government advocacy, research initiatives, and national awareness campaigns.
\end{description}

\subsection{United States}
\label{subsec:us-orgs}

\begin{description}[style=nextline]
\item[American ME and CFS Society (AMMES)] \hfill \url{https://ammes.org/}\\
Serves patients and caregivers through support, advocacy, and education. Channels patient perspectives to government agencies and initiatives. Comprehensive website with links to international organizations.

\item[U.S. ME/CFS Clinician Coalition] \hfill \url{https://mecfscliniciancoalition.org/}\\
Provides resources for medical providers caring for ME/CFS patients. Developed clinical guidance documents.

\item[Bateman Horne Center] \hfill \url{https://batemanhornecenter.org/}\\
Medical center of excellence for ME/CFS and fibromyalgia. Founded by Dr. Lucinda Bateman. Focuses on diagnosis, treatment, research, and patient empowerment.
\end{description}

%===============================================================================
\section{Research Centers and Specialized Clinics}
\label{sec:research-centers}
%===============================================================================

\subsection{Leading Research Centers}

\begin{description}[style=nextline]
\item[Stanford ME/CFS Collaborative Research Center] \hfill \url{https://med.stanford.edu/chronicfatiguesyndrome/}\\
Established 2013, directed by Ron Davis, PhD. Part of Stanford Genome Technology Center. Focus on developing objective diagnostic tests and treatments. Known for nanoneedle diagnostics development.

\item[Columbia Center for Infection and Immunity] \hfill Columbia University.\\
Directed by W. Ian Lipkin, MD. ME/CFS research focus on infectious triggers and immune dysfunction.

\item[Cornell ME/CFS Center for Enervating NeuroImmune Disease] \hfill Directed by Maureen Hanson, PhD. Research on mitochondrial function, immune cells, and microbiome.

\item[Charité Fatigue Centrum] \hfill Berlin, Germany. \url{https://cfc.charite.de/}\\
Major European ME/CFS research and clinical center. Led by Prof. Carmen Scheibenbogen.

\item[Uppsala University ME/CFS Collaboration] \hfill Sweden.\\
Led by Jonas Bergquist, MD, PhD. Focus on neurochemistry and analytical approaches.
\end{description}

\subsection{Specialized Clinical Centers}

\begin{description}[style=nextline]
\item[Bateman Horne Center] \hfill Salt Lake City, Utah.\\
Clinical care with research integration. Founded by Dr. Lucinda Bateman.

\item[Open Medicine Clinic] \hfill Mountain View, California.\\
Run by Dr. David Kaufman. Known for complex chronic illness expertise.

\item[Haukeland University Hospital] \hfill Bergen, Norway.\\
Site of rituximab trials and ongoing autoimmunity research.
\end{description}

%===============================================================================
\section{Online Communities and Forums}
\label{sec:online-communities}
%===============================================================================

\subsection{Discussion Forums}

\begin{description}[style=nextline]
\item[Phoenix Rising] \hfill \url{https://phoenixrising.me/} and \url{https://forums.phoenixrising.me/}\\
Founded by Cort Johnson. One of the most visited ME/CFS websites. Approximately 19,000 member accounts with 600 daily active members (2017). Covers dysautonomia, hormones, methylation, lifestyle management, relationships, and caregiver support.

\item[Science for ME (S4ME)] \hfill \url{https://www.s4me.info/}\\
Independent, patient-led, international forum. Founded by Andy Devereux-Cooke. Each thread typically dedicated to a single research paper, enabling in-depth discussion. Notable members have included Jonathan Edwards, Tom Kindlon, Simon McGrath, and David Tuller. Advocates for patients as research partners.

\item[Health Rising Forums] \hfill \url{https://www.healthrising.org/forums/}\\
Companion to Health Rising blog. Discussion of ME/CFS, fibromyalgia, chronic pain, IBS, and dysautonomia.

\item[MEpedia] \hfill \url{https://me-pedia.org/}\\
Crowd-sourced encyclopedia of ME/CFS science and history. Creative Commons licensed. Categories include notable patients, citizen scientists, and ME/CFS history. Valuable reference for terminology, research summaries, and advocacy history.
\end{description}

\subsection{Reddit Communities}

\begin{description}[style=nextline]
\item[r/cfs] \hfill \url{https://www.reddit.com/r/cfs/}\\
Primary ME/CFS subreddit. Research discussions, treatment experiences, and personal support. Active moderation maintaining distinction between ME/CFS and chronic fatigue symptom.

\item[r/covidlonghaulers] \hfill Related community for Long COVID with significant ME/CFS overlap.
\end{description}

\subsection{Facebook Groups}

Major ME/CFS Facebook groups (search on Facebook):
\begin{itemize}
    \item Chronic Fatigue Syndrome \& Myalgic Encephalomyelitis ME Self Help Group (founded 2012)
    \item \#MEAction state/regional chapters
    \item Pregnancy and Parenting with ME/CFS
    \item Caregiver Support groups
    \item Severe ME support groups
    \item Youth ME/CFS support (ages 13--21)
\end{itemize}

\subsection{Other Platforms}

\begin{description}[style=nextline]
\item[Smart Patients ME/CFS Community] \hfill \url{https://www.smartpatients.com/communities/me-cfs}\\
Peer-to-peer support where patients and families share experiences and research.

\item[NURA] \hfill Social network platform specifically for Long COVID, ME/CFS, and fibromyalgia patients, created by people with these conditions.
\end{description}

%===============================================================================
\section{Prominent Patient Advocates and Content Creators}
\label{sec:patient-advocates}
%===============================================================================

\begin{observation}[The Patient Expertise Network]
ME/CFS advocacy is largely driven by patients themselves, often working with extremely limited energy. The individuals listed here represent a fraction of the patient community dedicating their scarce functional capacity to improving conditions for all patients. Many severely ill patients contribute via social media, writing single tweets or posts that may represent their entire energy expenditure for a day.
\end{observation}

\subsection{Filmmakers and Documentarians}

\subsubsection{Jennifer Brea}
\begin{description}[style=sameline]
\item[Background] American documentary filmmaker and activist. PhD student at Harvard when sudden illness left her bedridden.
\item[Key work] \textit{Unrest} (2017)---Sundance award-winning documentary, Emmy-nominated, shortlisted for Academy Award. Available free on YouTube (May 2023). Produced largely from bed, directing remotely with crews worldwide.
\item[Advocacy] Co-founder of \#MEAction. Delivered highest-rated TED Talk at 2016 TED Summit (nearly 2 million views, 25+ languages).
\item[Personal journey] Later discovered craniocervical instability (CCI) and underwent spinal fusion surgery, experiencing significant improvement---highlighting ME/CFS subgroup heterogeneity.
\item[Website] \url{https://www.jenniferbrea.com/}
\end{description}

\subsubsection{Dianna Cowern (Physics Girl)}
\begin{description}[style=sameline]
\item[Background] Science educator, YouTube channel with 2.8+ million subscribers.
\item[Illness] Contracted COVID-19, developed Long COVID/ME/CFS. Currently completely bedbound, unable to care for herself. Also developed MCAS.
\item[Advocacy] 12-hour livestream (July 6, 2024) showing ``a day in her life'' with severe ME/CFS. Co-hosted by Ian Hecox and Simone Giertz. Raised \$150,000+ for Open Medicine Foundation. Livestream became top post on r/videos (27M subscribers).
\item[Impact] Brought ME/CFS awareness to mainstream audience unfamiliar with the condition.
\item[Platform] YouTube: Physics Girl; Twitter/X: @thephysicsgirl
\end{description}

\subsection{Writers and Bloggers}

\subsubsection{Cort Johnson}
\begin{description}[style=sameline]
\item[Background] Developed ME/CFS/FM in 1980s while in Environmental Studies program at UC Santa Cruz. MS in Environmental Studies from San Jose State University (2000).
\item[Key work] Founded Phoenix Rising (2004)---became most visited ME/CFS website by 2010. Left to found Health Rising (2012), broadening focus to include fibromyalgia. Produced 1000+ comprehensive blogs on ME/CFS and FM.
\item[Recognition] ProHealth's Advocate of the Year (2015). IACFS/ME Special Services Award (2016). Described as ``the quintessential patient advocate, breaking more news about this illness than many professional journalists.''
\item[Website] \url{https://www.healthrising.org/}
\item[Social media] Twitter/X: @CortJohnson
\end{description}

\subsubsection{Jamison Hill}
\begin{description}[style=sameline]
\item[Background] Former bodybuilder and certified personal trainer at Sonoma State University. Developed ME after mononucleosis in senior year (2010).
\item[Illness severity] By age 28, bedridden, unable to speak, eat solid food, or elevate body. Wrote on cellphone wearing tanning goggles to block light.
\item[Publications] \textit{When Force Meets Fate: A Mission to Solve an Invisible Illness} (2021 memoir). Written for The Washington Post, The New York Times, Los Angeles Times, Men's Journal, Vox, VICE, and others.
\item[Media] Featured in \textit{Forgotten Plague} documentary, Netflix series (2018), WBUR Modern Love podcast, Dax Shepard's Armchair Expert podcast.
\item[Current status] Improved with anti-virals, hydrocortisone, IV saline---not fully recovered but able to tell his story.
\item[Website] \url{https://jamisonwrites.com/}
\end{description}

\subsubsection{Whitney Dafoe}
\begin{description}[style=sameline]
\item[Background] Son of Dr. Ron Davis (Stanford geneticist) and Dr. Janet Dafoe. Former adventurer and photographer who traveled to all 50 states, India, Nepal, Ecuador.
\item[Diagnosis] ME/CFS diagnosed 2010.
\item[Current severity] One of the most severe ME/CFS cases documented. Cannot speak. Cannot tolerate contact with anyone but parents due to visual dysfunction. Fed by tube directly into stomach. Hasn't spoken in years.
\item[Advocacy] Despite severity, maintains blog, Facebook page, Instagram documenting life with severe ME/CFS. Won Gold at European Photography Awards (2022) for documentary series ``The Living Death.''
\item[Impact] His illness catalyzed his father's complete redirection of research focus: ``I decided to terminate everything I was working on before Whitney got sick. Everything is ME/CFS now.''
\item[Website] \url{https://www.whitneydafoe.com/}
\item[Patreon] \url{https://www.patreon.com/whitneydafoe}
\end{description}

\subsubsection{Other Notable Bloggers}

\begin{description}[style=nextline]
\item[Suzan Jackson (Live with CFS)] \hfill Has ME/CFS since 2002; both sons also developed ME/CFS at ages 6 and 10. Blog focuses on living well despite chronic illness.

\item[Mary M. Schweitzer, PhD (Slightly Alive)] \hfill Former history professor. Maintains ME and CFS Information Page with essays, reports, and conference summaries.

\item[Naomi Whittingham (A Life Hidden)] \hfill UK-based, severe ME since age 12. Does interviews and supports brother Tom Whittingham's marathon fundraising for ME Research UK.

\item[Laura's Pen] \hfill Blog covering Lyme disease, ME/CFS, and endometriosis awareness.

\item[Super Pooped] \hfill ME/CFS awareness through art, crafts, and humor.
\end{description}

\subsection{Researchers Who Are Patients or Family Members}

\subsubsection{Ron Davis, PhD}
\begin{description}[style=sameline]
\item[Position] Professor of Biochemistry and Genetics, Director of Stanford Genome Technology Center.
\item[Background] Pioneered technology that powered the Human Genome Project. Over 64 biotechnology patents.
\item[Personal connection] Son Whitney has very severe ME/CFS.
\item[Research pivot] ``I decided to terminate everything I was working on before Whitney got sick. Everything is ME/CFS now.''
\item[Leadership] Director of OMF Scientific Advisory Board. Established Stanford ME/CFS Collaborative Research Center (2013). His work helped prove ME/CFS is a biological disease.
\end{description}

\subsubsection{Tom Kindlon}
\begin{description}[style=sameline]
\item[Background] Very active young man (soccer, tennis, cricket, cross-country) until ME at age 16 (February 1989).
\item[Current status] Housebound for 22+ years. Uses wheelchair. Full-time carer: his mother Vera.
\item[Expertise] Studied Mathematical Sciences at Trinity College Dublin before dropping out. Extensive analysis and publications on PACE trial and harms of graded exercise therapy. Work available on ResearchGate and PubMed.
\item[Role] Assistant Chairperson, Irish ME/CFS Association.
\item[Recognition] OMF certificate of merit. Described as ``a leader in the global ME/CFS community'' who ``initiated patient-led efforts to take a scientific approach to analyzing ME/CFS research.'' Nominated for honorary degree at Trinity College Dublin.
\end{description}

\subsubsection{Andy Devereux-Cooke}
\begin{description}[style=sameline]
\item[Role] Patient, founder of Science for ME forum.
\item[Research] Research investigator on DecodeME study---demonstrating patient partnership in research.
\end{description}

\subsection{Celebrity Advocates}

\begin{description}[style=nextline]
\item[Laura Hillenbrand] \hfill Author of \textit{Seabiscuit} and \textit{Unbroken}. Candid about ME/CFS struggles and medical misunderstanding.

\item[Karin Alvtegen] \hfill Scandinavian author of psychological thrillers (\textit{Missing}, \textit{Betrayal}). ME/CFS has significantly shaped her life and career.
\end{description}

\subsection{Euthanasia and End-of-Life Discussions}

\begin{remark}[Community Discourse on Quality of Life]
The severity of ME/CFS has led to difficult conversations within the patient community about quality of life and end-of-life options. Some patients with very severe ME/CFS have publicly discussed or pursued medical assistance in dying in countries where it is legal (Belgium, Netherlands, Switzerland, Canada). These discussions reflect the profound suffering experienced by the most severely affected patients and the lack of effective treatments.
\end{remark}

\begin{itemize}
    \item \textbf{Samuel (Austria, 2004--2026)} -- Samuel developed very severe ME/CFS following a COVID-19 infection. In his final public statement, posted to Reddit 12 days before his death, he described his condition: ``I must lie in bed 24 hours a day and cannot move too much, it must be permanently dark because I cannot tolerate light. I wear double hearing protection because I also cannot tolerate sounds. I cannot watch television or videos for even a second, as moving images overwhelm my nervous system and trigger unbearable suffering. I cannot listen to music or podcasts. I cannot even talk to my own mother, who cares for me, because listening is too exhausting and speaking has become completely impossible. I must communicate with pen and paper.'' He described the physical suffering as ``like drowning and burning at the same time'' and noted that any exertion beyond his energy limits caused crashes leading to permanent deterioration. He highlighted systemic failures: no approved medications exist for ME/CFS, Austria has no dedicated treatment centers despite 1\% of the population being affected, most physicians do not recognize the disease, and the pension authority (PVA) called patients ``charlatans and freeloaders.'' Samuel chose medical assistance in dying in Austria, passing away on January 30, 2026---his 22nd birthday. His final message: ``ME/CFS kills!'' (\textit{ME/CFS TÖTET!})

    Sources: Original Reddit post: \url{https://www.reddit.com/r/Austria/comments/1qg8oit/}; News coverage: \url{https://www.heute.at/s/samuel-21-ist-tot-er-starb-an-seinem-geburtstag-120159259}

    \item \textbf{Austrian Policy Discussion} -- Reddit r/Austria community discussion on euthanasia policies: \url{https://www.reddit.com/r/Austria/s/JUzy5LM6O7}
\end{itemize}

%===============================================================================
\section{Podcasts}
\label{sec:podcasts}
%===============================================================================

\begin{description}[style=nextline]
\item[The Understanding ME/CFS Podcast] \hfill Apple Podcasts, Spotify\\
Hosted by Patrick Ussher (7-year ME/CFS patient, author of \textit{Understanding ME/CFS \& Strategies for Healing}). Weekly interviews with patients and experts. Covers research, treatments, quality of life, and recovery stories.

\item[Chronically Complex: The \#MEAction Podcast] \hfill \url{https://www.meaction.net/chronically-complex-meaction-podcast/}\\
Interviews influential voices in ME/CFS and Long COVID. Topics include books on complex chronic disease, \#MillionsMissing, \#StopRestPace, disability activism, and art from disabled artists. Notable guests: Ryan Prior (CNN journalist, \textit{Forgotten Plague} filmmaker), Cynthia Adinig (Long COVID advocate, SolveME board member).

\item[CFS Unravelled] \hfill By Dan Neuffer (recovered from ME/CFS). Interviews with recovered patients and expert practitioners.

\item[Discomfort Zone (Invisible Not Broken)] \hfill Hosted by Jason, engineering graduate who developed fibromyalgia, ME/CFS, and POTS. Each episode explores what it means to be chronically ill and disabled.

\item[This Podcast Will Kill You -- Episode 137] \hfill ``ME/CFS: What's in a name? (A lot, actually)'' (April 2024). Deep dive into biology, history, and current research.

\item[Hope and Help for Fatigue \& Chronic Illness] \hfill Mission to help people with post-viral syndromes including Long COVID and ME/CFS.
\end{description}

%===============================================================================
\section{YouTube Channels and Video Resources}
\label{sec:youtube}
%===============================================================================

\subsection{Patient-Focused Channels}

\begin{description}[style=nextline]
\item[CFS Health] \hfill 32.7K subscribers. Founded by Toby Morrison. Multi-dimensional approach to ME/CFS and fibromyalgia recovery.

\item[CFS Unravelled] \hfill 53.5K subscribers. Dan Neuffer shares insights on ME/CFS, POTS, and fibromyalgia healing.

\item[Understanding ME-CFS] \hfill Patrick Ussher's channel accompanying his podcast.
\end{description}

\subsection{Organization Channels}

\begin{description}[style=nextline]
\item[Open Medicine Foundation] \hfill Research updates, patient stories, educational content.

\item[Solve ME/CFS Initiative] \hfill 10K subscribers. Research and advocacy updates.

\item[Bateman Horne Center] \hfill 15.6K subscribers. Clinical education and patient resources.

\item[MEAction] \hfill 3.3K subscribers. Advocacy updates and \#MillionsMissing content.
\end{description}

\subsection{Documentaries}

\begin{description}[style=nextline]
\item[\textit{Unrest}] \hfill (2017) Jennifer Brea. Available free on YouTube (since May 2023). Essential viewing for understanding patient experience.

\item[\textit{Forgotten Plague}] \hfill Co-directed by Ryan Prior. Features Jamison Hill and other patients.

\item[\textit{What About ME?}] \hfill Earlier documentary on ME/CFS.

\item[\textit{Hope to our Hands}] \hfill (2020) Documentary about ME/CFS patients in Japan struggling for acknowledgment.

\item[\textit{Living with Chronic Fatigue Syndrome}] \hfill German/French documentary premiered on ARTE. Available in German and French.
\end{description}

%===============================================================================
\section{Social Media Hashtags and Campaigns}
\label{sec:social-media}
%===============================================================================

\subsection{Key Hashtags}

\begin{tabular}{ll}
\texttt{\#MillionsMissing} & Primary advocacy hashtag for global protests \\
\texttt{\#MECFS} & Standard disease hashtag \\
\texttt{\#pwME} & ``People with ME'' \\
\texttt{\#SevereME} & Focusing on severe/very severe patients \\
\texttt{\#MyalgicE} & Short form for myalgic encephalomyelitis \\
\texttt{\#May12th} & ME Awareness Day \\
\texttt{\#StopRestPace} & Pacing advocacy \\
\texttt{\#TeachMETreatME} & 2024 campaign theme \\
\texttt{\#LongCovid} & Related condition with significant overlap \\
\end{tabular}

\subsection{\#MillionsMissing Campaign}

Annual global campaign for ME health equality, organized by \#MEAction. May 12th is ME Awareness Day---patients gather (in-person and virtually) to demand recognition, research, and clinical care. In 2025, communities joined at the U.S. Capitol to advocate for protecting Medicaid, home care support, research funding, and open science.

The campaign highlights the ``millions missing'' from their own lives due to illness, and the millions of research dollars missing from funding.

%===============================================================================
\section{Books by Patients and Advocates}
\label{sec:books}
%===============================================================================

\subsection{Patient Memoirs}

\begin{description}[style=nextline]
\item[\textit{When Force Meets Fate: A Mission to Solve an Invisible Illness}] \hfill Jamison Hill (2021). Former bodybuilder's journey through severe ME/CFS.

\item[\textit{The Puzzle Solver: A Scientist's Desperate Quest to Cure the Illness that Stole His Son}] \hfill Tracie White and Ron Davis (2021). Story of Ron Davis and Whitney Dafoe.

\item[\textit{Understanding ME/CFS \& Strategies for Healing}] \hfill Patrick Ussher. Guide by a patient, companion to podcast.

\item[\textit{The Long Haul}] \hfill Ryan Prior. On Long COVID and ME/CFS advocacy.
\end{description}

\subsection{Clinical and Reference Works}

See also the scientific literature cited throughout this document. Patient organizations often maintain curated reading lists of accessible scientific overviews.

%===============================================================================
\section{Clinical Trial Registries}
\label{sec:clinical-trials}
%===============================================================================

\begin{description}[style=nextline]
\item[ClinicalTrials.gov] \hfill \url{https://clinicaltrials.gov/}\\
Search for ``myalgic encephalomyelitis'' or ``chronic fatigue syndrome.'' Filter by recruiting status and location.

\item[EU Clinical Trials Register] \hfill \url{https://www.clinicaltrialsregister.eu/}\\
European clinical trials database.

\item[ME/CFS Research Register] \hfill \url{https://mecfs-research.org/}\\
Specialized registry tracking ME/CFS research internationally.
\end{description}

\textbf{Questions to ask before participating:}
\begin{itemize}
    \item What are inclusion/exclusion criteria?
    \item What is the time commitment?
    \item Will travel be required? Is remote participation possible?
    \item What accommodations exist for severely ill participants?
    \item How will participant safety be monitored?
    \item Will results be shared with participants?
\end{itemize}

%===============================================================================
\section{Patient Registries and Biobanks}
\label{sec:registries}
%===============================================================================

\begin{description}[style=nextline]
\item[You+ME Registry and Biobank] \hfill \url{https://youandme.solvecfs.org/}\\
Solve M.E.'s patient registry. Collects patient-reported data and biospecimens. International participation.

\item[AusME Registry] \hfill Australian ME/CFS and Long COVID registry, partnership between Emerge Australia and Solve M.E.

\item[UK ME/CFS Biobank] \hfill London School of Hygiene \& Tropical Medicine. Funded partly by ME Association.

\item[Netherlands ME/CFS Cohort and Biobank (NMCB)] \hfill National Dutch infrastructure for ME/CFS research.

\item[DecodeME] \hfill UK-based genetic study with 15,000+ participants. Largest ME/CFS study ever conducted.
\end{description}

%===============================================================================
\section{Disability and Legal Resources}
\label{sec:disability-resources}
%===============================================================================

\subsection{General Guidance}

Most national patient organizations provide country-specific guidance on:
\begin{itemize}
    \item Disability benefits applications
    \item Workplace accommodations
    \item Educational accommodations
    \item Healthcare rights
    \item Insurance issues
\end{itemize}

\subsection{Key Considerations}

\begin{itemize}
    \item ME/CFS is classified as a neurological disease by WHO (ICD-11: 8E49)
    \item Documentation of functional limitations is essential
    \item Some countries recognize ME/CFS for disability; others require extensive advocacy
    \item Patient organizations often provide template letters and case examples
    \item Legal advocacy organizations exist in some countries
\end{itemize}

The \textbf{Steungroep ME en Arbeidsongeschiktheid} (Netherlands) specifically focuses on employment and disability issues. The \textbf{ME Association} (UK) provides extensive guidance on UK benefits system.

%===============================================================================
\section{Resource Evaluation Guidelines}
\label{sec:evaluation-guidelines}
%===============================================================================

\begin{observation}[Navigating Information Quality]
The ME/CFS information landscape includes high-quality patient-led resources alongside misinformation and exploitation. The patient community has developed sophisticated evaluation skills born of necessity.
\end{observation}

\subsection{Indicators of Reliable Resources}

\begin{itemize}
    \item Connection to established patient organizations
    \item Citation of peer-reviewed research
    \item Acknowledgment of uncertainty and ME/CFS heterogeneity
    \item Clear distinction between established knowledge and speculation
    \item Avoidance of cure claims
    \item Transparency about funding sources
    \item Respect for patient autonomy and pacing needs
\end{itemize}

\subsection{Warning Signs}

\begin{itemize}
    \item Guaranteed cures
    \item Pressure to commit quickly or pay upfront
    \item Hostility to questions
    \item Rejection of biomedical model without evidence
    \item Promotion of graded exercise therapy without acknowledging PEM risks
    \item Claims that contradict major patient surveys
    \item No outcome data available
    \item Primarily selling products or services
\end{itemize}

\begin{remark}[For Wallonia/Belgium Residents]
French-language resources from ASFC (France), Swiss organizations, and Canadian French resources may supplement the primarily Flemish Belgian organizations. The European ME Alliance provides pan-European perspective and advocacy regardless of language.
\end{remark}