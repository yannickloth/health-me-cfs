\chapter{Medical Recommendations and Analysis}
\label{app:recommendations}

\begin{warning}[Preliminary Medical Analysis]
All recommendations in this appendix are generated through AI-assisted analysis
of case data and medical literature. They represent \textbf{preliminary evidence
summaries for discussion with qualified healthcare providers}, not final medical
advice. Every recommendation must be reviewed and approved by the patient's
treating physician before implementation.
\end{warning}

\section*{About This Appendix}

This appendix contains:

\begin{itemize}
\item Evidence-based treatment recommendations from \texttt{medical-advisor}
\item Statistical analyses of treatment effectiveness from \texttt{treatment-analyst}
\item Subtype and mechanistic hypotheses from \texttt{hypothesis-generator}
\item Crisis management protocols from \texttt{crisis-manager}
\end{itemize}

Each recommendation includes:
\begin{itemize}
\item Evidence base with citations and quality ratings
\item Specific protocols and dosing guidance
\item Monitoring parameters
\item Risks and contraindications
\item Questions for physician discussion
\item Expected timeline for results
\end{itemize}

\section*{How to Use This Appendix}

\subsection*{For Patients}

Review recommendations with your treating physician. Bring relevant sections to
medical appointments. Use the ``Questions for Doctor'' lists to facilitate
informed discussions about treatment options.

\subsection*{For Healthcare Providers}

This appendix provides:
\begin{itemize}
\item Systematic review of patient's case data
\item Evidence synthesis from current ME/CFS literature
\item Patient's treatment response history
\item Specific questions requiring clinical judgment
\end{itemize}

Please review recommendations in context of complete medical history,
contraindications, and individual patient factors not captured in automated
analysis.

\section*{Certainty Levels}

Recommendations are categorized by evidence quality:

\begin{itemize}
\item \textbf{High Certainty:} Large studies (n>100), peer-reviewed, replicated,
consistent results, low bias
\item \textbf{Medium Certainty:} Moderate studies (n=20-100), peer-reviewed,
limited replication, or single high-quality study
\item \textbf{Low Certainty:} Small studies (n<20), preprints, mechanistic
rationale only, or conflicting evidence
\end{itemize}

\section*{Status Indicators}

\begin{itemize}
\item ⚠️ \textbf{Preliminary:} Awaiting physician review
\item ✓ \textbf{Approved:} Physician has reviewed and approved for trial
\item ✗ \textbf{Declined:} Physician has declined or determined inappropriate
\item 🔄 \textbf{In Progress:} Currently trialing
\item ✅ \textbf{Completed:} Trial finished, see treatment-analyst results
\end{itemize}

\newpage

%% ============================================================================
%% Medical Advisor Recommendations
%% ============================================================================

\section{Treatment Recommendations}

\textit{Recommendations from \texttt{medical-advisor} agent will be added here
as they are generated. Each recommendation follows a structured format with
evidence base, protocol, monitoring plan, and physician discussion points.}

\subsection*{Placeholder for Future Recommendations}

When you request treatment recommendations from the \texttt{medical-advisor}
agent, they will appear in this section with the following structure:

\begin{enumerate}
\item Problem statement (current symptom pattern)
\item Evidence base (cited research with quality ratings)
\item Specific recommendations with protocols
\item Monitoring plan
\item Risks and contraindications
\item Questions for physician
\item Status indicator
\end{enumerate}

Example structure:
\begin{verbatim}
\section{Recommendation: [Title]}
\label{rec:[short-name]-[date]}

\subsection{Problem Statement}
[Current symptom/pattern from case data]

\subsection{Evidence Base}
[Citations with achievement/hypothesis/warning environments]

\subsection{Recommendations}
[Numbered list with detailed protocols]

\subsection{Monitoring Plan}
[What to track and how]

\subsection{Red Flags}
[When to seek urgent care]

\subsection{Status}
⚠️ PRELIMINARY - Requires physician approval
\end{verbatim}

%% ============================================================================
%% Treatment Effectiveness Analyses
%% ============================================================================

\section{Treatment Effectiveness Analyses}

\textit{Statistical analyses from \texttt{treatment-analyst} agent will be added
here after completing treatment trials. Each analysis includes effect sizes,
statistical significance, and continue/modify/discontinue recommendations.}

\subsection*{Placeholder for Future Analyses}

When you complete a treatment trial and request analysis from the
\texttt{treatment-analyst} agent, results will appear here with:

\begin{itemize}
\item Baseline vs treatment period comparison
\item Effect sizes (Cohen's d) with interpretations
\item Statistical significance (p-values)
\item Time series visualizations
\item Comparative ranking of all treatments tried
\item Responder profile analysis
\item Evidence-based recommendations for continuing or stopping
\end{itemize}

%% ============================================================================
%% Hypothesis and Subtype Analysis
%% ============================================================================

\section{Subtype and Mechanistic Hypotheses}

\textit{Analyses from \texttt{hypothesis-generator} agent will be added here.
These provide theoretical frameworks for understanding the patient's specific
ME/CFS presentation and guiding diagnostic and treatment strategies.}

\subsection*{Placeholder for Future Hypotheses}

When you request subtype analysis from the \texttt{hypothesis-generator} agent,
it will appear here with:

\begin{itemize}
\item Symptom pattern analysis
\item Proposed subtype classification
\item Mechanistic hypotheses
\item Testable predictions
\item Recommended diagnostic tests
\item Treatment response predictions
\item Confidence assessment
\end{itemize}

%% ============================================================================
%% Crisis and Recovery Protocols
%% ============================================================================

\section{Crisis Management Protocols}

\textit{Emergency protocols from \texttt{crisis-manager} agent will be added
here as needed during severe symptom exacerbations.}

\subsection*{Placeholder for Crisis Documentation}

When severe crashes occur and you invoke the \texttt{crisis-manager} agent,
documentation will appear here including:

\begin{itemize}
\item Crash overview and severity assessment
\item Immediate management protocol
\item Recovery tracking data
\item Lessons learned and prevention strategies
\item Emergency department documentation (if needed)
\end{itemize}

%% ============================================================================
%% Research Updates Relevant to Case
%% ============================================================================

\section{Research Updates}

\textit{Monthly summaries from \texttt{research-monitor} agent highlighting new
research findings relevant to this patient's case.}

\subsection*{Placeholder for Research Summaries}

The \texttt{research-monitor} agent will periodically add summaries here
covering:

\begin{itemize}
\item Breakthrough findings in ME/CFS research
\item New biomarker studies relevant to patient's symptom profile
\item Treatment trials and results
\item Clinical trials patient may be eligible for
\item Updates to evidence base for current treatments
\end{itemize}

%% ============================================================================
%% Notes for Future Sections
%% ============================================================================

\section*{Usage Notes}

\begin{enumerate}
\item \textbf{Regular updates:} This appendix grows as medical agents generate
new analyses and recommendations.

\item \textbf{Version control:} Each section is dated and labeled, allowing
tracking of how recommendations evolve over time.

\item \textbf{Cross-references:} Recommendations reference case data from
Appendix I and cite literature from the main bibliography.

\item \textbf{Physician collaboration:} Share relevant sections with healthcare
providers for informed decision-making.

\item \textbf{Treatment timeline:} This appendix creates a chronological record
of treatment decisions, rationales, and outcomes.
\end{enumerate}

\section*{Getting Started}

To populate this appendix, use the medical agent system:

\begin{verbatim}
# For treatment recommendations:
"medical-advisor: review my symptoms and suggest treatment priorities"

# For treatment analysis (after 8+ weeks trial):
"treatment-analyst: analyze my [treatment name] trial"

# For subtype understanding:
"hypothesis-generator: analyze my case and propose my ME/CFS subtype"

# For crash management:
"crisis-manager: I'm having a severe crash, generate management protocol"

# For research updates:
"research-monitor: generate monthly summary of relevant new research"
\end{verbatim}

See \texttt{.claude/systems/medical-agent-system.md} for complete documentation
of the medical agent system.
