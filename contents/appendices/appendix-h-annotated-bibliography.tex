\chapter{Annotated Bibliography of ME/CFS Literature}
\label{app:annotated-bibliography}

This appendix provides a comprehensive annotated bibliography of scientific literature on Myalgic Encephalomyelitis/Chronic Fatigue Syndrome (ME/CFS). Sources are organized by research domain and include full identifying information where available.

% =============================================================================
\section{Primary Research: NIH Deep Phenotyping Study}
\label{sec:bib-nih-deep-phenotyping}
% =============================================================================

\subsection{Walitt et al.\ 2024 --- The Foundational Deep Phenotyping Study}

\begin{description}
    \item[Full Citation:] Walitt B, Singh K, LaMunion SR, et al.\ Deep phenotyping of post-infectious myalgic encephalomyelitis/chronic fatigue syndrome. \textit{Nature Communications}. 2024;15(1):907.
    \item[DOI:] \href{https://doi.org/10.1038/s41467-024-45107-3}{10.1038/s41467-024-45107-3}
    \item[PMID:] 38383456
    \item[PMCID:] PMC10881493
    \item[Published:] February 21, 2024
    \item[Study Design:] Cross-sectional deep phenotyping study
    \item[Sample Size:] 17 PI-ME/CFS patients, 21 matched healthy controls
    \item[Key Findings:]
    \begin{itemize}
        \item Altered effort preference rather than physical or central fatigue (OR=1.65, $p$=0.04)
        \item Decreased brain activation in right temporal-parietal junction during motor tasks
        \item CSF catechol abnormalities: decreased DOPA ($p$=0.021), DOPAC ($p$=0.025), DHPG ($p$=0.006)
        \item Reduced peak VO$_2$ on cardiopulmonary exercise testing ($p$=0.004)
        \item Chronotropic incompetence (5/8 PI-ME/CFS vs 1/9 controls, $p$=0.03)
        \item B-cell abnormalities: increased na\"ive B-cells ($p$=0.037), decreased switched memory B-cells ($p$=0.008)
        \item Sex-specific gene expression differences with $<$5\% overlap between sexes
    \end{itemize}
    \item[Conclusion:] ME/CFS appears to be a centrally mediated disorder characterized by altered effort preference, potentially associated with central catecholamine dysregulation.
    \item[Limitations:] Small sample size (80\% power only detects effects $\geq$0.94 SD); cross-sectional design; recruitment terminated due to COVID-19 pandemic.
\end{description}

% =============================================================================
\section{Diagnostic Criteria and Clinical Guidelines}
\label{sec:bib-diagnostic-criteria}
% =============================================================================

\subsection{Institute of Medicine (IOM) 2015 Criteria}

\begin{description}
    \item[Full Citation:] Institute of Medicine (US) Committee on the Diagnostic Criteria for Myalgic Encephalomyelitis/Chronic Fatigue Syndrome. Beyond Myalgic Encephalomyelitis/Chronic Fatigue Syndrome: Redefining an Illness. Washington, DC: National Academies Press; 2015.
    \item[URL:] \url{https://www.cdc.gov/me-cfs/hcp/diagnosis/iom-2015-diagnostic-criteria-1.html}
    \item[ISBN:] 978-0-309-31689-7
    \item[Key Requirements:]
    \begin{itemize}
        \item Three required symptoms: (1) substantial reduction in functioning with fatigue $\geq$6 months, (2) post-exertional malaise, (3) unrefreshing sleep
        \item Plus at least one of: cognitive impairment OR orthostatic intolerance
        \item Symptoms must be present $\geq$50\% of the time with moderate-to-severe intensity
    \end{itemize}
    \item[Significance:] Proposed renaming to Systemic Exertion Intolerance Disease (SEID); currently used by CDC.
\end{description}

\subsection{NICE 2021 Guidelines (NG206)}

\begin{description}
    \item[Full Citation:] National Institute for Health and Care Excellence. Myalgic encephalomyelitis (or encephalopathy)/chronic fatigue syndrome: diagnosis and management. NICE guideline [NG206]. London: NICE; 2021.
    \item[URL:] \url{https://www.nice.org.uk/guidance/ng206}
    \item[Published:] October 29, 2021
    \item[Key Changes from 2007 Guideline:]
    \begin{itemize}
        \item All four core symptoms required: debilitating fatiguability, PEM, unrefreshing sleep, cognitive difficulties
        \item Symptoms must persist $\geq$3 months (suspected from 6 weeks in adults, 4 weeks in children)
        \item Graded Exercise Therapy (GET) \textbf{no longer recommended}
        \item CBT not considered a treatment for ME/CFS itself
        \item Recognition of PEM as the cardinal symptom
    \end{itemize}
    \item[Adoption:] Endorsed in Northern Ireland (2022); default guidance in Scotland (2025).
\end{description}

\subsection{Canadian Consensus Criteria (2003)}

\begin{description}
    \item[Full Citation:] Carruthers BM, Jain AK, De Meirleir KL, et al.\ Myalgic Encephalomyelitis/Chronic Fatigue Syndrome: Clinical Working Case Definition, Diagnostic and Treatment Protocols. \textit{Journal of Chronic Fatigue Syndrome}. 2003;11(1):7--115.
    \item[DOI:] \href{https://doi.org/10.1300/J092v11n01_02}{10.1300/J092v11n01\_02}
    \item[Significance:] First criteria to require PEM; more restrictive than Fukuda 1994; widely used in research.
\end{description}

\subsection{Fukuda et al.\ 1994 (CDC Criteria)}

\begin{description}
    \item[Full Citation:] Fukuda K, Straus SE, Hickie I, Sharpe MC, Dobbins JG, Komaroff A. The chronic fatigue syndrome: a comprehensive approach to its definition and study. \textit{Annals of Internal Medicine}. 1994;121(12):953--959.
    \item[DOI:] \href{https://doi.org/10.7326/0003-4819-121-12-199412150-00009}{10.7326/0003-4819-121-12-199412150-00009}
    \item[PMID:] 7978722
    \item[Significance:] Most widely used research criteria historically; criticized for being too broad.
\end{description}

% =============================================================================
\section{Systematic Reviews and Meta-Analyses}
\label{sec:bib-systematic-reviews}
% =============================================================================

\subsection{Prevalence and Epidemiology}

\begin{description}
    \item[Full Citation:] Lim E-J, Ahn Y-C, Jang E-S, Lee S-W, Lee S-H, Son C-G. Systematic review and meta-analysis of the prevalence of chronic fatigue syndrome/myalgic encephalomyelitis (CFS/ME). \textit{Journal of Translational Medicine}. 2020;18(1):100.
    \item[DOI:] \href{https://doi.org/10.1186/s12967-020-02269-0}{10.1186/s12967-020-02269-0}
    \item[PMID:] 32093722
    \item[PMCID:] PMC7038594
    \item[Key Findings:] Pooled prevalence 0.89\% (95\% CI: 0.60--1.33\%); women 1.36\% vs men 0.86\%; children/adolescents 0.55\%.
\end{description}

\begin{description}
    \item[Full Citation:] Centers for Disease Control and Prevention. Myalgic Encephalomyelitis/Chronic Fatigue Syndrome in Adults: United States, 2021--2022. NCHS Data Brief No.\ 488. Hyattsville, MD: National Center for Health Statistics; 2023.
    \item[URL:] \url{https://www.cdc.gov/nchs/products/databriefs/db488.htm}
    \item[Key Findings:] 1.3\% of US adults have ME/CFS; prevalence increases with age through 60--69 years; 84--91\% remain undiagnosed.
\end{description}

\subsection{Cognitive Impairment}

\begin{description}
    \item[Full Citation:] Sebaiti MA, Hainselin M, Gounden Y, et al.\ Systematic review and meta-analysis of cognitive impairment in myalgic encephalomyelitis/chronic fatigue syndrome (ME/CFS). \textit{Scientific Reports}. 2022;12(1):2157.
    \item[DOI:] \href{https://doi.org/10.1038/s41598-021-04764-w}{10.1038/s41598-021-04764-w}
    \item[PMID:] 35145174
    \item[Key Findings:] Large effect size for verbal working memory deficits; no significant difference in visual working memory.
\end{description}

\subsection{Long COVID and ME/CFS Overlap}

\begin{description}
    \item[Full Citation:] Wong TL, Weitzer DJ. Long COVID and Myalgic Encephalomyelitis/Chronic Fatigue Syndrome (ME/CFS)---A Systematic Review and Comparison of Clinical Presentation and Symptomatology. \textit{Medicina}. 2021;57(5):418.
    \item[DOI:] \href{https://doi.org/10.3390/medicina57050418}{10.3390/medicina57050418}
    \item[PMCID:] PMC8145228
\end{description}

\begin{description}
    \item[Full Citation:] The persistence of myalgic encephalomyelitis/chronic fatigue syndrome (ME/CFS) after SARS-CoV-2 infection: A systematic review and meta-analysis. \textit{Journal of Infection}. 2024;89(4):101231.
    \item[DOI:] \href{https://doi.org/10.1016/j.jinf.2024.106231}{10.1016/j.jinf.2024.106231}
    \item[PMID:] 39353473
    \item[Key Findings:] Approximately half of Long COVID patients fulfill ME/CFS diagnostic criteria.
\end{description}

\subsection{Sleep Abnormalities}

\begin{description}
    \item[Full Citation:] Baig S, Engelbrecht K, Engelbrecht F, et al.\ Objective sleep measures in chronic fatigue syndrome patients: A systematic review and meta-analysis. \textit{Sleep Medicine Reviews}. 2023;69:101775.
    \item[DOI:] \href{https://doi.org/10.1016/j.smrv.2023.101775}{10.1016/j.smrv.2023.101775}
    \item[PMID:] 37116254
    \item[PMCID:] PMC10281648
    \item[Sample:] 24 studies; 801 adults (426 ME/CFS, 375 controls); 477 adolescents
    \item[Key Findings:] Longer sleep latency, reduced sleep efficiency, longer REM latency, altered sleep microstructure.
\end{description}

\begin{description}
    \item[Full Citation:] Maksoud R, du Preez S, Eaton-Fitch N, et al.\ Systematic Review of Sleep Characteristics in Myalgic Encephalomyelitis/Chronic Fatigue Syndrome. \textit{Healthcare}. 2021;9(5):568.
    \item[DOI:] \href{https://doi.org/10.3390/healthcare9050568}{10.3390/healthcare9050568}
    \item[PMCID:] PMC8150292
\end{description}

\subsection{Evidence Mapping}

\begin{description}
    \item[Full Citation:] Toogood PL, Clauw DJ, Engel CC, et al.\ Recent research in myalgic encephalomyelitis/chronic fatigue syndrome: an evidence map. \textit{BMC Medicine}. 2025;23(1):134.
    \item[PMCID:] PMC11973615
    \item[Scope:] Mapping ME/CFS evidence from 2018--2023.
\end{description}

% =============================================================================
\section{Pathophysiology: Immune Dysfunction}
\label{sec:bib-immune-dysfunction}
% =============================================================================

\subsection{Autoantibodies and G-Protein Coupled Receptors}

\begin{description}
    \item[Full Citation:] Wirth K, Scheibenbogen C. Autoantibodies to Vasoregulative G-Protein-Coupled Receptors Correlate with Symptom Severity, Autonomic Dysfunction and Disability in Myalgic Encephalomyelitis/Chronic Fatigue Syndrome. \textit{Journal of Clinical Medicine}. 2021;10(16):3675.
    \item[DOI:] \href{https://doi.org/10.3390/jcm10163675}{10.3390/jcm10163675}
    \item[PMID:] 34441971
    \item[PMCID:] PMC8397061
    \item[Key Findings:] Anti-$\beta$2, M3, M4 receptor antibodies elevated; correlate with fatigue and muscle pain severity.
\end{description}

\begin{description}
    \item[Full Citation:] M\"uller JA, Subburayalu J, Winkler F, et al.\ Dysregulated autoantibodies targeting vaso- and immunoregulatory receptors in Post COVID Syndrome correlate with symptom severity. \textit{Frontiers in Immunology}. 2022;13:981532.
    \item[DOI:] \href{https://doi.org/10.3389/fimmu.2022.981532}{10.3389/fimmu.2022.981532}
\end{description}

\begin{description}
    \item[Full Citation:] Stein E, Heindrich C, Wittke K, et al.\ Efficacy of repeated immunoadsorption in patients with post-COVID myalgic encephalomyelitis/chronic fatigue syndrome and elevated $\beta$2-adrenergic receptor autoantibodies: a prospective cohort study. \textit{The Lancet Regional Health -- Europe}. 2024;46:101330.
    \item[DOI:] \href{https://doi.org/10.1016/j.lanepe.2024.101330}{10.1016/j.lanepe.2024.101330}
    \item[Significance:] Demonstrates therapeutic potential of immunoadsorption targeting autoantibodies.
\end{description}

\subsection{TRPM3 Ion Channel Dysfunction}

\begin{description}
    \item[Full Citation:] Sasso E, Smith P, Marshall-Gradisnik S, et al.\ Multi-site validation of TRPM3 ion channel dysfunction in Myalgic Encephalomyelitis/Chronic Fatigue Syndrome. \textit{Frontiers in Medicine}. 2026.
    \item[DOI:] \href{https://doi.org/10.3389/fmed.2025.1703924}{10.3389/fmed.2025.1703924}
    \item[Published:] January 13, 2026
    \item[Institution:] Griffith University, National Centre for Neuroimmunology and Emerging Diseases (NCNED)
    \item[Study Design:] Multi-site validation study using gold-standard techniques
    \item[Key Findings:]
    \begin{itemize}
        \item TRPM3 ion channel (a calcium-permeable channel in immune cells) functions abnormally in ME/CFS patients compared to healthy controls
        \item Defect reproducibly observed across independent laboratories over 4,000 km apart (Gold Coast and Perth, Australia)
        \item Faulty ion channels act like ``stuck doors,'' preventing cells from receiving calcium needed for immune function
    \end{itemize}
    \item[Significance:] Provides robust, multi-site validated evidence of measurable cellular abnormalities in ME/CFS. Supports development of diagnostic tests and identifies potential therapeutic targets. Offers greater recognition of ME/CFS as a legitimate medical condition with objective biological markers.
    \item[Lead Researchers:] Professor Sonya Marshall-Gradisnik (Director), Dr.\ Etianne Sasso (Lead Author), Dr.\ Peter Smith
\end{description}

\subsection{Immune Exhaustion and Chronic Activation}

\begin{description}
    \item[Full Citation:] Immune exhaustion in ME/CFS and long COVID. \textit{JCI Insight}. 2024;9(19):e183810.
    \item[DOI:] \href{https://doi.org/10.1172/jci.insight.183810}{10.1172/jci.insight.183810}
\end{description}

\subsection{Cytokine Biomarkers and Immune Signatures}

\paragraph{Hornig et al.\ 2015 --- Duration-Dependent Cytokine Signatures}

\cite{Hornig2015}

\paragraph{Key Findings:}
This landmark study in \emph{Science Advances} identified distinct immune signatures in ME/CFS that vary by disease duration. Among 298 ME/CFS patients and 348 controls, early-stage patients (<3 years) showed prominent activation of both pro- and anti-inflammatory cytokines, including elevated IL-1$\alpha$, IL-8, IL-10, IL-12p40, IL-17F, IFN-$\gamma$, CXCL1, CXCL9, and IL-5. In stark contrast, patients with longer disease duration (>3 years) had normalized cytokine levels. A 17-cytokine panel distinguished early ME/CFS from controls with high accuracy.

\paragraph{Relevance:}
Provides the strongest evidence to date that ME/CFS immunopathology evolves over time, potentially from initial immune activation to exhaustion or adaptation. This duration-dependent pattern explains heterogeneity in previous cytokine studies that failed to stratify by illness duration and suggests therapeutic windows where early intervention may be more effective.

\paragraph{Certainty Assessment:}
\begin{itemize}
    \item \textbf{Quality:} High (published in \emph{Science Advances}, large sample size, rigorous methodology)
    \item \textbf{Sample:} n=646 total (298 ME/CFS, 348 controls)
    \item \textbf{Replication:} Partially replicated in Montoya 2017 and Che 2025
    \item \textbf{Limitations:} Cross-sectional design cannot track individual progression; mechanism of cytokine normalization unclear
\end{itemize}

\paragraph{Montoya et al.\ 2017 --- Cytokine-Severity Correlation}

\cite{Montoya2017}

\paragraph{Key Findings:}
This \emph{PNAS} study demonstrated dose-response relationships between cytokines and symptom severity. Although only two cytokines differed overall between patients and controls (TGF-$\beta$ higher, resistin lower), 17 cytokines showed significant upward linear trends correlating with disease severity. Thirteen of these 17 are proinflammatory: CCL11, CXCL1, CXCL10, IFN-$\gamma$, IL-4, IL-5, IL-7, IL-12p70, IL-13, IL-17F, G-CSF, GM-CSF, and TGF-$\alpha$. CXCL9 inversely correlated with fatigue duration, consistent with Hornig 2015's duration-dependent findings.

\paragraph{Relevance:}
Provides evidence that immune activation tracks with symptom burden. The dose-response relationship (rather than binary patient-control comparison) suggests cytokine profiling could stratify patients for clinical trials and identify those likely to benefit from immunomodulatory therapies. Complements Hornig 2015 by focusing on severity rather than duration.

\paragraph{Certainty Assessment:}
\begin{itemize}
    \item \textbf{Quality:} High (published in \emph{PNAS}, large sample, Stanford affiliation)
    \item \textbf{Sample:} n=584 (192 ME/CFS, 392 controls)
    \item \textbf{Replication:} Partially replicated in Che 2025
    \item \textbf{Limitations:} Cross-sectional; cannot determine causality; severity assessment subjective
\end{itemize}

\paragraph{Che et al.\ 2025 --- Sex-Specific Immune Dysregulation}

\cite{Che2025}

\paragraph{Key Findings:}
Multi-omics study from the Walitt/Lipkin group demonstrated exaggerated innate immune responses to microbial stimulation in ME/CFS, with IL-6 and other proinflammatory cytokines elevated before and markedly increased after exercise. Critically, hyperinflammatory responses were amplified in women over 45 years with diminished estradiol levels, suggesting sex hormone-immune interactions. The study also identified impaired energy production (TCA cycle dysfunction, fatty acid oxidation defects) that worsened post-exercise.

\paragraph{Relevance:}
Extends previous cytokine findings to reveal sex- and age-specific patterns. The estradiol-cytokine relationship provides mechanistic insight into female predominance of ME/CFS and suggests potential therapeutic interventions (estrogen supplementation for older women). Integrates immune and metabolic dysfunction, supporting multi-system pathophysiology model.

\paragraph{Certainty Assessment:}
\begin{itemize}
    \item \textbf{Quality:} High (Nature portfolio journal, rigorous multi-omics approach)
    \item \textbf{Sample:} Specific n not provided in abstract
    \item \textbf{Replication:} Confirms and extends Hornig/Montoya cytokine findings
    \item \textbf{Limitations:} Sex-hormone mechanism needs further validation
\end{itemize}

\paragraph{Giloteaux et al.\ 2023 --- IL-2 in Extracellular Vesicles}

\cite{Giloteaux2023}

\paragraph{Key Findings:}
Novel study examining cytokine content in extracellular vesicles (EVs) rather than free plasma. IL-2 was significantly elevated in ME/CFS patient EVs. Proinflammatory cytokines CSF2 and TNF$\alpha$ correlated with physical and fatigue symptom severity. EVs represent cell-to-cell signaling mechanism and may better reflect active immune communication.

\paragraph{Relevance:}
Identifies IL-2 as potentially important cytokine in ME/CFS pathophysiology. Notably, Hunter 2025 independently identified IL-2 pathway using epigenetic methods, providing convergent evidence from different methodologies. EV-based analysis may reveal immune signals missed by conventional plasma assays.

\paragraph{Certainty Assessment:}
\begin{itemize}
    \item \textbf{Quality:} Medium-High (novel methodology, peer-reviewed)
    \item \textbf{Sample:} n=98 (49 ME/CFS, 49 controls)
    \item \textbf{Replication:} IL-2 finding supported by Hunter 2025; EV method needs replication
    \item \textbf{Limitations:} Single study with novel methodology; EV assays less standardized than plasma
\end{itemize}

\paragraph{Hunter et al.\ 2025 --- Epigenetic Biomarkers and IL-2 Pathway}

\cite{Hunter2025}

\paragraph{Key Findings:}
Developed blood-based diagnostic test using EpiSwitch\textregistered\ technology identifying 200 chromosome conformation markers that distinguish ME/CFS from controls with 92\% sensitivity and 98\% specificity. Pathway analysis revealed involvement of IL-2, TNF$\alpha$, toll-like receptor signaling, and JAK/STAT mechanisms. IL-2 identified as shared pathway with existing therapies (Rituximab, glatiramer acetate).

\paragraph{Relevance:}
Provides epigenetic validation of immune pathways identified by cytokine studies. High diagnostic specificity (98\%) suggests potential clinical utility. IL-2 pathway finding converges with Giloteaux 2023, supporting IL-2 as therapeutic target. Study focused on severely affected patients.

\paragraph{Certainty Assessment:}
\begin{itemize}
    \item \textbf{Quality:} Medium-High (peer-reviewed, high diagnostic accuracy)
    \item \textbf{Sample:} n=108 (47 ME/CFS, 61 controls)
    \item \textbf{Replication:} Single study; proprietary technology limits independent validation
    \item \textbf{Limitations:} Severe patients only; EpiSwitch technology not widely available; needs independent cohort validation
\end{itemize}

\subsection{Comprehensive Immune Reviews}

\begin{description}
    \item[Full Citation:] Komaroff AL, Lipkin WI. ME/CFS and Long COVID share similar symptoms and biological abnormalities: road map to the literature. \textit{Frontiers in Medicine}. 2023;10:1187163.
    \item[DOI:] \href{https://doi.org/10.3389/fmed.2023.1187163}{10.3389/fmed.2023.1187163}
    \item[PMCID:] PMC10278546
    \item[Significance:] Comprehensive comparison of ME/CFS and Long COVID biological abnormalities.
\end{description}

\begin{description}
    \item[Full Citation:] Komaroff AL, Lipkin WI. Myalgic Encephalomyelitis/Chronic Fatigue Syndrome: the biology of a neglected disease. \textit{Frontiers in Immunology}. 2024;15:1386607.
    \item[DOI:] \href{https://doi.org/10.3389/fimmu.2024.1386607}{10.3389/fimmu.2024.1386607}
    \item[PMCID:] PMC11180809
\end{description}

% =============================================================================
\section{Pathophysiology: Neurological Abnormalities}
\label{sec:bib-neurological}
% =============================================================================

\subsection{Neuroinflammation}

\begin{description}
    \item[Full Citation:] Nakatomi Y, Mizuno K, Ishii A, et al.\ Neuroinflammation in Patients with Chronic Fatigue Syndrome/Myalgic Encephalomyelitis: An $^{11}$C-(R)-PK11195-PET Study. \textit{Journal of Nuclear Medicine}. 2014;55(6):945--950.
    \item[DOI:] \href{https://doi.org/10.2967/jnumed.113.131045}{10.2967/jnumed.113.131045}
    \item[PMID:] 24665088
    \item[Key Findings:] PET imaging demonstrates widespread neuroinflammation correlating with symptom severity.
\end{description}

\begin{description}
    \item[Full Citation:] Renz-Polster H, Tremblay M-E, Engel D, Scheibenbogen C, Brehm JU. Molecular Mechanisms of Neuroinflammation in ME/CFS and Long COVID to Sustain Disease and Promote Relapses. \textit{Frontiers in Neurology}. 2022;13:877772.
    \item[DOI:] \href{https://doi.org/10.3389/fneur.2022.877772}{10.3389/fneur.2022.877772}
\end{description}

\subsection{Neuroimaging Reviews}

\begin{description}
    \item[Full Citation:] Shan ZY, Barnden LR, Kwiatek RA, Bhuta S, Groszmann M, Blumbergs PC. Neuroimaging characteristics of myalgic encephalomyelitis/chronic fatigue syndrome (ME/CFS): a systematic review. \textit{Journal of Translational Medicine}. 2020;18(1):335.
    \item[DOI:] \href{https://doi.org/10.1186/s12967-020-02506-6}{10.1186/s12967-020-02506-6}
    \item[Key Findings:] Evidence for structural, functional, and metabolic brain abnormalities; hypoperfusion in multiple regions.
\end{description}

\begin{description}
    \item[Full Citation:] Metabolic neuroimaging of myalgic encephalomyelitis/chronic fatigue syndrome and Long-COVID. \textit{Immunometabolism}. 2025;10:e00068.
    \item[DOI:] \href{https://doi.org/10.1097/IN9.0000000000000068}{10.1097/IN9.0000000000000068}
\end{description}

\subsection{Brainstem and Autonomic Dysfunction}

\begin{description}
    \item[Full Citation:] van Campen CLMC, Rowe PC, Visser FC. Similar Patterns of Dysautonomia in Myalgic Encephalomyelitis/Chronic Fatigue and Post-COVID-19 Syndromes. \textit{Pathophysiology}. 2024;31(1):1--17.
    \item[DOI:] \href{https://doi.org/10.3390/pathophysiology31010001}{10.3390/pathophysiology31010001}
    \item[PMCID:] PMC10801610
\end{description}

\begin{description}
    \item[Full Citation:] Wells R, Spurrier AJ, Linz D, et al.\ Is postural orthostatic tachycardia syndrome (POTS) a central nervous system disorder? \textit{Journal of Neurology, Neurosurgery \& Psychiatry}. 2021;92(11):1196--1207.
    \item[DOI:] \href{https://doi.org/10.1136/jnnp-2020-325932}{10.1136/jnnp-2020-325932}
    \item[PMCID:] PMC7936931
\end{description}

\begin{description}
    \item[Full Citation:] Dysautonomia and small fiber neuropathy in post-COVID condition and Chronic Fatigue Syndrome. \textit{Journal of Neurology}. 2024;271(1):40--48.
    \item[PMCID:] PMC10648633
\end{description}

% =============================================================================
\section{Pathophysiology: Metabolic and Mitochondrial Dysfunction}
\label{sec:bib-metabolic}
% =============================================================================

\subsection{Mitochondrial Dysfunction}

\begin{description}
    \item[Full Citation:] Holden S, Maksoud R, Eaton-Fitch N, et al.\ Mitochondrial Dysfunction in Myalgic Encephalomyelitis/Chronic Fatigue Syndrome. \textit{Physiology}. 2025;40(2):89--102.
    \item[DOI:] \href{https://doi.org/10.1152/physiol.00056.2024}{10.1152/physiol.00056.2024}
    \item[PMCID:] PMC12151296
    \item[Key Topics:] Impaired oxidative phosphorylation, reduced ATP production, WASF3 dysregulation.
\end{description}

\paragraph{Wang et al.\ 2023 --- WASF3 Disrupts Mitochondrial Respiration}

\cite{wang2023wasf3}

\paragraph{Key Findings:}
This \emph{PNAS} study identifies a specific molecular mechanism for mitochondrial dysfunction in ME/CFS. ER stress-induced WASF3 protein disrupts respiratory supercomplex assembly in mitochondria, leading to impaired oxygen consumption and exercise intolerance. Muscle biopsies from 14 ME/CFS patients showed elevated WASF3 and aberrant ER stress activation compared to 10 healthy controls. Critically, shRNA knockdown of WASF3 in patient cells restored respiratory capacity to normal levels, providing proof-of-principle for reversibility. Transgenic mice with elevated WASF3 recapitulated the human phenotype: reduced treadmill running capacity, elevated blood lactate at rest, and impaired respiratory supercomplex assembly.

\paragraph{Relevance:}
Establishes a mechanistic link from viral triggers (ER stress) through WASF3 to mitochondrial dysfunction and exercise intolerance. Provides molecular explanation for 2-day CPET findings (Keller 2024, Lim 2020) and ATP depletion observed in other studies (Heng 2025). Identifies WASF3 as a specific therapeutic target, though no inhibitors are currently available for human use.

\paragraph{Certainty Assessment:}
\begin{itemize}
    \item \textbf{Quality:} High (published in \emph{PNAS}, rigorous methodology, multi-level validation)
    \item \textbf{Sample:} n=14 ME/CFS patients, n=10 controls (small but adequate for mechanistic study)
    \item \textbf{Replication:} Pending (published 2023, too recent for independent validation)
    \item \textbf{Limitations:} Unknown whether WASF3 elevation applies to all ME/CFS patients or specific subgroup; therapeutic compounds not yet developed
\end{itemize}

\begin{description}
    \item[Full Citation:] Morris G, Maes M. Mitochondrial dysfunctions in myalgic encephalomyelitis/chronic fatigue syndrome explained by activated immuno-inflammatory, oxidative and nitrosative stress pathways. \textit{Metabolic Brain Disease}. 2014;29(1):19--36.
    \item[DOI:] \href{https://doi.org/10.1007/s11011-013-9435-x}{10.1007/s11011-013-9435-x}
    \item[PMID:] 24557875
\end{description}

\begin{description}
    \item[Full Citation:] Myhill S, Booth NE, McLaren-Howard J. Chronic fatigue syndrome and mitochondrial dysfunction. \textit{International Journal of Clinical and Experimental Medicine}. 2009;2(1):1--16.
    \item[PMCID:] PMC2680051
\end{description}

\subsection{Metabolomics and Metabolic Traps}

\paragraph{Phair et al.\ 2019 --- The IDO Metabolic Trap Hypothesis}

\cite{phair2019ido}

\paragraph{Key Findings:}
Mathematical model proposing bistability in tryptophan metabolism as an etiological mechanism for ME/CFS. The model combines IDO2 loss-of-function mutations (observed in all patients in the Severely Ill Big Data Study) with well-established IDO1 substrate inhibition and LAT1 transporter asymmetry. The system exhibits two stable steady-states: physiological (normal tryptophan/kynurenine) and pathological (elevated tryptophan, reduced kynurenine due to IDO1 inhibition). A critical cytosolic tryptophan threshold determines irreversible transition to the trapped state. Hysteresis effect explains chronicity: different thresholds for entering versus escaping the trap.

\paragraph{Relevance:}
Provides theoretical framework for understanding chronic ME/CFS and suggests testable therapeutic interventions (reducing cytosolic tryptophan below critical threshold). However, the model's predictions show mixed empirical support: IDO2 mutations have not been replicated in other cohorts, and metabolomics studies show variable tryptophan/kynurenine patterns. \textbf{Critical contradiction:} Guo et al.\ 2023 found \emph{opposite} mechanism in long COVID (IDO2 \emph{gain}-of-function with low tryptophan, high kynurenine), suggesting different mechanisms may operate in different patient subgroups or diseases.

\paragraph{Certainty Assessment:}
\begin{itemize}
    \item \textbf{Quality:} High (rigorous mathematical modeling)
    \item \textbf{Empirical Support:} Low-Moderate (genetic findings not replicated; metabolomics inconsistent)
    \item \textbf{Validation:} Pending (therapeutic predictions untested; contradicted by Guo 2023)
    \item \textbf{Limitations:} Theoretical model with limited validation; IDO2 mutation ubiquity not confirmed in independent cohorts; use as speculative hypothesis for subset of patients
\end{itemize}

\begin{description}
    \item[Full Citation:] Baraniuk JN, Kern G, Engel S, Engel G. Cerebrospinal fluid metabolomics, lipidomics and serine pathway dysfunction in myalgic encephalomyelitis/chronic fatigue syndrome (ME/CFS). \textit{Scientific Reports}. 2025;15(1):6789.
    \item[DOI:] \href{https://doi.org/10.1038/s41598-025-91324-1}{10.1038/s41598-025-91324-1}
    \item[PMCID:] PMC11873053
    \item[Key Findings:] Elevated serine, reduced 5-MTHF in CSF; altered phospholipid synthesis.
\end{description}

\begin{description}
    \item[Full Citation:] Naviaux RK, Naviaux JC, Li K, et al.\ Metabolic features of chronic fatigue syndrome. \textit{Proceedings of the National Academy of Sciences}. 2016;113(37):E5472--E5480.
    \item[DOI:] \href{https://doi.org/10.1073/pnas.1607571113}{10.1073/pnas.1607571113}
    \item[Key Findings:] Chemical signature with approximately 40 metabolic abnormalities; hypometabolic state.
\end{description}

\begin{description}
    \item[Full Citation:] Germain A, Barupal DK, Levine SM. Comprehensive Circulatory Metabolomics in ME/CFS Reveals Disrupted Metabolism of Acyl Lipids and Steroids. \textit{Metabolites}. 2020;10(1):34.
    \item[DOI:] \href{https://doi.org/10.3390/metabo10010034}{10.3390/metabo10010034}
    \item[PMID:] 31947545
    \item[Key Findings:] Acyl cholines consistently reduced across cohorts.
\end{description}

% =============================================================================
\section{Pathophysiology: Gut Microbiome}
\label{sec:bib-microbiome}
% =============================================================================

\begin{description}
    \item[Full Citation:] Lupo GFD, Rocchetti G, Lucini L, et al.\ Potential role of microbiome in Chronic Fatigue Syndrome/Myalgic Encephalomyelitis (CFS/ME). \textit{Scientific Reports}. 2021;11(1):7043.
    \item[DOI:] \href{https://doi.org/10.1038/s41598-021-86425-6}{10.1038/s41598-021-86425-6}
\end{description}

\begin{description}
    \item[Full Citation:] Giloteaux L, Goodrich JK, Walters WA, Levine SM, Ley RE, Hanson MR. Reduced diversity and altered composition of the gut microbiome in individuals with myalgic encephalomyelitis/chronic fatigue syndrome. \textit{Microbiome}. 2016;4(1):30.
    \item[DOI:] \href{https://doi.org/10.1186/s40168-016-0171-4}{10.1186/s40168-016-0171-4}
    \item[Key Findings:] Reduced \textit{Faecalibacterium prausnitzii} and \textit{Eubacterium rectale} (butyrate producers).
\end{description}

\begin{description}
    \item[Full Citation:] K\"onig RS, Albrich WC, Kahlert CR, et al.\ The Gut Microbiome in Myalgic Encephalomyelitis (ME)/Chronic Fatigue Syndrome (CFS). \textit{Frontiers in Immunology}. 2022;12:628741.
    \item[DOI:] \href{https://doi.org/10.3389/fimmu.2021.628741}{10.3389/fimmu.2021.628741}
    \item[PMCID:] PMC8761622
\end{description}

\begin{description}
    \item[Full Citation:] Varesi A, Campagnoli LIM, Frasca A, et al.\ The gastrointestinal microbiota in the development of ME/CFS: a critical view and potential perspectives. \textit{Frontiers in Immunology}. 2024;15:1352744.
    \item[DOI:] \href{https://doi.org/10.3389/fimmu.2024.1352744}{10.3389/fimmu.2024.1352744}
\end{description}

\begin{description}
    \item[Full Citation:] Ciregia F, Rahmania F, Semenova-Ziga V, Ortega-Molina M, Rodrigues M, Gonzalez-Lopez E. Increased gut permeability and bacterial translocation are associated with fibromyalgia and myalgic encephalomyelitis/chronic fatigue syndrome: implications for disease-related biomarker discovery. \textit{Frontiers in Immunology}. 2023;14:1253121.
    \item[DOI:] \href{https://doi.org/10.3389/fimmu.2023.1253121}{10.3389/fimmu.2023.1253121}
    \item[Key Findings:] Elevated markers of gut permeability and bacterial translocation.
\end{description}

% =============================================================================
\section{Pathophysiology: Viral Persistence and Reactivation}
\label{sec:bib-viral}
% =============================================================================

\subsection{Enterovirus and Chronic Persistence}

\paragraph{Chia 2005 --- Enterovirus in Chronic Fatigue Syndrome}

\begin{description}
    \item[Full Citation:] Chia JKS. The role of enterovirus in chronic fatigue syndrome. \textit{Journal of Clinical Pathology}. 2005;58(11):1126--1132.
    \item[DOI:] \href{https://doi.org/10.1136/jcp.2004.020255}{10.1136/jcp.2004.020255}
    \item[PMID:] 16254097
    \item[PMCID:] PMC1770761
    \item[Type:] Review article
\end{description}

\paragraph{Key Findings:}
This comprehensive review article synthesizes evidence for chronic enteroviral infection as an etiologic factor in a subset of ME/CFS patients. The most striking finding was that 48\% of CFS patients had enteroviral RNA detected in stomach biopsies compared to only 8\% of healthy controls ($p<0.001$, n=165 CFS patients). Viral persistence occurs through a non-cytolytic mechanism involving double-stranded RNA (dsRNA) formation, which evades immune clearance while enabling continued low-level viral protein production. Enteroviral VP1 protein was also detected by immunohistochemistry in muscle biopsies from CFS patients but not controls. Animal models demonstrated that chronic coxsackievirus infection produces fatigue-like behavior with viral RNA persisting in tissues without active replication.

\paragraph{Relevance:}
Provides mechanistic explanation for post-viral ME/CFS onset, particularly in patients with GI symptoms and enteroviral exposure history. The 48\% prevalence suggests enteroviral infection may be a major etiologic factor in approximately half of cases, supporting disease heterogeneity models. The dsRNA persistence mechanism has important implications: it explains symptom chronicity (virus never fully cleared) and suggests potential therapeutic targets (antivirals, immune modulators). Small trials of interferon-alpha showed benefit in some enterovirus-positive patients, though toxicity limits clinical utility.

\paragraph{Certainty Assessment:}
\begin{itemize}
    \item \textbf{Quality:} Medium (review article synthesizing multiple studies; some primary studies well-designed, others smaller)
    \item \textbf{Sample:} Primary stomach biopsy study n=165 CFS (adequate); muscle studies smaller (n=10--30)
    \item \textbf{Replication:} Multiple independent groups detected enteroviral RNA/protein; some negative studies exist
    \item \textbf{Limitations:} RT-PCR can yield false positives; 8\% control positivity unclear (latent infection? contamination?); causation vs association not definitively proven; not all CFS patients affected (52\% negative); author potential bias (runs antiviral treatment clinic)
\end{itemize}

\paragraph{Modern Context:}
This 2005 work gains renewed relevance with Long COVID, which may involve similar viral persistence mechanisms (SARS-CoV-2 reservoirs). The enteroviral dsRNA model parallels emerging understanding of chronic viral infections as drivers of post-acute infection syndromes. Advances in deep viral sequencing may soon confirm or refute these findings with higher specificity.

\subsection{Viral Etiology Meta-Analysis}

\paragraph{Hwang et al.\ 2023 --- Systematic Review of Viral Associations}

\cite{hwang2023viral}

\paragraph{Key Findings:}
Comprehensive systematic review and meta-analysis of 64 studies with 4,971 ME/CFS patients and 9,221 controls, examining 18 viral species. Five viruses showed odds ratios $>$2.0 indicating moderate to strong associations: Borna disease virus (OR$\geq$3.47, strongest association), HHV-7 (OR$>$2.0), parvovirus B19 (OR$>$2.0), enterovirus (OR$>$2.0), and coxsackie B virus (OR$>$2.0). Notably, EBV and enterovirus showed high heterogeneity ($>$50\%) across studies, suggesting subgroup effects or methodological variability. BDV association strongest but controversial due to concerns about human pathogenicity and possible laboratory contamination.

\paragraph{Relevance:}
Provides quantitative meta-analytic evidence for viral associations in ME/CFS etiology. Multiple viral triggers implicated, suggesting diverse pathways to chronic illness rather than single causative agent. High heterogeneity for some viruses (EBV, enterovirus) explains inconsistent findings in individual studies and supports hypothesis of viral-onset subgroups within ME/CFS. Complements mechanistic viral papers (Ruiz-Pablos 2021 EBV, O'Neal 2021 enterovirus, Nunes 2024 herpesvirus endothelial hypothesis) with epidemiological quantification.

\paragraph{Certainty Assessment:}
\begin{itemize}
    \item \textbf{Quality:} High (systematic review, large sample across 64 studies)
    \item \textbf{Effect Size:} Moderate (OR 2.0--3.47, not extremely strong)
    \item \textbf{Causation:} Unclear (associations do not prove causation; could be trigger, consequence, or shared susceptibility)
    \item \textbf{Limitations:} High heterogeneity for key viruses; BDV findings require validation; methodological variability across included studies; publication bias possible
\end{itemize}

\subsection{Specific Viral Mechanisms}

\begin{description}
    \item[Full Citation:] Rasa S, Nora-Krukle Z, Henning N, et al.\ Chronic viral infections in myalgic encephalomyelitis/chronic fatigue syndrome (ME/CFS). \textit{Journal of Translational Medicine}. 2018;16(1):268.
    \item[DOI:] \href{https://doi.org/10.1186/s12967-018-1644-y}{10.1186/s12967-018-1644-y}
    \item[PMCID:] PMC6167797
    \item[Viruses Covered:] EBV, HHV-6, CMV, enteroviruses, B19V.
\end{description}

\begin{description}
    \item[Full Citation:] Williams MV, Cox B, Ariza ME. Chronic Reactivation of Persistent Human Herpesviruses EBV, HHV-6 and VZV and Heightened Anti-dUTPase IgG Antibodies Are a Recurrent Hallmark in Post-Infectious ME/CFS and is Associated With Fatigue. \textit{Frontiers in Immunology}. 2025;(in press).
    \item[PMID:] 41451845
    \item[Key Findings:] 72.5\% of ME/CFS patients have antibodies to multiple herpesvirus dUTPases vs 31\% controls.
\end{description}

\begin{description}
    \item[Full Citation:] Kasimir F, Toomey D, Liu Z, et al.\ Tissue specific signature of HHV-6 infection in ME/CFS. \textit{Frontiers in Molecular Biosciences}. 2022;9:1044964.
    \item[DOI:] \href{https://doi.org/10.3389/fmolb.2022.1044964}{10.3389/fmolb.2022.1044964}
    \item[PMCID:] PMC9795011
    \item[Key Findings:] Viral miRNA detected in brain and spinal cord tissue only in ME/CFS patients.
\end{description}

\begin{description}
    \item[Full Citation:] Ruiz-Pab\'on JF, Montoya JG, Lupo J, Epstein-Barr Virus and the Origin of Myalgic Encephalomyelitis or Chronic Fatigue Syndrome. \textit{Frontiers in Immunology}. 2021;12:656797.
    \item[DOI:] \href{https://doi.org/10.3389/fimmu.2021.656797}{10.3389/fimmu.2021.656797}
    \item[PMCID:] PMC8634673
\end{description}

\begin{description}
    \item[Full Citation:] Ruiz-Pab\'on JF, Henao E, Pinto F, Estrada S, Corredor V. Epstein--Barr virus-acquired immunodeficiency in myalgic encephalomyelitis---Is it present in long COVID? \textit{Journal of Translational Medicine}. 2023;21:633.
    \item[DOI:] \href{https://doi.org/10.1186/s12967-023-04515-7}{10.1186/s12967-023-04515-7}
\end{description}

% =============================================================================
\section{Pathophysiology: Genetics and Epigenetics}
\label{sec:bib-genetics}
% =============================================================================

\begin{description}
    \item[Full Citation:] de Vega WC, Vernon SD, McGowan PO. Identification of Myalgic Encephalomyelitis/Chronic Fatigue Syndrome-associated DNA methylation patterns. \textit{PLOS ONE}. 2018;13(7):e0201066.
    \item[DOI:] \href{https://doi.org/10.1371/journal.pone.0201066}{10.1371/journal.pone.0201066}
    \item[Key Findings:] 17,296 differentially methylated CpG sites; 307 differentially methylated promoters; immune-related pathways.
\end{description}

\begin{description}
    \item[Full Citation:] de Vega WC, Herber S, Ghaseminejad Tafreshi M, et al.\ Epigenetic modifications and glucocorticoid sensitivity in Myalgic Encephalomyelitis/Chronic Fatigue Syndrome (ME/CFS). \textit{BMC Medical Genomics}. 2017;10(1):11.
    \item[DOI:] \href{https://doi.org/10.1186/s12920-017-0248-3}{10.1186/s12920-017-0248-3}
\end{description}

\begin{description}
    \item[Full Citation:] Wang T, Yin J, Miller AH, Xiao C. Genetic risk factors for ME/CFS identified using combinatorial analysis. \textit{Journal of Translational Medicine}. 2022;20:598.
    \item[DOI:] \href{https://doi.org/10.1186/s12967-022-03815-8}{10.1186/s12967-022-03815-8}
    \item[Key Findings:] 199 SNPs in 14 genes associated with 91\% of ME/CFS cases.
\end{description}

\begin{description}
    \item[Full Citation:] Dissecting the genetic complexity of myalgic encephalomyelitis/chronic fatigue syndrome via deep learning-powered genome analysis. \textit{Nature Communications}. 2025.
    \item[PMCID:] PMC12047926
    \item[Key Findings:] 115 ME/CFS-risk genes identified; intolerance to loss-of-function mutations.
\end{description}

\begin{description}
    \item[Full Citation:] Trivedi MS, Oltra E, Engelbrecht B, et al.\ Recursive ensemble feature selection provides a robust mRNA expression signature for myalgic encephalomyelitis/chronic fatigue syndrome. \textit{Scientific Reports}. 2021;11(1):4541.
    \item[DOI:] \href{https://doi.org/10.1038/s41598-021-83660-9}{10.1038/s41598-021-83660-9}
\end{description}

% =============================================================================
\section{Biomarkers: Tetrahydrobiopterin (BH4) and Orthostatic Intolerance}
\label{sec:bib-bh4-biomarkers}
% =============================================================================

\subsection{BH4 Elevation in ME/CFS with Orthostatic Intolerance}

\paragraph{Gottschalk et al.\ 2023 --- BH4 Detection in ME/CFS + OI}

\begin{description}
    \item[Full Citation:] Gottschalk CG, Whelan R, Peterson D, Roy A. Detection of Elevated Level of Tetrahydrobiopterin in Serum Samples of ME/CFS Patients with Orthostatic Intolerance: A Pilot Study. \textit{International Journal of Molecular Sciences}. 2023;24(10):8713.
    \item[DOI:] \href{https://doi.org/10.3390/ijms24108713}{10.3390/ijms24108713}
    \item[PMID:] 37240059
    \item[Published:] May 12, 2023
    \item[Study Design:] Cross-sectional pilot study
    \item[Sample Size:] Total n=66 (CFS n=32, CFS+OI n=10, CFS+OI+SFN n=12, controls n=30)
\end{description}

\paragraph{Key Findings:}
Serum tetrahydrobiopterin (BH4) levels were significantly elevated in ME/CFS patients compared to age- and gender-matched controls, with the strongest elevation in patients with orthostatic intolerance. Specifically: general CFS group ($p=0.033$), CFS+OI group ($p=0.0223$, most significant), and CFS+OI+SFN group ($p=0.0269$) all showed significant BH4 elevation. A moderately positive correlation existed between BH4 levels and reactive oxygen species (ROS) production in microglial cell assays, suggesting a link between BH4 elevation and oxidative stress.

\paragraph{Bulbule et al.\ 2024 --- Mechanistic Study of BH4 Dysregulation}

\begin{description}
    \item[Full Citation:] Bulbule S, Gottschalk CG, Drosen ME, Peterson D, Arnold LA, Roy A. Dysregulation of tetrahydrobiopterin metabolism in myalgic encephalomyelitis/chronic fatigue syndrome by pentose phosphate pathway. \textit{Journal of Central Nervous System Disease}. 2024;16:11795735241271675.
    \item[DOI:] \href{https://doi.org/10.1177/11795735241271675}{10.1177/11795735241271675}
    \item[PMID:] 39161795
    \item[PMCID:] PMC11331476
    \item[Published:] August 19, 2024
    \item[Study Design:] Pilot mechanistic study
    \item[Sample Size:] ME+OI n=10, healthy controls n=10
\end{description}

\paragraph{Key Findings:}
This companion study to Gottschalk 2023 elucidated the molecular mechanism underlying BH4 elevation. The non-oxidative pentose phosphate pathway (PPP) was confirmed to drive upregulation of both BH4 and its oxidized derivative BH2 via the purine biosynthetic pathway. The level of GTP cyclohydrolase I (GCH1), the rate-limiting enzyme in BH4 synthesis, was quantified in peripheral blood mononuclear cells (PBMCs) and found to be dysregulated in ME+OI patients. Critically, plasma from ME+OI patients with high BH4 upregulated inducible nitric oxide synthase (iNOS) and nitric oxide (NO) production in human microglial cells \emph{in vitro}, suggesting elevated BH4 may trigger neuroinflammatory responses.

\paragraph{Integrated Relevance:}
These two studies together identify BH4 as a potential biomarker for the orthostatic intolerance subgroup of ME/CFS and provide mechanistic insight linking metabolic dysregulation (PPP activation) to inflammatory processes (iNOS/NO pathway). The findings are particularly notable because they present a paradox: BH4 is normally a beneficial cofactor for nitric oxide synthase and neurotransmitter synthesis, yet appears pathologically elevated in ME/CFS. Possible explanations include preferential activation of inflammatory iNOS (rather than protective eNOS), oxidation of BH4 to dysfunctional BH2, NOS uncoupling, or compartmentalization issues. This paradox must be resolved before therapeutic targeting can be attempted.

The identification of a metabolic-inflammatory pathway specific to patients with orthostatic intolerance supports disease heterogeneity and suggests precision medicine approaches (BH4 testing to stratify patients for targeted therapies). However, therapeutic direction remains unclear: should BH4 be supplemented (sapropterin) or reduced (PPP inhibition)? The iNOS activation finding suggests reduction might be beneficial, but this contradicts BH4's normal protective role.

\paragraph{Certainty Assessment:}
\begin{itemize}
    \item \textbf{BH4 Elevation:} Moderate certainty (consistent across two studies, statistically significant, mechanistic depth)
    \item \textbf{Sample Size:} Small (2023: n=32 general CFS, n=10 CFS+OI; 2024: n=10 ME+OI) --- pilot studies only
    \item \textbf{Replication:} Same research group (Peterson, Roy, Gottschalk); needs independent validation
    \item \textbf{Mechanism:} Low-moderate certainty (in vitro validation, but n=10 very small; mechanism needs in vivo confirmation)
    \item \textbf{Clinical Utility:} Low certainty (not yet validated as biomarker; no established cutoffs; therapeutic direction unclear)
    \item \textbf{Generalizability:} OI subgroup only; unclear if applies to broader ME/CFS population or is specific to orthostatic intolerance regardless of underlying disease
    \item \textbf{Limitations:} Very small samples, single research group, BH4 paradox unresolved, cross-sectional design, no longitudinal tracking, therapeutic implications unknown
\end{itemize}

\paragraph{Research Priorities:}
High-priority validation needed: (1) Independent replication in larger cohort (n$>$100), (2) Clarification of BH4 paradox (why is normally-beneficial BH4 elevated and apparently harmful?), (3) BH4/BH2 ratio analysis, (4) Longitudinal tracking to assess stability as biomarker, (5) Correlation with objective measures of orthostatic intolerance (tilt table, CPET), (6) In vivo confirmation of microglial iNOS activation. Therapeutic trials should NOT proceed until mechanism is clarified and direction determined (supplement vs reduce).

% =============================================================================
\section{Exercise Physiology and Post-Exertional Malaise}
\label{sec:bib-exercise}
% =============================================================================

\begin{description}
    \item[Full Citation:] Franklin JD, Graham M, the Workwell Foundation. The Prospects of the Two-Day Cardiopulmonary Exercise Test (CPET) in ME/CFS Patients: A Meta-Analysis. \textit{International Journal of Environmental Research and Public Health}. 2020;17(24):9575.
    \item[DOI:] \href{https://doi.org/10.3390/ijerph17249575}{10.3390/ijerph17249575}
    \item[PMCID:] PMC7765094
    \item[Key Findings:] Day 2 CPET shows decreased VO$_2$max and workload unique to ME/CFS.
\end{description}

\begin{description}
    \item[Full Citation:] Stevens S, Snell C, Stevens J, Keller B, VanNess JM. Cardiopulmonary Exercise Test Methodology for Assessing Exertion Intolerance in Myalgic Encephalomyelitis/Chronic Fatigue Syndrome. \textit{Frontiers in Pediatrics}. 2018;6:242.
    \item[DOI:] \href{https://doi.org/10.3389/fped.2018.00242}{10.3389/fped.2018.00242}
\end{description}

\begin{description}
    \item[Full Citation:] Two-day cardiopulmonary exercise testing in long COVID post-exertional malaise diagnosis. \textit{Respiratory Medicine and Research}. 2024;85:101551.
    \item[DOI:] \href{https://doi.org/10.1016/j.resmer.2024.101551}{10.1016/j.resmer.2024.101551}
\end{description}

\begin{description}
    \item[Full Citation:] Recovery time from two-day CPET in ME/CFS. Cornell Center for Enervating NeuroImmune Disease. 2024.
    \item[URL:] \url{https://neuroimmune.cornell.edu/news/recovery-from-two-day-cpet-in-me-cfs/}
    \item[Key Findings:] Recovery $\sim$13 days in ME/CFS vs $\sim$2 days in sedentary controls.
\end{description}

% =============================================================================
\section{Treatment Evidence}
\label{sec:bib-treatment}
% =============================================================================

\subsection{Immunological Therapies: Rituximab and Cyclophosphamide}

\paragraph{Fluge et al.\ 2019 --- Rituximab Phase III Trial (NEGATIVE)}

\begin{description}
    \item[Full Citation:] Fluge Ø, Rekeland IG, Lien K, et al.\ B-Lymphocyte Depletion in Patients With Myalgic Encephalomyelitis/Chronic Fatigue Syndrome: A Randomized, Double-Blind, Placebo-Controlled Trial. \textit{Annals of Internal Medicine}. 2019;170(9):585--593.
    \item[DOI:] \href{https://doi.org/10.7326/M18-1451}{10.7326/M18-1451}
    \item[PMID:] 30934066
    \item[Trial Registration:] ClinicalTrials.gov NCT02229942
    \item[Study Design:] Phase III randomized, double-blind, placebo-controlled, multicenter trial
    \item[Sample Size:] 151 patients (77 rituximab, 74 placebo)
\end{description}

\paragraph{Key Findings:}
\textbf{This trial was NEGATIVE.} Overall response rates were 35.1\% in the placebo group versus 26.0\% in the rituximab group (difference 9.2 percentage points [95\% CI: $-5.5$ to 23.3]; $p=0.22$). The treatment groups showed no differences in fatigue scores over 24 months (difference in average score 0.02 [CI: $-0.27$ to 0.31]; $p=0.80$) or any secondary endpoints (SF-36, physical function, activity levels). Serious adverse events occurred in 26.0\% of rituximab patients versus 18.9\% of placebo patients. Notably, the placebo response rate of 35\% demonstrates substantial natural fluctuation or expectation effects in ME/CFS.

\paragraph{Relevance:}
This landmark negative trial definitively refutes B-cell depletion as a therapeutic strategy for ME/CFS, contradicting earlier promising Phase II open-label studies from the same research group. The high placebo response rate (35\%) has critical implications for trial design: it demonstrates that even large apparent improvements in uncontrolled studies may not represent true drug effects. The study serves as a cautionary tale about extrapolating from small early-phase trials and emphasizes the necessity of rigorous placebo-controlled validation. \textbf{Rituximab should NOT be used for ME/CFS.}

\paragraph{Certainty Assessment:}
\begin{itemize}
    \item \textbf{Quality:} High (Phase III RCT, double-blind, placebo-controlled, multicenter, published in \emph{Annals of Internal Medicine})
    \item \textbf{Sample:} n=151 (adequate for Phase III efficacy trial)
    \item \textbf{Replication:} This \emph{was} the replication---contradicted earlier positive Phase II results from same group
    \item \textbf{Funding:} Publicly funded (Norwegian Research Council, health trusts), no industry bias
    \item \textbf{Limitations:} Self-reported outcomes (though standard for ME/CFS); possible heterogeneity (small subset might respond but undetectable in overall analysis)
\end{itemize}

\paragraph{Rekeland et al.\ 2024 --- 6-Year Follow-up}

\begin{description}
    \item[Full Citation:] Rekeland IG, Sørland K, Neteland LL, et al.\ Six-year follow-up of participants in two clinical trials of rituximab or cyclophosphamide in Myalgic Encephalomyelitis/Chronic Fatigue Syndrome. \textit{PLoS One}. 2024;19(7):e0307484.
    \item[DOI:] \href{https://doi.org/10.1371/journal.pone.0307484}{10.1371/journal.pone.0307484}
    \item[PMID:] 39042627
    \item[PMCID:] PMC11265720
    \item[Study Type:] Long-term observational follow-up of RituxME (Phase III RCT) and CycloME (Phase II open-label) trials
\end{description}

\paragraph{Key Findings:}
At 6-year follow-up, rituximab showed no sustained benefit over placebo: 27.6\% of rituximab-treated patients achieved SF-36 Physical Function $\geq$70 compared to 20.4\% of placebo patients (not statistically significant). In contrast, the open-label cyclophosphamide group showed 44.1\% achieving SF-36 PF $\geq$70, with 17.6\% reaching normal function (PF $\geq$90). However, the authors explicitly caution: ``cyclophosphamide carries toxicity concerns and should not be used for ME/CFS patients outside clinical trials.'' The placebo group data provides valuable natural history information: approximately 20\% of patients improved substantially over 6 years without specific treatment, while 15\% worsened significantly.

\paragraph{Relevance:}
Confirms long-term lack of benefit for rituximab. The cyclophosphamide results are intriguing but \textbf{cannot be interpreted as evidence of efficacy} due to absence of placebo control, open-label design, small sample (n=34 at 6 years), and potential selection bias (94\% follow-up rate may favor responders). Given cyclophosphamide's severe toxicity (cancer risk, infertility, life-threatening infections), the uncertain benefit based solely on open-label data is insufficient to justify clinical use. The findings do, however, support the hypothesis of a possible immune-mediated subgroup and warrant investigation of safer immune-modulating agents with proper placebo-controlled trials.

\paragraph{Certainty Assessment:}
\begin{itemize}
    \item \textbf{Rituximab data:} High certainty of lack of benefit (follow-up of rigorous RCT)
    \item \textbf{Cyclophosphamide data:} Low certainty (no placebo control, open-label, small sample, selection bias)
    \item \textbf{Natural history data:} Moderate certainty (from placebo arm, but 24\% loss to follow-up)
    \item \textbf{Limitations:} Cyclophosphamide findings are hypothesis-generating only; different patient populations between trials complicate cross-comparison
\end{itemize}

\subsection{Low-Dose Naltrexone}

\paragraph{Polo et al.\ 2019 --- Retrospective Observational Study}

\begin{description}
    \item[Full Citation:] Polo O, Pesonen P, Tuominen E. Low-dose naltrexone in the treatment of myalgic encephalomyelitis/chronic fatigue syndrome (ME/CFS). \textit{Fatigue: Biomedicine, Health \& Behavior}. 2019;7(4):207--217.
    \item[DOI:] \href{https://doi.org/10.1080/21641846.2019.1692770}{10.1080/21641846.2019.1692770}
    \item[Published:] November 19, 2019
    \item[Study Design:] Retrospective chart review
    \item[Sample Size:] 218 ME/CFS patients
\end{description}

\paragraph{Key Findings:}
In this large retrospective analysis, 73.9\% (n=161/218) of ME/CFS patients reported subjective improvement with low-dose naltrexone (3.0--4.5 mg/day) over mean 1.7-year follow-up. Specific improvements included vigilance/alertness (51.4\%), physical performance (23.9\%), and cognitive function (21.1\%). No severe adverse events were reported; mild transient side effects (insomnia, nausea) occurred at treatment initiation but typically resolved. The authors explicitly acknowledge the study's limitations, concluding: ``placebo-controlled studies are needed to confirm these findings.''

\paragraph{Relevance:}
This is the largest observational study of LDN in ME/CFS, suggesting potential benefit with an excellent safety profile. However, \textbf{the absence of placebo control is a critical limitation.} Given that the rituximab trial demonstrated 35\% placebo response, the 74\% response rate to LDN in an open-label setting cannot be assumed to represent true drug effect. Additional concerns include retrospective design, subjective outcomes, selection bias (which patients were prescribed LDN?), and lack of validated outcome measures. That said, LDN's favorable safety profile, low cost (generic), and mechanistic plausibility (opioid receptor modulation, immune effects) make it a high-priority candidate for rigorous placebo-controlled RCT testing. Given the contrast with rituximab (both looked promising in early studies; rituximab failed RCT), this study should be viewed as hypothesis-generating rather than evidence of efficacy.

\paragraph{Certainty Assessment:}
\begin{itemize}
    \item \textbf{Safety:} High certainty (large sample, long follow-up, no serious adverse events)
    \item \textbf{Efficacy:} Low certainty (no placebo control, retrospective design, subjective outcomes)
    \item \textbf{Clinical Use:} May be reasonable for treatment-refractory patients with informed consent about uncertain evidence
    \item \textbf{Research Priority:} High (safe, cheap, worth rigorous RCT validation)
    \item \textbf{Limitations:} Retrospective, no placebo control (disqualifying for efficacy claims), undefined response criteria, no standardized dosing, single geographic location (Finland)
\end{itemize}

\subsection{Graded Exercise Therapy (Negative Evidence)}

\begin{description}
    \item[Full Citation:] Geraghty K, Hann M, Kurtev S. The Updated NICE Guidance Exposed the Serious Flaws in CBT and Graded Exercise Therapy Trials for ME/CFS. \textit{Healthcare}. 2022;10(5):898.
    \item[DOI:] \href{https://doi.org/10.3390/healthcare10050898}{10.3390/healthcare10050898}
    \item[PMCID:] PMC9141828
    \item[Key Findings:] Methodological flaws and biases in trials; patient surveys show harm from GET.
\end{description}

\begin{description}
    \item[Full Citation:] Vink M, Vink-Niese A. The PACE Trial's GET Manual for Therapists Exposes the Fixed Incremental Nature of Graded Exercise Therapy for ME/CFS. \textit{Life}. 2025;15(4):584.
    \item[DOI:] \href{https://doi.org/10.3390/life15040584}{10.3390/life15040584}
\end{description}

\begin{description}
    \item[Full Citation:] Vink M, Vink-Niese A. Graded exercise therapy does not restore the ability to work in ME/CFS -- Rethinking of a Cochrane review. \textit{Work}. 2020;66(2):283--308.
    \item[DOI:] \href{https://doi.org/10.3233/WOR-203174}{10.3233/WOR-203174}
    \item[PMID:] 32568149
\end{description}

\subsection{Pacing and Energy Management}

\begin{description}
    \item[Full Citation:] Goudsmit EM, Nijs J, Jason LA, Wallman KE. A scoping review of `Pacing' for management of Myalgic Encephalomyelitis/Chronic Fatigue Syndrome (ME/CFS): lessons learned for the long COVID pandemic. \textit{Journal of Translational Medicine}. 2023;21:738.
    \item[DOI:] \href{https://doi.org/10.1186/s12967-023-04586-6}{10.1186/s12967-023-04586-6}
    \item[PMCID:] PMC10576275
\end{description}

\begin{description}
    \item[Full Citation:] Jason LA, Brown M, Brown A, et al.\ Energy Conservation/Envelope Theory Interventions to Help Patients with Myalgic Encephalomyelitis/Chronic Fatigue Syndrome. \textit{Fatigue: Biomedicine, Health \& Behavior}. 2013;1(1--2):65--78.
    \item[DOI:] \href{https://doi.org/10.1080/21641846.2012.733602}{10.1080/21641846.2012.733602}
    \item[PMCID:] PMC3596172
\end{description}

\subsection{Patient-Reported Treatment Outcomes}

\begin{description}
    \item[Full Citation:] Davis HE, McCorkell L, Vogel JM, et al.\ Patient-reported treatment outcomes in ME/CFS and long COVID. \textit{Proceedings of the National Academy of Sciences}. 2025;122(26):e2426874122.
    \item[DOI:] \href{https://doi.org/10.1073/pnas.2426874122}{10.1073/pnas.2426874122}
    \item[PMCID:] PMC12280984
    \item[Sample:] $>$3,900 patients
    \item[Key Findings:] Treatment responses highly correlated ($R^2$=0.68) between ME/CFS and Long COVID.
\end{description}

% =============================================================================
\section{Long COVID and ME/CFS Overlap}
\label{sec:bib-long-covid}
% =============================================================================

\begin{description}
    \item[Full Citation:] Thapaliya K, Marshall-Gradisnik S, Barber PA, Eaton-Fitch N, Staines D. Unravelling shared mechanisms: insights from recent ME/CFS research to illuminate long COVID pathologies. \textit{Trends in Molecular Medicine}. 2024;30(5):443--458.
    \item[DOI:] \href{https://doi.org/10.1016/j.molmed.2024.02.003}{10.1016/j.molmed.2024.02.003}
    \item[PMID:] 38443223
\end{description}

\begin{description}
    \item[Full Citation:] Mapping the complexity of ME/CFS: Evidence for abnormal energy metabolism, altered immune profile, and vascular dysfunction. \textit{Cell Reports Medicine}. 2025;6(12):101587.
    \item[DOI:] \href{https://doi.org/10.1016/j.xcrm.2025.101587}{10.1016/j.xcrm.2025.101587}
\end{description}

% =============================================================================
\section{Historical Background and Epidemics}
\label{sec:bib-history}
% =============================================================================

\begin{description}
    \item[Full Citation:] Underhill RA. Myalgic encephalomyelitis, chronic fatigue syndrome: An infectious disease. \textit{Medical Hypotheses}. 2015;85(6):765--773.
    \item[DOI:] \href{https://doi.org/10.1016/j.mehy.2015.10.011}{10.1016/j.mehy.2015.10.011}
    \item[Topics:] Historical outbreaks from 1934 onwards.
\end{description}

\begin{description}
    \item[Full Citation:] Underhill RA, O'Gorman R. The viral origin of myalgic encephalomyelitis/chronic fatigue syndrome. \textit{Journal of the Royal Society of Medicine}. 2023;116(8):269--282.
    \item[DOI:] \href{https://doi.org/10.1177/01410768231176937}{10.1177/01410768231176937}
    \item[PMCID:] PMC10434940
\end{description}

\begin{description}
    \item[Full Citation:] Brurberg KG, Fønhus MS, Larun L, Flottorp S, Malterud K. Myalgic Encephalomyelitis/Chronic Fatigue Syndrome: Organic Disease or Psychosomatic Illness? A Re-Examination of the Royal Free Epidemic of 1955. \textit{Medicina}. 2021;57(1):12.
    \item[DOI:] \href{https://doi.org/10.3390/medicina57010012}{10.3390/medicina57010012}
    \item[PMID:] 33375343
    \item[Key Findings:] First-hand accounts confirm organic infectious disease, not hysteria.
\end{description}

\begin{description}
    \item[Full Citation:] Jason LA, Lapp CW, Engel S, et al.\ Myalgic Encephalomyelitis (ME) outbreaks can be modelled as an infectious disease: a mathematical reconsideration of the Royal Free Epidemic of 1955. \textit{Fatigue: Biomedicine, Health \& Behavior}. 2020;8(2):99--109.
    \item[DOI:] \href{https://doi.org/10.1080/21641846.2020.1793058}{10.1080/21641846.2020.1793058}
\end{description}

% =============================================================================
\section{Research Roadmaps and Policy Documents}
\label{sec:bib-policy}
% =============================================================================

\begin{description}
    \item[Full Citation:] National Institute of Neurological Disorders and Stroke. Report of the ME/CFS Research Roadmap Working Group of Council. Bethesda, MD: NINDS; May 15, 2024.
    \item[URL:] \url{https://www.ninds.nih.gov/sites/default/files/2024-05/Report\%20of\%20the\%20MECFS\%20Research\%20Roadmap\%20Working\%20Group\%20of\%20Council_508C.pdf}
    \item[Significance:] Official NIH research priorities and funding recommendations.
\end{description}

\begin{description}
    \item[Full Citation:] Reframing Myalgic Encephalomyelitis/Chronic Fatigue Syndrome (ME/CFS): Biological Basis of Disease and Recommendations for Supporting Patients. 2025.
    \item[PMCID:] PMC12346739
\end{description}

% =============================================================================
\section{Comprehensive Reviews}
\label{sec:bib-comprehensive-reviews}
% =============================================================================

\begin{description}
    \item[Full Citation:] Cortes Rivera M, Mastronardi C, Silva-Aldana CT, Arcos-Burgos M, Lidbury BA. Myalgic Encephalomyelitis/Chronic Fatigue Syndrome: A Comprehensive Review. \textit{Diagnostics}. 2019;9(3):91.
    \item[DOI:] \href{https://doi.org/10.3390/diagnostics9030091}{10.3390/diagnostics9030091}
    \item[PMCID:] PMC6787585
\end{description}

% =============================================================================
% =============================================================================
\section{Mast Cell Activation and Antihistamine Therapies}
\label{sec:bib-mast-cell-antihistamines}
% =============================================================================

\subsection{Hardcastle et al.\ 2016 --- Mast Cell Phenotype Abnormalities in ME/CFS}

\begin{description}
    \item[Full Citation:] Hardcastle SL, Brenu EW, Johnston S, et al.\ Novel characterisation of mast cell phenotypes from peripheral blood mononuclear cells in chronic fatigue syndrome/myalgic encephalomyelitis patients. \textit{BMC Immunology}. 2016;17(1):30.
    \item[DOI:] \href{https://doi.org/10.1186/s12865-016-0167-z}{10.1186/s12865-016-0167-z}
    \item[PMID:] 27362406
    \item[PMCID:] PMC4928291
    \item[Published:] June 29, 2016
    \item[Study Design:] Cross-sectional immunophenotyping study
    \item[Sample Size:] 18 ME/CFS patients (12 moderate, 6 severe), 13 matched healthy controls
    \item[Key Findings:]
    \begin{itemize}
        \item Significant increase in naïve mast cells (CD117$^+$CD34$^+$Fc$\varepsilon$RI$^-$chymase$^-$) in moderate and severe ME/CFS ($p<0.05$)
        \item Elevated CD40 ligand and MHC-II receptors on differentiated mast cells in severe ME/CFS
        \item Demonstrates measurable mast cell abnormalities at cellular level
        \item Supports hypothesis that mast cells may be involved in ME/CFS pathophysiology
    \end{itemize}
    \item[Certainty:] High (well-designed study, statistically significant findings)
    \item[Clinical Relevance:] Provides biological basis for mast cell involvement in ME/CFS; supports rationale for mast cell-targeted therapies
\end{description}

\subsection{Wirth \& Scheibenbogen 2023 --- Mast Cell Activation and Vascular Pathomechanisms}

\begin{description}
    \item[Full Citation:] Wirth K, Scheibenbogen C. Myalgic Encephalomyelitis/Chronic Fatigue Syndrome (ME/CFS) and Comorbidities: Linked by Vascular Pathomechanisms and Vasoactive Mediators? \textit{Healthcare}. 2023;11(7):978.
    \item[DOI:] \href{https://doi.org/10.3390/healthcare11070978}{10.3390/healthcare11070978}
    \item[PMID:] 37046903
    \item[PMCID:] PMC10224216
    \item[Published:] March 27, 2023
    \item[Study Type:] Review and hypothesis paper
    \item[Key Mechanisms:]
    \begin{itemize}
        \item Mast cell activation shares pathogenic mechanisms with ME/CFS through excessive histamine, heparin, prostaglandins, leukotrienes, and protease release
        \item Spillover of vasoactive mediators into systemic circulation worsens orthostatic intolerance via histamine's vascular effects
        \item $\beta_2$-adrenergic receptor dysfunction amplifies symptoms
        \item ME/CFS patients with MCAS and orthostatic intolerance reported symptom alleviation significantly more often following mast cell-targeted treatment ($p<0.0001$)
    \end{itemize}
    \item[Certainty:] Medium (mechanistic hypothesis with clinical correlation)
    \item[Clinical Relevance:] Links mast cell activation to orthostatic intolerance; suggests mast cell-targeted therapies may benefit subset of ME/CFS patients with vascular/autonomic symptoms
\end{description}

\subsection{Steinberg et al.\ 1996 --- Terfenadine Trial (Negative)}

\begin{description}
    \item[Full Citation:] Steinberg P, McNutt BE, Marshall P, et al.\ A double-blind placebo-controlled study of the efficacy of oral terfenadine in the chronic fatigue syndrome. \textit{J Allergy Clin Immunol}. 1996;97(1 Pt 1):119--126.
    \item[DOI:] \href{https://doi.org/10.1016/S0091-6749(96)80212-6}{10.1016/S0091-6749(96)80212-6}
    \item[PMID:] 8568124
    \item[Published:] January 1996
    \item[Study Design:] Double-blind, placebo-controlled RCT
    \item[Sample Size:] 30 CFS patients enrolled, 28 completed
    \item[Intervention:] Terfenadine 60 mg twice daily for 8 weeks (H1 antihistamine only)
    \item[Results:]
    \begin{itemize}
        \item \textbf{NO therapeutic benefit detected}
        \item No improvement in symptom amelioration
        \item No improvement in physical or social functioning
        \item No improvement in health perceptions or mental health
        \item Additional finding: 73\% had atopy, 53\% had positive immediate skin test results
    \end{itemize}
    \item[Conclusion:] ``Terfenadine is unlikely to be of clinical benefit in treating CFS symptoms''
    \item[Certainty:] High (well-designed RCT with negative results)
    \item[Clinical Implications:] H1 antihistamine alone insufficient; suggests combination therapy (H1+H2 or H1+mast cell stabilizer) may be necessary
\end{description}

\subsection{Davis et al.\ 2023 --- Long COVID Case with H1/H2 Combination Success}

\begin{description}
    \item[Full Citation:] Davis HE, McCorkell L, Vogel JM, Topol EJ. Case Study of ME/CFS Care Applied to Long COVID: Hypothesis Regarding Exercise Intolerance, Orthostatic Intolerance, Mast Cell Activation, Sleep Dysfunction, Neuropathy, and Viral Persistence. \textit{Healthcare}. 2023;11(6):896.
    \item[DOI:] \href{https://doi.org/10.3390/healthcare11060896}{10.3390/healthcare11060896}
    \item[PMID:] 36981567
    \item[PMCID:] PMC10048325
    \item[Published:] March 21, 2023
    \item[Study Type:] Single case report (n=1)
    \item[Patient:] Long COVID patient meeting ME/CFS criteria
    \item[Interventions and Outcomes:]
    \begin{itemize}
        \item \textbf{H1 blockers} (loratadine 10 mg OR fexofenadine 180 mg): ``helpful with energy and cognitive dysfunction''
        \item \textbf{H2 blocker} (famotidine 40 mg BID): ``helpful with energy and cognitive dysfunction''
        \item \textbf{Discontinuation test}: Stopping fexofenadine and famotidine $\rightarrow$ ``increased fatigue and increased cognitive dysfunction, both of which improved rapidly upon resumption''
        \item \textbf{Cromolyn} (400 mg QID): Peak heart rate during walking fell from 130--140 bpm to 100--105 bpm
        \item \textbf{Quercetin} (1000 mg BID): ``Improvement in fatigue and allergic symptoms''
    \end{itemize}
    \item[Certainty:] Low (n=1 case report, but dramatic response with discontinuation-rechallenge confirmation)
    \item[Clinical Relevance:] Demonstrates potential for H1+H2 combination therapy; suggests mast cell-targeted approach may benefit post-viral fatigue syndromes
\end{description}

\subsection{Theoharides et al.\ 2012 --- Quercetin Superior to Cromolyn}

\begin{description}
    \item[Full Citation:] Theoharides TC, Asadi S, Panagiotidou S. Quercetin in combination with IL-6 inhibits histamine and TNF release from mast cells through interaction with the IL-6 receptor. \textit{PLOS ONE}. 2012;7(3):e33805.
    \item[DOI:] \href{https://doi.org/10.1371/journal.pone.0033805}{10.1371/journal.pone.0033805}
    \item[PMID:] 22470478
    \item[PMCID:] PMC3314669
    \item[Published:] March 29, 2012
    \item[Study Design:] In vitro comparison + clinical pilot trials
    \item[Concentration:] Quercetin 100 $\mu$M (approximated by 2 g/day oral dosing)
    \item[Key Findings:]
    \begin{itemize}
        \item \textbf{IgE/Anti-IgE stimulation}: Quercetin inhibited histamine (82\% vs 67\%), PGD$_2$ (77\% vs 75\%), leukotrienes (99\% vs 88\%) comparably to cromolyn
        \item \textbf{Substance P stimulation}: Quercetin dramatically outperformed cromolyn --- IL-8 reduced from 437.2 to 115.4 pg/mL (quercetin) vs 362.9 pg/mL (cromolyn)
        \item \textbf{Mechanism}: Quercetin worked prophylactically (30 min pre-stimulus); cromolyn required simultaneous addition
        \item \textbf{Clinical trial --- Contact dermatitis}: Quercetin 2 g/day for 3 days reduced nickel patch reactions $>$50\% in 8 of 10 patients; pruritus eliminated completely
        \item \textbf{Clinical trial --- Photosensitivity}: Quercetin 1 g increased minimal erythema dose in all patients ($p$=0.002)
    \end{itemize}
    \item[Certainty:] Medium-High (strong in vitro data, pilot clinical success)
    \item[Clinical Relevance:] Quercetin may be superior to prescription cromolyn for mast cell stabilization; available over-the-counter; well-tolerated
\end{description}

\subsection{Clemons et al.\ 2011 --- Amitriptyline Mast Cell Inhibition}

\begin{description}
    \item[Full Citation:] Clemons A, Vasiadi M, Kempuraj D, et al.\ Amitriptyline and prochlorperazine inhibit proinflammatory mediator release from human mast cells: possible relevance to chronic fatigue syndrome. \textit{J Clin Psychopharmacol}. 2011;31(3):385--387.
    \item[DOI:] \href{https://doi.org/10.1097/JCP.0b013e3182196e50}{10.1097/JCP.0b013e3182196e50}
    \item[PMID:] 21532369
    \item[PMCID:] PMC3498825
    \item[Published:] June 2011
    \item[Study Design:] In vitro study on human mast cells
    \item[Key Findings:]
    \begin{itemize}
        \item Amitriptyline (AMI) and prochlorperazine (PRO) at 25 $\mu$M significantly reduced IL-8, VEGF, and IL-6 release from stimulated human mast cells
        \item Bupropion, citalopram, and atomoxetine did NOT inhibit mast cells
        \item Mechanism involves modulation of intracellular calcium (FURA2 AM calcium indicator assays)
        \item AMI inhibits histamine release while permitting serotonin release
    \end{itemize}
    \item[Conclusion:] ``The ability of amitriptyline, but not other antidepressants, to inhibit human mast cell release of pro-inflammatory cytokines may be relevant to their apparent benefit in CFS''
    \item[Certainty:] Medium (mechanistic in vitro study, explains clinical observations)
    \item[Clinical Relevance:] Amitriptyline's benefit in ME/CFS may involve mast cell inhibition beyond pain/sleep effects; specific pharmacological mechanism
\end{description}

\subsection{Rupatadine --- Dual H1/PAF Antagonist with Mast Cell Stabilization}

\begin{description}
    \item[Full Citations:]
    \begin{itemize}
        \item Piñero-González J, et al.\ Rupatadine inhibits proinflammatory mediator secretion from human mast cells triggered by different stimuli. \textit{J Investig Allergol Clin Immunol}. 2017;27(3):161--168. PMID: 19672095; PMCID: PMC7065400.
        \item Mullol J, Bousquet J, Bachert C, et al.\ Rupatadine in allergic rhinitis and chronic urticaria. \textit{Allergy}. 2008;63(Suppl 87):5--28. PMID: 18339040.
    \end{itemize}
    \item[Mechanism:] Triple action --- (1) H1 receptor antagonist, (2) PAF (platelet-activating factor) antagonist, (3) Direct mast cell stabilizer
    \item[Mast Cell Effects:]
    \begin{itemize}
        \item Rupatadine (10--50 $\mu$M) inhibited IL-8 (80\%), VEGF (73\%), histamine (88\%) release from LAD2 mast cell line
        \item Also inhibited IL-6, IL-8, IL-10, IL-13, and TNF release from human cord blood-derived cultured mast cells
        \item More effective than levocetirizine and desloratadine at PAF-induced mast cell inhibition
    \end{itemize}
    \item[PAF Antagonism Potency:]
    \begin{itemize}
        \item Rupatadine IC$_{50}$ = 4.6 $\mu$M (most potent)
        \item Loratadine IC$_{50}$ = 142 $\mu$M ($\sim$31$\times$ less potent)
        \item Cetirizine IC$_{50}$ $>$200 $\mu$M ($>$43$\times$ less potent)
        \item Fexofenadine IC$_{50}$ $>$200 $\mu$M ($>$43$\times$ less potent)
    \end{itemize}
    \item[Efficacy Ranking:] Network meta-analysis for allergic rhinitis (SUCRA scores):
    \begin{itemize}
        \item Rupatadine 20 mg: 99.7\% (highest rank)
        \item Rupatadine 10 mg: 76.3\%
        \item Fexofenadine, cetirizine: moderate
        \item Loratadine 10 mg: lowest (inferior to all others)
    \end{itemize}
    \item[Certainty:] High (multiple RCTs, network meta-analysis, in vitro mechanistic data)
    \item[Clinical Relevance:] Superior to standard H1 antihistamines; unique PAF antagonism may benefit ME/CFS patients with mast cell activation and vascular/orthostatic symptoms
    \item[Note:] PAF is a key inflammatory mediator in ME/CFS contributing to vascular leakage, brain fog, and orthostatic issues
\end{description}

\subsection{Moldofsky et al.\ 2015 --- Ketotifen in Fibromyalgia (Negative)}

\begin{description}
    \item[Full Citation:] Moldofsky H, Harris HW, Archambault WT, Kwong T, Lederman S. A randomized, double-blind, placebo-controlled Phase 1 trial of ketotifen in fibromyalgia. \textit{J Rheumatol}. 2015;42(12):2505--2513.
    \item[DOI:] \href{https://doi.org/10.3899/jrheum.150460}{10.3899/jrheum.150460}
    \item[PMID:] 26472411
    \item[PMCID:] PMC4417653
    \item[Published:] December 2015
    \item[Study Design:] Phase 1 RCT, double-blind, placebo-controlled
    \item[Sample Size:] 51 fibromyalgia patients (24 ketotifen, 27 placebo)
    \item[Intervention:] Ketotifen 2 mg BID for 8 weeks (after 1-week titration)
    \item[Results:] \textbf{NO significant differences} in primary outcomes:
    \begin{itemize}
        \item Pain intensity: ketotifen $-1.3$ vs placebo $-1.5$ ($p$=0.7)
        \item FIQR scores: $-12.1$ vs $-12.2$ ($p$=0.9)
        \item Side effect: Transient sedation 28.6\% vs 4\%
    \end{itemize}
    \item[Certainty:] High (well-designed RCT showing no benefit)
    \item[Clinical Relevance:] Mast cell stabilization alone may not address core pathophysiology in central pain syndromes like fibromyalgia; relevance to ME/CFS unclear
    \item[Note:] Despite this negative finding, retrospective ME/CFS study (not included here) showed 77\% of continuers had significant PEM reduction with ketotifen
\end{description}

\section{Additional Key Resources}
\label{sec:bib-resources}
% =============================================================================

\subsection{Patient Advocacy and Information}

\begin{description}
    \item[MEpedia:] \url{https://me-pedia.org/} --- Comprehensive patient-edited wiki on ME/CFS.
    \item[ME Association (UK):] \url{https://meassociation.org.uk/} --- Patient support and research summaries.
    \item[Bateman Horne Center:] \url{https://batemanhornecenter.org/} --- Clinical and educational resources.
    \item[Open Medicine Foundation:] \url{https://www.openmedicinefoundation.ngo/} --- Research funding and updates.
    \item[Solve ME/CFS Initiative:] \url{https://solvecfs.org/} --- US-based research and advocacy.
\end{description}

\subsection{Research Centers}

\begin{description}
    \item[Cornell Center for Enervating NeuroImmune Disease:] \url{https://neuroimmune.cornell.edu/}
    \item[Griffith University National Centre for Neuroimmunology and Emerging Diseases:] Queensland, Australia
    \item[Charit\'e Fatigue Center:] Berlin, Germany
    \item[Stanford ME/CFS Initiative:] Stanford University, California
\end{description}

\vspace{1cm}
\begin{center}
\rule{0.5\textwidth}{0.4pt}
\end{center}
\vspace{0.5cm}

\noindent\textit{Note: This bibliography was compiled in January 2025. The field of ME/CFS research is rapidly evolving, particularly with insights from Long COVID research. Readers are encouraged to search PubMed and preprint servers for the most current literature.}