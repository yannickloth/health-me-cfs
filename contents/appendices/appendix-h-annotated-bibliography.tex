\chapter{Annotated Bibliography of ME/CFS Literature}
\label{app:annotated-bibliography}

This appendix provides a comprehensive annotated bibliography of scientific literature on Myalgic Encephalomyelitis/Chronic Fatigue Syndrome (ME/CFS). Sources are organized by research domain and include full identifying information where available.

% =============================================================================
\section{Primary Research: NIH Deep Phenotyping Study}
\label{sec:bib-nih-deep-phenotyping}
% =============================================================================

\subsection{Walitt et al.\ 2024 --- The Foundational Deep Phenotyping Study}

\begin{description}
    \item[Full Citation:] Walitt B, Singh K, LaMunion SR, et al.\ Deep phenotyping of post-infectious myalgic encephalomyelitis/chronic fatigue syndrome. \textit{Nature Communications}. 2024;15(1):907.
    \item[DOI:] \href{https://doi.org/10.1038/s41467-024-45107-3}{10.1038/s41467-024-45107-3}
    \item[PMID:] 38383456
    \item[PMCID:] PMC10881493
    \item[Published:] February 21, 2024
    \item[Study Design:] Cross-sectional deep phenotyping study
    \item[Sample Size:] 17 PI-ME/CFS patients, 21 matched healthy controls
    \item[Key Findings:]
    \begin{itemize}
        \item Altered effort preference rather than physical or central fatigue (OR=1.65, $p$=0.04)
        \item Decreased brain activation in right temporal-parietal junction during motor tasks
        \item CSF catechol abnormalities: decreased DOPA ($p$=0.021), DOPAC ($p$=0.025), DHPG ($p$=0.006)
        \item Reduced peak VO$_2$ on cardiopulmonary exercise testing ($p$=0.004)
        \item Chronotropic incompetence (5/8 PI-ME/CFS vs 1/9 controls, $p$=0.03)
        \item B-cell abnormalities: increased na\"ive B-cells ($p$=0.037), decreased switched memory B-cells ($p$=0.008)
        \item Sex-specific gene expression differences with $<$5\% overlap between sexes
    \end{itemize}
    \item[Conclusion:] ME/CFS appears to be a centrally mediated disorder characterized by altered effort preference, potentially associated with central catecholamine dysregulation.
    \item[Limitations:] Small sample size (80\% power only detects effects $\geq$0.94 SD); cross-sectional design; recruitment terminated due to COVID-19 pandemic.
\end{description}

% =============================================================================
\section{Diagnostic Criteria and Clinical Guidelines}
\label{sec:bib-diagnostic-criteria}
% =============================================================================

\subsection{Institute of Medicine (IOM) 2015 Criteria}

\begin{description}
    \item[Full Citation:] Institute of Medicine (US) Committee on the Diagnostic Criteria for Myalgic Encephalomyelitis/Chronic Fatigue Syndrome. Beyond Myalgic Encephalomyelitis/Chronic Fatigue Syndrome: Redefining an Illness. Washington, DC: National Academies Press; 2015.
    \item[URL:] \url{https://www.cdc.gov/me-cfs/hcp/diagnosis/iom-2015-diagnostic-criteria-1.html}
    \item[ISBN:] 978-0-309-31689-7
    \item[Key Requirements:]
    \begin{itemize}
        \item Three required symptoms: (1) substantial reduction in functioning with fatigue $\geq$6 months, (2) post-exertional malaise, (3) unrefreshing sleep
        \item Plus at least one of: cognitive impairment OR orthostatic intolerance
        \item Symptoms must be present $\geq$50\% of the time with moderate-to-severe intensity
    \end{itemize}
    \item[Significance:] Proposed renaming to Systemic Exertion Intolerance Disease (SEID); currently used by CDC.
\end{description}

\subsection{NICE 2021 Guidelines (NG206)}

\begin{description}
    \item[Full Citation:] National Institute for Health and Care Excellence. Myalgic encephalomyelitis (or encephalopathy)/chronic fatigue syndrome: diagnosis and management. NICE guideline [NG206]. London: NICE; 2021.
    \item[URL:] \url{https://www.nice.org.uk/guidance/ng206}
    \item[Published:] October 29, 2021
    \item[Key Changes from 2007 Guideline:]
    \begin{itemize}
        \item All four core symptoms required: debilitating fatiguability, PEM, unrefreshing sleep, cognitive difficulties
        \item Symptoms must persist $\geq$3 months (suspected from 6 weeks in adults, 4 weeks in children)
        \item Graded Exercise Therapy (GET) \textbf{no longer recommended}
        \item CBT not considered a treatment for ME/CFS itself
        \item Recognition of PEM as the cardinal symptom
    \end{itemize}
    \item[Adoption:] Endorsed in Northern Ireland (2022); default guidance in Scotland (2025).
\end{description}

\subsection{Canadian Consensus Criteria (2003)}

\begin{description}
    \item[Full Citation:] Carruthers BM, Jain AK, De Meirleir KL, et al.\ Myalgic Encephalomyelitis/Chronic Fatigue Syndrome: Clinical Working Case Definition, Diagnostic and Treatment Protocols. \textit{Journal of Chronic Fatigue Syndrome}. 2003;11(1):7--115.
    \item[DOI:] \href{https://doi.org/10.1300/J092v11n01_02}{10.1300/J092v11n01\_02}
    \item[Significance:] First criteria to require PEM; more restrictive than Fukuda 1994; widely used in research.
\end{description}

\subsection{Fukuda et al.\ 1994 (CDC Criteria)}

\begin{description}
    \item[Full Citation:] Fukuda K, Straus SE, Hickie I, Sharpe MC, Dobbins JG, Komaroff A. The chronic fatigue syndrome: a comprehensive approach to its definition and study. \textit{Annals of Internal Medicine}. 1994;121(12):953--959.
    \item[DOI:] \href{https://doi.org/10.7326/0003-4819-121-12-199412150-00009}{10.7326/0003-4819-121-12-199412150-00009}
    \item[PMID:] 7978722
    \item[Significance:] Most widely used research criteria historically; criticized for being too broad.
\end{description}

% =============================================================================
\section{Systematic Reviews and Meta-Analyses}
\label{sec:bib-systematic-reviews}
% =============================================================================

\subsection{Prevalence and Epidemiology}

\begin{description}
    \item[Full Citation:] Lim E-J, Ahn Y-C, Jang E-S, Lee S-W, Lee S-H, Son C-G. Systematic review and meta-analysis of the prevalence of chronic fatigue syndrome/myalgic encephalomyelitis (CFS/ME). \textit{Journal of Translational Medicine}. 2020;18(1):100.
    \item[DOI:] \href{https://doi.org/10.1186/s12967-020-02269-0}{10.1186/s12967-020-02269-0}
    \item[PMID:] 32093722
    \item[PMCID:] PMC7038594
    \item[Key Findings:] Pooled prevalence 0.89\% (95\% CI: 0.60--1.33\%); women 1.36\% vs men 0.86\%; children/adolescents 0.55\%.
\end{description}

\begin{description}
    \item[Full Citation:] Centers for Disease Control and Prevention. Myalgic Encephalomyelitis/Chronic Fatigue Syndrome in Adults: United States, 2021--2022. NCHS Data Brief No.\ 488. Hyattsville, MD: National Center for Health Statistics; 2023.
    \item[URL:] \url{https://www.cdc.gov/nchs/products/databriefs/db488.htm}
    \item[Key Findings:] 1.3\% of US adults have ME/CFS; prevalence increases with age through 60--69 years; 84--91\% remain undiagnosed.
\end{description}

\subsection{Cognitive Impairment}

\begin{description}
    \item[Full Citation:] Sebaiti MA, Hainselin M, Gounden Y, et al.\ Systematic review and meta-analysis of cognitive impairment in myalgic encephalomyelitis/chronic fatigue syndrome (ME/CFS). \textit{Scientific Reports}. 2022;12(1):2157.
    \item[DOI:] \href{https://doi.org/10.1038/s41598-021-04764-w}{10.1038/s41598-021-04764-w}
    \item[PMID:] 35145174
    \item[Key Findings:] Large effect size for verbal working memory deficits; no significant difference in visual working memory.
\end{description}

\subsection{Long COVID and ME/CFS Overlap}

\begin{description}
    \item[Full Citation:] Wong TL, Weitzer DJ. Long COVID and Myalgic Encephalomyelitis/Chronic Fatigue Syndrome (ME/CFS)---A Systematic Review and Comparison of Clinical Presentation and Symptomatology. \textit{Medicina}. 2021;57(5):418.
    \item[DOI:] \href{https://doi.org/10.3390/medicina57050418}{10.3390/medicina57050418}
    \item[PMCID:] PMC8145228
\end{description}

\begin{description}
    \item[Full Citation:] The persistence of myalgic encephalomyelitis/chronic fatigue syndrome (ME/CFS) after SARS-CoV-2 infection: A systematic review and meta-analysis. \textit{Journal of Infection}. 2024;89(4):101231.
    \item[DOI:] \href{https://doi.org/10.1016/j.jinf.2024.106231}{10.1016/j.jinf.2024.106231}
    \item[PMID:] 39353473
    \item[Key Findings:] Approximately half of Long COVID patients fulfill ME/CFS diagnostic criteria.
\end{description}

\subsection{Sleep Abnormalities}

\begin{description}
    \item[Full Citation:] Baig S, Engelbrecht K, Engelbrecht F, et al.\ Objective sleep measures in chronic fatigue syndrome patients: A systematic review and meta-analysis. \textit{Sleep Medicine Reviews}. 2023;69:101775.
    \item[DOI:] \href{https://doi.org/10.1016/j.smrv.2023.101775}{10.1016/j.smrv.2023.101775}
    \item[PMID:] 37116254
    \item[PMCID:] PMC10281648
    \item[Sample:] 24 studies; 801 adults (426 ME/CFS, 375 controls); 477 adolescents
    \item[Key Findings:] Longer sleep latency, reduced sleep efficiency, longer REM latency, altered sleep microstructure.
\end{description}

\begin{description}
    \item[Full Citation:] Maksoud R, du Preez S, Eaton-Fitch N, et al.\ Systematic Review of Sleep Characteristics in Myalgic Encephalomyelitis/Chronic Fatigue Syndrome. \textit{Healthcare}. 2021;9(5):568.
    \item[DOI:] \href{https://doi.org/10.3390/healthcare9050568}{10.3390/healthcare9050568}
    \item[PMCID:] PMC8150292
\end{description}

\subsection{Evidence Mapping}

\begin{description}
    \item[Full Citation:] Toogood PL, Clauw DJ, Engel CC, et al.\ Recent research in myalgic encephalomyelitis/chronic fatigue syndrome: an evidence map. \textit{BMC Medicine}. 2025;23(1):134.
    \item[PMCID:] PMC11973615
    \item[Scope:] Mapping ME/CFS evidence from 2018--2023.
\end{description}

% =============================================================================
\section{Pathophysiology: Immune Dysfunction}
\label{sec:bib-immune-dysfunction}
% =============================================================================

\subsection{Autoantibodies and G-Protein Coupled Receptors}

\begin{description}
    \item[Full Citation:] Wirth K, Scheibenbogen C. Autoantibodies to Vasoregulative G-Protein-Coupled Receptors Correlate with Symptom Severity, Autonomic Dysfunction and Disability in Myalgic Encephalomyelitis/Chronic Fatigue Syndrome. \textit{Journal of Clinical Medicine}. 2021;10(16):3675.
    \item[DOI:] \href{https://doi.org/10.3390/jcm10163675}{10.3390/jcm10163675}
    \item[PMID:] 34441971
    \item[PMCID:] PMC8397061
    \item[Key Findings:] Anti-$\beta$2, M3, M4 receptor antibodies elevated; correlate with fatigue and muscle pain severity.
\end{description}

\begin{description}
    \item[Full Citation:] M\"uller JA, Subburayalu J, Winkler F, et al.\ Dysregulated autoantibodies targeting vaso- and immunoregulatory receptors in Post COVID Syndrome correlate with symptom severity. \textit{Frontiers in Immunology}. 2022;13:981532.
    \item[DOI:] \href{https://doi.org/10.3389/fimmu.2022.981532}{10.3389/fimmu.2022.981532}
\end{description}

\begin{description}
    \item[Full Citation:] Stein E, Heindrich C, Wittke K, et al.\ Efficacy of repeated immunoadsorption in patients with post-COVID myalgic encephalomyelitis/chronic fatigue syndrome and elevated $\beta$2-adrenergic receptor autoantibodies: a prospective cohort study. \textit{The Lancet Regional Health -- Europe}. 2024;46:101330.
    \item[DOI:] \href{https://doi.org/10.1016/j.lanepe.2024.101330}{10.1016/j.lanepe.2024.101330}
    \item[Significance:] Demonstrates therapeutic potential of immunoadsorption targeting autoantibodies.
\end{description}

\subsection{Immune Exhaustion and Chronic Activation}

\begin{description}
    \item[Full Citation:] Immune exhaustion in ME/CFS and long COVID. \textit{JCI Insight}. 2024;9(19):e183810.
    \item[DOI:] \href{https://doi.org/10.1172/jci.insight.183810}{10.1172/jci.insight.183810}
\end{description}

\subsection{Comprehensive Immune Reviews}

\begin{description}
    \item[Full Citation:] Komaroff AL, Lipkin WI. ME/CFS and Long COVID share similar symptoms and biological abnormalities: road map to the literature. \textit{Frontiers in Medicine}. 2023;10:1187163.
    \item[DOI:] \href{https://doi.org/10.3389/fmed.2023.1187163}{10.3389/fmed.2023.1187163}
    \item[PMCID:] PMC10278546
    \item[Significance:] Comprehensive comparison of ME/CFS and Long COVID biological abnormalities.
\end{description}

\begin{description}
    \item[Full Citation:] Komaroff AL, Lipkin WI. Myalgic Encephalomyelitis/Chronic Fatigue Syndrome: the biology of a neglected disease. \textit{Frontiers in Immunology}. 2024;15:1386607.
    \item[DOI:] \href{https://doi.org/10.3389/fimmu.2024.1386607}{10.3389/fimmu.2024.1386607}
    \item[PMCID:] PMC11180809
\end{description}

% =============================================================================
\section{Pathophysiology: Neurological Abnormalities}
\label{sec:bib-neurological}
% =============================================================================

\subsection{Neuroinflammation}

\begin{description}
    \item[Full Citation:] Nakatomi Y, Mizuno K, Ishii A, et al.\ Neuroinflammation in Patients with Chronic Fatigue Syndrome/Myalgic Encephalomyelitis: An $^{11}$C-(R)-PK11195-PET Study. \textit{Journal of Nuclear Medicine}. 2014;55(6):945--950.
    \item[DOI:] \href{https://doi.org/10.2967/jnumed.113.131045}{10.2967/jnumed.113.131045}
    \item[PMID:] 24665088
    \item[Key Findings:] PET imaging demonstrates widespread neuroinflammation correlating with symptom severity.
\end{description}

\begin{description}
    \item[Full Citation:] Renz-Polster H, Tremblay M-E, Engel D, Scheibenbogen C, Brehm JU. Molecular Mechanisms of Neuroinflammation in ME/CFS and Long COVID to Sustain Disease and Promote Relapses. \textit{Frontiers in Neurology}. 2022;13:877772.
    \item[DOI:] \href{https://doi.org/10.3389/fneur.2022.877772}{10.3389/fneur.2022.877772}
\end{description}

\subsection{Neuroimaging Reviews}

\begin{description}
    \item[Full Citation:] Shan ZY, Barnden LR, Kwiatek RA, Bhuta S, Groszmann M, Blumbergs PC. Neuroimaging characteristics of myalgic encephalomyelitis/chronic fatigue syndrome (ME/CFS): a systematic review. \textit{Journal of Translational Medicine}. 2020;18(1):335.
    \item[DOI:] \href{https://doi.org/10.1186/s12967-020-02506-6}{10.1186/s12967-020-02506-6}
    \item[Key Findings:] Evidence for structural, functional, and metabolic brain abnormalities; hypoperfusion in multiple regions.
\end{description}

\begin{description}
    \item[Full Citation:] Metabolic neuroimaging of myalgic encephalomyelitis/chronic fatigue syndrome and Long-COVID. \textit{Immunometabolism}. 2025;10:e00068.
    \item[DOI:] \href{https://doi.org/10.1097/IN9.0000000000000068}{10.1097/IN9.0000000000000068}
\end{description}

\subsection{Brainstem and Autonomic Dysfunction}

\begin{description}
    \item[Full Citation:] van Campen CLMC, Rowe PC, Visser FC. Similar Patterns of Dysautonomia in Myalgic Encephalomyelitis/Chronic Fatigue and Post-COVID-19 Syndromes. \textit{Pathophysiology}. 2024;31(1):1--17.
    \item[DOI:] \href{https://doi.org/10.3390/pathophysiology31010001}{10.3390/pathophysiology31010001}
    \item[PMCID:] PMC10801610
\end{description}

\begin{description}
    \item[Full Citation:] Wells R, Spurrier AJ, Linz D, et al.\ Is postural orthostatic tachycardia syndrome (POTS) a central nervous system disorder? \textit{Journal of Neurology, Neurosurgery \& Psychiatry}. 2021;92(11):1196--1207.
    \item[DOI:] \href{https://doi.org/10.1136/jnnp-2020-325932}{10.1136/jnnp-2020-325932}
    \item[PMCID:] PMC7936931
\end{description}

\begin{description}
    \item[Full Citation:] Dysautonomia and small fiber neuropathy in post-COVID condition and Chronic Fatigue Syndrome. \textit{Journal of Neurology}. 2024;271(1):40--48.
    \item[PMCID:] PMC10648633
\end{description}

% =============================================================================
\section{Pathophysiology: Metabolic and Mitochondrial Dysfunction}
\label{sec:bib-metabolic}
% =============================================================================

\subsection{Mitochondrial Dysfunction}

\begin{description}
    \item[Full Citation:] Holden S, Maksoud R, Eaton-Fitch N, et al.\ Mitochondrial Dysfunction in Myalgic Encephalomyelitis/Chronic Fatigue Syndrome. \textit{Physiology}. 2025;40(2):89--102.
    \item[DOI:] \href{https://doi.org/10.1152/physiol.00056.2024}{10.1152/physiol.00056.2024}
    \item[PMCID:] PMC12151296
    \item[Key Topics:] Impaired oxidative phosphorylation, reduced ATP production, WASF3 dysregulation.
\end{description}

\begin{description}
    \item[Full Citation:] Morris G, Maes M. Mitochondrial dysfunctions in myalgic encephalomyelitis/chronic fatigue syndrome explained by activated immuno-inflammatory, oxidative and nitrosative stress pathways. \textit{Metabolic Brain Disease}. 2014;29(1):19--36.
    \item[DOI:] \href{https://doi.org/10.1007/s11011-013-9435-x}{10.1007/s11011-013-9435-x}
    \item[PMID:] 24557875
\end{description}

\begin{description}
    \item[Full Citation:] Myhill S, Booth NE, McLaren-Howard J. Chronic fatigue syndrome and mitochondrial dysfunction. \textit{International Journal of Clinical and Experimental Medicine}. 2009;2(1):1--16.
    \item[PMCID:] PMC2680051
\end{description}

\subsection{Metabolomics}

\begin{description}
    \item[Full Citation:] Baraniuk JN, Kern G, Engel S, Engel G. Cerebrospinal fluid metabolomics, lipidomics and serine pathway dysfunction in myalgic encephalomyelitis/chronic fatigue syndrome (ME/CFS). \textit{Scientific Reports}. 2025;15(1):6789.
    \item[DOI:] \href{https://doi.org/10.1038/s41598-025-91324-1}{10.1038/s41598-025-91324-1}
    \item[PMCID:] PMC11873053
    \item[Key Findings:] Elevated serine, reduced 5-MTHF in CSF; altered phospholipid synthesis.
\end{description}

\begin{description}
    \item[Full Citation:] Naviaux RK, Naviaux JC, Li K, et al.\ Metabolic features of chronic fatigue syndrome. \textit{Proceedings of the National Academy of Sciences}. 2016;113(37):E5472--E5480.
    \item[DOI:] \href{https://doi.org/10.1073/pnas.1607571113}{10.1073/pnas.1607571113}
    \item[Key Findings:] Chemical signature with approximately 40 metabolic abnormalities; hypometabolic state.
\end{description}

\begin{description}
    \item[Full Citation:] Germain A, Barupal DK, Levine SM. Comprehensive Circulatory Metabolomics in ME/CFS Reveals Disrupted Metabolism of Acyl Lipids and Steroids. \textit{Metabolites}. 2020;10(1):34.
    \item[DOI:] \href{https://doi.org/10.3390/metabo10010034}{10.3390/metabo10010034}
    \item[PMID:] 31947545
    \item[Key Findings:] Acyl cholines consistently reduced across cohorts.
\end{description}

% =============================================================================
\section{Pathophysiology: Gut Microbiome}
\label{sec:bib-microbiome}
% =============================================================================

\begin{description}
    \item[Full Citation:] Lupo GFD, Rocchetti G, Lucini L, et al.\ Potential role of microbiome in Chronic Fatigue Syndrome/Myalgic Encephalomyelitis (CFS/ME). \textit{Scientific Reports}. 2021;11(1):7043.
    \item[DOI:] \href{https://doi.org/10.1038/s41598-021-86425-6}{10.1038/s41598-021-86425-6}
\end{description}

\begin{description}
    \item[Full Citation:] Giloteaux L, Goodrich JK, Walters WA, Levine SM, Ley RE, Hanson MR. Reduced diversity and altered composition of the gut microbiome in individuals with myalgic encephalomyelitis/chronic fatigue syndrome. \textit{Microbiome}. 2016;4(1):30.
    \item[DOI:] \href{https://doi.org/10.1186/s40168-016-0171-4}{10.1186/s40168-016-0171-4}
    \item[Key Findings:] Reduced \textit{Faecalibacterium prausnitzii} and \textit{Eubacterium rectale} (butyrate producers).
\end{description}

\begin{description}
    \item[Full Citation:] K\"onig RS, Albrich WC, Kahlert CR, et al.\ The Gut Microbiome in Myalgic Encephalomyelitis (ME)/Chronic Fatigue Syndrome (CFS). \textit{Frontiers in Immunology}. 2022;12:628741.
    \item[DOI:] \href{https://doi.org/10.3389/fimmu.2021.628741}{10.3389/fimmu.2021.628741}
    \item[PMCID:] PMC8761622
\end{description}

\begin{description}
    \item[Full Citation:] Varesi A, Campagnoli LIM, Frasca A, et al.\ The gastrointestinal microbiota in the development of ME/CFS: a critical view and potential perspectives. \textit{Frontiers in Immunology}. 2024;15:1352744.
    \item[DOI:] \href{https://doi.org/10.3389/fimmu.2024.1352744}{10.3389/fimmu.2024.1352744}
\end{description}

\begin{description}
    \item[Full Citation:] Ciregia F, Rahmania F, Semenova-Ziga V, Ortega-Molina M, Rodrigues M, Gonzalez-Lopez E. Increased gut permeability and bacterial translocation are associated with fibromyalgia and myalgic encephalomyelitis/chronic fatigue syndrome: implications for disease-related biomarker discovery. \textit{Frontiers in Immunology}. 2023;14:1253121.
    \item[DOI:] \href{https://doi.org/10.3389/fimmu.2023.1253121}{10.3389/fimmu.2023.1253121}
    \item[Key Findings:] Elevated markers of gut permeability and bacterial translocation.
\end{description}

% =============================================================================
\section{Pathophysiology: Viral Persistence and Reactivation}
\label{sec:bib-viral}
% =============================================================================

\begin{description}
    \item[Full Citation:] Rasa S, Nora-Krukle Z, Henning N, et al.\ Chronic viral infections in myalgic encephalomyelitis/chronic fatigue syndrome (ME/CFS). \textit{Journal of Translational Medicine}. 2018;16(1):268.
    \item[DOI:] \href{https://doi.org/10.1186/s12967-018-1644-y}{10.1186/s12967-018-1644-y}
    \item[PMCID:] PMC6167797
    \item[Viruses Covered:] EBV, HHV-6, CMV, enteroviruses, B19V.
\end{description}

\begin{description}
    \item[Full Citation:] Williams MV, Cox B, Ariza ME. Chronic Reactivation of Persistent Human Herpesviruses EBV, HHV-6 and VZV and Heightened Anti-dUTPase IgG Antibodies Are a Recurrent Hallmark in Post-Infectious ME/CFS and is Associated With Fatigue. \textit{Frontiers in Immunology}. 2025;(in press).
    \item[PMID:] 41451845
    \item[Key Findings:] 72.5\% of ME/CFS patients have antibodies to multiple herpesvirus dUTPases vs 31\% controls.
\end{description}

\begin{description}
    \item[Full Citation:] Kasimir F, Toomey D, Liu Z, et al.\ Tissue specific signature of HHV-6 infection in ME/CFS. \textit{Frontiers in Molecular Biosciences}. 2022;9:1044964.
    \item[DOI:] \href{https://doi.org/10.3389/fmolb.2022.1044964}{10.3389/fmolb.2022.1044964}
    \item[PMCID:] PMC9795011
    \item[Key Findings:] Viral miRNA detected in brain and spinal cord tissue only in ME/CFS patients.
\end{description}

\begin{description}
    \item[Full Citation:] Ruiz-Pab\'on JF, Montoya JG, Lupo J, Epstein-Barr Virus and the Origin of Myalgic Encephalomyelitis or Chronic Fatigue Syndrome. \textit{Frontiers in Immunology}. 2021;12:656797.
    \item[DOI:] \href{https://doi.org/10.3389/fimmu.2021.656797}{10.3389/fimmu.2021.656797}
    \item[PMCID:] PMC8634673
\end{description}

\begin{description}
    \item[Full Citation:] Ruiz-Pab\'on JF, Henao E, Pinto F, Estrada S, Corredor V. Epstein--Barr virus-acquired immunodeficiency in myalgic encephalomyelitis---Is it present in long COVID? \textit{Journal of Translational Medicine}. 2023;21:633.
    \item[DOI:] \href{https://doi.org/10.1186/s12967-023-04515-7}{10.1186/s12967-023-04515-7}
\end{description}

% =============================================================================
\section{Pathophysiology: Genetics and Epigenetics}
\label{sec:bib-genetics}
% =============================================================================

\begin{description}
    \item[Full Citation:] de Vega WC, Vernon SD, McGowan PO. Identification of Myalgic Encephalomyelitis/Chronic Fatigue Syndrome-associated DNA methylation patterns. \textit{PLOS ONE}. 2018;13(7):e0201066.
    \item[DOI:] \href{https://doi.org/10.1371/journal.pone.0201066}{10.1371/journal.pone.0201066}
    \item[Key Findings:] 17,296 differentially methylated CpG sites; 307 differentially methylated promoters; immune-related pathways.
\end{description}

\begin{description}
    \item[Full Citation:] de Vega WC, Herber S, Ghaseminejad Tafreshi M, et al.\ Epigenetic modifications and glucocorticoid sensitivity in Myalgic Encephalomyelitis/Chronic Fatigue Syndrome (ME/CFS). \textit{BMC Medical Genomics}. 2017;10(1):11.
    \item[DOI:] \href{https://doi.org/10.1186/s12920-017-0248-3}{10.1186/s12920-017-0248-3}
\end{description}

\begin{description}
    \item[Full Citation:] Wang T, Yin J, Miller AH, Xiao C. Genetic risk factors for ME/CFS identified using combinatorial analysis. \textit{Journal of Translational Medicine}. 2022;20:598.
    \item[DOI:] \href{https://doi.org/10.1186/s12967-022-03815-8}{10.1186/s12967-022-03815-8}
    \item[Key Findings:] 199 SNPs in 14 genes associated with 91\% of ME/CFS cases.
\end{description}

\begin{description}
    \item[Full Citation:] Dissecting the genetic complexity of myalgic encephalomyelitis/chronic fatigue syndrome via deep learning-powered genome analysis. \textit{Nature Communications}. 2025.
    \item[PMCID:] PMC12047926
    \item[Key Findings:] 115 ME/CFS-risk genes identified; intolerance to loss-of-function mutations.
\end{description}

\begin{description}
    \item[Full Citation:] Trivedi MS, Oltra E, Engelbrecht B, et al.\ Recursive ensemble feature selection provides a robust mRNA expression signature for myalgic encephalomyelitis/chronic fatigue syndrome. \textit{Scientific Reports}. 2021;11(1):4541.
    \item[DOI:] \href{https://doi.org/10.1038/s41598-021-83660-9}{10.1038/s41598-021-83660-9}
\end{description}

% =============================================================================
\section{Exercise Physiology and Post-Exertional Malaise}
\label{sec:bib-exercise}
% =============================================================================

\begin{description}
    \item[Full Citation:] Franklin JD, Graham M, the Workwell Foundation. The Prospects of the Two-Day Cardiopulmonary Exercise Test (CPET) in ME/CFS Patients: A Meta-Analysis. \textit{International Journal of Environmental Research and Public Health}. 2020;17(24):9575.
    \item[DOI:] \href{https://doi.org/10.3390/ijerph17249575}{10.3390/ijerph17249575}
    \item[PMCID:] PMC7765094
    \item[Key Findings:] Day 2 CPET shows decreased VO$_2$max and workload unique to ME/CFS.
\end{description}

\begin{description}
    \item[Full Citation:] Stevens S, Snell C, Stevens J, Keller B, VanNess JM. Cardiopulmonary Exercise Test Methodology for Assessing Exertion Intolerance in Myalgic Encephalomyelitis/Chronic Fatigue Syndrome. \textit{Frontiers in Pediatrics}. 2018;6:242.
    \item[DOI:] \href{https://doi.org/10.3389/fped.2018.00242}{10.3389/fped.2018.00242}
\end{description}

\begin{description}
    \item[Full Citation:] Two-day cardiopulmonary exercise testing in long COVID post-exertional malaise diagnosis. \textit{Respiratory Medicine and Research}. 2024;85:101551.
    \item[DOI:] \href{https://doi.org/10.1016/j.resmer.2024.101551}{10.1016/j.resmer.2024.101551}
\end{description}

\begin{description}
    \item[Full Citation:] Recovery time from two-day CPET in ME/CFS. Cornell Center for Enervating NeuroImmune Disease. 2024.
    \item[URL:] \url{https://neuroimmune.cornell.edu/news/recovery-from-two-day-cpet-in-me-cfs/}
    \item[Key Findings:] Recovery $\sim$13 days in ME/CFS vs $\sim$2 days in sedentary controls.
\end{description}

% =============================================================================
\section{Treatment Evidence}
\label{sec:bib-treatment}
% =============================================================================

\subsection{Graded Exercise Therapy (Negative Evidence)}

\begin{description}
    \item[Full Citation:] Geraghty K, Hann M, Kurtev S. The Updated NICE Guidance Exposed the Serious Flaws in CBT and Graded Exercise Therapy Trials for ME/CFS. \textit{Healthcare}. 2022;10(5):898.
    \item[DOI:] \href{https://doi.org/10.3390/healthcare10050898}{10.3390/healthcare10050898}
    \item[PMCID:] PMC9141828
    \item[Key Findings:] Methodological flaws and biases in trials; patient surveys show harm from GET.
\end{description}

\begin{description}
    \item[Full Citation:] Vink M, Vink-Niese A. The PACE Trial's GET Manual for Therapists Exposes the Fixed Incremental Nature of Graded Exercise Therapy for ME/CFS. \textit{Life}. 2025;15(4):584.
    \item[DOI:] \href{https://doi.org/10.3390/life15040584}{10.3390/life15040584}
\end{description}

\begin{description}
    \item[Full Citation:] Vink M, Vink-Niese A. Graded exercise therapy does not restore the ability to work in ME/CFS -- Rethinking of a Cochrane review. \textit{Work}. 2020;66(2):283--308.
    \item[DOI:] \href{https://doi.org/10.3233/WOR-203174}{10.3233/WOR-203174}
    \item[PMID:] 32568149
\end{description}

\subsection{Pacing and Energy Management}

\begin{description}
    \item[Full Citation:] Goudsmit EM, Nijs J, Jason LA, Wallman KE. A scoping review of `Pacing' for management of Myalgic Encephalomyelitis/Chronic Fatigue Syndrome (ME/CFS): lessons learned for the long COVID pandemic. \textit{Journal of Translational Medicine}. 2023;21:738.
    \item[DOI:] \href{https://doi.org/10.1186/s12967-023-04586-6}{10.1186/s12967-023-04586-6}
    \item[PMCID:] PMC10576275
\end{description}

\begin{description}
    \item[Full Citation:] Jason LA, Brown M, Brown A, et al.\ Energy Conservation/Envelope Theory Interventions to Help Patients with Myalgic Encephalomyelitis/Chronic Fatigue Syndrome. \textit{Fatigue: Biomedicine, Health \& Behavior}. 2013;1(1--2):65--78.
    \item[DOI:] \href{https://doi.org/10.1080/21641846.2012.733602}{10.1080/21641846.2012.733602}
    \item[PMCID:] PMC3596172
\end{description}

\subsection{Patient-Reported Treatment Outcomes}

\begin{description}
    \item[Full Citation:] Davis HE, McCorkell L, Vogel JM, et al.\ Patient-reported treatment outcomes in ME/CFS and long COVID. \textit{Proceedings of the National Academy of Sciences}. 2025;122(26):e2426874122.
    \item[DOI:] \href{https://doi.org/10.1073/pnas.2426874122}{10.1073/pnas.2426874122}
    \item[PMCID:] PMC12280984
    \item[Sample:] $>$3,900 patients
    \item[Key Findings:] Treatment responses highly correlated ($R^2$=0.68) between ME/CFS and Long COVID.
\end{description}

% =============================================================================
\section{Long COVID and ME/CFS Overlap}
\label{sec:bib-long-covid}
% =============================================================================

\begin{description}
    \item[Full Citation:] Thapaliya K, Marshall-Gradisnik S, Barber PA, Eaton-Fitch N, Staines D. Unravelling shared mechanisms: insights from recent ME/CFS research to illuminate long COVID pathologies. \textit{Trends in Molecular Medicine}. 2024;30(5):443--458.
    \item[DOI:] \href{https://doi.org/10.1016/j.molmed.2024.02.003}{10.1016/j.molmed.2024.02.003}
    \item[PMID:] 38443223
\end{description}

\begin{description}
    \item[Full Citation:] Mapping the complexity of ME/CFS: Evidence for abnormal energy metabolism, altered immune profile, and vascular dysfunction. \textit{Cell Reports Medicine}. 2025;6(12):101587.
    \item[DOI:] \href{https://doi.org/10.1016/j.xcrm.2025.101587}{10.1016/j.xcrm.2025.101587}
\end{description}

% =============================================================================
\section{Historical Background and Epidemics}
\label{sec:bib-history}
% =============================================================================

\begin{description}
    \item[Full Citation:] Underhill RA. Myalgic encephalomyelitis, chronic fatigue syndrome: An infectious disease. \textit{Medical Hypotheses}. 2015;85(6):765--773.
    \item[DOI:] \href{https://doi.org/10.1016/j.mehy.2015.10.011}{10.1016/j.mehy.2015.10.011}
    \item[Topics:] Historical outbreaks from 1934 onwards.
\end{description}

\begin{description}
    \item[Full Citation:] Underhill RA, O'Gorman R. The viral origin of myalgic encephalomyelitis/chronic fatigue syndrome. \textit{Journal of the Royal Society of Medicine}. 2023;116(8):269--282.
    \item[DOI:] \href{https://doi.org/10.1177/01410768231176937}{10.1177/01410768231176937}
    \item[PMCID:] PMC10434940
\end{description}

\begin{description}
    \item[Full Citation:] Brurberg KG, Fønhus MS, Larun L, Flottorp S, Malterud K. Myalgic Encephalomyelitis/Chronic Fatigue Syndrome: Organic Disease or Psychosomatic Illness? A Re-Examination of the Royal Free Epidemic of 1955. \textit{Medicina}. 2021;57(1):12.
    \item[DOI:] \href{https://doi.org/10.3390/medicina57010012}{10.3390/medicina57010012}
    \item[PMID:] 33375343
    \item[Key Findings:] First-hand accounts confirm organic infectious disease, not hysteria.
\end{description}

\begin{description}
    \item[Full Citation:] Jason LA, Lapp CW, Engel S, et al.\ Myalgic Encephalomyelitis (ME) outbreaks can be modelled as an infectious disease: a mathematical reconsideration of the Royal Free Epidemic of 1955. \textit{Fatigue: Biomedicine, Health \& Behavior}. 2020;8(2):99--109.
    \item[DOI:] \href{https://doi.org/10.1080/21641846.2020.1793058}{10.1080/21641846.2020.1793058}
\end{description}

% =============================================================================
\section{Research Roadmaps and Policy Documents}
\label{sec:bib-policy}
% =============================================================================

\begin{description}
    \item[Full Citation:] National Institute of Neurological Disorders and Stroke. Report of the ME/CFS Research Roadmap Working Group of Council. Bethesda, MD: NINDS; May 15, 2024.
    \item[URL:] \url{https://www.ninds.nih.gov/sites/default/files/2024-05/Report\%20of\%20the\%20MECFS\%20Research\%20Roadmap\%20Working\%20Group\%20of\%20Council_508C.pdf}
    \item[Significance:] Official NIH research priorities and funding recommendations.
\end{description}

\begin{description}
    \item[Full Citation:] Reframing Myalgic Encephalomyelitis/Chronic Fatigue Syndrome (ME/CFS): Biological Basis of Disease and Recommendations for Supporting Patients. 2025.
    \item[PMCID:] PMC12346739
\end{description}

% =============================================================================
\section{Comprehensive Reviews}
\label{sec:bib-comprehensive-reviews}
% =============================================================================

\begin{description}
    \item[Full Citation:] Cortes Rivera M, Mastronardi C, Silva-Aldana CT, Arcos-Burgos M, Lidbury BA. Myalgic Encephalomyelitis/Chronic Fatigue Syndrome: A Comprehensive Review. \textit{Diagnostics}. 2019;9(3):91.
    \item[DOI:] \href{https://doi.org/10.3390/diagnostics9030091}{10.3390/diagnostics9030091}
    \item[PMCID:] PMC6787585
\end{description}

% =============================================================================
\section{Additional Key Resources}
\label{sec:bib-resources}
% =============================================================================

\subsection{Patient Advocacy and Information}

\begin{description}
    \item[MEpedia:] \url{https://me-pedia.org/} --- Comprehensive patient-edited wiki on ME/CFS.
    \item[ME Association (UK):] \url{https://meassociation.org.uk/} --- Patient support and research summaries.
    \item[Bateman Horne Center:] \url{https://batemanhornecenter.org/} --- Clinical and educational resources.
    \item[Open Medicine Foundation:] \url{https://www.openmedicinefoundation.ngo/} --- Research funding and updates.
    \item[Solve ME/CFS Initiative:] \url{https://solvecfs.org/} --- US-based research and advocacy.
\end{description}

\subsection{Research Centers}

\begin{description}
    \item[Cornell Center for Enervating NeuroImmune Disease:] \url{https://neuroimmune.cornell.edu/}
    \item[Griffith University National Centre for Neuroimmunology and Emerging Diseases:] Queensland, Australia
    \item[Charit\'e Fatigue Center:] Berlin, Germany
    \item[Stanford ME/CFS Initiative:] Stanford University, California
\end{description}

\vspace{1cm}
\begin{center}
\rule{0.5\textwidth}{0.4pt}
\end{center}
\vspace{0.5cm}

\noindent\textit{Note: This bibliography was compiled in January 2025. The field of ME/CFS research is rapidly evolving, particularly with insights from Long COVID research. Readers are encouraged to search PubMed and preprint servers for the most current literature.}