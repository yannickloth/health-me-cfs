% FILE: Personal clinical findings and test results — diagnostic tests, clinical assessments, objective findings
\chapter{Clinical Findings \& Medical History}
\label{app:clinical-findings}

This appendix documents objective clinical data, laboratory findings, and medical history. For symptom descriptions, see Appendix~\ref{app:personal-symptoms}. For current management protocols, see Appendix~\ref{app:medical-management}.

% DOCUMENTED CLINICAL FINDINGS
%%%%%%%%%%%%%%%%%%%%%%%%%%%%%%%%%%%%%%%%%%%%%%%%%%%%%%%%%%%%%%%%%%%%%%%%%%%%%%%

\section{Documented Clinical Findings}
\label{sec:documented-findings}

This section records objective clinical data from medical records, laboratory tests, and specialist evaluations.

\subsection{Laboratory Findings (2025)}
\label{subsec:lab-findings-2025}

\subsubsection{Hematology and Iron Status}

\begin{table}[htbp]
\centering
\caption{Iron Status and Hematology (2025)}
\label{tab:iron-status}
\begin{tabular}{lccl}
\toprule
\textbf{Parameter} & \textbf{Result} & \textbf{Reference} & \textbf{Clinical Note} \\
\midrule
Hemoglobin & 15.6 g/dL & 13.5--17.6 & Normal \\
Ferritin & 40--55 $\mu$g/L & 20--300 & \textbf{Suboptimal for ME/CFS} \\
Iron & 107 $\mu$g/dL & 65--175 & Normal \\
Transferrin & 3.12 g/L & 1.74--3.64 & Normal \\
Transferrin saturation & 25\% & 15--50 & Normal \\
Vitamin B12 & 383--424 ng/L & 187--883 & Normal \\
Folate & 2.8--4.2 $\mu$g/L & 2.3--17.6 & Low-normal \\
\bottomrule
\end{tabular}
\end{table}

\paragraph{Ferritin Interpretation.}
While ferritin 40--55 $\mu$g/L falls within the standard reference range, a consulting somnologist specifically noted: \emph{``Un taux supérieur à 70--75 $\mu$g/L est recommandé''} in the context of periodic limb movements during sleep. This target is also recommended for ME/CFS patients given iron's role in:
\begin{itemize}
    \item Dopamine synthesis (tyrosine hydroxylase cofactor)
    \item Mitochondrial electron transport chain (cytochromes)
    \item Restless legs syndrome management
\end{itemize}

\subsubsection{Immune and Inflammatory Markers}

\begin{table}[htbp]
\centering
\caption{Immune Markers (October--November 2025)}
\label{tab:immune-markers}
\begin{tabular}{lccl}
\toprule
\textbf{Parameter} & \textbf{Result} & \textbf{Reference} & \textbf{Clinical Note} \\
\midrule
\multicolumn{4}{l}{\textit{Rheumatoid markers}} \\
Rheumatoid Factor & 119--176 IU/mL & $<$14--20 & \textbf{Strongly positive} \\
Anti-CCP & $<$0.8 U/mL & $<$7 & Negative \\
ANA & Negative & $<$1/80 & Normal \\
\midrule
\multicolumn{4}{l}{\textit{Inflammation}} \\
CRP & 1.6--3.6 mg/L & $<$5--8.5 & Normal \\
\midrule
\multicolumn{4}{l}{\textit{Complement}} \\
C3 & 1.39--1.49 g/L & 0.82--1.85 & Normal \\
C4 & 0.39--0.42 g/L & 0.10--0.53 & Upper normal \\
\midrule
\multicolumn{4}{l}{\textit{Immunoglobulins}} \\
IgG & 14.4 g/L & 5.40--18.22 & Normal \\
IgA & 2.80 g/L & 0.63--4.84 & Normal \\
IgM & 0.95 g/L & 0.22--2.40 & Normal \\
\bottomrule
\end{tabular}
\end{table}

\paragraph{Rheumatoid Factor Interpretation.}
The strongly elevated RF (119--176 IU/mL) with \textbf{negative} Anti-CCP effectively rules out rheumatoid arthritis. Elevated RF without Anti-CCP occurs in:
\begin{itemize}
    \item Chronic infections (including post-viral states)
    \item Other autoimmune conditions
    \item ME/CFS (non-specific immune activation)
    \item Healthy individuals (false positive, especially older adults)
\end{itemize}
The negative ANA further argues against systemic autoimmune disease.

\subsubsection{Viral Serology}

\begin{table}[htbp]
\centering
\caption{Viral Serology (October 2025)}
\label{tab:viral-serology}
\begin{tabular}{lccl}
\toprule
\textbf{Virus} & \textbf{IgG} & \textbf{IgM} & \textbf{Interpretation} \\
\midrule
EBV (VCA) & $>$750 U/mL & Negative & Past infection, very high titer \\
Parvovirus B19 & 61.0 U/mL & Negative & Past infection \\
CMV & 0.9 U/mL & Negative & No exposure \\
Hepatitis B & Negative & --- & No infection/immunity \\
Hepatitis C & Negative & --- & No infection \\
Toxoplasmosis & $<$0.5 UI/mL & Negative & No exposure \\
Borrelia (Lyme) & 6.7 U/mL & Negative & No infection \\
Bartonella & 1/64 & Negative & At detection threshold \\
\bottomrule
\end{tabular}
\end{table}

\paragraph{EBV Interpretation.}
The very high EBV VCA IgG ($>$750 U/mL) indicates past EBV infection with robust antibody response. EBV is one of the most common triggers for ME/CFS. The high titer suggests either:
\begin{itemize}
    \item Strong initial immune response to past infection
    \item Possible ongoing low-level viral reactivation
    \item Persistent immune stimulation from EBV antigens
\end{itemize}
This finding supports the post-infectious etiology model for ME/CFS.

\subsubsection{Metabolic Panel}

\begin{table}[htbp]
\centering
\caption{Metabolic Parameters (2025)}
\label{tab:metabolic-panel}
\begin{tabular}{lccl}
\toprule
\textbf{Parameter} & \textbf{Result} & \textbf{Reference} & \textbf{Clinical Note} \\
\midrule
\multicolumn{4}{l}{\textit{Glucose metabolism}} \\
Fasting glucose & 104 mg/dL & 70--100 & Impaired fasting glucose \\
\midrule
\multicolumn{4}{l}{\textit{Lipids}} \\
Total cholesterol & 202--208 mg/dL & $<$190 & Elevated \\
LDL cholesterol & 132--137 mg/dL & $<$100 & Elevated \\
HDL cholesterol & 42--49 mg/dL & $>$40 & Low-normal \\
Triglycerides & 117--135 mg/dL & 40--150 & Normal \\
\midrule
\multicolumn{4}{l}{\textit{Liver}} \\
Total bilirubin & 1.52 mg/dL & 0.2--1.2 & Elevated (indirect) \\
Direct bilirubin & 0.45 mg/dL & 0--0.5 & Normal \\
AST/ALT & 31/40 U/L & 5--34/$<$55 & Normal \\
GGT & 23--26 U/L & 11--59 & Normal \\
\midrule
\multicolumn{4}{l}{\textit{Renal}} \\
Creatinine & 1.09--1.10 mg/dL & 0.72--1.25 & Normal \\
eGFR (EKFC) & 81--82 mL/min & 59--137 & Normal \\
\bottomrule
\end{tabular}
\end{table}

\paragraph{Fasting Glucose Interpretation.}
Fasting glucose of 104 mg/dL falls in the ``impaired fasting glucose'' range (100--125 mg/dL). In the context of ME/CFS, this may reflect:
\begin{itemize}
    \item Mitochondrial dysfunction affecting glucose metabolism
    \item Metabolic ``safe mode'' with altered fuel utilization
    \item Stress response/cortisol effects
    \item True early insulin resistance
\end{itemize}
Recommend HbA1c testing to assess longer-term glucose control.

\paragraph{Bilirubin Interpretation.}
Elevated total bilirubin (1.52 mg/dL) with normal direct bilirubin and liver enzymes suggests unconjugated hyperbilirubinemia. While this pattern is consistent with Gilbert syndrome, \textbf{no clinical symptoms have been observed}. This finding is of uncertain clinical significance and does not require treatment.

\subsubsection{Hormonal and Nutritional Status}

\begin{table}[htbp]
\centering
\caption{Hormonal and Nutritional Parameters (2025)}
\label{tab:hormonal-nutritional}
\begin{tabular}{lccl}
\toprule
\textbf{Parameter} & \textbf{Result} & \textbf{Reference} & \textbf{Clinical Note} \\
\midrule
\multicolumn{4}{l}{\textit{Thyroid}} \\
TSH & 2.10--2.51 mU/L & 0.3--4.2 & Normal \\
Free T4 & 11.6 pmol/L & 10.3--20.6 & Normal \\
\midrule
\multicolumn{4}{l}{\textit{Adrenal}} \\
Cortisol (morning) & 6.3 $\mu$g/dL & 7--25 & \textbf{Low-normal} \\
\midrule
\multicolumn{4}{l}{\textit{Gonadal}} \\
Testosterone & 469 ng/dL & 240--870 & Normal \\
\midrule
\multicolumn{4}{l}{\textit{Vitamins/Minerals}} \\
Vitamin D (25-OH) & 27--42 $\mu$g/L & 30--60 & Improved (was deficient) \\
Selenium & 78 $\mu$g/L & 60--120 & \textbf{Suboptimal} (rec.\ 90--143) \\
Zinc & 106 $\mu$g/dL & 60--130 & Suboptimal (rec.\ $>$110) \\
Calcium & 2.60 mmol/L & 2.10--2.55 & Slightly elevated \\
Magnesium & 0.92 mmol/L & 0.66--1.07 & Normal \\
\bottomrule
\end{tabular}
\end{table}

\paragraph{Cortisol Interpretation.}
Morning cortisol of 6.3 $\mu$g/dL is at the low end of the reference range (7--25 for morning). In ME/CFS, blunted cortisol awakening response and low-normal cortisol are common findings reflecting HPA axis dysfunction. This may contribute to:
\begin{itemize}
    \item Morning fatigue and difficulty waking
    \item Reduced stress tolerance
    \item Impaired inflammatory regulation
\end{itemize}

\subsubsection{Allergy Panel}

\begin{table}[htbp]
\centering
\caption{Allergy Testing (August 2025)}
\label{tab:allergy-panel}
\begin{tabular}{lcc}
\toprule
\textbf{Allergen Panel} & \textbf{Result (kUA/L)} & \textbf{Interpretation} \\
\midrule
Total IgE & 63 kU/L & Normal ($<$114) \\
Trees TX5 (alder, hazel, elm, willow, poplar) & 1.60 & Positive \\
Trees TX6 (maple, birch, beech, oak, walnut) & 2.11 & Positive \\
Grasses GX3 & 8.89 & \textbf{Strongly positive} \\
Feathers EX71 & $<$0.10 & Negative \\
Nuts FX1 (peanut, hazelnut, Brazil, almond, coconut) & 3.33 & Positive \\
Cat epithelium & $<$0.10 & Negative \\
Soy IgG & 88 mg/L & \textbf{Elevated} (ref $<$5) \\
\bottomrule
\end{tabular}
\end{table}

\subsection{Polysomnography Findings (December 2018)}
\label{subsec:psg-findings}

Full polysomnography with Multiple Sleep Latency Test (MSLT) performed at CHA Libramont, Sleep Laboratory, 07--08 December 2018.

\subsubsection{Patient Characteristics at Time of Study}

\begin{itemize}
    \item Age: 37 years
    \item Weight: 72 kg; Height: 175 cm; BMI: 23.5
    \item Chief complaint: \emph{``Fatigue présente depuis l'adolescence''} (fatigue since adolescence)
    \item No caffeine, no tobacco, no alcohol
    \item Physical activity: Swimming 4$\times$/week
    \item Chronotype: Evening type
    \item Sleep need: 8 hours + 1.5-hour nap
    \item Recently stopped Concerta (July 2018), gained 4 kg in 3 months
\end{itemize}

\subsubsection{Questionnaire Scores}

\begin{table}[htbp]
\centering
\caption{Sleep Questionnaire Results (2018 and 2021)}
\label{tab:sleep-questionnaires}
\begin{tabular}{lccc}
\toprule
\textbf{Scale} & \textbf{2018} & \textbf{2021} & \textbf{Interpretation} \\
\midrule
Epworth Sleepiness Scale & 16/24 & 14/24 & Pathological ($>$10) \\
Fatigue Severity Score & 4.5 & --- & Abnormal fatigue \\
Pichot Depression & --- & 10/13 & Mood disorder suggested \\
Goldberg Anxiety & --- & 6/7 & Anxiety disorder suggested \\
Insomnia Severity Index & --- & 18/28 & Moderate (16 pts daytime) \\
\bottomrule
\end{tabular}
\end{table}

\subsubsection{Nocturnal Polysomnography Results}

\begin{table}[htbp]
\centering
\caption{Polysomnography Parameters (December 2018)}
\label{tab:psg-results}
\begin{tabular}{lccc}
\toprule
\textbf{Parameter} & \textbf{Result} & \textbf{Normal} & \textbf{Assessment} \\
\midrule
\multicolumn{4}{l}{\textit{Sleep Duration}} \\
Time in bed & 518 min & --- & --- \\
Total sleep time (TST) & 429 min & --- & Normal \\
Sleep period & 515 min & --- & --- \\
\midrule
\multicolumn{4}{l}{\textit{Sleep Quality Indices}} \\
Sleep efficiency (TST/TRS) & 82.8\% & $>$86\% & \textbf{Reduced} \\
Sleep continuity (TST/TPS) & 83.3\% & $>$95\% & \textbf{Insufficient} \\
Sleep quality index (SWS+REM/TST) & 54.9\% & $>$35\% & Good \\
\midrule
\multicolumn{4}{l}{\textit{Sleep Architecture}} \\
N1 (light sleep) & 2 min (0.5\%) & 2--5\% & Low \\
N2 (intermediate) & 191 min (44.6\%) & 45--55\% & Normal \\
N3 (deep/SWS) & 141 min (32.8\%) & 15--33\% & Normal-high \\
REM sleep & 95 min (22.1\%) & 20--25\% & Normal \\
\midrule
\multicolumn{4}{l}{\textit{Sleep Fragmentation}} \\
Stage changes & 131 & --- & \textbf{Elevated} \\
WASO (wake after sleep onset) & 86 min & $<$30 min & \textbf{Excessive} \\
Number of awakenings & 25/night & --- & Elevated \\
Micro-arousal index & 6.1/h & $<$10/h & Normal \\
\midrule
\multicolumn{4}{l}{\textit{Sleep Latencies}} \\
Sleep onset latency & 13 min & $<$30 min & Normal \\
REM latency & 72 min & 70--120 min & Normal \\
\bottomrule
\end{tabular}
\end{table}

\subsubsection{Periodic Limb Movements}

\begin{table}[htbp]
\centering
\caption{Periodic Limb Movement Analysis}
\label{tab:plm-analysis}
\begin{tabular}{lcc}
\toprule
\textbf{Parameter} & \textbf{Result} & \textbf{Normal} \\
\midrule
PLM index (during sleep) & 13.3/h & $<$5/h \\
PLM index (during N1) & 30.0/h & --- \\
PLM index (during N2) & 10.7/h & --- \\
PLM index (during N3) & 11.9/h & --- \\
PLM duration (mean) & 10.2 sec & --- \\
\bottomrule
\end{tabular}
\end{table}

\paragraph{PLM Interpretation.}
The PLM index of 13.3/h is elevated (normal $<$5/h) and contributes to sleep fragmentation. The consulting somnologist specifically noted that ferritin $>$70--75 $\mu$g/L is recommended for patients with periodic limb movements.

\subsubsection{Respiratory Events}

\begin{table}[htbp]
\centering
\caption{Respiratory Analysis}
\label{tab:respiratory-analysis}
\begin{tabular}{lcc}
\toprule
\textbf{Parameter} & \textbf{Result} & \textbf{Interpretation} \\
\midrule
Apnea-Hypopnea Index (AHI) & 3.8/h & Normal ($<$5/h) \\
AHI in REM & 9.5/h & Mild \\
AHI supine & 7.7/h & Mild positional \\
Central apneas & 4 events & Minimal \\
Obstructive apneas & 3 events & Minimal \\
Obstructive hypopneas & 24 events & Predominant type \\
Mean SpO$_2$ & 95.9\% & Normal \\
Time SpO$_2$ $<$90\% & 0 min & Normal \\
\bottomrule
\end{tabular}
\end{table}

\paragraph{Respiratory Interpretation.}
Overall AHI is within normal limits. The study concluded: \emph{``L'analyse de la respiration ne met pas en évidence d'apnées, d'hypopnées ou de désaturation.''} Respiratory events are not the primary cause of sleep disruption.

\subsubsection{Multiple Sleep Latency Test (MSLT)}

\begin{table}[htbp]
\centering
\caption{MSLT Results (December 2018)}
\label{tab:mslt-results}
\begin{tabular}{lcccl}
\toprule
\textbf{Nap Time} & \textbf{Sleep Latency} & \textbf{Stages Reached} & \textbf{SOREMP} & \textbf{Note} \\
\midrule
09:00 & 0.5 min & N1, N2, N3 & No & Extremely rapid \\
11:00 & 3.0 min & N1, N2, N3 & No & Rapid \\
13:00 & 12.0 min & N1, N2 & No & Normal \\
15:00 & No sleep & --- & No & Did not fall asleep \\
\midrule
\textbf{Mean latency} & \textbf{9.0 min} & --- & \textbf{0/4} & \textbf{Pathological} \\
\bottomrule
\end{tabular}
\end{table}

\paragraph{MSLT Interpretation.}
\begin{itemize}
    \item Mean sleep latency of 9 minutes is pathological ($<$10 min indicates excessive daytime sleepiness)
    \item Absence of sleep-onset REM periods (SOREMPs) rules out narcolepsy
    \item Pattern shows \textbf{morning-predominant somnolence}---fell asleep in 30 seconds at 9h, 3 minutes at 11h
    \item Afternoon improvement (12 min at 13h, no sleep at 15h)
\end{itemize}

Report conclusion: \emph{``Présence de somnolence pathologique essentiellement en matinée (endormissement rapide et présence de sommeil lent profond).''}

\subsubsection{Official Diagnosis (2018 Sleep Study)}

\begin{tcolorbox}[colback=gray!5!white,colframe=gray!75!black,title=Polysomnography Diagnosis]
\textbf{Dyssomnia} characterized by:
\begin{itemize}
    \item Sleep fragmentation
    \item High number of stage changes (131)
    \item Periodic limb movements during sleep (index 13.3/h)
    \item No significant respiratory events
\end{itemize}

\textbf{Excessive daytime somnolence} (Epworth 16/24) with:
\begin{itemize}
    \item Risk of falling asleep while driving
    \item Pathological MSLT (mean latency 9 min)
    \item Morning-predominant pattern
    \item No narcolepsy features (no SOREMPs)
\end{itemize}

\textbf{Abnormal fatigue complaint} (Fatigue Severity Score 4.5)
\end{tcolorbox}

\subsection{Somnology Consultation (November 2021)}
\label{subsec:collet-assessment}

Sleep pathology consultation at Clinique Saint-Luc Bouge, November 2021.

\subsubsection{Key Clinical Observations}

\begin{itemize}
    \item \textbf{Fatigue onset}: Age 15--16 years (adolescence)
    \item \textbf{Fatigue pattern}: Fluctuating, with phases of 6--10 days of extreme physical and mental fatigue, headaches, brain fog, irritability
    \item \textbf{Burnout}: End of 2017
    \item \textbf{Family history}: Mother and two sisters diagnosed with ADHD
    \item \textbf{Cognitive}: IQ $>$135, skipped 6th grade primary, excellent academic facility
    \item \textbf{Weight}: 74 kg at 173 cm (BMI 24.7)---5--6 kg gain over 3 years
\end{itemize}

\subsubsection{Clinical Conclusion}

\begin{quote}
\emph{``Votre patient présente un tableau complexe de fatigue chronique d'étiologie indéterminée. Le bilan du sommeil réalisé au CHA n'a pas été décisif quant à un trouble du sommeil spécifique. L'hypersomnie idiopathique suspectée est un trouble se caractérisant par un allongement anormal du temps de sommeil avec persistance de fatigue/somnolence durant les phases d'éveil.''}

---Consulting somnologist
\end{quote}

\subsubsection{Clinical Recommendations}

\begin{enumerate}
    \item Ferritin target: $>$70--75 $\mu$g/L for PLM management
    \item Consider complete hypersomnia re-evaluation (actigraphy + PSG + MSLT + bedrest)
    \item ADHD/HP evaluation suggested (Dr.\ Linsmeaux, ADHD clinic)
    \item Continued Provigil treatment (100 mg $\times$3/day)
\end{enumerate}

\subsection{Disease Evolution Timeline}
\label{subsec:disease-timeline}

This subsection documents major milestones, changes in severity, and significant events in the disease course.

\begin{description}
    \item[Constitutional Phase (Childhood--2017):] Lifelong fatigue, idiopathic hypersomnia
    \begin{itemize}
        \item Early childhood: Required afternoon naps through age 7--8
        \item \textbf{Adolescence (age $\sim$13--15):} Onset of recurrent brain fog; constant tiredness but maintained academic performance
        \item \textbf{Age $\sim$20 (circa 2001):} Onset of spontaneous muscle cramps (nocturnal, throat/neck, without exertion)
        \item Young adulthood: University difficulties despite high IQ (>135) - cognitive impairment from energy deficit, not intellectual limitation
        \begin{itemize}
            \item Frequently slept during lectures throughout the day (not only after lunch)
            \item Sleep was involuntary response to overwhelming exhaustion, not simple drowsiness
            \item Academic struggles reflected energy deficit preventing sustained attention, not lack of intellectual capacity
        \end{itemize}
        \item \textbf{Work years:} Barely maintaining employment through unsustainable compensatory strategies
        \begin{itemize}
            \item Spent entire Saturdays sleeping (morning + afternoon) to recover for evening table tennis matches (not for work week)
            \item Experienced mid-match energy collapse leading to performance decline and losses
            \item Already too exhausted for proper work engagement during the week; just going through the motions
            \item Progressive difficulty maintaining even this unsustainable level of compensatory effort
            \item Employment was survival mode, not functional work performance
        \end{itemize}
        \item \textbf{Historical exercise tolerance:} At some point could swim 1\,km daily
        \begin{itemize}
            \item Physical fitness improved (better table tennis performance)
            \item Cognitive symptoms (fog, sleepiness) persisted during the day
            \item Exercise provided net benefit despite not eliminating underlying dysfunction
        \end{itemize}
        \item Status: Severely impaired but maintaining employment through extreme, unsustainable compensatory effort; already too exhausted for normal social/work engagement
    \end{itemize}

    \item[Triggering Event (Late 2017):] Severe burnout
    \begin{itemize}
        \item Burnout documented end of 2017 (per clinical sleep assessment)
        \item \textbf{Causal uncertainty}: Whether burnout was the trigger remains unclear; however, it was undeniably a profoundly depressive event
        \item Likely precipitated transition to full ME/CFS phenotype
        \item Burnout involves HPA axis dysregulation, cortisol dysfunction
        \item May have ``locked'' the metabolic safe mode described in speculative hypotheses
    \end{itemize}

    \item[Post-Trigger Phase (2018--Present):] Severe ME/CFS with disabling PEM
    \begin{itemize}
        \item \textbf{Important:} PEM itself is not new---it has been present for decades (weekend crash-recovery cycles, mid-match collapses)
        \item What changed: \textbf{Severity escalation} from ``manageable with extreme effort'' to ``disabling''
        \item \textbf{29 June 2018:} Concussion (commotion cérébrale) --- Clinique Saint-Joseph, Arlon
        \begin{itemize}
            \item \textbf{Mechanism}: Vagal syncope in public place $\rightarrow$ fall from chair $\rightarrow$ head trauma
            \item \textbf{Post-traumatic amnesia}: 5 hours (significant)
            \item \textbf{Clinical note}: ``Syncopes répétées'' (recurrent syncopes) --- not an isolated event
            \item \textbf{Imaging}: CT crâne + cervical: negative for post-traumatic lesions
            \item \textbf{Diagnosis}: ``Commotion cérébrale très probable'' (consulting emergency physician)
            \item \textbf{Follow-up ordered}: EEG (2/7/2018), Holter monitoring (16/7/2018)
            \item \textbf{Treatment}: Litican (piracetam --- nootropic for post-TBI cognitive support)
            \item \textbf{Relevant lab findings at admission}:
            \begin{itemize}
                \item Lactic acid: \textbf{3.18 mmol/L} (ref.\ 0.50--2.20) --- elevated at baseline
                \item CPK: \textbf{254 U/L} (ref.\ 5--195) --- muscle damage marker elevated
                \item LDH: \textbf{249 U/L} (ref.\ 135--225) --- upper limit
                \item Prolactin: \textbf{93.3 $\mu$g/L} (ref.\ 4.0--15.2) --- markedly elevated (post-ictal?)
                \item Glucose: 148 mg/dL (ref.\ 70--105) --- elevated (stress response)
            \end{itemize}
            \item \textbf{ME/CFS relevance}:
            \begin{itemize}
                \item Elevated baseline lactic acid supports metabolic dysfunction hypothesis
                \item Recurrent vagal syncopes consistent with dysautonomia
                \item Post-concussion syndrome shares features with ME/CFS: cognitive dysfunction, fatigue, exercise intolerance
                \item TBI can trigger or exacerbate neuroimmune dysfunction
                \item Timeline: 6 months after burnout trigger, during early deterioration phase
            \end{itemize}
        \end{itemize}
        \item Transition from ``tired but functional with compensatory strategies'' to ``unable to compensate''
        \item Unable to maintain employment consistently
        \item \textbf{2025/2026:} Attempted to resume swimming regimen (4--5 months duration)
        \begin{itemize}
            \item Previously: 1\,km daily swimming improved physical fitness (despite persistent cognitive symptoms)
            \item Current attempt: Resulted in \textbf{constant mental fog} severe enough to eliminate work function
            \item Consequence: Work underperformance leading to job loss
            \item Demonstrates disease progression: exercise changed from ``net benefit with symptoms'' to ``disabling cognitive PEM outweighing any fitness gains''
        \end{itemize}
        \item Current functional status: Severe functional impairment despite preserved basic mobility
        \begin{itemize}
            \item \textit{Can perform}: Drive children to school, buy groceries, sit at computer on better days
            \item \textit{Requires stimulants}: For any function; without stimulants, completely non-functional
            \item \textit{Profound exhaustion}: Despite stimulants, too tired for social engagement, eye contact, smiling, laughing
            \item \textit{Isolation preference}: Human interaction requires energy that doesn't exist; prefer distance over engagement
            \item \textit{Summary}: Can execute essential tasks but no energy for anything that makes life meaningful; ``too tired to be human''
        \end{itemize}
    \end{itemize}

    \item[Diagnoses:]
    \begin{itemize}
        \item Idiopathic hypersomnia (sleep study confirmed)
        \item Restless legs syndrome
        \item Sleep apnea (some degree)
        \item ME/CFS features: PEM, cognitive dysfunction, unrefreshing sleep
    \end{itemize}

    \item[Treatment milestones:]
    \begin{itemize}
        \item Methylphenidate (Rilatine): Effective for arousal/function
        \item Modafinil (Provigil): Effective for wakefulness
        \item LDN: Current status and effect to be documented
    \end{itemize}

    \item[Functional status changes:]
    \begin{itemize}
        \item Pre-2018: Maintaining employment through unsustainable effort; already too exhausted for proper work engagement; required extreme weekend recovery (full-day Saturday sleep)
        \item Post-2018: Unable to maintain employment consistently
        \item 2025/2026: Job loss following exercise-induced cognitive PEM (swimming regimen)
        \item Current (2026): Severe impairment; can perform essential tasks (drive, groceries, limited computer work) but too exhausted for social engagement or meaningful activities despite stimulants
    \end{itemize}
\end{description}

\subsection{Medication History}
\label{subsec:medication-history}

\begin{table}[htbp]
\centering
\caption{Medication History Log}
\label{tab:medication-history}
\begin{tabular}{lllp{4cm}}
\toprule
\textbf{Medication} & \textbf{Started} & \textbf{Stopped} & \textbf{Notes} \\
\midrule
LDN 3\,mg & 2026-01-05 & ongoing & Morning dosing (atypical); rapid escalation from starting dose due to good tolerance; plan to increase to 4--4.5\,mg pending prescription \\
Methylphenidate MR 30\,mg & & ongoing & 1--2 doses daily, max 3 pills total with modafinil \\
Modafinil 100\,mg & & ongoing & 1--2 doses daily, max 3 pills total with methylphenidate \\
% Add rows as needed
\bottomrule
\end{tabular}
\end{table}

\subsection{Supplement Trial Log}
\label{subsec:supplement-log}

\begin{table}[htbp]
\centering
\caption{Supplement Trial History}
\label{tab:supplement-history}
\begin{tabular}{llllp{3.5cm}}
\toprule
\textbf{Supplement} & \textbf{Dose} & \textbf{Started} & \textbf{Stopped} & \textbf{Effect/Notes} \\
\midrule
Acetyl-L-carnitine (Bandini) & 1000\,mg & 2026-01-21 & ongoing & Targets carnitine shuttle dysfunction for both muscle cramps and cognitive fog; monitor for GI effects \\
Riboflavin (B2) & 400\,mg & 2026-01-21 & ongoing & Migraine prevention (4--12 week timeline); supports FAD production for mitochondrial function; take separate from methylphenidate \\
Magnesium glycinate & 300--400\,mg & 2026-01-21 & ongoing & Replaces Magnecaps Dynatonic to avoid methylphenidate interaction; bedtime dosing for cramps; separate from methylphenidate by 2--4 hours \\
% CoQ10 (Ubiquinol) & 100\,mg & & & \\
\bottomrule
\end{tabular}
\end{table}

\subsection{Pattern Recognition Notes}
\label{subsec:pattern-notes}

Use this section to document observed patterns, correlations, and insights derived from the journal entries.

\paragraph{Identified Triggers.}
\begin{itemize}
    \item \textbf{PEM episodes not reliably linked to identifiable exertion}: Day 2 severe crash (2026-01-22) with intense joint pain occurred without obvious activity trigger. Patient managed normal childcare duties but no extraordinary physical or cognitive exertion identified.
    \item Suggests very low PEM threshold or delayed accumulation effect (multiple days of baseline activity triggering crash)
\end{itemize}

\paragraph{Helpful Interventions.}
\begin{itemize}
    \item \textbf{Electrolyte solution (500\,mL daily, days 1--3)}: Marked subjective improvement in cognitive function and focus by day 3 (2026-01-23)
    \begin{itemize}
        \item Able to maintain focus without methylphenidate on day 3 (took only modafinil morning dose)
        \item Suggests electrolyte/blood volume component to cognitive dysfunction
        \item Orthostatic tolerance remained acceptable throughout (no dizziness on standing)
    \end{itemize}
    \item \textbf{Magnesium glycinate (started 2026-01-21)}: Rapid joint pain resolution
    \begin{itemize}
        \item Day 2 (2026-01-22): Severe joint pain from morning onward (knees, shoulders) --- \emph{``the kind where you just want it gone in any possible way''}
        \item Day 3 (2026-01-23): Most joint pain resolved
        \item Suggests either magnesium deficiency or strong anti-inflammatory/muscle relaxant effect
    \end{itemize}
\end{itemize}

\paragraph{Crash Pattern Observations.}
\begin{itemize}
    \item \textbf{Afternoon vulnerability window}: Consistent afternoon fatigue/crash risk (1200--1430 window on day 2)
    \begin{itemize}
        \item Day 2: Severe crash with sleep 1200--1430, extreme joint pain
        \item Day 3: Afternoon fatigue present but manageable (no collapse)
        \item Pattern timing suggests circadian/HPA axis involvement or energy depletion pattern
        \item Not clearly linked to meal timing or standing duration
    \end{itemize}
    \item \textbf{Joint pain as crash indicator}: During severe PEM episodes, joint pain intensity becomes extreme
    \begin{itemize}
        \item Suggests inflammatory/cytokine component to crashes
        \item Joint pain resolved rapidly with magnesium, but crash vulnerability remains
    \end{itemize}
\end{itemize}

\paragraph{Medication Observations.}
\begin{itemize}
    \item \textbf{Methylphenidate dependence variable}: Day 3 demonstrated ability to maintain focus with only modafinil, suggesting electrolyte/metabolic improvement can reduce stimulant requirement
    \item \textbf{Stimulant fatigue-masking confirmed}: Patient notes tiredness in afternoon but ``currently OK'' --- classic stimulant masking effect (fatigue present but functional capacity maintained)
\end{itemize}

\paragraph{Temporal Patterns.}
\begin{itemize}
    \item \textbf{Afternoon timing consistent but not absolute}: Crash window 1200--1430 occurred day 2, afternoon fatigue day 3, but not predictably every day based on sleep quality alone
    \item \textbf{3-day intervention response timeline}: Subjective improvement in cognitive function noticeable by day 3 of electrolyte protocol; joint pain improvement within 24--48 hours of magnesium glycinate initiation
    \item \textbf{Sleep debt correlation}: Sleep debt DOES predictably correlate with symptom severity in this patient (contrary to some ME/CFS cases where sleep is non-restorative regardless of duration)
\end{itemize}

\subsection{Comparative Observation: Electrolyte Response Across Diagnoses}
\label{subsec:comparative-electrolyte}

\paragraph{Research Note.}

This section documents parallel electrolyte trials in two household members with different diagnoses (ME/CFS vs.\ narcolepsy) but overlapping cognitive dysfunction. The comparative observation provides insight into shared metabolic bottlenecks across neurological disorders and may inform future research on electrolyte optimization in chronic fatigue conditions.

\subsubsection{Clinical Context}

\begin{itemize}
    \item \textbf{Patient A} (primary case): ME/CFS, documented mitochondrial dysfunction features
    \item \textbf{Patient B} (spouse): Narcolepsy confirmed by polysomnography with SOREMPs
    \item \textbf{Patient B baseline treatment}: Following structured ``Light Energy Protocol'' (comprehensive mitochondrial support: acetyl-L-carnitine 1000\,mg, CoQ10 100--200\,mg, B-complex, magnesium glycinate 200--400\,mg, MCT oil, vitamin D3, D-Ribose)
    \item \textbf{Shared symptom}: Severe cognitive dysfunction (brain fog, impaired focus)
    \item \textbf{Intervention}: Identical electrolyte solution added to Patient B's existing protocol, 500\,mL daily (250\,mL $\times$ 2)
    \item \textbf{Timeline}: Concurrent 3-day trial (2026-01-21 through 2026-01-23)
\end{itemize}

\paragraph{Significance of Patient B Context.}

Patient B was already on a comprehensive energy/mitochondrial support protocol addressing:
\begin{itemize}
    \item Fat metabolism (acetyl-L-carnitine, MCT oil)
    \item Mitochondrial function (CoQ10, B vitamins, magnesium)
    \item ATP production (D-Ribose)
    \item Vitamin absorption optimization
\end{itemize}

The fact that electrolytes provided \emph{additional} cognitive benefit on top of this existing comprehensive protocol suggests:
\begin{enumerate}
    \item Electrolytes address a distinct bottleneck not covered by mitochondrial support alone
    \item Cellular hydration/ionic homeostasis may be an independent limiting factor
    \item Even with optimal metabolic support, impaired ionic balance can limit neuronal function
\end{enumerate}

This strengthens the hypothesis that electrolyte optimization is a complementary intervention, not redundant with mitochondrial/metabolic support.

\subsubsection{Convergent Responses}

Both patients reported:
\begin{itemize}
    \item Subjective improvement in mental focus and concentration
    \item Rapid timeline (improvement noticeable by day 3)
    \item Same formulation effective (sodium + potassium + glucose)
\end{itemize}

\subsubsection{Divergent Features}

\paragraph{Patient A (ME/CFS).}
\begin{itemize}
    \item Joint pain dramatically improved (9/10 $\rightarrow$ 1/10)
    \item Reduced stimulant requirement (focus maintained without methylphenidate)
    \item Sleep debt correlates predictably with symptoms
\end{itemize}

\paragraph{Patient B (Narcolepsy).}
\begin{itemize}
    \item Morning headaches from sleep deprivation persisted despite electrolytes
    \item Cognitive improvement but headaches signal uncompensated sleep debt
    \item Sleep architecture disruption requires actual restorative sleep for recovery
\end{itemize}

\subsubsection{Mechanistic Interpretation}

\paragraph{Convergence Point.}

Different root causes converge on shared metabolic consequence:

\begin{enumerate}
    \item \textbf{ME/CFS}: Mitochondrial dysfunction $\rightarrow$ inadequate ATP $\rightarrow$ impaired ionic homeostasis $\rightarrow$ neuronal dysfunction

    \item \textbf{Narcolepsy}: Sleep disruption $\rightarrow$ inadequate restoration $\rightarrow$ energy deficit $\rightarrow$ impaired ionic homeostasis $\rightarrow$ neuronal dysfunction

    \item \textbf{Both}: Cellular energy crisis affecting neuronal function
\end{enumerate}

\paragraph{Why Electrolytes Help Both.}

Electrolytes address downstream consequences, not root causes:
\begin{itemize}
    \item Reduce ATP demand on Na\textsuperscript{+}/K\textsuperscript{+}-ATPase pumps
    \item Improve cellular hydration and waste clearance
    \item Stabilize neuronal electrical function (brain highly sensitive)
    \item \textbf{Do not fix}: ME/CFS mitochondria or narcolepsy hypocretin deficiency
\end{itemize}

\paragraph{Divergent Features Explained.}

\begin{itemize}
    \item \textbf{Joint pain (Patient A only)}: Magnesium deficiency specific to ME/CFS mitochondrial dysfunction; inflammatory component to crashes
    \item \textbf{Headaches (Patient B only)}: Narcolepsy requires actual sleep for recovery; electrolytes cannot substitute for sleep architecture restoration
\end{itemize}

\subsubsection{Research Implications}

\paragraph{Broader Applicability.}
\begin{enumerate}
    \item Electrolyte optimization may benefit multiple fatigue-associated disorders
    \item Cellular energy crisis is a shared endpoint across diverse pathophysiologies
    \item Symptomatic benefit (metabolic support) vs.\ disease-modifying distinction
    \item Research question: Are chronic neurological/energy disorders associated with systematic electrolyte/cellular hydration deficits?
\end{enumerate}

\paragraph{Testable Hypotheses.}
\begin{itemize}
    \item Measure intracellular Na\textsuperscript{+}/K\textsuperscript{+} ratios before/after supplementation
    \item Assess extracellular fluid volume in ME/CFS, narcolepsy, Long COVID
    \item Controlled trials of electrolyte optimization across fatigue disorders
    \item Identify biomarkers predicting responders vs.\ non-responders
\end{itemize}

\paragraph{Clinical Utility.}
\begin{itemize}
    \item Low-cost, low-risk intervention with rapid assessment (3 days)
    \item May reduce stimulant requirement or enhance effectiveness
    \item Particularly relevant for cognitive symptoms
    \item Does not replace disease-specific treatments
\end{itemize}

\paragraph{Limitations.}
\begin{itemize}
    \item Sample size: Two individuals; cannot generalize
    \item Confounding: Other supplements/medications present
    \item Subjective assessment: No objective cognitive testing
    \item Short duration: 3-day trial; long-term effects unknown
\end{itemize}

%%%%%%%%%%%%%%%%%%%%%%%%%%%%%%%%%%%%%%%%%%%%%%%%%%%%%%%%%%%%%%%%%%%%%%%%%%%%%%%
