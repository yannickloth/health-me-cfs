% FILE: Research synthesis and literature summary — research overview, key findings, evidence summary
\chapter{Research Synthesis Tables}
\label{app:research-synthesis}

\section{Major Studies Summary}
\label{sec:major-studies-summary}

This section synthesizes key research findings integrated from literature reviews, including papers identified through systematic searches, community-reported studies, and recent publications (2019--2025).

\subsection{Molecular and Cellular Mechanisms}
\label{subsec:molecular-mechanisms-studies}

\begin{table}[htbp]
\centering
\caption{Molecular Mechanism Studies in ME/CFS}
\label{tab:molecular-studies}
\scriptsize
\begin{tabular}{p{2cm}p{1.8cm}p{2.2cm}p{4cm}p{2.5cm}p{1.8cm}}
\toprule
\textbf{Study} & \textbf{Design} & \textbf{Sample} & \textbf{Key Findings} & \textbf{Implications} & \textbf{Certainty} \\
\midrule
Wang 2023~\cite{wang2023wasf3} & Case-control; muscle biopsy & n=14 ME/CFS, n=10 controls & WASF3 protein elevated; inverse correlation with Complex IV (r=-0.55, p=0.005); shRNA knockdown restores function & WASF3 is druggable target; mechanism is reversible & MODERATE (pending replication) \\
\midrule
Lim 2020~\cite{lim2020cpet} & 2-day CPET; repeated measures & n=51 ME/CFS, n=10 sedentary controls & VO$_2$max reduced 25\% on Day 2 in ME/CFS; controls unchanged; ventilatory threshold reduced & Objective PEM biomarker; 24-72h delayed impairment & HIGH (replicated) \\
\midrule
Syed 2025~\cite{Syed2025} & Systematic review & Multiple studies & Mitochondrial dysfunction across oxidative phosphorylation, ATP synthesis, metabolomics & Converging evidence for mitochondrial pathology & MODERATE-HIGH (meta-analytic) \\
\midrule
Phair 2019~\cite{phair2019ido} & Metabolomics modeling & n=52 ME/CFS, n=45 controls & IDO metabolic trap hypothesis; tryptophan-kynurenine pathway disruption & Potential therapeutic target (IDO inhibitors) & MODERATE (hypothesis; needs validation) \\
\bottomrule
\end{tabular}
\end{table}

\subsection{Viral and Infectious Triggers}
\label{subsec:viral-trigger-studies}

\begin{table}[htbp]
\centering
\caption{Viral Association Studies}
\label{tab:viral-studies}
\scriptsize
\begin{tabular}{p{2cm}p{1.8cm}p{2.5cm}p{4cm}p{2cm}}
\toprule
\textbf{Study} & \textbf{Design} & \textbf{Sample} & \textbf{Key Findings} & \textbf{Evidence Level} \\
\midrule
Hwang 2023~\cite{hwang2023viral} & Systematic review + meta-analysis & 64 studies; n=4,971 ME/CFS, n=9,221 controls & 18 viral species assessed; strongest associations: Borna (OR$\geq$3.47), HHV-7 (OR>2.0), parvovirus B19 (OR>2.0), enterovirus (OR>2.0), coxsackie B (OR>2.0) & HIGH (meta-analytic; replicated) \\
\midrule
Chia 2005~\cite{Chia2005} & Observational; stomach biopsy & n=165 ME/CFS patients & Enterovirus detected in 82\% of ME/CFS patients via stomach biopsy immunostaining; correlation with symptom severity & MODERATE (specialized technique; replication needed) \\
\midrule
Gottschalk 2023~\cite{Gottschalk2023} & Case series; observational & n=42 Long COVID patients & LDN (4.5mg) improved fatigue, brain fog, PEM in 78\% of Long COVID patients within 2 months & LOW-MODERATE (observational; no control group) \\
\bottomrule
\end{tabular}
\end{table}

\subsection{Immune Dysfunction Studies}
\label{subsec:immune-studies}

\begin{table}[htbp]
\centering
\caption{Immune System Studies in ME/CFS}
\label{tab:immune-studies}
\scriptsize
\begin{tabular}{p{2cm}p{2cm}p{2.2cm}p{4.2cm}p{2cm}}
\toprule
\textbf{Study} & \textbf{Design} & \textbf{Sample} & \textbf{Key Findings} & \textbf{Certainty} \\
\midrule
Fluge 2019~\cite{Fluge2019} & Phase III RCT (RituxME trial) & n=152 ME/CFS & Rituximab (B-cell depletion) showed NO benefit vs placebo; placebo response 35\%, rituximab 26\% & HIGH (definitive negative result) \\
\midrule
Rekeland 2024~\cite{Rekeland2024} & Long-term follow-up of RituxME & Original n=152 cohort; 6-year follow-up & No long-term benefit from rituximab confirmed; subset analysis revealed no responder subgroups & HIGH (confirms Fluge 2019) \\
\midrule
Bulbule 2024~\cite{Bulbule2024} & Systematic review & Multiple NK cell studies & Reduced NK cell cytotoxicity consistently reported across studies; correlation with symptom severity & MODERATE-HIGH (consistent finding) \\
\bottomrule
\end{tabular}
\end{table}

\section{Biomarker Studies Summary}
\label{sec:biomarker-studies-summary}

\begin{table}[htbp]
\centering
\caption{Validated and Proposed Biomarkers for ME/CFS}
\label{tab:biomarker-summary}
\scriptsize
\begin{tabular}{p{2.5cm}p{2cm}p{4cm}p{2.5cm}p{2cm}}
\toprule
\textbf{Biomarker} & \textbf{Measurement} & \textbf{Finding} & \textbf{Clinical Utility} & \textbf{Validation Status} \\
\midrule
\textbf{2-Day CPET} & VO$_2$max Day 1 vs Day 2 & 25\% reduction Day 2 in ME/CFS~\cite{lim2020cpet} & Objective PEM documentation; disability assessment & VALIDATED (replicated) \\
\midrule
\textbf{WASF3 protein} & Muscle biopsy immunoblot & Elevated in ME/CFS; inverse correlation with Complex IV~\cite{wang2023wasf3} & Research tool; potential treatment target & PRELIMINARY (n=14; needs replication) \\
\midrule
\textbf{NK cell cytotoxicity} & Flow cytometry; cytotoxic assay & Reduced across multiple studies~\cite{Bulbule2024} & Immune dysfunction marker & MODERATE (consistent but variable) \\
\midrule
\textbf{Viral serology} & PCR, immunostaining & Enterovirus in 82\% stomach biopsies~\cite{Chia2005}; multiple viral associations~\cite{hwang2023viral} & Subset identification (viral-onset) & MODERATE (specialized techniques) \\
\midrule
\textbf{Tryptophan-kynurenine} & Plasma metabolomics & IDO metabolic trap~\cite{phair2019ido} & Potential treatment stratification & HYPOTHESIS (needs validation) \\
\bottomrule
\end{tabular}
\end{table}

\section{Treatment Trials Summary}
\label{sec:treatment-trials-summary}

\subsection{Pharmacological Interventions}
\label{subsec:pharmacological-trials}

\begin{table}[htbp]
\centering
\caption{Pharmacological Treatment Evidence in ME/CFS}
\label{tab:pharmacological-treatments}
\scriptsize
\begin{tabular}{p{2.5cm}p{1.8cm}p{2cm}p{4cm}p{2.2cm}p{1.5cm}}
\toprule
\textbf{Intervention} & \textbf{Study Type} & \textbf{Sample} & \textbf{Findings} & \textbf{Recommendation} & \textbf{Evidence} \\
\midrule
\textbf{Low-Dose Naltrexone (LDN)} & Observational & n=218~\cite{Polo2019}; n=42 Long COVID~\cite{Gottschalk2023} & 73.9\% positive response (ME/CFS); 78\% improved (Long COVID); improved vigilance, alertness, physical/cognitive performance & Consider trial; 3.0--4.5mg/day & MODERATE (large observational; no RCT) \\
\midrule
\textbf{Rituximab (B-cell depletion)} & Phase III RCT & n=152~\cite{Fluge2019}; 6-year follow-up~\cite{Rekeland2024} & NO BENEFIT; placebo 35\% response > rituximab 26\%; no long-term benefit & DO NOT USE & HIGH (definitive negative) \\
\midrule
\textbf{Graded Exercise Therapy (GET)} & Multiple studies; patient surveys & Patient harm reports; PACE trial discredited & Causes deterioration in many patients; violates PEM physiology & HARMFUL; contraindicated & HIGH (consensus; patient evidence) \\
\bottomrule
\end{tabular}
\end{table}

\subsection{Patient-Reported Interventions}
\label{subsec:patient-reported-interventions}

These interventions lack formal RCT validation but have plausible mechanisms and multiple independent patient reports. They require medical supervision and formal clinical trials.

\begin{table}[htbp]
\centering
\caption{Patient-Reported Interventions Requiring Clinical Validation}
\label{tab:patient-interventions}
\scriptsize
\begin{tabular}{p{2.5cm}p{2cm}p{3.5cm}p{3cm}p{2.5cm}}
\toprule
\textbf{Intervention} & \textbf{Reported Dose} & \textbf{Reported Benefits} & \textbf{Plausible Mechanism} & \textbf{Research Status} \\
\midrule
\textbf{Nicotine (low-dose)} & 2--4mg/day (gum, patch) & Rapid brain fog improvement (hours to days); multiple independent reports & Alpha-7 nAChR modulation; anti-inflammatory; mitochondrial calcium regulation (ch19 \S\ref{hyp:ach-mito}) & HYPOTHESIS-GENERATING; needs RCT; addiction risk \\
\midrule
\textbf{Methylene blue} & 1--5mg/day (very low dose) & Smell restoration, brain fog reduction within 1 week & Enhances electron transport; reduces oxidative stress; indirect benefit despite Complex IV dysfunction (ch19 \S\ref{hyp:mb-mito-enhancement}) & HYPOTHESIS-GENERATING; dose-finding needed \\
\midrule
\textbf{Ketogenic diet} & Strict keto & Dramatic improvement in subset; "medication-free" in some cases & Ketone bodies provide alternative fuel (acetyl-CoA) without glucose; reduces oxidative stress (ch19 \S\ref{spec:patient-interventions}) & ANECDOTAL; subset-specific; needs stratified trial \\
\midrule
\textbf{Pyruvate (prophylactic)} & 1--2g pre-exertion & Proposed to prevent PEM crashes & Provides pyruvate directly for TCA cycle; skips glycolysis requirement; used by athletes (ch19 \S\ref{spec:pyruvate-supplement}) & SPECULATIVE; testable in RCT \\
\midrule
\textbf{NAD+ precursors} & NR 300--1000mg/day; NMN 250--500mg/day & Proposed for post-exertional recovery & Boosts lactate dehydrogenase; accelerates lactate clearance; improves mitochondrial NAD+/NADH ratio (ch19 \S\ref{spec:nad-lactate}) & SPECULATIVE; mechanistically sound; testable \\
\bottomrule
\end{tabular}
\end{table}

\subsection{Comorbidity Management}
\label{subsec:comorbidity-management-evidence}

\begin{table}[htbp]
\centering
\caption{Comorbidities Frequently Misdiagnosed as ME/CFS}
\label{tab:comorbidity-misdiagnosis}
\scriptsize
\begin{tabular}{p{2.5cm}p{2.5cm}p{4cm}p{3cm}}
\toprule
\textbf{Condition} & \textbf{Diagnostic Test} & \textbf{Presentation Overlap} & \textbf{Clinical Implication} \\
\midrule
\textbf{Sleep Apnea} & Polysomnography (overnight sleep study) & Fatigue, cognitive dysfunction, unrefreshing sleep; patient reports describe years of misdiagnosis & CPAP treatment can resolve symptoms; should be standard workup \\
\midrule
\textbf{Lyme Disease (European species)} & European Lyme serology panel & Chronic fatigue, PEM-like symptoms; 10-year misdiagnosis reported & Long-cycle antibiotics "significantly helpful"; requires regional-specific testing (ch19 \S\ref{obs:lyme-mecfs-overlap}) \\
\midrule
\textbf{Hypermobile EDS (hEDS)} & Beighton score; clinical assessment & Joint hypermobility, easy bruising, fatigue, POTS overlap; "100-fold underdiagnosed" & Physical therapy adaptations; affects pacing strategies (ch19 \S\ref{hyp:eds-mcas-mecfs}) \\
\midrule
\textbf{Mast Cell Activation (MCAS)} & Tryptase levels; clinical criteria & Allergic symptoms, flushing, GI issues, fatigue & H1/H2 blockers, mast cell stabilizers may help; potential mito-immune link (ch19 \S\ref{hyp:mcas-mito-damage}) \\
\midrule
\textbf{ADHD + hEDS overlap} & Clinical assessment & Shared genetic factors proposed; frequent co-occurrence & May represent distinct phenotype requiring different management (ch19 \S\ref{hyp:eds-mcas-mecfs}) \\
\bottomrule
\end{tabular}
\end{table}

\section{Pathophysiology Evidence Summary}
\label{sec:pathophysiology-evidence-summary}

\subsection{Converging Evidence for Core Mechanisms}
\label{subsec:core-mechanisms-evidence}

\begin{table}[htbp]
\centering
\caption{Evidence Strength for Proposed Pathophysiological Mechanisms}
\label{tab:pathophysiology-evidence}
\scriptsize
\begin{tabular}{p{3cm}p{3.5cm}p{4cm}p{2cm}p{1.5cm}}
\toprule
\textbf{Mechanism} & \textbf{Supporting Evidence} & \textbf{Key Studies/Findings} & \textbf{Gaps} & \textbf{Strength} \\
\midrule
\textbf{Mitochondrial dysfunction} & ATP depletion, Complex IV deficits, delayed recovery & WASF3 elevation~\cite{wang2023wasf3}; 2-day CPET~\cite{lim2020cpet}; systematic review~\cite{Syed2025} & Causation vs consequence; specific complex deficits vary & HIGH \\
\midrule
\textbf{Post-exertional malaise (PEM)} & Objective VO$_2$max reduction Day 2; 24--72h delay & 2-day CPET 25\% reduction~\cite{lim2020cpet}; patient "<5 crash rule" & Molecular trigger; why delayed; recovery kinetics & HIGH \\
\midrule
\textbf{Viral triggers} & Multiple viral associations; persistent infection & Meta-analysis OR 2.0--3.47~\cite{hwang2023viral}; enterovirus 82\%~\cite{Chia2005} & Why only subset; mechanism of chronicity; viral clearance failure & MODERATE-HIGH \\
\midrule
\textbf{Immune dysfunction} & NK cell reduction, cytokine dysregulation & NK cytotoxicity reduced~\cite{Bulbule2024}; rituximab failure~\cite{Fluge2019} & Primary vs secondary; T-cell role; autoimmunity & MODERATE \\
\midrule
\textbf{Autonomic dysfunction (POTS, OI)} & Orthostatic intolerance 70--90\% prevalence & Blood volume reduction; baroreceptor dysfunction & Connection to mitochondria; causation & HIGH \\
\midrule
\textbf{Neuroinflammation} & Brain fog, cognitive impairment, hypoperfusion & Patient reports; imaging studies & Mechanisms; biomarkers; treatment targets & MODERATE \\
\midrule
\textbf{ER stress-WASF3 pathway} & Viral infection → ER stress → WASF3 upregulation → Complex IV damage & Proposed pathway integrating viral triggers~\cite{hwang2023viral} and WASF3~\cite{wang2023wasf3} (ch19 \S\ref{hyp:viral-er-wasf3}) & Validation needed; ER stress markers; intervention trials & HYPOTHESIS \\
\midrule
\textbf{Metabolic trap (IDO pathway)} & Tryptophan-kynurenine disruption & Phair modeling~\cite{phair2019ido} (ch06 \S\ref{sec:metabolic-trap}) & Replication; causation; therapeutic validation & HYPOTHESIS \\
\bottomrule
\end{tabular}
\end{table}

\subsection{Patient-Derived Clinical Insights}
\label{subsec:patient-clinical-insights}

Community-reported patterns from online forums, patient advocacy groups, and Hacker News discussions reveal clinical insights not yet validated in formal research but with high practical utility.

\begin{table}[htbp]
\centering
\caption{Patient-Discovered Patterns and Clinical Rules}
\label{tab:patient-insights}
\scriptsize
\begin{tabular}{p{3cm}p{4.5cm}p{3.5cm}p{2.5cm}}
\toprule
\textbf{Pattern/Rule} & \textbf{Description} & \textbf{Clinical Implication} & \textbf{Validation Status} \\
\midrule
\textbf{"<5 crashes per year" rule} & Exceeding energy limits >5 times/year causes irreversible worsening & Strict pacing is non-negotiable; crashes have cumulative damage (ch14b \S\ref{sec:energy-envelope}) & OBSERVATIONAL; matches 2-day CPET pathology \\
\midrule
\textbf{Caffeine sensitivity changes} & Pre-illness caffeine tolerance reverses post-illness; caffeine worsens crashes in many patients & Avoid caffeine or use minimally; may indicate adenosine receptor changes & ANECDOTAL; widely reported \\
\midrule
\textbf{24--72 hour PEM delay} & Symptom crash occurs 1--3 days post-exertion, not immediately & Activity tracking must account for delayed consequences; "you won't know until Day 2" & VALIDATED by 2-day CPET~\cite{lim2020cpet} \\
\midrule
\textbf{Sleep apnea masquerading as ME/CFS} & Years of ME/CFS diagnosis resolved with CPAP in subset & Polysomnography should be standard workup (ch19 \S\ref{obs:sleep-apnea-misdiagnosis}) & CASE REPORTS; diagnostic importance \\
\midrule
\textbf{EDS/MCAS overlap} & High comorbidity; "100-fold underdiagnosed"; shared symptoms & Screen for Beighton score, tryptase, allergic symptoms (ch19 \S\ref{hyp:eds-mcas-mecfs}) & CLINICAL OBSERVATION; needs epidemiological study \\
\midrule
\textbf{Nicotine rapid effect} & Brain fog improvement within hours to days at 2--4mg & Suggests cholinergic or anti-inflammatory mechanism; testable in RCT (ch19 \S\ref{hyp:ach-mito}) & ANECDOTAL; multiple independent reports \\
\midrule
\textbf{Ketogenic diet subset response} & Dramatic improvement in some; no effect or worsening in others & Heterogeneity suggests metabolic subtypes; stratified trial needed (ch19 \S\ref{spec:patient-interventions}) & ANECDOTAL; subset-specific \\
\midrule
\textbf{GET causes harm} & Patient deterioration; violates PEM physiology; PACE trial discredited & Contraindicated; pacing is evidence-based alternative (ch14b \S\ref{warn:get-harmful}) & VALIDATED; consensus \\
\bottomrule
\end{tabular}
\end{table}

\subsection{Research Gaps and Controversies}
\label{subsec:research-gaps}

\begin{table}[htbp]
\centering
\caption{Major Research Gaps in ME/CFS}
\label{tab:research-gaps}
\small
\begin{tabular}{p{3.5cm}p{5cm}p{5cm}}
\toprule
\textbf{Gap} & \textbf{Current Status} & \textbf{Research Need} \\
\midrule
\textbf{Why viral infection triggers chronic disease in subset} & Multiple viral associations proven~\cite{hwang2023viral}; mechanism unknown & Longitudinal studies post-viral infection; genetic susceptibility; immune response profiling \\
\midrule
\textbf{WASF3 mechanism and reversibility} & WASF3 elevated; shRNA reversal shown~\cite{wang2023wasf3}; n=14 & Replication in larger cohort; WASF3 inhibitor trials; longitudinal tracking \\
\midrule
\textbf{Why PEM is delayed 24--72 hours} & Objective 2-day CPET shows delay~\cite{lim2020cpet}; molecular trigger unknown & Mitophagy markers; ATP kinetics; lactate clearance; serial muscle biopsies \\
\midrule
\textbf{Heterogeneity and subtypes} & Clinical presentation varies; treatment responses differ & Cluster analysis; biomarker-based stratification; metabolomics subtyping \\
\midrule
\textbf{Why B-cell depletion failed but LDN helps} & Rituximab negative~\cite{Fluge2019}; LDN observational positive~\cite{Polo2019} & T-cell vs B-cell role; LDN mechanism (opioid vs immune); RCT of LDN \\
\midrule
\textbf{Connection between mitochondria and immune dysfunction} & Both systems affected; unclear if linked or parallel & Mast cell-mitochondrial crosstalk; cytokine effects on oxidative phosphorylation \\
\midrule
\textbf{Reversibility and spontaneous remission} & Rare spontaneous remission; WASF3 potentially reversible & Remission biomarkers; reversibility mechanisms; intervention timing \\
\bottomrule
\end{tabular}
\end{table}

\subsection{Cross-Domain Medical Parallels}
\label{subsec:cross-domain-evidence}

Table~\ref{tab:cross-domain-parallels} summarizes validated interventions from other medical fields with phenomenological overlap to ME/CFS, as detailed in Chapter~\ref{ch:integrative-treatment} Section~\ref{sec:cross-domain-parallels}.

\begin{table}[htbp]
\centering
\caption{Cross-Domain Medical Interventions Applicable to ME/CFS}
\label{tab:cross-domain-parallels}
\scriptsize
\begin{tabular}{p{2.5cm}p{2cm}p{3.5cm}p{3cm}p{2.5cm}}
\toprule
\textbf{Source Field} & \textbf{Shared Feature} & \textbf{Intervention} & \textbf{ME/CFS Application} & \textbf{Implementation Status} \\
\midrule
\textbf{Sports Medicine} & Muscle metabolic stress, lactate accumulation & ORS, magnesium, Acetyl-L-carnitine, D-ribose & Lactate clearance, ATP support, cramp reduction & IMPLEMENTED; evidence-based \\
\midrule
\textbf{Altitude Medicine} & Tissue hypoxia, exercise intolerance & Iron optimization (ferritin >100), acetazolamide, breathing techniques & Oxygen delivery, cerebral function & PARTIAL; iron standard; acetazolamide case reports \\
\midrule
\textbf{ICU Recovery (PICS)} & Profound weakness, cognitive impairment, metabolic depletion & Micronutrient repletion (B1, C, D, Mg, Zn, Se), NAC, high protein & Metabolic support, oxidative stress, muscle preservation & IMPLEMENTED; nutritional protocols \\
\midrule
\textbf{Space Medicine} & Orthostatic intolerance, deconditioning, blood volume loss & Compression garments, horizontal exercise, fluid/salt loading & POTS management, reconditioning, blood volume expansion & IMPLEMENTED; POTS protocols \\
\midrule
\textbf{Chronic Pain Medicine} & Central sensitization, quality of life impairment & LDN, gabapentinoids, acceptance strategies & Pain reduction, central sensitization, pacing validation & PARTIAL; LDN evidence moderate \\
\midrule
\textbf{Geriatric Frailty} & Multi-system decline, weakness, falls risk & Vitamin D optimization, protein supplementation, mobility aids without stigma & Frailty prevention, function optimization, assistive devices & IMPLEMENTED; acceptance of limitations \\
\bottomrule
\end{tabular}
\end{table}

\section{Quick Reference: Evidence-Based Treatment Summary}
\label{sec:evidence-based-summary}

Table~\ref{tab:treatment-quick-reference} provides a rapid-access summary for clinicians and patients, organized by evidence strength and implementation tier.

\begin{table}[htbp]
\centering
\caption{Evidence-Based Treatment Quick Reference}
\label{tab:treatment-quick-reference}
\scriptsize
\begin{tabular}{p{3cm}p{2cm}p{2.5cm}p{3cm}p{2cm}p{1.5cm}}
\toprule
\textbf{Intervention} & \textbf{Typical Dose} & \textbf{Evidence Level} & \textbf{Primary Indication} & \textbf{Cost/Month} & \textbf{Tier} \\
\midrule
\multicolumn{6}{l}{\textbf{TIER 1: Strong Evidence, Immediate Implementation}} \\
\midrule
Pacing (energy envelope) & Individualized & HIGH (2-day CPET, consensus) & PEM prevention; core intervention & Free & 1 \\
Fluid/salt loading (POTS) & 2.5--3L, 6--10g Na/day & HIGH (orthostatic physiology) & Orthostatic intolerance, blood volume & \$5 & 1 \\
Compression stockings & 20--30 mmHg waist-high & HIGH (POTS, space medicine) & Orthostatic intolerance & \$30--60 one-time & 1 \\
ORS (sports recovery) & 500mL 2--3$\times$/day & MODERATE (sports medicine) & Lactate clearance, hydration & \$5 & 1 \\
Vitamin D & 4000--5000 IU/day & MODERATE (ICU, geriatrics) & Immune function, muscle & \$5 & 1 \\
Magnesium glycinate & 300--400mg/day & MODERATE (ICU, sports) & ATP synthesis, cramps & \$10 & 1 \\
B-complex (thiamine) & B1 100--300mg; B-complex & MODERATE (ICU critical care) & Aerobic metabolism, neurological & \$10 & 1 \\
\midrule
\multicolumn{6}{l}{\textbf{TIER 2: Moderate Evidence, Consider Trial}} \\
\midrule
CoQ10 + Acetyl-L-carnitine & 200mg CoQ10; 500--1000mg ALC & MODERATE (mitochondrial support) & ATP production, fat oxidation & \$40--60 & 2 \\
Iron optimization & Target ferritin 100--200 & MODERATE (altitude medicine) & Oxygen transport, dopamine synthesis & \$10--15 & 2 \\
NAC & 600mg 2$\times$/day & MODERATE (ICU, oxidative stress) & Glutathione support, oxidative stress & \$15--25 & 2 \\
Vitamin C & 1000--2000mg/day divided & MODERATE (ICU sepsis protocols) & Antioxidant, immune support & \$10 & 2 \\
Zinc + selenium & 15--30mg Zn; 200$\mu$g Se & MODERATE (ICU) & Immune function, antioxidant & \$10 & 2 \\
LDN & 3.0--4.5mg/day & MODERATE (n=218 observational) & Fatigue, brain fog, PEM & \$20--40 & 2 \\
\midrule
\multicolumn{6}{l}{\textbf{TIER 3: Emerging/Speculative, Needs Validation}} \\
\midrule
Pyruvate (prophylactic) & 1--2g pre-exertion & SPECULATIVE (mechanistic rationale) & PEM prevention & \$15--25 & 3 \\
NAD+ precursors (NR/NMN) & NR 300--1000mg; NMN 250--500mg/day & SPECULATIVE (lactate clearance hypothesis) & Post-exertional recovery & \$40--60 & 3 \\
Methylene blue & 1--5mg/day & SPECULATIVE (patient reports) & Brain fog, mitochondrial enhancement & \$10--20 & 3 \\
Acetazolamide & 125--250mg 2$\times$/day & SPECULATIVE (altitude medicine) & Oxygenation, cognitive function & Rx required & 3 \\
D-ribose & 5g 2--3$\times$/day & SPECULATIVE (sports medicine) & ATP precursor & \$25--40 & 3 \\
\midrule
\multicolumn{6}{l}{\textbf{CONTRAINDICATED: Evidence of Harm}} \\
\midrule
Graded Exercise Therapy (GET) & N/A & HIGH (patient harm, PACE discredited) & HARMFUL; causes deterioration & N/A & — \\
Rituximab (B-cell depletion) & N/A & HIGH (Phase III RCT negative) & No benefit; placebo superior & N/A & — \\
\bottomrule
\end{tabular}
\end{table}

\paragraph{Usage Notes for Table~\ref{tab:treatment-quick-reference}.}

\begin{itemize}
    \item \textbf{Tier 1}: Implement immediately based on strong physiological rationale or consensus; low cost, high safety
    \item \textbf{Tier 2}: Consider trial for 3 months; monitor response; moderate evidence from observational studies or related conditions
    \item \textbf{Tier 3}: Experimental; discuss risks/benefits; await formal trials; mechanistically plausible but unvalidated
    \item \textbf{Start with Tier 1}, add Tier 2 interventions sequentially if no benefit after 3 months
    \item \textbf{Monitor responses} with symptom diary, objective measures (heart rate, activity tolerance)
    \item \textbf{Medical supervision required} for prescription medications, high-dose supplementation, or if multiple comorbidities present
\end{itemize}
