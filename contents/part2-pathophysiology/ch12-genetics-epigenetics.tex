\chapter{Genetic and Epigenetic Factors}
\label{ch:genetics-epigenetics}

\section{Genetic Predisposition}
\label{sec:genetics}

\subsection{Family Studies}

Familial clustering of ME/CFS provides evidence for genetic contribution while highlighting the complexity of inheritance patterns. Unlike simple Mendelian disorders, ME/CFS appears to follow a polygenic model where multiple genetic variants interact with environmental triggers.

\subsubsection{Familial Clustering Evidence}

Multiple studies document increased ME/CFS prevalence among first-degree relatives of affected individuals, though precise risk estimates vary. The pattern suggests genetic susceptibility rather than purely environmental causation, as familial clustering persists even when controlling for shared household exposures.

\subsubsection{Risk for Children of ME/CFS Parents}

The question of genetic risk is particularly salient for families planning children or concerned about pediatric cases. Current evidence suggests:

\begin{itemize}
    \item \textbf{Moderate inherited risk}: Children of ME/CFS parents have elevated risk compared to general population, but most do not develop the condition
    \item \textbf{Environmental trigger still required}: Genetic susceptibility alone appears insufficient---viral infection, trauma, or severe stress typically precipitates onset
    \item \textbf{Gene-environment interaction model}: The NIH RECOVER study found 4.5\% of COVID-19 survivors developed ME/CFS~\cite{Komaroff2023}, meaning 95.5\% did not despite identical viral exposure, suggesting genetic factors influence who progresses from acute infection to chronic illness
\end{itemize}

\subsubsection{Inherited Susceptibility Patterns}

Children of ME/CFS parents may inherit:

\begin{itemize}
    \item \textbf{Immune gene variants}: Affecting cytokine production profiles, HLA types, and immune regulation
    \item \textbf{Mitochondrial susceptibility}: Recent evidence identifies WASF3 pathway dysregulation in ME/CFS~\cite{Syed2025}, potentially affecting cellular energy production capacity
    \item \textbf{Autonomic nervous system sensitivity}: Increased risk for orthostatic intolerance and POTS, which co-occurs in 60\% of ME/CFS patients~\cite{Natelson2022}
\end{itemize}

\subsubsection{Preventive Strategies for At-Risk Children}

While genetic risk cannot be eliminated, evidence-based prevention strategies may reduce conversion from acute infection to chronic illness:

\begin{warning}[Prevention is Not Guarantee]
These strategies reflect prudent health practices but cannot eliminate ME/CFS risk. They should not create anxiety or overprotection, but rather inform appropriate response to illness.
\end{warning}

\textbf{Post-Viral Vigilance:}
\begin{itemize}
    \item Aggressive rest during and after significant infections (mononucleosis, COVID-19, influenza)
    \item Monitoring for prolonged fatigue persisting beyond 3 months post-infection
    \item Avoiding premature return to full activity levels
\end{itemize}

\textbf{Early Pacing Education:}
\begin{itemize}
    \item Teaching energy envelope management before illness onset
    \item Recognizing that athletic children may be at higher risk if they habitually push through warning signals
    \item Emphasizing that rest during illness is health-preserving, not weakness
\end{itemize}

\textbf{Early Warning Signs in At-Risk Children:}
\begin{itemize}
    \item Exercise intolerance disproportionate to peers
    \item Orthostatic symptoms (dizziness upon standing)
    \item Slow recovery from minor illnesses
    \item Early fatigue compared to siblings or classmates
    \item Development of food sensitivities or allergic-type symptoms (potential MCAS)
\end{itemize}

\subsubsection{Twin Studies and Heritability}

Twin study data, while limited in ME/CFS, supports moderate heritability. Concordance rates between monozygotic twins exceed dizygotic twins, but remain well below 100\%, confirming that genetic factors contribute to but do not determine disease development.

\subsection{Genetic Variants}
% Single nucleotide polymorphisms (SNPs)
% Genes implicated in ME/CFS
% Immune system genes
% Metabolic genes
% Neurotransmitter genes

\subsection{Genome-Wide Association Studies (GWAS)}
% Findings to date
% Limitations and challenges

\section{Epigenetic Modifications}
\label{sec:epigenetics}

\subsection{DNA Methylation}
% Patterns in ME/CFS
% Genes affected
% Functional consequences

\subsection{Histone Modifications}
% Chromatin remodeling
% Gene expression changes

\subsection{MicroRNAs}
% Altered miRNA profiles
% Regulatory effects
% Potential as biomarkers

\section{Gene Expression Patterns}
\label{sec:gene-expression}

% Transcriptomics studies
% Differentially expressed genes
% Pathway analysis
% Cell type-specific expression
