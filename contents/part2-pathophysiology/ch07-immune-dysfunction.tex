\chapter{Immune System Dysfunction}
\label{ch:immune-dysfunction}

Immune abnormalities are among the most consistently documented features of ME/CFS and likely play a central role in disease pathogenesis. The 2024 NIH deep phenotyping study by Walitt et al.\ provided definitive evidence for specific immune abnormalities, including characteristic B cell population shifts and sex-specific patterns of immune dysregulation~\cite{walitt2024deep}. This chapter provides a comprehensive examination of immune dysfunction across the innate and adaptive immune systems, inflammatory mediators, and potential autoimmune mechanisms.

\section{Innate Immunity}
\label{sec:innate-immunity}

The innate immune system provides immediate, non-specific defense against pathogens and plays a critical role in initiating and shaping adaptive immune responses. Multiple components of innate immunity show abnormalities in ME/CFS.

\subsection{Natural Killer (NK) Cell Dysfunction}
\label{sec:nk-cells}

Natural killer cell abnormalities represent one of the most replicated findings in ME/CFS research, with impaired NK cell function reported across numerous independent studies spanning decades.

\subsubsection{Reduced NK Cell Cytotoxicity}

NK cells eliminate virus-infected and malignant cells through direct cytotoxic mechanisms. ME/CFS patients consistently demonstrate:

\begin{itemize}
    \item \textbf{Decreased cytotoxic activity}: Reduced ability to kill target cells (typically K562 erythroleukemia cells in standard assays)
    \item \textbf{Magnitude of impairment}: Cytotoxicity often reduced by 40--60\% compared to healthy controls
    \item \textbf{Correlation with severity}: Lower NK cell function correlates with greater symptom severity in some studies
    \item \textbf{Persistence}: Abnormalities remain stable over time, suggesting a chronic rather than transient dysfunction
\end{itemize}

\subsubsection{Mechanisms of Impaired Cytotoxicity}

Several mechanisms may underlie reduced NK cell function:

\paragraph{Perforin and Granzyme Deficiency}
NK cells kill targets by releasing cytotoxic granules containing perforin (which creates pores in target cell membranes) and granzymes (which trigger apoptosis). Studies have found:
\begin{itemize}
    \item Reduced intracellular perforin content in ME/CFS NK cells
    \item Decreased granzyme B expression
    \item Impaired degranulation despite target cell recognition
    \item Abnormal granule trafficking and release
\end{itemize}

\paragraph{Receptor Abnormalities}
NK cell activation is regulated by a balance between activating and inhibitory receptors:
\begin{itemize}
    \item Altered expression of activating receptors (NKG2D, NKp46, NKp30)
    \item Changed inhibitory receptor profiles
    \item Impaired signaling downstream of activating receptors
    \item Disrupted calcium flux following receptor engagement
\end{itemize}

\paragraph{Metabolic Dysfunction}
NK cells require substantial energy for cytotoxic function:
\begin{itemize}
    \item Impaired glycolytic metabolism in ME/CFS NK cells
    \item Mitochondrial dysfunction affecting ATP production
    \item Reduced metabolic reserve limiting sustained activity
\end{itemize}

\subsubsection{NK Cell Subsets}

Human NK cells are divided into functionally distinct subsets:

\begin{itemize}
    \item \textbf{CD56$^{\text{bright}}$ NK cells}: Primarily produce cytokines; found mainly in lymphoid tissues
    \item \textbf{CD56$^{\text{dim}}$ NK cells}: Primarily cytotoxic; predominate in peripheral blood
\end{itemize}

ME/CFS studies have reported:
\begin{itemize}
    \item Altered CD56$^{\text{bright}}$/CD56$^{\text{dim}}$ ratios
    \item Increased proportion of CD56$^{\text{bright}}$ cells in some studies
    \item Reduced absolute numbers of CD56$^{\text{dim}}$ cytotoxic cells
    \item Abnormal maturation patterns
\end{itemize}

\subsubsection{Clinical Significance of NK Cell Dysfunction}

Impaired NK cell function may contribute to ME/CFS through several mechanisms:

\begin{enumerate}
    \item \textbf{Viral reactivation}: Inadequate control of latent herpesviruses (EBV, HHV-6, CMV)
    \item \textbf{Tumor surveillance}: Theoretical increased cancer risk (though not clearly demonstrated)
    \item \textbf{Immune regulation}: NK cells modulate other immune cells; dysfunction may permit chronic inflammation
    \item \textbf{Infection susceptibility}: Reduced defense against new infections
\end{enumerate}

\subsection{Neutrophil and Monocyte Function}
\label{sec:neutrophils-monocytes}

\subsubsection{Neutrophil Abnormalities}

Neutrophils are the most abundant circulating white blood cells and serve as first responders to infection. ME/CFS-associated abnormalities include:

\paragraph{Phagocytosis Impairment}
\begin{itemize}
    \item Reduced uptake of bacteria and particles
    \item Impaired phagosome formation
    \item Decreased acidification of phagolysosomes
\end{itemize}

\paragraph{Respiratory Burst Defects}
The respiratory burst produces reactive oxygen species to kill ingested pathogens:
\begin{itemize}
    \item Reduced superoxide production in some studies
    \item Impaired NADPH oxidase function
    \item Altered baseline oxidative status
\end{itemize}

\paragraph{Chemotaxis Impairment}
\begin{itemize}
    \item Reduced migration toward chemoattractants
    \item Impaired directional sensing
    \item Decreased expression of chemokine receptors
\end{itemize}

\paragraph{Neutrophil Extracellular Traps (NETs)}
NETs are web-like structures of DNA and antimicrobial proteins released by neutrophils:
\begin{itemize}
    \item Altered NET formation in ME/CFS
    \item Potential contribution to inflammation if excessive
    \item May explain some autoimmune features
\end{itemize}

\subsubsection{Monocyte and Macrophage Dysfunction}

Monocytes and their tissue-resident derivatives (macrophages) bridge innate and adaptive immunity:

\paragraph{Monocyte Subset Alterations}
Human monocytes are classified into three subsets:
\begin{itemize}
    \item \textbf{Classical (CD14$^{++}$CD16$^{-}$)}: Phagocytic, antimicrobial
    \item \textbf{Intermediate (CD14$^{++}$CD16$^{+}$)}: Antigen presentation, cytokine production
    \item \textbf{Non-classical (CD14$^{+}$CD16$^{++}$)}: Patrolling, vascular surveillance
\end{itemize}

ME/CFS studies have found:
\begin{itemize}
    \item Increased intermediate monocytes (associated with inflammation)
    \item Altered cytokine production profiles
    \item Abnormal response to stimulation
    \item Changed expression of activation markers
\end{itemize}

\paragraph{Macrophage Polarization}
Tissue macrophages can adopt pro-inflammatory (M1) or anti-inflammatory (M2) phenotypes:
\begin{itemize}
    \item Evidence for M1 polarization in ME/CFS
    \item Impaired transition to resolving M2 phenotype
    \item Chronic inflammatory macrophage activation
\end{itemize}

\subsection{Complement System}
\label{sec:complement}

The complement system consists of plasma proteins that enhance (``complement'') antibody and phagocyte function. Abnormalities in ME/CFS include:

\subsubsection{Complement Activation Patterns}

\begin{itemize}
    \item \textbf{Elevated activation products}: Increased C3a, C4a, and C5a fragments indicating ongoing activation
    \item \textbf{Reduced complement components}: Decreased C3 and C4 levels suggesting consumption
    \item \textbf{Altered regulation}: Abnormal levels of complement regulatory proteins
\end{itemize}

\subsubsection{Clinical Implications}

Complement abnormalities may contribute to:
\begin{itemize}
    \item Inflammation through anaphylatoxin (C3a, C5a) production
    \item Impaired pathogen clearance
    \item Autoimmune manifestations
    \item Mast cell activation (complement fragments trigger mast cell degranulation)
\end{itemize}

\subsection{Dendritic Cells}
\label{sec:dendritic-cells}

Dendritic cells (DCs) are professional antigen-presenting cells that initiate adaptive immune responses:

\begin{itemize}
    \item \textbf{Altered maturation}: Abnormal expression of co-stimulatory molecules
    \item \textbf{Changed cytokine production}: Skewed toward pro-inflammatory profiles
    \item \textbf{Impaired antigen presentation}: May contribute to inadequate pathogen clearance
    \item \textbf{Plasmacytoid DC abnormalities}: Altered type I interferon production
\end{itemize}

\section{Adaptive Immunity}
\label{sec:adaptive-immunity}

The adaptive immune system provides specific, long-lasting responses through T and B lymphocytes. The NIH deep phenotyping study identified characteristic abnormalities in B cell populations that may represent a biomarker signature for ME/CFS~\cite{walitt2024deep}.

\subsection{T Cell Abnormalities}
\label{sec:t-cells}

T lymphocytes coordinate adaptive immune responses and directly eliminate infected cells.

\subsubsection{T Cell Subset Distribution}

\paragraph{CD4/CD8 Ratio Changes}
The ratio of helper (CD4$^{+}$) to cytotoxic (CD8$^{+}$) T cells is altered in some ME/CFS patients:
\begin{itemize}
    \item Variable findings across studies
    \item Some report decreased CD4/CD8 ratio
    \item Others find increased ratio
    \item Heterogeneity may reflect patient subgroups
\end{itemize}

\paragraph{Helper T Cell Subsets}
CD4$^{+}$ T cells differentiate into functional subsets:
\begin{itemize}
    \item \textbf{Th1 cells}: Produce interferon-gamma; promote cell-mediated immunity
    \item \textbf{Th2 cells}: Produce IL-4, IL-5, IL-13; promote antibody responses
    \item \textbf{Th17 cells}: Produce IL-17; involved in autoimmunity and mucosal defense
    \item \textbf{Regulatory T cells (Tregs)}: Suppress immune responses; maintain tolerance
\end{itemize}

ME/CFS findings include:
\begin{itemize}
    \item Th1/Th2 imbalance (variable direction across studies)
    \item Elevated Th17 cells in some patients
    \item Reduced Treg numbers or function
    \item Altered cytokine profiles reflecting subset imbalances
\end{itemize}

\subsubsection{T Cell Exhaustion Markers}

Chronic antigen exposure can lead to T cell exhaustion, characterized by:
\begin{itemize}
    \item \textbf{Increased PD-1 expression}: Programmed death-1, an inhibitory receptor
    \item \textbf{Elevated Tim-3}: T cell immunoglobulin and mucin domain-3
    \item \textbf{CTLA-4 upregulation}: Cytotoxic T-lymphocyte-associated protein 4
    \item \textbf{Reduced proliferative capacity}: Impaired response to stimulation
    \item \textbf{Decreased cytokine production}: Despite activation marker expression
\end{itemize}

These findings suggest chronic immune stimulation in ME/CFS, consistent with persistent infection or autoimmune processes.

\subsubsection{Regulatory T Cell Dysfunction}

Tregs maintain immune tolerance and prevent autoimmunity. ME/CFS abnormalities include:
\begin{itemize}
    \item Reduced Treg numbers (CD4$^{+}$CD25$^{+}$FoxP3$^{+}$ cells)
    \item Impaired suppressive function
    \item Altered Treg/effector T cell ratios
    \item Potential contribution to autoimmune features
\end{itemize}

\subsubsection{Sex-Specific T Cell Findings from the NIH Study}

The Walitt et al.\ deep phenotyping study revealed striking sex differences in T cell abnormalities~\cite{walitt2024deep}:

\paragraph{Male Patients}
Men with PI-ME/CFS demonstrated:
\begin{itemize}
    \item Altered T cell activation patterns
    \item Changes in markers of innate immunity
    \item Distinct inflammatory signatures compared to female patients
\end{itemize}

These findings suggest that immune pathophysiology may differ fundamentally between sexes, with implications for treatment approaches.

\subsection{B Cell Function and Antibodies}
\label{sec:b-cells}

B lymphocytes produce antibodies and present antigens to T cells. The NIH deep phenotyping study provided definitive evidence for characteristic B cell abnormalities in PI-ME/CFS~\cite{walitt2024deep}.

\subsubsection{B Cell Population Shifts: Key NIH Findings}

The Walitt et al.\ study documented a specific pattern of B cell subset abnormalities that may represent a diagnostic signature:

\paragraph{Increased Naïve B Cells}
Naïve B cells have not yet encountered their cognate antigen and can respond to any new threat:
\begin{itemize}
    \item Significantly elevated in PI-ME/CFS patients compared to controls
    \item Reflects either increased production or impaired maturation
    \item May indicate abnormal B cell development or survival
    \item Could represent immune system ``reset'' following infection
\end{itemize}

\paragraph{Decreased Switched Memory B Cells}
Switched memory B cells have undergone class-switch recombination and provide rapid, specific responses to previously encountered pathogens:
\begin{itemize}
    \item Significantly reduced in PI-ME/CFS patients
    \item Suggests impaired generation of long-term humoral immunity
    \item May explain susceptibility to recurrent infections
    \item Could result from chronic antigenic stimulation ``exhausting'' the memory pool
\end{itemize}

\paragraph{Interpretation: Chronic Antigenic Stimulation}
The NIH study concluded that this B cell pattern ``suggested chronic antigenic stimulation''~\cite{walitt2024deep}. This interpretation implies:
\begin{itemize}
    \item Persistent immune activation, possibly from ongoing infection or autoimmunity
    \item Continuous recruitment of naïve B cells into responses
    \item Depletion of the memory B cell compartment through sustained activation
    \item Potential for developing autoantibodies through aberrant B cell selection
\end{itemize}

\begin{open_question}[Naïve vs.\ Memory B Cell Imbalance]
The NIH study found elevated naïve B cells and reduced memory B cells in PI-ME/CFS patients. Does this represent an immune system ``stuck'' in early activation, continuously attempting new responses but failing to consolidate immunological memory? If so, what maintains this state---persistent antigen, aberrant signaling, or microenvironmental factors? Could interventions promoting B cell maturation (e.g., targeted cytokine support, germinal center modulation) restore normal immune function and break the cycle of chronic activation?
\end{open_question}

\subsubsection{Autoantibodies in ME/CFS}

Multiple autoantibodies have been identified in ME/CFS patients:

\paragraph{Anti-Nuclear Antibodies (ANA)}
\begin{itemize}
    \item Present in 20--30\% of ME/CFS patients (compared to 5--10\% of healthy individuals)
    \item Usually low titer
    \item Various patterns (homogeneous, speckled, nucleolar)
    \item Clinical significance unclear
\end{itemize}

\paragraph{Receptor Autoantibodies}
Functionally significant autoantibodies targeting cellular receptors have been identified:

\begin{itemize}
    \item \textbf{$\beta$-adrenergic receptor antibodies}: Target $\beta_1$ and $\beta_2$ adrenergic receptors; may cause cardiovascular and autonomic symptoms
    \item \textbf{Muscarinic acetylcholine receptor antibodies}: Target M1--M5 receptors; may cause autonomic, cognitive, and gastrointestinal symptoms
    \item \textbf{$\alpha_1$-adrenergic receptor antibodies}: May affect vascular function
    \item \textbf{Angiotensin II type 1 receptor antibodies}: May affect blood pressure regulation
\end{itemize}

These receptor autoantibodies can either:
\begin{itemize}
    \item Activate receptors (agonistic), causing overstimulation
    \item Block receptors (antagonistic), preventing normal signaling
    \item Modulate receptor function in complex ways
\end{itemize}

\paragraph{Anti-Neuronal Antibodies}
Autoantibodies targeting nervous system components:
\begin{itemize}
    \item Anti-ganglioside antibodies
    \item Anti-neuronal nuclear antibodies
    \item Antibodies against ion channels
    \item May contribute to neurological symptoms
\end{itemize}

\subsubsection{Immunoglobulin Levels}

Serum immunoglobulin levels show variable abnormalities:
\begin{itemize}
    \item \textbf{IgG}: May be low (selective IgG subclass deficiency) or elevated
    \item \textbf{IgA}: Sometimes reduced, particularly secretory IgA
    \item \textbf{IgM}: Variable findings
    \item \textbf{IgE}: May be elevated in patients with allergic features
\end{itemize}

\subsubsection{Sex-Specific B Cell Findings from the NIH Study}

The deep phenotyping study revealed that female patients showed distinct B cell abnormalities~\cite{walitt2024deep}:

\paragraph{Female Patients}
Women with PI-ME/CFS demonstrated:
\begin{itemize}
    \item Abnormal B cell proliferation patterns
    \item Distinct white blood cell growth characteristics
    \item Different inflammatory markers compared to male patients
\end{itemize}

These sex-specific findings underscore that ME/CFS may involve fundamentally different immunological processes in men and women, potentially requiring sex-specific therapeutic approaches.

\section{Cytokines and Inflammatory Mediators}
\label{sec:cytokines}

Cytokines are signaling proteins that coordinate immune responses. Cytokine abnormalities in ME/CFS have been extensively studied, though findings vary considerably across studies.

\subsection{Pro-inflammatory Cytokines}
\label{sec:pro-inflammatory}

\subsubsection{Interleukin-1 (IL-1)}

IL-1 is a master regulator of inflammation:
\begin{itemize}
    \item \textbf{IL-1$\beta$}: Often elevated in ME/CFS
    \item \textbf{Effects}: Fever, fatigue, muscle breakdown, acute phase response
    \item \textbf{CNS effects}: Produces ``sickness behavior'' closely resembling ME/CFS symptoms
    \item \textbf{Correlation}: Levels may correlate with symptom severity
\end{itemize}

\subsubsection{Interleukin-6 (IL-6)}

IL-6 has both pro- and anti-inflammatory effects:
\begin{itemize}
    \item Frequently elevated in ME/CFS, particularly in early illness
    \item Induces acute phase proteins
    \item Promotes B cell differentiation
    \item Crosses blood-brain barrier to affect CNS function
    \item Correlation with fatigue in other conditions
\end{itemize}

\subsubsection{Tumor Necrosis Factor-Alpha (TNF-$\alpha$)}

TNF-$\alpha$ is a central inflammatory cytokine:
\begin{itemize}
    \item Elevated in some ME/CFS studies
    \item Causes fatigue, malaise, cognitive dysfunction
    \item Affects mitochondrial function
    \item Promotes muscle wasting (cachexia)
    \item Variable findings may reflect patient heterogeneity
\end{itemize}

\subsubsection{Interferons}

Type I interferons (IFN-$\alpha$, IFN-$\beta$) are antiviral cytokines:
\begin{itemize}
    \item Elevated in some ME/CFS patients
    \item Cause profound fatigue (known from therapeutic use)
    \item May indicate ongoing viral activation
    \item Interferon-induced gene expression patterns observed
\end{itemize}

Type II interferon (IFN-$\gamma$):
\begin{itemize}
    \item Activates macrophages and promotes Th1 responses
    \item Variable findings in ME/CFS
    \item May be elevated or reduced depending on disease stage
\end{itemize}

\subsubsection{Cytokine Patterns Across Disease Duration}

A landmark study found distinct cytokine profiles based on illness duration:

\begin{itemize}
    \item \textbf{Early illness ($<$3 years)}: Multiple elevated pro-inflammatory cytokines
    \item \textbf{Longer illness ($>$3 years)}: Cytokine levels normalize or decrease
    \item \textbf{Interpretation}: Initial inflammatory phase may transition to immunosuppression or exhaustion
\end{itemize}

This temporal pattern has implications for treatment timing and biomarker development.

\subsection{Anti-inflammatory Cytokines}
\label{sec:anti-inflammatory}

\subsubsection{Interleukin-10 (IL-10)}

IL-10 is a potent immunosuppressive cytokine:
\begin{itemize}
    \item Variable findings in ME/CFS
    \item May be elevated (attempting to control inflammation) or reduced (permitting inflammation)
    \item Important for resolving immune responses
    \item Produced by regulatory T cells and other cell types
\end{itemize}

\subsubsection{Transforming Growth Factor-Beta (TGF-$\beta$)}

TGF-$\beta$ has immunosuppressive and tissue remodeling functions:
\begin{itemize}
    \item Often elevated in ME/CFS
    \item May represent attempt to control inflammation
    \item Can promote fibrosis if chronically elevated
    \item Important for Treg development
\end{itemize}

\subsubsection{Balance Between Pro- and Anti-inflammatory Signals}

The key issue in ME/CFS may not be absolute cytokine levels but rather:
\begin{itemize}
    \item Imbalanced pro-/anti-inflammatory ratios
    \item Inappropriate cytokine responses to stimuli
    \item Failure to resolve inflammation
    \item Chronic low-grade immune activation
\end{itemize}

\subsection{Chemokines}
\label{sec:chemokines}

Chemokines direct immune cell migration to sites of infection or inflammation:

\subsubsection{Recruitment Patterns}

\begin{itemize}
    \item \textbf{CCL2 (MCP-1)}: Monocyte recruitment; often elevated
    \item \textbf{CCL5 (RANTES)}: T cell and NK cell recruitment
    \item \textbf{CXCL8 (IL-8)}: Neutrophil recruitment
    \item \textbf{CXCL10 (IP-10)}: Interferon-induced; T cell recruitment
\end{itemize}

\subsubsection{Tissue Infiltration}

Elevated chemokines may promote:
\begin{itemize}
    \item Immune cell infiltration into tissues (muscle, brain, gut)
    \item Local inflammation
    \item Tissue damage
    \item Symptom generation through inflammatory mediators
\end{itemize}

\section{Immune Activation and Inflammation}
\label{sec:immune-activation}

\subsection{Chronic Immune Activation}
\label{sec:chronic-activation}

Evidence for ongoing immune activation in ME/CFS includes:

\subsubsection{Activation Markers}

\begin{itemize}
    \item \textbf{Neopterin}: Produced by activated macrophages; elevated in ME/CFS
    \item \textbf{$\beta_2$-microglobulin}: Marker of immune cell turnover; often elevated
    \item \textbf{Soluble CD25 (sIL-2R)}: Released by activated T cells
    \item \textbf{Soluble CD14}: Marker of monocyte/macrophage activation
\end{itemize}

\subsubsection{Consequences for Energy Metabolism}

Chronic immune activation is metabolically expensive:
\begin{itemize}
    \item Immune cells are highly metabolically active
    \item Cytokines alter whole-body metabolism
    \item Competition for nutrients between immune and other tissues
    \item May explain fatigue through metabolic drain
\end{itemize}

\subsubsection{Connection to Symptoms}

Cytokines and inflammatory mediators directly cause many ME/CFS symptoms:
\begin{itemize}
    \item \textbf{Fatigue}: IL-1, IL-6, TNF-$\alpha$, interferons
    \item \textbf{Cognitive dysfunction}: Pro-inflammatory cytokines cross BBB
    \item \textbf{Pain}: Sensitization of nociceptors by inflammatory mediators
    \item \textbf{Sleep disturbance}: Cytokine effects on sleep regulation
    \item \textbf{Fever/chills}: Pyrogenic cytokines
\end{itemize}

\subsection{Neuroinflammation}
\label{sec:neuroinflammation}

The brain was traditionally considered ``immune privileged,'' but it is now recognized that peripheral inflammation affects brain function.

\subsubsection{Microglial Activation}

Microglia are the brain's resident immune cells:
\begin{itemize}
    \item PET imaging shows increased TSPO binding (marker of microglial activation)
    \item Activation persists years after initial infection
    \item Produces local cytokines affecting neuronal function
    \item May explain cognitive symptoms
\end{itemize}

\subsubsection{Blood-Brain Barrier Dysfunction}

BBB compromise permits:
\begin{itemize}
    \item Entry of peripheral cytokines
    \item Infiltration of immune cells
    \item Exposure of brain to circulating autoantibodies
    \item Direct pathogen entry in some cases
\end{itemize}

\subsubsection{Cytokine Effects on Brain Function}

Peripheral cytokines affect the brain through:
\begin{itemize}
    \item Transport across BBB
    \item Signaling via vagal afferents
    \item Acting at circumventricular organs (lacking BBB)
    \item Inducing local cytokine production by glia
\end{itemize}

Brain effects include:
\begin{itemize}
    \item Altered neurotransmitter synthesis and release
    \item Changed receptor expression
    \item Modified synaptic plasticity
    \item ``Sickness behavior'' (fatigue, social withdrawal, anhedonia)
\end{itemize}

\subsubsection{Neuroimaging Evidence}

Studies have demonstrated:
\begin{itemize}
    \item Increased microglial activation on PET
    \item Elevated CSF inflammatory markers
    \item Correlation between brain inflammation and symptoms
    \item Persistence of neuroinflammation
\end{itemize}

\section{Viral Reactivation and Persistence}
\label{sec:viral}

Many ME/CFS cases follow acute infections, and evidence suggests ongoing viral activity in some patients.

\subsection{Herpesviruses}
\label{sec:herpesviruses}

Human herpesviruses establish lifelong latent infections with potential for reactivation.

\subsubsection{Epstein-Barr Virus (EBV)}

EBV infects B cells and establishes latency:
\begin{itemize}
    \item \textbf{Acute infection}: Infectious mononucleosis is a common ME/CFS trigger
    \item \textbf{Reactivation markers}: Elevated early antigen (EA) antibodies, viral load
    \item \textbf{Prevalence}: 10--20\% of ME/CFS patients show evidence of reactivation
    \item \textbf{Mechanism}: May drive chronic B cell activation and autoantibody production
\end{itemize}

\subsubsection{Human Herpesvirus 6 (HHV-6)}

HHV-6 infects T cells and can integrate into chromosomes:
\begin{itemize}
    \item Two species: HHV-6A and HHV-6B
    \item Evidence for active infection in some ME/CFS patients
    \item Can affect mitochondrial function
    \item Neurotropic (infects brain tissue)
\end{itemize}

\subsubsection{Cytomegalovirus (CMV)}

CMV establishes latency in monocytes and other cells:
\begin{itemize}
    \item Reactivation documented in some ME/CFS patients
    \item Can cause significant inflammation upon reactivation
    \item Associated with T cell exhaustion
\end{itemize}

\subsubsection{Reactivation Patterns}

Herpesvirus reactivation in ME/CFS may be:
\begin{itemize}
    \item \textbf{Consequence}: Result of impaired immune control (NK cell dysfunction)
    \item \textbf{Cause}: Driver of ongoing immune activation
    \item \textbf{Both}: Part of a vicious cycle of immune dysfunction and viral activation
\end{itemize}

\subsection{Other Implicated Viruses}
\label{sec:other-viruses}

\subsubsection{Enteroviruses}

Enteroviruses (Coxsackieviruses, Echoviruses) have been implicated:
\begin{itemize}
    \item Detection of viral RNA in muscle and gut biopsies
    \item Elevated antibodies in some patients
    \item Possible persistent low-level infection
    \item Historical associations with epidemic ME/CFS outbreaks
\end{itemize}

\subsubsection{Parvovirus B19}

Parvovirus B19 can cause chronic arthritis and fatigue:
\begin{itemize}
    \item Associated with ME/CFS onset in some patients
    \item Viral DNA detectable in tissues years after infection
    \item May persist in bone marrow and synovium
\end{itemize}

\subsubsection{SARS-CoV-2 and Long COVID}

The COVID-19 pandemic highlighted viral triggers for ME/CFS-like illness:
\begin{itemize}
    \item Long COVID shares many features with ME/CFS
    \item Viral persistence documented in some patients
    \item Similar immune abnormalities observed
    \item Provides opportunity to study post-infectious ME/CFS from known onset
\end{itemize}

\section{Autoimmunity in ME/CFS}
\label{sec:autoimmunity}

Evidence increasingly supports autoimmune mechanisms in at least a subset of ME/CFS patients.

\subsection{Autoantibodies Identified}
\label{sec:autoantibodies}

\subsubsection{Anti-Nuclear Antibodies}

ANA prevalence is elevated in ME/CFS:
\begin{itemize}
    \item 20--30\% positive (vs. 5--10\% in healthy individuals)
    \item Various patterns observed
    \item Significance unclear; may indicate general immune dysregulation
\end{itemize}

\subsubsection{Receptor Autoantibodies}

Functionally relevant autoantibodies have been identified:

\paragraph{$\beta$-Adrenergic Receptor Antibodies}
\begin{itemize}
    \item Target $\beta_1$ and $\beta_2$ receptors
    \item Present in 25--30\% of ME/CFS patients in some studies
    \item May cause cardiovascular symptoms
    \item Potential treatment target (immunoadsorption)
\end{itemize}

\paragraph{Muscarinic Acetylcholine Receptor Antibodies}
\begin{itemize}
    \item Target M1--M5 muscarinic receptors
    \item Found in significant proportion of patients
    \item May cause autonomic, cognitive, and GI symptoms
    \item Correlate with symptom severity in some studies
\end{itemize}

\subsubsection{Anti-Neuronal Antibodies}

Antibodies targeting nervous system components:
\begin{itemize}
    \item Anti-ganglioside antibodies
    \item Antibodies against voltage-gated ion channels
    \item Anti-neuronal surface antigen antibodies
    \item May contribute to neurological symptoms
\end{itemize}

\subsection{Autoimmune Mechanisms}
\label{sec:autoimmune-mechanisms}

\subsubsection{Molecular Mimicry}

Structural similarity between pathogen and self-antigens:
\begin{itemize}
    \item Antibodies or T cells generated against infection cross-react with self
    \item Documented for several viruses associated with ME/CFS
    \item May explain link between infection and autoimmunity
\end{itemize}

\subsubsection{Epitope Spreading}

Tissue damage exposes new antigens:
\begin{itemize}
    \item Initial immune response causes tissue injury
    \item Released self-antigens trigger new autoimmune responses
    \item Progressive expansion of autoimmune targets
\end{itemize}

\subsubsection{Loss of Self-Tolerance}

Regulatory mechanisms fail:
\begin{itemize}
    \item Treg dysfunction permits autoreactive cells
    \item B cell tolerance checkpoints fail
    \item Chronic inflammation promotes autoimmunity
\end{itemize}

\section{Connections to Allergies and Mast Cell Activation}
\label{sec:allergies-mast-cells}

Many ME/CFS patients report increased sensitivity to foods, medications, and environmental factors.

\subsection{Mast Cell Activation Syndrome (MCAS)}
\label{sec:mcas}

\subsubsection{Overlap with ME/CFS}

MCAS involves inappropriate mast cell degranulation:
\begin{itemize}
    \item Substantial symptom overlap with ME/CFS
    \item Fatigue, cognitive dysfunction, pain common in both
    \item May represent comorbidity or shared pathophysiology
    \item Estimated 30--50\% of ME/CFS patients may have MCAS features
\end{itemize}

\subsubsection{Histamine and Other Mediators}

Mast cells release numerous mediators:
\begin{itemize}
    \item \textbf{Histamine}: Causes many typical symptoms
    \item \textbf{Tryptase}: Marker of mast cell activation
    \item \textbf{Prostaglandins}: Inflammatory mediators
    \item \textbf{Leukotrienes}: Cause bronchoconstriction and inflammation
    \item \textbf{Cytokines}: IL-6, TNF-$\alpha$, and others
\end{itemize}

\subsubsection{Diagnostic Criteria}

MCAS diagnosis requires:
\begin{itemize}
    \item Typical symptoms
    \item Elevated mast cell mediators (tryptase, histamine metabolites)
    \item Response to mast cell-directed therapy
\end{itemize}

\subsubsection{Treatment Implications}

If MCAS is present, mast cell stabilizers and antihistamines may help:
\begin{itemize}
    \item H1 antihistamines (cetirizine, fexofenadine)
    \item H2 antihistamines (famotidine)
    \item Mast cell stabilizers (cromolyn sodium, ketotifen)
    \item Low-histamine diet
\end{itemize}

\begin{observation}[Patient-Reported MCAS Treatment Benefits]
Patient communities consistently report that a subset of ME/CFS and Long COVID patients experience meaningful symptom improvement with MCAS-directed therapies, even absent formal MCAS diagnosis. Commonly reported benefits include reduced ``brain fog,'' fewer panic-like episodes, decreased flushing, and improved gastrointestinal symptoms. A typical empirical approach involves a 2-week trial of strict low-histamine diet combined with H1/H2 antihistamines. While controlled trial data are lacking, the low risk profile and potential for significant benefit in the MCAS-overlap subgroup justify consideration of empirical trials in patients with compatible symptom patterns (flushing, urticaria, food reactions, autonomic episodes). Formal prevalence studies and randomized controlled trials are needed to establish which patients benefit and optimal treatment protocols.
\end{observation}

\subsection{Allergic Responses}
\label{sec:allergic-responses}

\subsubsection{Food Sensitivities}

Many ME/CFS patients report food intolerances:
\begin{itemize}
    \item May be IgE-mediated (true allergy) or non-IgE-mediated
    \item Common triggers: gluten, dairy, histamine-rich foods
    \item Mechanism may involve mast cell activation or gut barrier dysfunction
    \item Elimination diets help some patients
\end{itemize}

\subsubsection{Environmental Allergies}

Increased sensitivity to:
\begin{itemize}
    \item Pollen, dust mites, mold
    \item Chemical sensitivities (fragrances, cleaning products)
    \item Medication sensitivities
    \item May reflect mast cell hyperreactivity or neurogenic inflammation
\end{itemize}

\subsubsection{Shared Immune Pathways}

Links between allergy and ME/CFS:
\begin{itemize}
    \item Th2 skewing in some patients
    \item Elevated IgE in subsets
    \item Mast cell dysfunction
    \item Neurogenic inflammation (sensory nerve-mast cell interactions)
\end{itemize}

\section{Summary: Integrated Model of Immune Dysfunction}
\label{sec:immune-summary}

The immune abnormalities in ME/CFS form a coherent, if complex, picture~\cite{walitt2024deep}:

\begin{enumerate}
    \item \textbf{Triggering event}: Infection or other immune challenge initiates the process

    \item \textbf{Innate immune dysfunction}: NK cells and other innate effectors fail to clear the pathogen or control reactivation

    \item \textbf{Chronic antigenic stimulation}: Persistent infection or autoimmunity drives ongoing B cell activation, producing the characteristic naïve B cell expansion and switched memory B cell depletion documented by the NIH study

    \item \textbf{Autoantibody development}: Aberrant B cell responses generate autoantibodies targeting receptors and other self-antigens

    \item \textbf{T cell exhaustion}: Chronic stimulation exhausts T cell responses

    \item \textbf{Cytokine dysregulation}: Ongoing inflammation produces symptom-causing cytokines

    \item \textbf{Sex-specific patterns}: Men and women show different immune abnormalities, suggesting distinct pathophysiological pathways

    \item \textbf{Neuroinflammation}: Peripheral immune signals affect brain function, contributing to fatigue and cognitive symptoms

    \item \textbf{Mast cell involvement}: Mast cell activation may amplify symptoms in susceptible individuals
\end{enumerate}

This model provides multiple potential therapeutic targets: antiviral agents for persistent infection, immunomodulators for autoimmunity, mast cell stabilizers for those with MCAS, and anti-inflammatory approaches for cytokine-mediated symptoms. The recognition of sex-specific immune patterns may eventually enable personalized treatment selection.
