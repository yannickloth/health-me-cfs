\chapter{Integrative Models and Related Phenomena}
\label{ch:integrative-models}

\begin{flushright}
\textit{``All models are wrong, but some are useful.''}\\
--- George E.P.\ Box
\end{flushright}

\vspace{1em}

This chapter attempts to synthesize the diverse findings presented in previous chapters into coherent models of ME/CFS pathophysiology. We present these models with explicit acknowledgment of their evidence levels, from well-established observations to speculative hypotheses. The goal is intellectual honesty: to distinguish what we know, what we suspect, and what we're guessing.

\section{Evidence Classification Framework}
\label{sec:evidence-classification}

Before presenting hypotheses, we define our evidence classification system. This framework is conservative---we classify based on the \textit{weakest} link in the evidence chain.

\begin{table}[htbp]
\centering
\caption{Evidence Level Definitions}
\label{tab:evidence-levels}
\begin{tabular}{p{2.5cm}p{4cm}p{6cm}}
\toprule
\textbf{Level} & \textbf{Definition} & \textbf{What This Means} \\
\midrule
\textbf{Established} & Replicated in multiple independent studies with consistent findings & High confidence this is real; disagreement is about interpretation, not existence \\
\addlinespace
\textbf{Probable} & Documented in $\geq$2 studies OR single large/well-designed study & Likely real, but replication needed; could be overturned \\
\addlinespace
\textbf{Preliminary} & Single study or small studies with suggestive findings & Interesting signal, but may not replicate; treat as hypothesis \\
\addlinespace
\textbf{Theoretical} & Biologically plausible based on known mechanisms, but not directly tested in ME/CFS & Reasonable extrapolation from other conditions; needs direct testing \\
\addlinespace
\textbf{Speculative} & Creative hypothesis without direct supporting data & May inspire research but should not guide treatment decisions \\
\bottomrule
\end{tabular}
\end{table}

\begin{observation}[Honest Uncertainty]
Much of what follows involves substantial uncertainty. The ME/CFS field has been plagued by premature certainty---both from those who dismissed the illness as psychological and from those who promoted specific biological theories without adequate evidence. We aim to avoid both errors by clearly labeling our confidence levels and acknowledging where we may be wrong.
\end{observation}


\section{Comprehensive Hypothesis Ranking}
\label{sec:hypothesis-ranking}

Table~\ref{tab:hypothesis-ranking} presents the major hypotheses about ME/CFS pathophysiology, ranked by our assessment of their likelihood of being substantially correct. This ranking is inherently subjective and will change as new evidence emerges. We weight: (1) quality and quantity of direct evidence, (2) explanatory power for core symptoms, (3) consistency with treatment responses, and (4) biological plausibility.

\begin{landscape}
\begin{longtable}{p{3.2cm}p{1.8cm}p{4.5cm}p{4.5cm}p{4cm}p{2.5cm}}
\caption{Ranked Hypotheses of ME/CFS Pathophysiology} \label{tab:hypothesis-ranking} \\
\toprule
\textbf{Hypothesis} & \textbf{Evidence Level} & \textbf{Key Supporting Evidence} & \textbf{Explains Which Symptoms/Observations} & \textbf{Treatment Implications} & \textbf{Potential for Rapid Benefit} \\
\midrule
\endfirsthead
\multicolumn{6}{c}{\tablename\ \thetable{} -- continued from previous page} \\
\toprule
\textbf{Hypothesis} & \textbf{Evidence Level} & \textbf{Key Supporting Evidence} & \textbf{Explains Which Symptoms/Observations} & \textbf{Treatment Implications} & \textbf{Potential for Rapid Benefit} \\
\midrule
\endhead
\midrule \multicolumn{6}{r}{Continued on next page} \\
\endfoot
\bottomrule
\endlastfoot

% TIER 1: ESTABLISHED
\multicolumn{6}{l}{\textbf{TIER 1: ESTABLISHED PHENOMENA}} \\
\addlinespace

Post-exertional malaise (PEM) as cardinal feature (\S\ref{sec:pem}) & Established & 2-day CPET studies; universal patient reports; objective physiological decline on day 2 & Exercise intolerance; delayed crashes; why GET harms & Pacing; energy management; avoid overexertion & High (pacing prevents crashes) \\
\addlinespace

Autonomic dysfunction (\S\ref{sec:autonomic}) & Established & Abnormal tilt table tests; HRV abnormalities; POTS prevalence $>$30\% & Orthostatic intolerance; tachycardia; temperature dysregulation; coat hanger pain & Salt/fluids; compression; fludrocortisone; midodrine; ivabradine & Moderate--High \\
\addlinespace

Sleep architecture abnormalities (\S\ref{sec:sleep}) & Established & Polysomnography showing reduced slow-wave, fragmented sleep; universal unrefreshing sleep & Unrefreshing sleep; cognitive dysfunction; fatigue & Sleep hygiene; low-dose trazodone; address comorbid sleep disorders & Moderate \\
\addlinespace

Immune dysregulation (ch07) & Established & Cytokine abnormalities; NK cell dysfunction; T cell subset changes; B cell abnormalities & Flu-like symptoms; susceptibility to infections; post-infectious onset & LDN; immunomodulators; avoid immune stressors & Moderate \\
\addlinespace

\multicolumn{6}{l}{\textbf{TIER 2: PROBABLE MECHANISMS}} \\
\addlinespace

Mitochondrial/energy metabolism dysfunction (\S\ref{sec:mitochondrial-dysfunction}) & Probable & ATP profile abnormalities; Heng 2025 AMP/ADP elevation; lactate abnormalities; metabolomic signatures & Fatigue; exercise intolerance; PEM; muscle weakness & CoQ10; NAD$^+$ precursors; D-ribose; B vitamins & Low--Moderate \\
\addlinespace

Neuroinflammation (ch08) & Probable & PET imaging (Nakatomi); CSF abnormalities; microglial activation markers & Brain fog; cognitive dysfunction; sensory sensitivities; headaches & Anti-inflammatory approaches; LDN; avoid neuroinflammatory triggers & Low--Moderate \\
\addlinespace

GPCR autoantibodies (\S\ref{sec:gpcr-autoantibodies}) & Probable & Elevated anti-$\beta$2, M3, M4 antibodies~\cite{Loebel2016,Bynke2020}; correlation with symptoms~\cite{Sotzny2021}; immunoadsorption responses~\cite{Stein2024immunoadsorption}; monocyte dysfunction~\cite{Hackel2025monocyte} & Autonomic dysfunction; fatigue; muscle symptoms; cytokine dysregulation; why some respond to IA & Immunoadsorption; BC007~\cite{Hohberger2021bc007}; daratumumab~\cite{Fluge2025daratumumab} & Moderate--High (in subset) \\
\addlinespace

Gut microbiome dysbiosis (ch14) & Probable & Reduced butyrate producers; altered diversity; correlation with symptoms & GI symptoms; systemic inflammation; food intolerances & Probiotics; dietary modification; possibly FMT & Low--Moderate \\
\addlinespace

Reduced cerebral blood flow & Probable & SPECT/MRI showing hypoperfusion; correlation with cognitive symptoms & Brain fog; cognitive dysfunction; orthostatic cognitive worsening & Address underlying POTS; potentially vasodilators & Moderate \\
\addlinespace

\multicolumn{6}{l}{\textbf{TIER 3: PRELIMINARY/EMERGING}} \\
\addlinespace

Plasma cell-mediated autoimmunity (\S\ref{hyp:plasma-cell-sanctuary}) & Preliminary & Daratumumab pilot (60\% response); explains rituximab failure; IgG reduction correlates with response & Autoimmune subset; why B-cell depletion failed but plasma cell depletion worked & Daratumumab; combined IA + plasma cell targeting & High (in autoimmune subset) \\
\addlinespace

Vascular-Immune-Energy Triad & Preliminary & Heng 2025 7-biomarker panel; coordinated abnormalities across 3 systems; 91\% diagnostic accuracy & Multi-system nature; why single-target treatments fail & Triple-target protocol; simultaneous intervention & Unknown (untested) \\
\addlinespace

Endothelial dysfunction / microclotting (ch14) & Preliminary & Elevated VWF, fibronectin, thrombospondin; Long COVID microclot findings & Exercise intolerance; brain fog; multi-system involvement & Anticoagulation; fibrinolytics; endothelial support & Moderate (if confirmed) \\
\addlinespace

Central catecholamine deficiency & Preliminary & Walitt 2024 CSF findings (reduced DOPA, DOPAC, DHPG); effort preference abnormality & Altered effort perception; motivation difficulties; why ``pushing through'' fails & Dopamine precursors?; stimulants with caution & Unknown \\
\addlinespace

NAD$^+$ depletion (ch14) & Preliminary & Metabolomic abnormalities; 2025 NR trial in Long COVID; theoretical PARP consumption & Energy failure; mitochondrial dysfunction; immune cell dysfunction & NR/NMN 1000--2000~mg; prolonged treatment ($>$10 weeks) & Low (slow onset) \\
\addlinespace

Small fiber neuropathy & Preliminary & Skin biopsy studies; correlation with dysautonomia; elevated in subset & Pain; autonomic symptoms; temperature regulation issues & IVIG (in some); immunomodulation; symptom management & Moderate (in subset) \\
\addlinespace

Viral persistence/reactivation (ch14) & Preliminary & HHV-6 miRNA in CNS; elevated herpesvirus antibodies; EBV reactivation markers & Post-infectious onset; relapsing course; why antivirals help some & Valacyclovir; valganciclovir; potentially IVIG & Low--Moderate \\
\addlinespace

EBV-driven CNS autoimmunity & Preliminary & EBV-infected B cells cross BBB~\cite{Pless2026ebv_demyelination}; LMP1 expression enables brain infiltration; complement/microglial activation & Post-EBV onset; neuroinflammation; brain fog distinct from peripheral fatigue & Antivirals; B cell depletion; complement inhibition & Moderate (in EBV+ subset) \\
\addlinespace

Autoantibody-monocyte reprogramming (\S\ref{hyp:autoantibody-monocyte}) & Preliminary & GPCR autoantibodies reprogram monocyte cytokine production~\cite{Hackel2025monocyte}; MIP-1$\delta$, PDGF-BB, TGF-$\beta$3 elevation & Systemic inflammation; why effects persist beyond receptor binding; tissue remodeling & Autoantibody removal + monocyte modulation (JAK inhibitors) & Moderate--High \\
\addlinespace

\multicolumn{6}{l}{\textbf{TIER 4: THEORETICAL}} \\
\addlinespace

Glymphatic clearance failure (\S\ref{sec:glymphatic}) & Theoretical & Sleep dysfunction; cognitive symptoms; craniocervical junction issues in subset & Brain fog; unrefreshing sleep; position-dependent symptoms & Address CCI if present; optimize slow-wave sleep & Unknown \\
\addlinespace

Tryptophan/kynurenine trap (\S\ref{sec:kynurenine-trap}) & Theoretical & IDO activation documented; tryptophan pathway abnormalities; elevated QUIN:KYNA ratio in some studies & Depression-like symptoms; neuroinflammation; NAD$^+$ depletion & IDO inhibitors?; shift pathway toward KYNA & Unknown \\
\addlinespace

Circadian desynchronization (ch14) & Theoretical & Cortisol rhythm abnormalities; sleep timing issues; fluctuating symptoms & Unpredictable symptom patterns; unrefreshing sleep; why timing matters & Chronotherapy; melatonin; time-restricted feeding; light therapy & Moderate \\
\addlinespace

Epigenetic ``lock'' & Theoretical & DNA methylation changes documented; duration predicts prognosis; why early intervention helps & Persistence; treatment resistance; why disease stabilizes & Epigenetic modifiers (experimental); early aggressive treatment & Unknown \\
\addlinespace

Purinergic signaling dysregulation & Theoretical & ATP is danger signal; P2X7 and inflammation; exercise releases ATP & PEM delay (24--72h matches DTH kinetics); pain sensitization; inflammation & P2X7 antagonists (experimental) & Unknown \\
\addlinespace

\multicolumn{6}{l}{\textbf{TIER 5: SPECULATIVE}} \\
\addlinespace

``Safe mode'' / stuck sickness behavior & Speculative & Fits symptom pattern; evolutionarily plausible; explains why pushing harms & All core symptoms as adaptive (but stuck) response & Reset hypothalamic setpoint?; break the ``lock'' & Unknown \\
\addlinespace

HERV reactivation & Speculative & HERVs can be de-silenced; would explain persistent immune activation without pathogen & Post-viral onset; autoimmunity; female predominance & Antiretrovirals?; epigenetic silencing? & Unknown \\
\addlinespace

Ion channel autoimmunity & Speculative & Precedent in other conditions (LEMS, MG); would explain ``wired but tired'' & Sensory sensitivities; autonomic dysfunction; muscle fatigue; cardiac symptoms & Plasmapheresis; IVIG; channel-specific interventions & Moderate (if confirmed) \\
\addlinespace

Receptor internalization (not blockade) & Speculative & NMDA receptor autoantibodies cause internalization~\cite{Kim2026nmdar_cryoem}; would explain lag between Ab removal and recovery & Why symptoms persist after immunoadsorption; need for receptor regeneration time & Autoantibody removal + time for receptor resynthesis & Moderate (delayed) \\
\addlinespace

Lactate compartmentalization (MCT dysfunction) & Speculative & Lactate abnormalities documented; would explain tissue-specific symptoms & PEM; muscle symptoms; brain fog; why systemic lactate seems okay & DCA?; lactate supplementation? & Unknown \\
\addlinespace

Ferroptosis susceptibility & Speculative & Lipid abnormalities; oxidative stress; iron dysregulation documented & Why high-energy tissues affected; why iron supplementation can harm & Ferroptosis inhibitors; careful with iron & Unknown \\
\addlinespace

Trained endotheliopathy & Speculative & Endothelial markers elevated (Heng 2025); innate immune training established; vascular symptoms & Multi-system involvement; persistent endothelial activation; microvascular dysfunction & Vascular-focused protocol; epigenetic reversal? & Unknown \\

\end{longtable}
\end{landscape}

\subsection{Interpretation Notes}

\begin{enumerate}
    \item \textbf{Ranking reflects current evidence, not ultimate truth.} The ``Speculative'' hypotheses may prove correct; the ``Established'' phenomena may be reinterpreted. Science is provisional.

    \item \textbf{Multiple hypotheses may be simultaneously true.} ME/CFS is almost certainly heterogeneous. Different patients may have different primary drivers, and individual patients may have multiple contributing mechanisms.

    \item \textbf{``Treatment implications'' does not mean ``proven treatment.''} We list logical therapeutic consequences of each hypothesis, not demonstrated efficacy. Very few ME/CFS treatments have robust RCT support.

    \item \textbf{``Potential for rapid benefit'' is our subjective assessment} of how quickly patients might improve \textit{if} the hypothesis is correct \textit{and} appropriate treatment is applied. ``Unknown'' means we cannot predict.

    \item \textbf{Severely ill patients face different considerations.} Some interventions (immunoadsorption, daratumumab) require hospital access impossible for bedbound patients. Others (pacing, supplements) are accessible. The table does not capture this dimension adequately.
\end{enumerate}


\section{Synthesis: What the Evidence Suggests}
\label{sec:synthesis}

Drawing together the ranked hypotheses, several patterns emerge:

\subsection{The Core Triad: Energy-Immune-Autonomic}

Three systems show consistent abnormalities across evidence levels:

\begin{enumerate}
    \item \textbf{Energy metabolism} (mitochondrial dysfunction, ATP depletion, metabolomic abnormalities)
    \item \textbf{Immune function} (cytokine dysregulation, autoantibodies, NK cell dysfunction)
    \item \textbf{Autonomic regulation} (POTS, HRV abnormalities, catecholamine changes)
\end{enumerate}

The Heng 2025 study~\cite{heng2025mecfs} suggests these are not independent---the 7-biomarker panel spanning all three systems achieved 91\% diagnostic accuracy, implying coordinated dysfunction. This has profound implications:

\begin{itemize}
    \item Treatments targeting only one system may fail because the others maintain dysfunction
    \item Patient subgroups may differ in which system predominates, not which system is involved
    \item A ``multi-lock'' model (see Chapter~\ref{ch:speculative-hypotheses}) may explain treatment resistance
\end{itemize}

\subsection{The Autoimmune Subgroup}

The daratumumab pilot trial (60\% response) provides the strongest evidence yet for an autoimmune mechanism in \textit{a subset} of patients. Key insights:

\begin{itemize}
    \item Rituximab (anti-CD20, targets B cells) failed in large trials
    \item Daratumumab (anti-CD38, targets plasma cells) succeeded in pilot
    \item This suggests \textbf{long-lived plasma cells}, not B cells, are the critical autoantibody source
    \item The 60\% response rate implies heterogeneity---not all ME/CFS is autoimmune
    \item Biomarkers for patient selection are urgently needed
\end{itemize}

\begin{observation}[The Rituximab Puzzle Solved?]
The daratumumab finding may explain one of ME/CFS research's biggest disappointments. Rituximab showed promise in early trials but failed in the large Norwegian RCT. If the critical autoantibodies come from long-lived plasma cells (CD38$^+$, CD20$^-$), rituximab would deplete the wrong cells. Existing plasma cells would continue producing autoantibodies for months, and by the time B cells returned, no improvement would be evident. The trial ``failed'' not because autoimmunity isn't involved, but because the wrong cells were targeted.
\end{observation}

\subsection{The Vascular Dimension}

Elevated VWF, fibronectin, and thrombospondin~\cite{heng2025mecfs} point to \textbf{endothelial activation}---the blood vessel lining is chronically stressed. This connects to:

\begin{itemize}
    \item Long COVID microclot findings
    \item Cerebral hypoperfusion documented in ME/CFS
    \item Exercise intolerance (endothelium cannot vasodilate properly)
    \item Multi-system involvement (endothelium is everywhere)
\end{itemize}

If ME/CFS is partly an \textbf{endotheliopathy}, vascular-targeted treatments (anticoagulation, fibrinolytics, endothelial support) might help---but this remains preliminary.

\subsection{The Central Nervous System}

The Walitt 2024 finding of altered \textbf{effort preference} (not physical fatigue) localizes part of the problem to the brain. Combined with:

\begin{itemize}
    \item CSF catecholamine deficiency
    \item Neuroinflammation on PET imaging
    \item Cognitive dysfunction correlating with perfusion
    \item Brainstem abnormalities
\end{itemize}

This suggests ME/CFS involves a \textbf{central state change}---the brain is computing effort-reward differently, possibly appropriately given peripheral metabolic dysfunction, but creating the subjective experience of profound unwillingness/inability to exert.

\subsection{The ``Stuck'' State}

Multiple hypotheses converge on the idea that ME/CFS represents a \textbf{stable pathological state} that resists perturbation:

\begin{itemize}
    \item Epigenetic changes may ``lock'' gene expression patterns
    \item Autoantibodies from long-lived plasma cells provide continuous dysfunction
    \item Metabolic pathway shifts may be self-perpetuating
    \item The brain's effort computation may be recalibrated
    \item Circadian rhythms may be desynchronized
\end{itemize}

This ``multi-lock'' concept (detailed in Chapter~\ref{ch:speculative-hypotheses}) suggests why:
\begin{itemize}
    \item Single interventions rarely produce cures
    \item Early treatment may prevent lock stabilization
    \item Disease duration correlates with prognosis
    \item Some patients spontaneously recover (locks didn't fully stabilize)
    \item Treatment may need to target multiple locks simultaneously
\end{itemize}


\section{Proposed Unifying Mechanisms}
\label{sec:unifying-mechanisms}

\subsection{Vicious Cycle Models}

Several vicious cycles may perpetuate ME/CFS:

\paragraph{Inflammation-Metabolism Cycle.}
\begin{enumerate}
    \item Inflammation activates IDO, shunting tryptophan toward kynurenine
    \item Kynurenine pathway produces neurotoxic quinolinic acid
    \item Neuroinflammation maintains cytokine production
    \item Cytokines perpetuate IDO activation
\end{enumerate}

\paragraph{Energy-Immune Cycle.}
\begin{enumerate}
    \item Mitochondrial dysfunction depletes ATP
    \item Immune cells cannot complete activation/maturation (ATP-dependent)
    \item Dysfunctional immune response fails to clear triggers
    \item Persistent triggers maintain inflammation
    \item Inflammation impairs mitochondria
\end{enumerate}

\paragraph{Autonomic-Vascular Cycle.}
\begin{enumerate}
    \item Autonomic dysfunction impairs vascular regulation
    \item Poor perfusion causes tissue hypoxia
    \item Hypoxia triggers HIF pathway and metabolic shifts
    \item Metabolic abnormalities affect autonomic centers
\end{enumerate}

\paragraph{Exertion-Crash Cycle.}
\begin{enumerate}
    \item Patient feels slightly better, increases activity
    \item Activity exceeds metabolic capacity
    \item Post-exertional crash (24--72 hours delayed)
    \item Crash worsens baseline, triggers immune/metabolic responses
    \item Partial recovery, patient attempts activity again
\end{enumerate}

Breaking these cycles is the goal of effective treatment---but which cycle to break, and how, likely differs between patients.

\subsection{Multisystem Failure Cascade}

A proposed sequence for ME/CFS development:

\paragraph{Phase 1: Triggering Event.}
\begin{itemize}
    \item Infection (EBV, enteroviruses, SARS-CoV-2, others)
    \item Severe stress (physical, psychological, surgical)
    \item Combination of factors in vulnerable individual
\end{itemize}

\paragraph{Phase 2: Acute Response.}
\begin{itemize}
    \item Normal sickness behavior program activates
    \item Metabolic suppression, immune activation, behavioral changes
    \item This is \textit{adaptive}---conserving resources for recovery
\end{itemize}

\paragraph{Phase 3: Failed Resolution.}
\begin{itemize}
    \item In most people, acute phase resolves in days to weeks
    \item In ME/CFS-susceptible individuals, resolution fails
    \item Possible reasons: genetic susceptibility, severity of insult, timing, comorbidities
\end{itemize}

\paragraph{Phase 4: Lock Establishment.}
\begin{itemize}
    \item Autoantibodies generated and plasma cells established
    \item Epigenetic changes stabilize ``sick'' gene expression
    \item Metabolic pathways shift to new equilibrium
    \item Brain recalibrates effort computation
    \item Autonomic setpoints shift
\end{itemize}

\paragraph{Phase 5: Stable Pathological State.}
\begin{itemize}
    \item Multiple locks reinforce each other
    \item Perturbations (exertion, stress, infection) trigger defensive responses
    \item Spontaneous recovery becomes unlikely
    \item Treatment must address multiple locks
\end{itemize}


\section{Hypothesis-Specific Treatment Implications}
\label{sec:treatment-implications}

Table~\ref{tab:treatment-by-hypothesis} maps hypotheses to their logical treatment implications, with honest assessment of evidence and accessibility.

\begin{longtable}{p{3.5cm}p{4cm}p{2cm}p{2.5cm}p{3cm}}
\caption{Treatment Implications by Hypothesis} \label{tab:treatment-by-hypothesis} \\
\toprule
\textbf{Hypothesis} & \textbf{Logical Treatment} & \textbf{Evidence for Treatment} & \textbf{Accessibility} & \textbf{Notes} \\
\midrule
\endfirsthead
\multicolumn{5}{c}{\tablename\ \thetable{} -- continued from previous page} \\
\toprule
\textbf{Hypothesis} & \textbf{Logical Treatment} & \textbf{Evidence for Treatment} & \textbf{Accessibility} & \textbf{Notes} \\
\midrule
\endhead
\midrule \multicolumn{5}{r}{Continued on next page} \\
\endfoot
\bottomrule
\endlastfoot

Autonomic dysfunction & Salt/fluids; compression; fludrocortisone; midodrine; ivabradine; beta-blockers (ch14b) & Moderate (POTS literature) & High & Often first-line; helps many \\
\addlinespace

GPCR autoantibodies & Immunoadsorption; BC007; daratumumab & Preliminary--Moderate & Very Low (specialized centers) & Most promising for autoimmune subset \\
\addlinespace

Plasma cell autoimmunity & Daratumumab; bortezomib & Preliminary (pilot study) & Very Low & 60\% response in pilot \\
\addlinespace

Mitochondrial dysfunction & CoQ10 (ubiquinol); NAD$^+$ precursors; D-ribose; B vitamins; PQQ (\S\ref{sec:mitochondrial-support}) & Low--Moderate & High & Widely used; modest benefit for many \\
\addlinespace

NAD$^+$ depletion & NR/NMN 1000--2000~mg/day for $\geq$10 weeks & Preliminary & Moderate (cost) & RCT in Long COVID showed NAD$^+$ increase \\
\addlinespace

Neuroinflammation & LDN; anti-inflammatories; avoid triggers & Low--Moderate & High (LDN) & LDN widely used; helps some \\
\addlinespace

Gut dysbiosis & Probiotics; dietary changes; possibly FMT & Low & High (probiotics) to Very Low (FMT) & Variable response \\
\addlinespace

Endothelial dysfunction & L-citrulline/arginine; statins; low-dose aspirin; omega-3s & Theoretical & High & Untested in ME/CFS specifically \\
\addlinespace

Viral persistence & Valacyclovir; valganciclovir (\S\ref{sec:antivirals}) & Low & Moderate & May help subset with viral markers \\
\addlinespace

Small fiber neuropathy & IVIG; immunomodulation & Preliminary & Low (IVIG access) & Helps some with documented SFN \\
\addlinespace

Circadian disruption & Melatonin; light therapy; time-restricted feeding; chronotherapy & Theoretical & High & Low risk; may help sleep \\
\addlinespace

Glymphatic failure & Address CCI if present; optimize sleep; position & Theoretical & Variable & CCI surgery controversial \\
\addlinespace

\end{longtable}

\begin{observation}[The Accessibility Problem]
The most promising emerging treatments (daratumumab, immunoadsorption) are essentially inaccessible to most patients---requiring specialized centers, costing tens of thousands of dollars, and often not covered by insurance. Meanwhile, accessible interventions (supplements, pacing) have modest effect sizes. This creates a cruel disparity where the sickest patients, often unable to travel or advocate for themselves, have the least access to potentially transformative treatments.
\end{observation}

\section{Relationships to Other Conditions}
\label{sec:related-conditions}

\subsection{Fibromyalgia}
% Symptom overlap
% Shared mechanisms
% Distinct features
% Comorbidity patterns

\subsection{Postural Orthostatic Tachycardia Syndrome (POTS)}
% Overlap and distinction
% Autonomic mechanisms
% Treatment considerations

\subsection{Mast Cell Activation Syndrome}
% Shared features
% Diagnostic challenges
% Combined treatment approaches

\subsection{Autoimmune Conditions}
% Systemic lupus erythematosus
% Sjögren's syndrome
% Multiple sclerosis
% Shared immunological features
% Why the immune system connection matters

\subsection{Ehlers-Danlos Syndrome}
% Hypermobility and ME/CFS
% Connective tissue involvement
% Prevalence of comorbidity

\subsection{Long COVID (Post-Acute Sequelae of SARS-CoV-2)}
% Symptom similarities
% Pathophysiological overlap
% Lessons from COVID-19 research
% Implications for ME/CFS understanding

\subsection{Multiple Chemical Sensitivity}
% Chemical intolerances in ME/CFS
% Proposed mechanisms
% Overlap with environmental illness

\subsection{Allergic and Atopic Conditions}
% Immune system involvement
% Mast cell connection
% Histamine intolerance
% Is there a mechanistic link?

\section{Systems Biology Approaches}
\label{sec:systems-biology}

% Network analysis
% Multi-omics integration
% Computational modeling
% Identifying key regulatory nodes

\section{Outstanding Questions}
\label{sec:questions}

% What triggers ME/CFS onset?
% Why do some people recover while others don't?
% What maintains the disease state?
% How do different subtypes differ mechanistically?
