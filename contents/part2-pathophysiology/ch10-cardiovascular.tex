\chapter{Cardiovascular Dysfunction}
\label{ch:cardiovascular}

Cardiovascular abnormalities are pervasive in ME/CFS and contribute substantially to disability, particularly through exercise intolerance and orthostatic symptoms. The 2024 NIH deep phenotyping study by Walitt et al.\ provided rigorous documentation of cardiopulmonary exercise testing abnormalities, including reduced peak oxygen consumption and chronotropic incompetence, establishing objective physiological correlates of the subjective exercise intolerance reported by patients~\cite{walitt2024deep}.

\section{Cardiac Function}
\label{sec:cardiac-function}

\subsection{Exercise Testing Abnormalities}
\label{sec:exercise-testing}

Cardiopulmonary exercise testing (CPET) provides objective measurement of integrated cardiovascular, pulmonary, and metabolic function during physical exertion. CPET findings in ME/CFS represent some of the most reproducible objective abnormalities documented in the illness.

\subsubsection{Cardiopulmonary Exercise Testing (CPET) Methodology}

CPET involves graded exercise (typically on a cycle ergometer or treadmill) with continuous measurement of:

\begin{itemize}
    \item \textbf{Oxygen consumption (VO$_2$)}: Volume of oxygen extracted from inspired air per unit time
    \item \textbf{Carbon dioxide production (VCO$_2$)}: Volume of CO$_2$ expired
    \item \textbf{Respiratory exchange ratio (RER)}: VCO$_2$/VO$_2$, indicating fuel substrate utilization
    \item \textbf{Minute ventilation (VE)}: Total volume of air breathed per minute
    \item \textbf{Heart rate}: Continuous electrocardiographic monitoring
    \item \textbf{Blood pressure}: Periodic measurements during exercise
    \item \textbf{Work rate}: Power output (watts) or speed/grade
\end{itemize}

Testing continues until volitional exhaustion or limiting symptoms, with criteria for maximal effort including RER $>$1.10, achievement of age-predicted maximal heart rate, or a plateau in VO$_2$ despite increasing work rate.

\subsubsection{Key NIH Deep Phenotyping CPET Findings}

The Walitt et al.\ study documented several critical cardiopulmonary abnormalities in PI-ME/CFS patients~\cite{walitt2024deep}:

\paragraph{Reduced Peak Oxygen Consumption (VO$_2$peak)}
Peak VO$_2$ represents maximal aerobic capacity and integrates cardiac output, oxygen delivery, and peripheral oxygen extraction:

\begin{itemize}
    \item PI-ME/CFS patients demonstrated significantly reduced VO$_2$peak compared to matched healthy controls
    \item Reduction indicates impaired aerobic capacity that cannot be explained by deconditioning alone
    \item Correlates with functional limitation and disability
    \item Objective confirmation of patient-reported exercise intolerance
\end{itemize}

The magnitude of VO$_2$peak reduction in ME/CFS typically ranges from 15--30\% below predicted values, with more severely affected patients showing greater reductions.

\paragraph{Chronotropic Incompetence}
Chronotropic incompetence refers to an inadequate heart rate response to exercise:

\begin{itemize}
    \item ME/CFS patients fail to achieve age-predicted maximal heart rate
    \item Heart rate rise is blunted relative to work rate increases
    \item Chronotropic index (proportion of heart rate reserve used) is reduced
    \item Indicates autonomic dysfunction affecting cardiac pacing
\end{itemize}

Chronotropic incompetence limits cardiac output augmentation during exercise, as cardiac output = heart rate $\times$ stroke volume. Without adequate heart rate increase, oxygen delivery to exercising muscles is compromised.

\paragraph{Mechanisms of Chronotropic Incompetence}
Several mechanisms may underlie the inadequate heart rate response:

\begin{enumerate}
    \item \textbf{Parasympathetic excess}: Sustained vagal tone preventing heart rate acceleration
    \item \textbf{Sympathetic dysfunction}: Impaired catecholamine release or receptor sensitivity
    \item \textbf{Sinoatrial node dysfunction}: Intrinsic pacemaker abnormality
    \item \textbf{Beta-adrenergic receptor autoantibodies}: Blocking receptor activation
    \item \textbf{Central nervous system dysfunction}: Impaired autonomic outflow
\end{enumerate}

\subsubsection{Two-Day CPET Protocol}

A particularly informative methodology involves repeat CPET on consecutive days:

\paragraph{Rationale}
Single CPET testing may not capture the distinctive post-exertional deterioration characteristic of ME/CFS. Two-day protocols assess recovery capacity and reproducibility of maximal effort.

\paragraph{Findings in ME/CFS}
\begin{itemize}
    \item \textbf{Day 1}: Reduced but measurable aerobic capacity
    \item \textbf{Day 2}: Further significant reductions in VO$_2$peak, anaerobic threshold, and work capacity
    \item \textbf{Healthy controls}: Reproduce or slightly improve Day 1 performance
    \item \textbf{Magnitude}: ME/CFS patients show 10--25\% decline on Day 2
\end{itemize}

This failure to reproduce exercise capacity is highly specific to ME/CFS and reflects the pathognomonic post-exertional malaise. The two-day protocol has been proposed as an objective diagnostic marker.

\paragraph{Mechanisms of Day 2 Decline}
\begin{itemize}
    \item Delayed recovery of metabolic substrates
    \item Persistent inflammatory activation
    \item Autonomic dysfunction exacerbation
    \item Mitochondrial damage from oxidative stress
    \item Central nervous system effects (increased perceived exertion)
\end{itemize}

\subsubsection{Anaerobic Threshold}
\label{sec:anaerobic-threshold}

The anaerobic threshold (AT, also called ventilatory threshold or lactate threshold) represents the exercise intensity at which anaerobic metabolism begins to supplement aerobic energy production:

\begin{itemize}
    \item \textbf{Reduced AT in ME/CFS}: Occurs at lower work rates and VO$_2$ levels
    \item \textbf{Early lactate accumulation}: Muscles rely on anaerobic glycolysis sooner
    \item \textbf{Implications}: Limited sustainable activity before symptom exacerbation
    \item \textbf{Mechanism}: Reflects impaired oxygen delivery, mitochondrial dysfunction, or both
\end{itemize}

The reduced AT has practical implications: patients exceed their aerobic capacity during activities that healthy individuals perform entirely aerobically, leading to metabolic stress and symptom generation.

\subsubsection{Ventilatory Efficiency}
\label{sec:ventilatory-efficiency}

Ventilatory efficiency describes how effectively ventilation eliminates CO$_2$, typically expressed as the VE/VCO$_2$ slope:

\begin{itemize}
    \item \textbf{Increased VE/VCO$_2$ slope}: More ventilation required per unit CO$_2$ eliminated
    \item \textbf{Causes}: Ventilation-perfusion mismatch, increased dead space, hyperventilation
    \item \textbf{Consequences}: Dyspnea at lower work rates, earlier exercise termination
    \item \textbf{ME/CFS findings}: Variable; some patients show ventilatory inefficiency
\end{itemize}

\subsection{Cardiac Output and Stroke Volume}
\label{sec:cardiac-output}

Cardiac output (CO) determines oxygen delivery capacity and is the product of heart rate and stroke volume.

\subsubsection{Preload Failure Hypothesis}

Multiple lines of evidence support inadequate cardiac preload (ventricular filling) as a contributor to ME/CFS cardiovascular dysfunction:

\begin{itemize}
    \item \textbf{Reduced end-diastolic volume}: Less blood fills the ventricles during diastole
    \item \textbf{Decreased stroke volume}: By Frank-Starling mechanism, reduced preload produces smaller stroke volume
    \item \textbf{Compensatory tachycardia}: Heart rate increases to maintain cardiac output (until chronotropic incompetence limits this)
    \item \textbf{Exercise limitation}: Inadequate cardiac output augmentation
\end{itemize}

\paragraph{Evidence for Preload Failure}
\begin{itemize}
    \item Echocardiographic studies showing reduced left ventricular end-diastolic volume
    \item Invasive hemodynamic measurements demonstrating low filling pressures
    \item Response to volume expansion (saline infusion) improving symptoms
    \item Correlation with blood volume measurements
\end{itemize}

\subsubsection{Reduced Blood Volume}

Blood volume deficits are well-documented in ME/CFS:

\begin{itemize}
    \item \textbf{Plasma volume}: Reduced by 10--20\% in most studies
    \item \textbf{Red cell mass}: Variable findings; may be proportionally reduced or relatively preserved
    \item \textbf{Total blood volume}: Typically 10--15\% below normal
    \item \textbf{Correlation with symptoms}: Lower blood volume correlates with worse orthostatic intolerance and fatigue
\end{itemize}

\paragraph{Mechanisms of Hypovolemia}
\begin{itemize}
    \item \textbf{RAAS dysfunction}: Impaired aldosterone response to hypovolemia
    \item \textbf{Natriuretic peptide elevation}: Promoting sodium and water excretion
    \item \textbf{Reduced erythropoietin}: Leading to mild anemia in some patients
    \item \textbf{Capillary leak}: Increased vascular permeability shifting fluid to interstitium
    \item \textbf{Inadequate fluid intake}: Secondary to nausea or other symptoms
\end{itemize}

\subsubsection{Venous Pooling}

Excessive venous pooling in dependent body parts reduces venous return:

\begin{itemize}
    \item \textbf{Lower extremity pooling}: Blood accumulates in leg veins during standing
    \item \textbf{Splanchnic pooling}: Blood redistributes to abdominal vasculature
    \item \textbf{Impaired venoconstriction}: Venous tone fails to increase appropriately
    \item \textbf{Consequences}: Reduced cardiac preload, orthostatic symptoms
\end{itemize}

\subsection{Cardiac Biomarkers}
\label{sec:cardiac-biomarkers}

\subsubsection{Troponin}

Cardiac troponins (cTnI, cTnT) are released from damaged cardiomyocytes:

\begin{itemize}
    \item \textbf{Baseline levels}: Generally normal in ME/CFS
    \item \textbf{Post-exercise}: Some studies report mild elevations after exertion
    \item \textbf{Interpretation}: May indicate subclinical myocardial stress or damage
    \item \textbf{Clinical significance}: Unclear; likely below threshold for clinical concern
\end{itemize}

\subsubsection{BNP and NT-proBNP}

B-type natriuretic peptide (BNP) and its N-terminal fragment are released in response to cardiac wall stress:

\begin{itemize}
    \item \textbf{Findings in ME/CFS}: Variable; some studies report mild elevations
    \item \textbf{Mechanism}: May reflect right heart strain from pulmonary issues or left ventricular stress
    \item \textbf{Correlation}: May correlate with fatigue severity in some studies
    \item \textbf{Clinical utility}: Not established as ME/CFS biomarker
\end{itemize}

\subsubsection{Evidence of Cardiac Strain}

Subclinical cardiac dysfunction may occur in ME/CFS:

\begin{itemize}
    \item \textbf{Diastolic dysfunction}: Impaired ventricular relaxation on echocardiography
    \item \textbf{Reduced contractile reserve}: Limited ability to augment function during stress
    \item \textbf{Right ventricular changes}: May occur secondary to pulmonary issues
    \item \textbf{Strain imaging}: Advanced echocardiographic techniques may detect subtle abnormalities
\end{itemize}

\section{Vascular Dysfunction}
\label{sec:vascular}

\subsection{Endothelial Dysfunction}
\label{sec:endothelial}

The vascular endothelium regulates vascular tone, coagulation, and inflammation. Endothelial dysfunction is increasingly recognized in ME/CFS.

\subsubsection{Nitric Oxide Bioavailability}

Nitric oxide (NO) is a critical vasodilator produced by endothelial NO synthase (eNOS):

\begin{itemize}
    \item \textbf{Reduced NO production}: Some ME/CFS studies report decreased NO metabolites
    \item \textbf{Increased NO scavenging}: Oxidative stress may inactivate NO
    \item \textbf{eNOS uncoupling}: Dysfunctional enzyme produces superoxide instead of NO
    \item \textbf{Consequences}: Impaired vasodilation, increased vascular resistance
\end{itemize}

\subsubsection{Flow-Mediated Dilation}

Flow-mediated dilation (FMD) measures endothelium-dependent vasodilation of the brachial artery following brief ischemia:

\begin{itemize}
    \item \textbf{Reduced FMD in ME/CFS}: Several studies report impaired endothelium-dependent dilation
    \item \textbf{Magnitude}: Typically 30--50\% reduction compared to healthy controls
    \item \textbf{Correlation}: May correlate with fatigue severity and autonomic dysfunction
    \item \textbf{Mechanism}: Reflects reduced NO bioavailability or vascular smooth muscle dysfunction
\end{itemize}

\subsubsection{Inflammatory Markers}

Endothelial inflammation contributes to dysfunction:

\begin{itemize}
    \item \textbf{Elevated adhesion molecules}: ICAM-1, VCAM-1, E-selectin
    \item \textbf{Increased inflammatory cytokines}: IL-6, TNF-$\alpha$ affect endothelial function
    \item \textbf{Oxidative stress markers}: Indicate endothelial damage
    \item \textbf{Circulating endothelial cells}: May be elevated, indicating endothelial injury
\end{itemize}

\subsection{Blood Volume Abnormalities}
\label{sec:blood-volume}

\subsubsection{Reduced Plasma Volume}

Plasma volume deficits have been consistently documented:

\begin{itemize}
    \item \textbf{Measurement methods}: Radioisotope dilution (gold standard), carbon monoxide rebreathing, dye dilution
    \item \textbf{Magnitude of reduction}: Typically 10--20\% below predicted
    \item \textbf{Correlation with symptoms}: Orthostatic intolerance, fatigue, cognitive dysfunction
    \item \textbf{Response to treatment}: Volume expansion may improve symptoms
\end{itemize}

\subsubsection{Red Blood Cell Mass}

Red cell mass findings are more variable:

\begin{itemize}
    \item \textbf{Some studies}: Report reduced red cell mass proportional to plasma volume reduction
    \item \textbf{Other studies}: Find relatively preserved red cell mass with disproportionate plasma volume loss
    \item \textbf{Hemoglobin/hematocrit}: May be normal or slightly elevated (hemoconcentration from low plasma volume)
    \item \textbf{Erythropoietin}: Sometimes reduced, potentially explaining mild anemia in some patients
\end{itemize}

\subsubsection{Mechanisms of Volume Depletion}

\paragraph{Renin-Angiotensin-Aldosterone System Dysfunction}
\begin{itemize}
    \item Blunted aldosterone response to hypovolemia
    \item Impaired sodium retention
    \item Inappropriate natriuresis despite low blood volume
\end{itemize}

\paragraph{Natriuretic Peptide Effects}
\begin{itemize}
    \item Elevated ANP or BNP promoting sodium/water excretion
    \item May result from cardiac filling abnormalities
\end{itemize}

\paragraph{Capillary Permeability}
\begin{itemize}
    \item Increased vascular permeability shifting fluid to interstitium
    \item May be inflammation-mediated
    \item Could explain edema in some patients despite hypovolemia
\end{itemize}

\subsection{Microcirculation}
\label{sec:microcirculation}

\subsubsection{Capillary Perfusion}

The microcirculation delivers oxygen and nutrients to tissues and removes metabolic waste:

\begin{itemize}
    \item \textbf{Capillary density}: May be reduced in ME/CFS patients
    \item \textbf{Capillary flow}: Abnormal flow patterns documented by nailfold capillaroscopy
    \item \textbf{Red cell deformability}: Impaired RBC flexibility may impede capillary transit
    \item \textbf{Capillary recruitment}: Inadequate increase in perfused capillaries during exercise
\end{itemize}

\subsubsection{Oxygen Extraction}

Peripheral oxygen extraction may be impaired:

\begin{itemize}
    \item \textbf{Widened arteriovenous O$_2$ difference}: In some studies, suggesting increased extraction to compensate for reduced delivery
    \item \textbf{Impaired extraction}: In others, suggesting mitochondrial dysfunction limiting oxygen utilization
    \item \textbf{Near-infrared spectroscopy}: Documents abnormal muscle oxygenation patterns during exercise
\end{itemize}

\subsubsection{Tissue Hypoxia}

Inadequate oxygen delivery produces tissue hypoxia:

\begin{itemize}
    \item \textbf{Muscle hypoxia}: Contributes to weakness and post-exertional symptoms
    \item \textbf{Cerebral hypoperfusion}: Causes cognitive dysfunction (see Chapter~\ref{ch:neurological})
    \item \textbf{Lactate accumulation}: Results from anaerobic metabolism
    \item \textbf{Symptom generation}: Hypoxia-sensitive nociceptors may trigger pain
\end{itemize}

\section{Blood Pressure Regulation}
\label{sec:blood-pressure}

Blood pressure dysregulation is common in ME/CFS and manifests as various orthostatic disorders.

\subsection{Orthostatic Hypotension}
\label{sec:orthostatic-hypotension}

Orthostatic hypotension (OH) is defined as a sustained reduction in systolic blood pressure $\geq$20 mmHg or diastolic $\geq$10 mmHg within 3 minutes of standing:

\begin{itemize}
    \item \textbf{Prevalence}: Occurs in a subset of ME/CFS patients
    \item \textbf{Symptoms}: Lightheadedness, visual disturbances, weakness, syncope
    \item \textbf{Mechanisms}: Autonomic failure, hypovolemia, medications
    \item \textbf{Initial OH}: Brief BP drop within first 15 seconds (common in ME/CFS)
\end{itemize}

\subsection{Neurally Mediated Hypotension}
\label{sec:nmh}

Neurally mediated hypotension (NMH, also called vasovagal syncope or neurocardiogenic syncope) involves paradoxical vasodilation and bradycardia during prolonged standing:

\begin{itemize}
    \item \textbf{Mechanism}: Vigorous ventricular contraction of underfilled heart triggers vagal reflex
    \item \textbf{Presentation}: Delayed BP drop after 10+ minutes of standing
    \item \textbf{Symptoms}: Nausea, diaphoresis, pallor preceding syncope
    \item \textbf{Testing}: Head-up tilt table testing
\end{itemize}

\subsection{Postural Orthostatic Tachycardia Syndrome (POTS)}
\label{sec:pots}

POTS is characterized by excessive heart rate increase upon standing without significant blood pressure drop:

\subsubsection{Diagnostic Criteria}
\begin{itemize}
    \item Heart rate increase $\geq$30 bpm (or $\geq$40 bpm in adolescents) within 10 minutes of standing
    \item Absence of orthostatic hypotension
    \item Symptoms of orthostatic intolerance
    \item Duration $>$6 months
\end{itemize}

\subsubsection{Prevalence in ME/CFS}
\begin{itemize}
    \item Estimated 25--50\% of ME/CFS patients meet POTS criteria
    \item Substantial symptom overlap between conditions
    \item May represent overlapping or related conditions
    \item Similar pathophysiological mechanisms
\end{itemize}

\subsubsection{POTS Subtypes}
Different pathophysiological mechanisms produce similar clinical phenotypes:

\paragraph{Neuropathic POTS}
\begin{itemize}
    \item Partial autonomic neuropathy affecting lower extremity vasoconstriction
    \item Blood pools in legs during standing
    \item Associated with small fiber neuropathy
    \item May be autoimmune in some cases
\end{itemize}

\paragraph{Hyperadrenergic POTS}
\begin{itemize}
    \item Excessive sympathetic activation
    \item Standing norepinephrine $>$600 pg/mL
    \item Associated with tremor, anxiety, hypertension during episodes
    \item May involve norepinephrine transporter deficiency
\end{itemize}

\paragraph{Hypovolemic POTS}
\begin{itemize}
    \item Low blood volume as primary driver
    \item Compensatory tachycardia to maintain cardiac output
    \item May respond to volume expansion
    \item Overlaps with ME/CFS blood volume deficits
\end{itemize}

\subsection{Hypertension in ME/CFS}
\label{sec:hypertension}

While hypotension is more commonly discussed, hypertension also occurs:

\begin{itemize}
    \item \textbf{Supine hypertension}: Some patients have elevated BP when lying down
    \item \textbf{Labile hypertension}: Wide BP fluctuations
    \item \textbf{Stress-related}: BP spikes during symptom exacerbations
    \item \textbf{Medication-related}: Sympathomimetics for orthostatic symptoms may raise BP
\end{itemize}

\section{Heart Rate Abnormalities}
\label{sec:heart-rate}

\subsection{Resting Tachycardia}
\label{sec:resting-tachycardia}

Many ME/CFS patients exhibit elevated resting heart rate:

\begin{itemize}
    \item \textbf{Mechanism}: Compensatory response to low stroke volume
    \item \textbf{Sympathetic activation}: Chronic low-grade sympathetic overdrive
    \item \textbf{Deconditioning}: Loss of cardiovascular fitness
    \item \textbf{Clinical significance}: Correlates with symptom severity
\end{itemize}

\subsection{Heart Rate Variability}
\label{sec:hrv}

Heart rate variability (HRV) reflects autonomic modulation of the sinoatrial node (see Chapter~\ref{ch:neurological} for detailed discussion). The NIH deep phenotyping study documented significantly reduced HRV in ME/CFS patients~\cite{walitt2024deep}:

\begin{itemize}
    \item \textbf{Reduced overall HRV}: Lower SDNN and total power
    \item \textbf{Diminished parasympathetic markers}: Reduced high-frequency power and RMSSD
    \item \textbf{Altered sympathovagal balance}: Changed LF/HF ratio
    \item \textbf{Prognostic implications}: Low HRV predicts poor health outcomes generally
\end{itemize}

\subsection{Heart Rate Recovery}
\label{sec:hr-recovery}

Heart rate recovery (HRR) after exercise reflects parasympathetic reactivation:

\begin{itemize}
    \item \textbf{Definition}: HR decrease from peak to 1 or 2 minutes post-exercise
    \item \textbf{ME/CFS findings}: Delayed HRR indicating impaired vagal reactivation
    \item \textbf{Clinical significance}: Abnormal HRR predicts mortality in other populations
    \item \textbf{Mechanism}: Consistent with parasympathetic dysfunction
\end{itemize}

\section{Coagulation and Rheological Abnormalities}
\label{sec:coagulation}

\subsection{Hypercoagulability}
\label{sec:hypercoagulability}

Some ME/CFS patients show evidence of increased coagulation activation:

\begin{itemize}
    \item \textbf{Elevated fibrinogen}: Acute phase reactant and clotting factor
    \item \textbf{Increased D-dimer}: Fibrin degradation product indicating clot turnover
    \item \textbf{Platelet activation}: Enhanced platelet aggregability
    \item \textbf{Thrombin generation}: Markers of coagulation cascade activation
\end{itemize}

\subsection{Fibrin Deposition}
\label{sec:fibrin}

Excessive fibrin deposition may impair microcirculation:

\begin{itemize}
    \item \textbf{Soluble fibrin monomer}: Elevated in some patients
    \item \textbf{Fibrin mesh formation}: May coat vessel walls and impede flow
    \item \textbf{Microclot hypothesis}: Recently proposed role of amyloid-like microclots
    \item \textbf{Treatment implications}: Anticoagulation investigated in small trials
\end{itemize}

\subsection{Red Blood Cell Deformability}
\label{sec:rbc-deformability}

Red blood cells must deform to traverse capillaries:

\begin{itemize}
    \item \textbf{Reduced deformability}: Documented in some ME/CFS studies
    \item \textbf{Mechanisms}: Membrane oxidative damage, altered lipid composition
    \item \textbf{Consequences}: Impaired capillary perfusion, tissue hypoxia
    \item \textbf{Measurement}: Ektacytometry, micropipette aspiration
\end{itemize}

\section{Summary: Integrated Cardiovascular Model}
\label{sec:cv-summary}

Cardiovascular dysfunction in ME/CFS involves multiple interacting abnormalities~\cite{walitt2024deep}:

\begin{enumerate}
    \item \textbf{Reduced blood volume}: Hypovolemia compromises cardiac preload and limits cardiac output reserve

    \item \textbf{Autonomic dysfunction}: Parasympathetic withdrawal reduces HRV and impairs baroreflex function; chronotropic incompetence limits exercise heart rate response

    \item \textbf{Endothelial dysfunction}: Impaired vasodilation reduces tissue perfusion

    \item \textbf{Cardiac limitation}: Preload failure and chronotropic incompetence reduce maximal cardiac output

    \item \textbf{Microcirculatory impairment}: Abnormal capillary perfusion and oxygen extraction limit peripheral oxygen delivery

    \item \textbf{Exercise intolerance}: The cumulative effect is reduced VO$_2$peak and early anaerobic threshold, objectively confirmed by the NIH study

    \item \textbf{Post-exertional deterioration}: Unique to ME/CFS, the failure to recover exercise capacity on day 2 CPET reflects pathological response to exertion

    \item \textbf{Orthostatic intolerance}: Blood pressure dysregulation (POTS, NMH, OH) produces symptoms with upright posture
\end{enumerate}

This cardiovascular dysfunction explains much of the disability in ME/CFS: patients cannot sustain physical activity because their cardiovascular system cannot deliver adequate oxygen to meet metabolic demands. The objective documentation of reduced VO$_2$peak and chronotropic incompetence in the NIH deep phenotyping study provides biological validation of patients' reported exercise intolerance.

Treatment approaches targeting cardiovascular dysfunction include volume expansion, medications for orthostatic intolerance, and careful activity management to avoid exceeding the reduced aerobic threshold. The recognition that cardiovascular abnormalities are objective and measurable helps counter misconceptions that ME/CFS exercise intolerance reflects psychological factors or simple deconditioning.
