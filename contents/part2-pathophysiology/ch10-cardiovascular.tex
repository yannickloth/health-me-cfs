% FILE: Cardiovascular mechanisms — POTS, orthostatic intolerance, cardiac output, blood flow dysregulation, autonomic nervous system
\chapter{Cardiovascular Dysfunction}
\label{ch:cardiovascular}

Cardiovascular abnormalities are pervasive in ME/CFS and contribute substantially to disability, particularly through exercise intolerance and orthostatic symptoms. The 2024 NIH deep phenotyping study by Walitt et al.\ provided rigorous documentation of cardiopulmonary exercise testing abnormalities, including reduced peak oxygen consumption and chronotropic incompetence, establishing objective physiological correlates of the subjective exercise intolerance reported by patients~\cite{walitt2024deep}.

\section{Cardiac Function}
\label{sec:cardiac-function}

\subsection{Exercise Testing Abnormalities}
\label{sec:exercise-testing}

Cardiopulmonary exercise testing (CPET) provides objective measurement of integrated cardiovascular, pulmonary, and metabolic function during physical exertion. CPET findings in ME/CFS represent some of the most reproducible objective abnormalities documented in the illness.

\subsubsection{Cardiopulmonary Exercise Testing (CPET) Methodology}

CPET involves graded exercise (typically on a cycle ergometer or treadmill) with continuous measurement of oxygen consumption (VO$_2$, the volume of oxygen extracted from inspired air per unit time), carbon dioxide production (VCO$_2$, the volume of CO$_2$ expired), and their ratio expressed as the respiratory exchange ratio (RER = VCO$_2$/VO$_2$), which indicates fuel substrate utilization. Simultaneously, the system records minute ventilation (VE, total air volume breathed per minute), heart rate via continuous electrocardiographic monitoring, periodic blood pressure measurements, and work rate (power output in watts or treadmill speed and grade).

Testing continues until volitional exhaustion or limiting symptoms. Criteria for maximal effort include an RER exceeding 1.10, achievement of age-predicted maximal heart rate, or a plateau in VO$_2$ despite increasing work rate.

\subsubsection{Key NIH Deep Phenotyping CPET Findings}

The Walitt et al.\ study documented several critical cardiopulmonary abnormalities in PI-ME/CFS patients~\cite{walitt2024deep}:

\paragraph{Reduced Peak Oxygen Consumption (VO$_2$peak)}
Peak VO$_2$ represents maximal aerobic capacity and integrates cardiac output, oxygen delivery, and peripheral oxygen extraction. PI-ME/CFS patients demonstrated significantly reduced VO$_2$peak compared to matched healthy controls~\cite{walitt2024deep}. This reduction indicates impaired aerobic capacity beyond what deconditioning alone would predict; ME/CFS patients showed greater deficits than sedentary controls matched for activity level, with reductions typically ranging from 15--30\% below predicted values~\cite{keller2024cpet,lim2020cpet}. More severely affected patients show greater reductions. The finding correlates with functional limitation and disability, providing objective confirmation of patient-reported exercise intolerance.

\paragraph{Chronotropic Incompetence}
Chronotropic incompetence refers to an inadequate heart rate response to exercise~\cite{walitt2024deep}:

\begin{itemize}
    \item ME/CFS patients fail to achieve age-predicted maximal heart rate
    \item Heart rate rise is blunted relative to work rate increases
    \item Chronotropic index (proportion of heart rate reserve used) is reduced
\end{itemize}

Chronotropic incompetence limits cardiac output augmentation during exercise, as cardiac output = heart rate $\times$ stroke volume. Without adequate heart rate increase, oxygen delivery to exercising muscles is compromised.

\paragraph{Mechanisms of Chronotropic Incompetence}
Several mechanisms have been proposed; their relative contributions in ME/CFS remain under investigation:

\begin{enumerate}
    \item \textbf{Parasympathetic excess}: Sustained vagal tone preventing heart rate acceleration---supported by HRV findings showing altered autonomic balance~\cite{Newton2007autonomicDysfunction}
    \item \textbf{Sympathetic dysfunction}: Impaired catecholamine release or receptor sensitivity
    \item \textbf{Sinoatrial node dysfunction}: Intrinsic pacemaker abnormality (hypothesized but not directly demonstrated in ME/CFS)
    \item \textbf{G-protein-coupled receptor (GPCR) autoantibodies}: A growing body of evidence implicates autoantibodies targeting G-protein-coupled receptors in ME/CFS cardiovascular dysfunction. Loebel et al.\ first documented elevated autoantibodies against beta-adrenergic ($\beta_1$, $\beta_2$) and muscarinic cholinergic (M3, M4) receptors in ME/CFS patients~\cite{Loebel2016}, findings subsequently validated in Swedish cohorts including cerebrospinal fluid samples~\cite{Bynke2020}.

    The cardiovascular effects of these autoantibodies are multifaceted. Beta-adrenergic receptor autoantibodies may exert either agonistic effects (causing inappropriate receptor activation) or antagonistic effects (blocking normal catecholamine signaling), depending on the specific antibody epitope and receptor subtype. Muscarinic receptor autoantibodies similarly can enhance or impair parasympathetic signaling to the heart and vasculature. This bidirectional potential explains why the same class of autoantibodies might produce different phenotypes across patients.

    Levels of vasoregulative GPCR autoantibodies correlate with symptom severity, autonomic dysfunction, and disability in ME/CFS~\cite{Sotzny2021}. The correlation with autonomic measures supports a direct pathophysiological role rather than an epiphenomenon of chronic illness. Beta-2 adrenergic receptor autoantibodies appear particularly relevant to cardiovascular symptoms, with immunoadsorption targeting these antibodies showing preliminary efficacy in post-COVID ME/CFS~\cite{Stein2024immunoadsorption}. BC007, a DNA aptamer that neutralizes GPCR autoantibodies, has shown promise in case reports for improving fatigue and microcirculatory function~\cite{Hohberger2021bc007}.
    \item \textbf{Central nervous system dysfunction}: Impaired autonomic outflow from brainstem centers
    \item \textbf{Gut-mediated vagal impairment (hypothesized)}: Butyrate enhances enterochromaffin cell serotonin production~\cite{Barton2025}, and enterochromaffin serotonin activates vagal afferents via 5-HT$_3$ receptors~\cite{Barton2023,Kaelberer2018}. Since ME/CFS patients show butyrate deficiency, reduced enterochromaffin serotonin could impair vagal afferent signaling, potentially weakening efferent vagal tone to the heart. No direct evidence yet links this pathway to cardiac chronotropy in ME/CFS. See Section~\ref{sec:gut-brain} of Chapter~\ref{ch:gut-microbiome} for the full evidence chain
\end{enumerate}

The finding of chronotropic incompetence, combined with reduced HRV and abnormal baroreflex sensitivity~\cite{Newton2007autonomicDysfunction}, indicates autonomic dysfunction affecting cardiac pacing, though the primary site of dysfunction (central vs.\ peripheral) remains debated.

\subsubsection{Two-Day CPET Protocol}

A particularly informative methodology involves repeat CPET on consecutive days:

\paragraph{Rationale}
Single CPET testing may not capture the distinctive post-exertional deterioration characteristic of ME/CFS. Two-day protocols assess recovery capacity and reproducibility of maximal effort.

\paragraph{Findings in ME/CFS}
\begin{itemize}
    \item \textbf{Day 1}: Reduced but measurable aerobic capacity
    \item \textbf{Day 2}: Further significant reductions in VO$_2$peak, anaerobic threshold, and work capacity
    \item \textbf{Healthy controls}: Reproduce or slightly improve Day 1 performance
    \item \textbf{Magnitude}: ME/CFS patients show 10--25\% decline on Day 2~\cite{keller2024cpet,lim2020cpet}
\end{itemize}

This failure to reproduce exercise capacity distinguishes ME/CFS from other fatiguing conditions and reflects the pathognomonic post-exertional malaise~\cite{Lim2020}. A meta-analysis of two-day CPET studies confirmed significant reductions in work capacity and oxygen consumption on Day 2, supporting this protocol as an objective marker of PEM~\cite{lim2020cpet}.

\paragraph{Mechanisms of Day 2 Decline}
\begin{itemize}
    \item Delayed recovery of metabolic substrates
    \item Persistent inflammatory activation
    \item Autonomic dysfunction exacerbation
    \item Mitochondrial damage from oxidative stress
    \item Central nervous system effects (increased perceived exertion)
\end{itemize}

\subsubsection{Anaerobic Threshold}
\label{sec:anaerobic-threshold}

The anaerobic threshold (AT, also called ventilatory threshold or lactate threshold) represents the exercise intensity at which anaerobic metabolism begins to supplement aerobic energy production:

\begin{itemize}
    \item \textbf{Reduced AT in ME/CFS}: Occurs at lower work rates and VO$_2$ levels
    \item \textbf{Early lactate accumulation}: Muscles rely on anaerobic glycolysis sooner
    \item \textbf{Implications}: Limited sustainable activity before symptom exacerbation
    \item \textbf{Mechanism}: Reflects impaired oxygen delivery, mitochondrial dysfunction, or both
\end{itemize}

The reduced AT has practical implications: patients exceed their aerobic capacity during activities that healthy individuals perform entirely aerobically, leading to metabolic stress and symptom generation.

\subsubsection{Ventilatory Efficiency}
\label{sec:ventilatory-efficiency}

Ventilatory efficiency describes how effectively ventilation eliminates CO$_2$, typically expressed as the VE/VCO$_2$ slope:

\begin{itemize}
    \item \textbf{Increased VE/VCO$_2$ slope}: More ventilation required per unit CO$_2$ eliminated
    \item \textbf{Causes}: Ventilation-perfusion mismatch, increased dead space, hyperventilation
    \item \textbf{Consequences}: Dyspnea at lower work rates, earlier exercise termination
    \item \textbf{ME/CFS findings}: Variable; some patients show ventilatory inefficiency
\end{itemize}

\subsection{Cardiac Output and Stroke Volume}
\label{sec:cardiac-output}

Cardiac output (CO) determines oxygen delivery capacity and is the product of heart rate and stroke volume.

\subsubsection{Preload Failure Hypothesis}

Multiple lines of evidence support inadequate cardiac preload (ventricular filling) as a contributor to ME/CFS cardiovascular dysfunction:

\begin{itemize}
    \item \textbf{Reduced end-diastolic volume}: Less blood fills the ventricles during diastole
    \item \textbf{Decreased stroke volume}: By Frank-Starling mechanism, reduced preload produces smaller stroke volume
    \item \textbf{Compensatory tachycardia at rest}: Heart rate increases to maintain resting cardiac output; however, during exercise, chronotropic incompetence prevents further adequate heart rate augmentation, creating a ceiling effect
    \item \textbf{Exercise limitation}: The combination of low stroke volume and inadequate heart rate response severely limits cardiac output augmentation during exertion
\end{itemize}

\paragraph{Evidence for Preload Failure}
\begin{itemize}
    \item Echocardiographic studies showing reduced left ventricular end-diastolic volume
    \item Invasive hemodynamic measurements demonstrating low filling pressures
    \item Response to volume expansion (saline infusion) improving symptoms
    \item Correlation with blood volume measurements
\end{itemize}

\paragraph{Supine Hemodynamic Abnormalities}
While cardiovascular dysfunction in ME/CFS is most apparent during orthostatic stress, some patients demonstrate abnormalities even at rest in the supine position. Newton et al.\ documented reduced cardiac volumes on cardiac MRI that correlated with blood volume deficits rather than deconditioning, with end-diastolic volume, end-systolic volume, and end-diastolic wall mass all significantly reduced~\cite{Newton2016}. Critically, these reductions showed no correlation with disease duration, arguing against deconditioning as the primary mechanism.

Reduced resting cardiac output in the supine position has been reported in some ME/CFS cohorts, with the magnitude of reduction correlating with symptom severity. This finding suggests that the cardiovascular impairment is not solely a failure of orthostatic compensation but reflects a baseline deficit in cardiac filling and output. Patients with more severe supine abnormalities tend to show greater decompensation during orthostatic challenge, as they have less hemodynamic reserve to mobilize when gravitational stress is applied.

\subsubsection{Reduced Blood Volume}

Blood volume deficits are well-documented in ME/CFS~\cite{Streeten1998blood,Newton2016,Raj2005}:

\begin{itemize}
    \item \textbf{Plasma volume}: Reduced by 10--20\% in most studies~\cite{Streeten1998blood}
    \item \textbf{Red cell mass}: Variable findings; may be proportionally reduced or relatively preserved
    \item \textbf{Total blood volume}: Typically 10--15\% below normal~\cite{Newton2016}
    \item \textbf{Correlation with symptoms}: Lower blood volume correlates with worse orthostatic intolerance and fatigue~\cite{Newton2016}
\end{itemize}

\paragraph{Mechanisms of Hypovolemia}
\begin{itemize}
    \item \textbf{RAAS dysfunction}: Impaired aldosterone response to hypovolemia
    \item \textbf{Natriuretic peptide elevation}: Promoting sodium and water excretion
    \item \textbf{Reduced erythropoietin}: Leading to mild anemia in some patients
    \item \textbf{Capillary leak}: Increased vascular permeability shifting fluid to interstitium
    \item \textbf{Inadequate fluid intake}: Secondary to nausea or other symptoms
\end{itemize}

\subsubsection{Venous Pooling}

Excessive venous pooling in dependent body parts reduces venous return:

\begin{itemize}
    \item \textbf{Lower extremity pooling}: Blood accumulates in leg veins during standing
    \item \textbf{Splanchnic pooling}: Blood redistributes to abdominal vasculature
    \item \textbf{Impaired venoconstriction}: Venous tone fails to increase appropriately
    \item \textbf{Consequences}: Reduced cardiac preload, orthostatic symptoms
\end{itemize}

\subsection{Cardiac Biomarkers}
\label{sec:cardiac-biomarkers}

\subsubsection{Troponin}

Cardiac troponins (cTnI, cTnT) are released from damaged cardiomyocytes:

\begin{itemize}
    \item \textbf{Baseline levels}: Generally normal in ME/CFS
    \item \textbf{Post-exercise}: Some studies report mild elevations after exertion
    \item \textbf{Interpretation}: May indicate subclinical myocardial stress or damage
    \item \textbf{Clinical significance}: Unclear; likely below threshold for clinical concern
\end{itemize}

\subsubsection{BNP and NT-proBNP}

B-type natriuretic peptide (BNP) and its N-terminal fragment are released in response to cardiac wall stress:

\begin{itemize}
    \item \textbf{Findings in ME/CFS}: Variable; some studies report mild elevations
    \item \textbf{Mechanism}: May reflect right heart strain from pulmonary issues or left ventricular stress
    \item \textbf{Correlation}: May correlate with fatigue severity in some studies
    \item \textbf{Clinical utility}: Not established as ME/CFS biomarker
\end{itemize}

\subsubsection{Cardiac Structure and Function}

Echocardiographic studies report variable findings, with evidence for subclinical dysfunction in some patients~\cite{Newton2016}:

\begin{itemize}
    \item \textbf{Reduced cardiac volumes}: Smaller left ventricular end-diastolic volume correlating with plasma volume deficits~\cite{Newton2016}
    \item \textbf{Diastolic dysfunction}: Some studies report impaired ventricular relaxation
    \item \textbf{Reduced contractile reserve}: Limited ability to augment function during stress echocardiography (preliminary data)
    \item \textbf{Strain imaging}: Speckle tracking echocardiography may detect subtle abnormalities not apparent on conventional imaging (requires further study)
\end{itemize}

These findings suggest that cardiac abnormalities in ME/CFS are primarily functional consequences of hypovolemia and autonomic dysfunction rather than primary myocardial disease.

\section{Vascular Dysfunction}
\label{sec:vascular}

\subsection{Endothelial Dysfunction}
\label{sec:endothelial}

The vascular endothelium regulates vascular tone, coagulation, and inflammation. Endothelial dysfunction is increasingly recognized in ME/CFS.

\subsubsection{Nitric Oxide Bioavailability}

Nitric oxide (NO) is a critical vasodilator produced by endothelial NO synthase (eNOS):

\begin{itemize}
    \item \textbf{Reduced NO production}: Some ME/CFS studies report decreased NO metabolites
    \item \textbf{Increased NO scavenging}: Oxidative stress may inactivate NO
    \item \textbf{eNOS uncoupling}: Dysfunctional enzyme produces superoxide instead of NO
    \item \textbf{Consequences}: Impaired vasodilation, increased vascular resistance
\end{itemize}

\subsubsection{Flow-Mediated Dilation}

Flow-mediated dilation (FMD) measures endothelium-dependent vasodilation of the brachial artery following brief ischemia:

\begin{itemize}
    \item \textbf{Reduced FMD in ME/CFS}: Multiple studies report impaired endothelium-dependent dilation, with peripheral endothelial dysfunction found in 51\% of ME/CFS patients versus 20\% of healthy controls~\cite{Scherbakov2020}
    \item \textbf{Correlation}: Associated with disease severity and severity of immune symptoms~\cite{Scherbakov2020}
    \item \textbf{Mechanism}: Reflects reduced NO bioavailability, elevated adhesion molecules, or chronic inflammatory state~\cite{Appel2025}
\end{itemize}

\subsubsection{Inflammatory Markers}

Endothelial inflammation contributes to dysfunction~\cite{Appel2025}:

\begin{itemize}
    \item \textbf{Elevated adhesion molecules}: ICAM-1, VCAM-1, E-selectin~\cite{Appel2025}
    \item \textbf{Increased inflammatory cytokines}: IL-6, TNF-$\alpha$ affect endothelial function
    \item \textbf{Oxidative stress markers}: Indicate endothelial damage
    \item \textbf{Circulating endothelial cells}: May be elevated, indicating endothelial injury
\end{itemize}

\subsection{Blood Volume Abnormalities}
\label{sec:blood-volume}

Blood volume deficits are among the most consistently documented abnormalities in ME/CFS (see also Section~\ref{sec:cardiac-output} for impact on cardiac preload). This section expands on measurement methods and pathophysiological mechanisms.

\subsubsection{Measurement and Magnitude}

\begin{itemize}
    \item \textbf{Measurement methods}: Radioisotope dilution (gold standard), carbon monoxide rebreathing, dye dilution
    \item \textbf{Plasma volume}: Typically 10--20\% below predicted~\cite{Streeten1998blood}
    \item \textbf{Red cell mass}: Variable; some studies report proportional reduction, others find preserved red cell mass with disproportionate plasma volume loss
    \item \textbf{Total blood volume}: 10--15\% below normal in most studies~\cite{Newton2016}
    \item \textbf{Hemoglobin/hematocrit}: May appear normal or elevated due to hemoconcentration
\end{itemize}

\subsubsection{Mechanisms of Volume Depletion}

\paragraph{Renin-Angiotensin-Aldosterone System Dysfunction}
Studies document a paradoxical RAAS response in POTS and ME/CFS, with elevated angiotensin II despite hypovolemia~\cite{Raj2005,Stewart2006}:
\begin{itemize}
    \item Blunted aldosterone response to hypovolemia~\cite{Raj2005}
    \item Impaired sodium retention leading to inappropriate natriuresis
    \item Elevated angiotensin II may contribute to symptoms through vasoconstriction~\cite{Stewart2006}
\end{itemize}

\paragraph{Natriuretic Peptide Effects}
\begin{itemize}
    \item Elevated ANP or BNP promoting sodium/water excretion
    \item May result from atrial stretch due to cardiac filling abnormalities
\end{itemize}

\paragraph{Capillary Permeability}
\begin{itemize}
    \item Increased vascular permeability shifting fluid to interstitium
    \item May be inflammation-mediated (cytokines increase endothelial permeability)
    \item Could explain peripheral edema in some patients despite intravascular hypovolemia
\end{itemize}

\subsection{Arterial Stiffness and Vascular Compliance}
\label{sec:arterial-stiffness}

Beyond endothelial function, arterial mechanical properties influence cardiovascular regulation. Pulse wave velocity (PWV), a measure of arterial stiffness, affects blood pressure regulation through its impact on baroreceptor function. The relationship between arterial stiffness and ME/CFS has not been extensively studied, but related conditions provide insight.

In hypermobile Ehlers-Danlos syndrome (hEDS), which frequently co-occurs with ME/CFS (Section~\ref{sec:septad}), central pulse wave velocity is significantly lower than controls (4.73 m/s versus normal values), indicating excessive arterial elasticity~\cite{Miller2020arterial}. This increased compliance paradoxically impairs blood pressure regulation: stretch receptors in vessel walls (baroreceptors) cannot accurately detect pressure changes when arterial walls are too compliant. The result is impaired baroreflex function and orthostatic intolerance despite (or because of) increased rather than decreased arterial elasticity.

Whether ME/CFS patients without connective tissue disorders show altered arterial stiffness remains unclear. Chronic inflammation typically increases arterial stiffness over time, while autonomic dysfunction could affect vascular smooth muscle tone. The interaction between these competing influences likely varies across patient subgroups. Pulse wave velocity measurement is non-invasive and could provide additional phenotyping data, though its clinical utility in ME/CFS management has not been established.

\subsection{Microcirculation}
\label{sec:microcirculation}

\subsubsection{Capillary Perfusion}

The microcirculation delivers oxygen and nutrients to tissues and removes metabolic waste:

\begin{itemize}
    \item \textbf{Capillary density}: May be reduced in ME/CFS patients
    \item \textbf{Capillary flow}: Abnormal flow patterns documented by nailfold capillaroscopy
    \item \textbf{Red cell deformability}: Impaired RBC flexibility may impede capillary transit
    \item \textbf{Capillary recruitment}: Inadequate increase in perfused capillaries during exercise
\end{itemize}

\subsubsection{Oxygen Extraction}

Peripheral oxygen extraction findings in ME/CFS appear contradictory across studies, likely reflecting patient heterogeneity or methodological differences:

\begin{itemize}
    \item \textbf{Widened arteriovenous O$_2$ difference}: Some studies report increased extraction, interpreted as compensation for reduced cardiac output and oxygen delivery
    \item \textbf{Impaired extraction}: Other studies find reduced extraction despite adequate delivery, suggesting mitochondrial dysfunction limits oxygen utilization at the tissue level
    \item \textbf{Reconciling the findings}: These patterns may represent distinct patient subgroups---those with primarily circulatory limitation (compensatory increased extraction) versus those with primarily metabolic/mitochondrial dysfunction (impaired extraction despite delivery)
    \item \textbf{Near-infrared spectroscopy}: Documents abnormal muscle oxygenation kinetics during exercise and delayed recovery, consistent with both delivery and utilization abnormalities
\end{itemize}

\subsubsection{Tissue Hypoxia}

Inadequate oxygen delivery produces tissue hypoxia:

\begin{itemize}
    \item \textbf{Muscle hypoxia}: Contributes to weakness and post-exertional symptoms
    \item \textbf{Cerebral hypoperfusion}: Causes cognitive dysfunction (see Chapter~\ref{ch:neurological})
    \item \textbf{Lactate accumulation}: Results from anaerobic metabolism
    \item \textbf{Symptom generation}: Hypoxia-sensitive nociceptors may trigger pain
\end{itemize}

\subsubsection{Cerebral Blood Flow During Orthostatic Stress}
\label{sec:cerebral-orthostatic}

While tissue hypoxia affects multiple organs, the brain is particularly vulnerable to perfusion deficits during orthostatic challenge. Van Campen and colleagues have systematically characterized cerebral blood flow (CBF) abnormalities in ME/CFS through a series of rigorous studies using transcranial Doppler during tilt-table testing~\cite{VanCampenEtAl2020,VanCampenEtAl2021,VanCampenEtAl2023,VanCampenEtAl2024}.

\begin{achievement}[Near-Universal CBF Decline in ME/CFS]
\label{ach:cbf-decline}
Van Campen et al.~\cite{VanCampenEtAl2020} demonstrated that ME/CFS patients show reduced cerebral blood flow during head-up tilt testing even in the absence of hypotension or tachycardia. The findings are striking in their consistency across orthostatic phenotypes (percentages represent distinct, non-overlapping subgroups stratified by vital sign response):
\begin{itemize}
    \item \textbf{82\%} of patients with normal HR/BP showed abnormal CBF reduction
    \item \textbf{98\%} of patients with delayed orthostatic hypotension showed abnormal CBF
    \item \textbf{100\%} of patients meeting POTS criteria showed abnormal CBF
    \item End-tilt CBF reduction: \textbf{26\% in ME/CFS vs.\ 7\% in controls} (3.7-fold greater)
\end{itemize}
Abnormal CBF reduction thus occurs across all orthostatic presentations---even in patients with entirely normal vital signs. In the largest study to date (n=534), \textbf{91\% of ME/CFS patients} with normal HR and BP responses demonstrated abnormal cardiac output and CBF reduction during tilt~\cite{VanCampenEtAl2024}, indicating that standard orthostatic vital signs miss the primary pathology in most patients.
\end{achievement}

\paragraph{Clinical Implications of CBF Findings}

The cognitive symptoms during orthostatic stress---including brain fog, difficulty concentrating, and word-finding problems---correlate directly with the degree of cerebral hypoperfusion~\cite{VanCampenEtAl2023}. Patients often report that cognitive function worsens progressively during prolonged standing and improves rapidly upon assuming a recumbent position. This positional dependence of cognitive symptoms provides clinical evidence for the cerebrovascular contribution to ME/CFS neurological dysfunction.

\begin{observation}[Impaired CBF Recovery]
\label{obs:cbf-recovery}
CBF reduction persists even after returning to supine position. Van Campen et al.~\cite{VanCampenEtAl2021} documented CBF reduction of $-29\%$ at end-tilt, improving to only $-16\%$ post-tilt. The degree of recovery correlated with disease severity rather than hemodynamic parameters, suggesting the CBF abnormality reflects intrinsic cerebrovascular or metabolic dysfunction rather than simple hemodynamic failure.
\end{observation}

\paragraph{Absence of Compensatory Vasodilation}

A particularly significant finding is the near 1:1 relationship between cardiac output reduction and CBF reduction in ME/CFS patients~\cite{VanCampenEtAl2024}. In healthy individuals, reduced cardiac output triggers compensatory cerebral vasodilation to maintain brain perfusion. The absence of this compensation in ME/CFS suggests possible endothelial dysfunction affecting cerebrovascular autoregulation. This may represent a critical vulnerability: the brain cannot protect itself from systemic hemodynamic perturbations.

\paragraph{Mechanisms of Cerebral Hypoperfusion}

Multiple mechanisms likely contribute to orthostatic cerebral hypoperfusion in ME/CFS. Evidence strength varies: \textbf{(documented)} = directly measured in ME/CFS studies; \textbf{(inferred)} = logically derived from observed relationships; \textbf{(hypothesized)} = proposed mechanism not yet directly tested.

\begin{itemize}
    \item \textbf{Reduced cardiac output} \textbf{(documented)}: Preload failure and chronotropic incompetence limit systemic perfusion pressure; directly measured in tilt studies showing parallel CO and CBF reduction~\cite{VanCampenEtAl2024} (see Section~\ref{sec:cardiac-output})
    \item \textbf{Impaired cerebral autoregulation} \textbf{(inferred)}: Failure of compensatory vasodilation during reduced perfusion pressure; inferred from near 1:1 CO-CBF relationship where healthy controls show compensatory vasodilation~\cite{VanCampenEtAl2024}
    \item \textbf{Endothelial dysfunction} \textbf{(hypothesized)}: May impair nitric oxide-mediated vasodilation; suggested by absence of compensatory response but not directly measured in CBF studies
    \item \textbf{Autonomic dysregulation} \textbf{(documented)}: Impaired sympathetic vasoconstriction in peripheral vascular beds allows excessive venous pooling; documented via HRV and catecholamine studies (see Chapter~\ref{ch:neurological} Section~\ref{sec:autonomic-imbalance})
    \item \textbf{Blood volume deficit} \textbf{(documented)}: Reduced circulating volume exacerbates orthostatic hemodynamic stress; documented in multiple studies showing 10--15\% blood volume reduction (see Section~\ref{sec:blood-volume})
\end{itemize}

In mast cell disorder patients, Novak et al.\ documented 20--24\% reduction in orthostatic cerebral blood flow velocity using transcranial Doppler~\cite{Novak2022}. Given the substantial overlap between mast cell activation and ME/CFS, histamine-mediated vasodilation during orthostatic stress may contribute to cerebral hypoperfusion in some patients. The combination of reduced blood volume, impaired vasoconstriction, and potentially histamine-induced vasodilation creates multiple mechanisms converging on inadequate cerebral perfusion during upright posture.

\paragraph{Integration with Selective Energy Dysfunction Hypothesis}

The profound and consistent CBF reduction during orthostatic challenge exemplifies the broader pattern of \emph{preserved baseline function with impaired challenge response} characteristic of ME/CFS (see Chapter~\ref{ch:energy-metabolism} Section~\ref{sec:selective-energy-dysfunction}). Resting cerebral perfusion may be adequate, but the system cannot maintain CBF during the increased demand of orthostatic stress. The brain---with its high energy demands and critical dependence on continuous perfusion---may serve as the ``canary in the coal mine'' for systemic energy coordination dysfunction.

See Chapter~\ref{ch:neurological} Section~\ref{sec:brain-bottleneck} for discussion of brain-centric pathophysiology and the role of CBF abnormalities in the broader ME/CFS disease model.

\section{Blood Pressure Regulation}
\label{sec:blood-pressure}

Blood pressure dysregulation is common in ME/CFS and manifests as various orthostatic disorders.

\subsection{Orthostatic Hypotension}
\label{sec:orthostatic-hypotension}

\begin{definition}[Orthostatic Hypotension]
A sustained reduction in systolic blood pressure $\geq$20 mmHg or diastolic blood pressure $\geq$10 mmHg within 3 minutes of standing. \emph{Initial orthostatic hypotension} refers to a brief BP drop within the first 15 seconds, which is common in ME/CFS.
\end{definition}

Orthostatic hypotension occurs in a subset of ME/CFS patients, presenting with lightheadedness, visual disturbances, weakness, and syncope. Contributing mechanisms include autonomic failure, hypovolemia, and medications.

\subsection{Neurally Mediated Hypotension}
\label{sec:nmh}

\begin{definition}[Neurally Mediated Hypotension]
Also called vasovagal syncope or neurocardiogenic syncope. Characterized by paradoxical vasodilation and bradycardia during prolonged standing, triggered when vigorous ventricular contraction of an underfilled heart activates vagal reflexes. Presents as delayed BP drop after 10+ minutes of standing, with prodromal nausea, diaphoresis, and pallor preceding syncope.
\end{definition}

Diagnosis is confirmed by head-up tilt table testing, which can provoke the characteristic hemodynamic response under controlled conditions.

\subsection{Postural Orthostatic Tachycardia Syndrome (POTS)}
\label{sec:pots}

\begin{definition}[Postural Orthostatic Tachycardia Syndrome]
A syndrome characterized by excessive heart rate increase upon standing without significant blood pressure drop. According to the 2015 Heart Rhythm Society expert consensus~\cite{Sheldon2015POTScriteria}, diagnostic criteria include: heart rate increase $\geq$30 bpm (or $\geq$40 bpm in adolescents) within 10 minutes of standing, absence of orthostatic hypotension, symptoms of orthostatic intolerance, and symptom duration exceeding 6 months.
\end{definition}

\subsubsection{Prevalence in ME/CFS}
\begin{itemize}
    \item Estimated 25--50\% of ME/CFS patients meet POTS criteria~\cite{Newton2008POTSprevalence}
    \item Substantial symptom overlap between conditions
    \item May represent overlapping or related conditions
    \item Similar pathophysiological mechanisms
\end{itemize}

POTS is one component of the ``Septad'' framework of frequently co-occurring conditions in ME/CFS (Section~\ref{sec:septad}). The interplay between dysautonomia, mast cell activation, EDS, and other Septad components suggests shared pathophysiological mechanisms warranting comprehensive evaluation.

\subsubsection{POTS Subtypes}
Different pathophysiological mechanisms produce similar clinical phenotypes:

\paragraph{Neuropathic POTS}
Neuropathic POTS results from partial autonomic neuropathy affecting lower extremity vasoconstriction, leading to blood pooling in the legs during standing. This subtype is associated with small fiber neuropathy (SFN) and may be autoimmune in some cases. SFN specifically affects the small-diameter autonomic nerve fibers that innervate blood vessels and sweat glands, and its prevalence in POTS has been confirmed through skin biopsy studies demonstrating reduced intraepidermal nerve fiber density~\cite{Azcue2023sfn}.

\subparagraph{Small Fiber Neuropathy in POTS and ME/CFS}
The connection between SFN and cardiovascular dysautonomia is increasingly recognized. Azcue et al.\ demonstrated that ME/CFS patients show heat response latencies indicating C-fiber denervation, with 31\% meeting POTS criteria and 34\% showing non-length-dependent SFN patterns~\cite{Azcue2023sfn}. This non-length-dependent pattern (affecting proximal and distal sites equally) suggests autoimmune or inflammatory etiology rather than length-dependent degeneration seen in metabolic neuropathies.

The autonomic consequences of SFN in ME/CFS are substantial. Damaged sympathetic sudomotor fibers contribute to temperature dysregulation, while damaged vasomotor fibers impair the normal vasoconstrictor response to orthostatic stress. When patients stand, intact baroreceptors detect the gravitational blood shift, but the effector arm of the reflex (sympathetic vasoconstriction mediated by small fibers) functions inadequately, resulting in venous pooling and compensatory tachycardia characteristic of neuropathic POTS.

In mast cell disorder patients, Novak et al.\ documented SFN in 80\% of cases, with universal dysautonomia when combining sympathetic, parasympathetic, and sudomotor testing~\cite{Novak2022}. Given the overlap between mast cell activation and ME/CFS (Section~\ref{sec:septad}), SFN may represent a common pathway linking immune dysregulation to cardiovascular symptoms in both conditions.

\paragraph{Hyperadrenergic POTS}
\begin{itemize}
    \item Excessive sympathetic activation~\cite{Raj2005}
    \item Standing norepinephrine $>$600 pg/mL
    \item Associated with tremor, anxiety, hypertension during episodes
    \item May involve norepinephrine transporter deficiency
\end{itemize}

\paragraph{Hypovolemic POTS}
\begin{itemize}
    \item Low blood volume as primary driver~\cite{Raj2005,Stewart2006}
    \item Compensatory tachycardia to maintain cardiac output
    \item May respond to volume expansion
    \item Overlaps with ME/CFS blood volume deficits
\end{itemize}

\begin{speculation}[Functional vs. Structural OI Distinction]
\label{spec:oi-distinction}
The higher prevalence but better reversibility of orthostatic intolerance in
pediatric ME/CFS (70--90\% prevalence with high response to treatment) compared
to adult disease suggests qualitatively different mechanisms. We speculate that
pediatric OI may primarily represent functional miscalibration of an autonomic
system still undergoing developmental tuning, while adult OI may involve
structural damage (small fiber neuropathy, receptor autoantibody-mediated
damage) accumulated over illness duration. This distinction would explain why
OI treatment in children often produces multi-domain improvement (fatigue,
cognition, wellbeing) while adult responses may be more limited. If correct,
this supports aggressive early OI treatment in both populations to prevent
functional miscalibration from progressing to structural damage.
\end{speculation}

\subsection{Hypertension in ME/CFS}
\label{sec:hypertension}

While hypotension is more commonly discussed, hypertension also occurs:

\begin{itemize}
    \item \textbf{Supine hypertension}: Some patients have elevated BP when lying down
    \item \textbf{Labile hypertension}: Wide BP fluctuations
    \item \textbf{Stress-related}: BP spikes during symptom exacerbations
    \item \textbf{Medication-related}: Sympathomimetics for orthostatic symptoms may raise BP
\end{itemize}

\section{Heart Rate Abnormalities}
\label{sec:heart-rate}

\subsection{Resting Tachycardia}
\label{sec:resting-tachycardia}

Many ME/CFS patients exhibit elevated resting heart rate:

\begin{itemize}
    \item \textbf{Mechanism}: Compensatory response to low stroke volume
    \item \textbf{Sympathetic activation}: Chronic low-grade sympathetic overdrive
    \item \textbf{Deconditioning}: Loss of cardiovascular fitness
    \item \textbf{Clinical significance}: Correlates with symptom severity
\end{itemize}

\subsection{Heart Rate Variability}
\label{sec:hrv}

Heart rate variability (HRV) reflects autonomic modulation of the sinoatrial node (see Chapter~\ref{ch:neurological} for detailed discussion). Multiple studies document autonomic dysfunction in ME/CFS~\cite{Newton2007autonomicDysfunction}, and the NIH deep phenotyping study confirmed significantly reduced HRV in ME/CFS patients~\cite{walitt2024deep}:

\begin{itemize}
    \item \textbf{Reduced overall HRV}: Lower SDNN and total power
    \item \textbf{Diminished parasympathetic markers}: Reduced high-frequency power and RMSSD
    \item \textbf{Altered sympathovagal balance}: Changed LF/HF ratio
    \item \textbf{Prognostic implications}: Low HRV predicts poor health outcomes generally
\end{itemize}

\subsection{Heart Rate Recovery}
\label{sec:hr-recovery}

Heart rate recovery (HRR) after exercise reflects parasympathetic reactivation:

\begin{itemize}
    \item \textbf{Definition}: HR decrease from peak to 1 or 2 minutes post-exercise
    \item \textbf{ME/CFS findings}: Delayed HRR indicating impaired vagal reactivation
    \item \textbf{Clinical significance}: Abnormal HRR predicts mortality in other populations
    \item \textbf{Mechanism}: Consistent with parasympathetic dysfunction
\end{itemize}

\section{Coagulation and Rheological Abnormalities}
\label{sec:coagulation}

\subsection{Hypercoagulability}
\label{sec:hypercoagulability}

Some ME/CFS patients show evidence of increased coagulation activation:

\begin{itemize}
    \item \textbf{Elevated fibrinogen}: Acute phase reactant and clotting factor
    \item \textbf{Increased D-dimer}: Fibrin degradation product indicating clot turnover
    \item \textbf{Platelet activation}: Enhanced platelet aggregability
    \item \textbf{Thrombin generation}: Markers of coagulation cascade activation
\end{itemize}

\subsection{Fibrin Deposition}
\label{sec:fibrin}

Excessive fibrin deposition may impair microcirculation:

\begin{itemize}
    \item \textbf{Soluble fibrin monomer}: Elevated in some patients
    \item \textbf{Fibrin mesh formation}: May coat vessel walls and impede flow
    \item \textbf{Microclot hypothesis}: Recently proposed role of amyloid-like microclots
    \item \textbf{Treatment implications}: Anticoagulation investigated in small trials
\end{itemize}

\subsection{Red Blood Cell Deformability}
\label{sec:rbc-deformability}

Red blood cells must deform to traverse capillaries~\cite{Saha2019}:

\begin{itemize}
    \item \textbf{Reduced deformability}: Red blood cell deformability is significantly diminished in ME/CFS patients~\cite{Saha2019}
    \item \textbf{Mechanisms}: Membrane oxidative damage, altered lipid composition
    \item \textbf{Consequences}: Impaired capillary perfusion and oxygen delivery, potentially contributing to exercise intolerance~\cite{Saha2019}
    \item \textbf{Measurement}: Ektacytometry, micropipette aspiration
\end{itemize}

\section{Cardiovascular Dysfunction in Post-COVID ME/CFS}
\label{sec:post-covid-cardiovascular}

The COVID-19 pandemic has provided an unfortunate natural experiment in post-infectious illness, with Long COVID (post-acute sequelae of SARS-CoV-2, PASC) showing remarkable overlap with ME/CFS cardiovascular manifestations. This convergence strengthens the evidence for shared pathophysiological mechanisms and may accelerate therapeutic development.

\subsection{Parallel Cardiovascular Findings}

Long COVID patients demonstrate cardiovascular abnormalities closely mirroring those documented in ME/CFS, including POTS (affecting 30--50\% of Long COVID patients with persistent symptoms), reduced exercise capacity on CPET with similar patterns of reduced VO$_2$peak and early anaerobic threshold, autonomic dysfunction with altered HRV and baroreflex sensitivity, and endothelial dysfunction with impaired flow-mediated dilation~\cite{Devigili2023SFN}.

Small fiber neuropathy has been documented in both conditions using identical methodologies. Azcue et al.\ found that ME/CFS and post-COVID patients showed comparable patterns of heat response latencies indicating C-fiber denervation~\cite{Azcue2023sfn}. The non-length-dependent pattern observed in both conditions suggests autoimmune or inflammatory etiology rather than metabolic neuropathy.

\subsection{GPCR Autoantibodies in Post-COVID}

The GPCR autoantibody hypothesis has received substantial support from post-COVID research. Elevated beta-adrenergic and muscarinic receptor autoantibodies have been documented in Long COVID patients with cardiovascular symptoms, and immunoadsorption targeting these autoantibodies has shown efficacy in post-COVID ME/CFS~\cite{Stein2024immunoadsorption}. The Scheibenbogen group demonstrated that repeated immunoadsorption in patients with post-COVID ME/CFS and elevated beta-2 adrenergic receptor autoantibodies produced significant improvements in fatigue and autonomic symptoms.

Hackel et al.\ demonstrated that GPCR autoantibodies reprogram monocyte function in post-COVID ME/CFS, altering cytokine production patterns and potentially explaining systemic inflammatory features~\cite{Hackel2025monocyte}. This finding links autoantibodies to immune dysfunction beyond direct receptor effects, suggesting multiple downstream consequences of the autoimmune process.

\subsection{Implications for Understanding ME/CFS}

The convergence of Long COVID and ME/CFS cardiovascular findings supports the hypothesis that both conditions share common post-infectious pathophysiology. The larger Long COVID research effort, driven by the pandemic's scale, is generating mechanistic insights likely applicable to ME/CFS. Therapeutic interventions developed for Long COVID---including immunoadsorption, GPCR-targeting aptamers, and mast cell stabilizers---may prove equally effective in ME/CFS patients with similar pathophysiology.

\section{Sex Differences in Cardiovascular Manifestations}
\label{sec:cv-sex-differences}

ME/CFS demonstrates a 3:1 to 4:1 female predominance, and emerging evidence suggests that cardiovascular manifestations may differ between sexes beyond simple prevalence differences.

POTS is more common in females, with cohort studies consistently showing 4:1 to 5:1 female-to-male ratios~\cite{Newton2008POTSprevalence}. This sex difference exceeds the overall ME/CFS female predominance, suggesting additional sex-specific factors in POTS pathophysiology. Potential contributors include differences in blood volume regulation (females have lower baseline blood volume per kilogram), hormonal effects on vascular tone and autonomic function, and sex differences in autoimmune propensity affecting GPCR autoantibody production.

The NIH deep phenotyping study revealed distinct immune abnormalities in male versus female ME/CFS patients~\cite{walitt2024deep}, and these differences likely extend to cardiovascular manifestations. Sex hormone effects on endothelial function, baroreflex sensitivity, and autonomic balance may modulate how the underlying disease process manifests cardiovascularly.

Blood volume deficits may be proportionally greater in females. van Campen et al.\ found that red blood cell mass was reduced in 93.8\% of female ME/CFS patients compared to 50\% of males, while plasma volume was subnormal in the majority of both sexes~\cite{vanCampen2018}. This sex difference in red cell mass reduction may reflect hormonal influences on erythropoiesis or differential inflammatory effects.

Clinical implications include the need for sex-stratified analysis in cardiovascular research and potentially different therapeutic thresholds. The higher prevalence of POTS in females may warrant lower diagnostic thresholds for autonomic testing referral in female patients with orthostatic symptoms.

\section{Mast Cell Activation and Cardiovascular Dysfunction}
\label{sec:mcas-cardiovascular}

Mast cell activation syndrome (MCAS) frequently co-occurs with POTS and ME/CFS, forming part of the ``Septad'' of overlapping conditions. The cardiovascular effects of mast cell degranulation provide a mechanistic link between immune activation and hemodynamic instability.

\subsection{Cardiovascular Mediators of Mast Cell Activation}

Mast cells release multiple vasoactive mediators upon degranulation:

\begin{itemize}
    \item \textbf{Histamine}: Causes vasodilation through H1 and H2 receptor activation on vascular smooth muscle, increasing vascular permeability and contributing to hypotension
    \item \textbf{Prostaglandin D$_2$}: Potent vasodilator that may contribute to flushing and hypotensive episodes
    \item \textbf{Tryptase}: Serine protease that can activate protease-activated receptors on endothelial cells, potentially contributing to endothelial dysfunction
    \item \textbf{Platelet-activating factor (PAF)}: Causes vasodilation, increases vascular permeability, and promotes platelet aggregation
    \item \textbf{Heparin}: Released during degranulation, may contribute to bleeding tendency and affect coagulation
\end{itemize}

During mast cell degranulation episodes, the sudden release of vasodilatory mediators can produce acute hypotensive episodes, flushing, and tachycardia. When mast cell activation is chronic and low-grade, the cumulative effect may include sustained endothelial dysfunction and impaired vascular reactivity.

\subsection{The MCAS-POTS Connection}

The relationship between MCAS and POTS is bidirectional. Mast cell mediators, particularly histamine, cause peripheral vasodilation that exacerbates venous pooling during orthostatic stress. Conversely, orthostatic stress may trigger mast cell degranulation in susceptible individuals, creating a feed-forward loop. This bidirectional interaction exemplifies the reinforcing pathophysiological cycles discussed in Section~\ref{sec:unifying-mechanisms} of Chapter~\ref{ch:integrative-models}.

Novak et al.\ documented that mast cell disorder patients universally showed dysautonomia when combining sympathetic, parasympathetic, and sudomotor testing~\cite{Novak2022}. The same patients showed 20--24\% reduction in orthostatic cerebral blood flow, directly linking mast cell activation to cerebral hypoperfusion during standing.

The high prevalence of small fiber neuropathy (80\%) in mast cell disorder patients~\cite{Novak2022} suggests that mast cell mediators may directly damage autonomic nerve fibers or that both findings reflect a common underlying autoimmune process. Tryptase and other mast cell proteases can cleave components of the extracellular matrix and potentially damage nerve terminals.

\subsection{Therapeutic Implications}

The mast cell-cardiovascular connection has therapeutic implications. H1 antihistamines (cetirizine, loratadine, rupatadine) and H2 blockers (famotidine) may improve orthostatic symptoms in patients with concurrent MCAS. Mast cell stabilizers (cromolyn sodium, ketotifen) may provide broader suppression of mediator release. In patients with ME/CFS and prominent flushing, episodic tachycardia, or symptom fluctuation temporally associated with meals or environmental triggers, evaluation for MCAS should be considered, and empiric antihistamine therapy may be warranted.

\section{Summary: Integrated Cardiovascular Model}
\label{sec:cv-summary}

Cardiovascular dysfunction in ME/CFS involves multiple interacting abnormalities~\cite{walitt2024deep}:

\begin{enumerate}
    \item \textbf{Reduced blood volume}: Hypovolemia compromises cardiac preload and limits cardiac output reserve

    \item \textbf{Autonomic dysfunction}: Parasympathetic withdrawal reduces HRV and impairs baroreflex function; chronotropic incompetence limits exercise heart rate response

    \item \textbf{Endothelial dysfunction}: Impaired vasodilation reduces tissue perfusion

    \item \textbf{Cardiac limitation}: Preload failure and chronotropic incompetence reduce maximal cardiac output

    \item \textbf{Microcirculatory impairment}: Abnormal capillary perfusion and oxygen extraction limit peripheral oxygen delivery

    \item \textbf{Exercise intolerance}: The cumulative effect is reduced VO$_2$peak and early anaerobic threshold, objectively confirmed by the NIH study

    \item \textbf{Post-exertional deterioration}: Unique to ME/CFS, the failure to recover exercise capacity on day 2 CPET reflects pathological response to exertion

    \item \textbf{Orthostatic intolerance}: Blood pressure dysregulation (POTS, NMH, OH) produces symptoms with upright posture
\end{enumerate}

This cardiovascular dysfunction explains much of the disability in ME/CFS: patients cannot sustain physical activity because their cardiovascular system cannot deliver adequate oxygen to meet metabolic demands. The objective documentation of reduced VO$_2$peak and chronotropic incompetence in the NIH deep phenotyping study provides biological validation of patients' reported exercise intolerance. These cardiovascular abnormalities integrate with metabolic dysfunction (Chapter~\ref{ch:energy-metabolism}), autonomic dysfunction (Chapter~\ref{ch:neurological}), and immune dysregulation (Chapter~\ref{ch:immune-dysfunction}) to produce the multi-system pathophysiology synthesized in Chapter~\ref{ch:integrative-models}.

Treatment approaches targeting cardiovascular dysfunction include volume expansion (fludrocortisone, increased fluid and salt intake), direct-acting autonomic agents (midodrine as alpha-agonist for vasoconstriction), and careful activity management to avoid exceeding the reduced aerobic threshold. The efficacy of pharmacological agents that bypass impaired CNS autonomic coordination (such as midodrine acting directly on peripheral alpha-receptors) provides indirect support for the selective energy dysfunction hypothesis discussed in Section~\ref{sec:selective-energy-dysfunction}. The recognition that cardiovascular abnormalities are objective and measurable helps counter misconceptions that ME/CFS exercise intolerance reflects psychological factors or simple deconditioning.

\begin{hypothesis}[Motor-Autonomic Coordination Overload Hypothesis]
\label{hyp:motor-autonomic-coordination}

Physical activity requires simultaneous CNS coordination of motor output (muscle recruitment, movement planning, proprioceptive feedback) and autonomic regulation (heart rate adjustment, blood pressure maintenance, thermoregulation, respiratory drive). In healthy individuals, these coordination tasks operate efficiently within the brain's energy budget. We hypothesize that in ME/CFS, where CNS energy is the primary bottleneck (Section~\ref{sec:selective-dysfunction}), motor and autonomic coordination compete for insufficient resources, causing both systems to fail under demand.

\paragraph{The Dual-Coordination Problem.}
During exercise, the CNS must:

\begin{itemize}
    \item \textbf{Motor coordination}: Generate movement commands, integrate proprioceptive feedback, maintain balance, adjust force output---all requiring continuous cortical, cerebellar, and spinal processing.
    \item \textbf{Autonomic coordination}: Increase heart rate, redistribute blood flow, maintain blood pressure during postural changes, regulate respiration, initiate sweating---all requiring brainstem and hypothalamic processing.
    \item \textbf{Integration}: Coordinate motor and autonomic outputs so that cardiovascular supply matches muscular demand in real time.
\end{itemize}

In ME/CFS, if total CNS energy available for coordination is reduced, attempting both tasks simultaneously will exceed the available budget sooner than either task alone. This explains the central governor theory observation~\cite{Noakes2004governor,StClairGibson2004fatigue}: the brain limits motor output to protect itself from energy depletion.

\paragraph{ME/CFS-Specific Predictions.}
This hypothesis explains several puzzling CPET findings:

\textit{Reduced VO$_2$peak beyond deconditioning.} The NIH deep phenotyping study documented VO$_2$peak reductions exceeding what deconditioning alone would predict~\cite{walitt2024deep}. If the brain limits motor output to preserve autonomic coordination capacity, peak exercise performance reflects the CNS energy budget, not peripheral muscle capacity.

\textit{Chronotropic incompetence.} The failure to achieve age-predicted maximal heart rate~\cite{walitt2024deep} may reflect CNS prioritization: under energy constraint, the brain may reduce autonomic drive to the heart in order to preserve motor coordination, or vice versa. The specific pattern of failure depends on which system the CNS prioritizes in a given individual.

\textit{Day-2 CPET deterioration.} The pathognomonic worsening on repeat CPET the following day reflects CNS energy depletion that has not recovered. The first test depletes CNS coordination reserves; insufficient recovery time means the second test starts from a lower baseline, producing objectively worse performance.

\textit{PEM as coordination exhaustion.} Post-exertional malaise may represent the downstream consequence of depleting CNS coordination reserves. Once exhausted, the brain cannot adequately coordinate autonomic function (causing orthostatic symptoms, heart rate irregularity) or motor output (causing weakness, poor coordination), producing the multi-system symptom exacerbation characteristic of PEM.

\paragraph{Testable Predictions.}
\begin{enumerate}
    \item \textbf{Cognitive-physical interference}: ME/CFS patients should show greater cognitive impairment during physical activity (dual-task paradigm) than healthy controls, reflecting competition for shared CNS resources.
    \item \textbf{Autonomic-motor trade-off}: During exercise, ME/CFS patients should show an inverse relationship between motor performance and autonomic function quality (e.g., better muscle output correlates with worse HRV, and vice versa).
    \item \textbf{Separate-task preservation}: Motor tasks without significant autonomic demand (e.g., seated fine motor tasks) and autonomic challenges without motor demand (e.g., passive tilt testing) should each show less impairment than combined motor-autonomic challenges (e.g., exercise).
    \item \textbf{Pharmacological bypass}: Agents that directly support autonomic function (midodrine, pyridostigmine) should improve exercise tolerance by offloading CNS autonomic coordination, freeing energy for motor output.
\end{enumerate}

\paragraph{Treatment Implications.}
\begin{itemize}
    \item \textbf{Pre-treatment with autonomic agents}: Taking autonomic-supporting medications before planned physical activity could extend exercise tolerance by reducing CNS autonomic coordination demands.
    \item \textbf{Activity design}: Activities that minimize simultaneous motor-autonomic demand (recumbent exercise, swimming) should be better tolerated than upright weight-bearing exercise.
    \item \textbf{Pacing rationale}: The coordination overload model provides a mechanistic rationale for pacing: staying below the threshold where motor and autonomic demands simultaneously exceed CNS capacity prevents the cascade of coordination failure that produces PEM.
\end{itemize}

\paragraph{Limitations.}
The hypothesis assumes CNS energy is the primary constraint, which remains debated. Peripheral factors (mitochondrial dysfunction, reduced blood volume, deconditioning) independently contribute to exercise intolerance. The dual-task prediction requires careful experimental design to distinguish CNS resource competition from general fatigue. Central governor theory itself remains controversial in exercise physiology.

\textbf{Certainty:} 0.55 (CPET findings well-documented; CNS coordination mechanism plausible; dual-task predictions testable but not yet tested in ME/CFS)
\end{hypothesis}

\begin{speculation}[Small Fiber Neuropathy Increases CNS Metabolic Load]
\label{spec:sfn-interface-failure}
Small fiber neuropathy affects approximately 30\% of ME/CFS patients~\cite{Azcue2023sfn,Azcue2025sfn}, creating a bidirectional communication burden between the peripheral nervous system and central nervous system that may amplify energy constraints.

\paragraph{Afferent Signal Degradation.}
SFN reduces the quality of autonomic afferent signals reaching the CNS---temperature sensing, baroreceptor feedback, visceral sensation. Degraded sensory input increases CNS processing demands to extract meaningful information. Neural systems must increase firing rates quadratically to achieve linear improvements in signal-to-noise ratio~\cite{Laughlin2003energy}, creating disproportionate metabolic costs when processing noisy peripheral signals. This is analogous to listening to conversation in a noisy environment: the brain expends more energy processing degraded input to achieve adequate perception.

\paragraph{Efferent Command Amplification.}
When efferent small fibers are damaged, the CNS must generate stronger, more frequent, or redundant autonomic commands to achieve target physiological responses. Fewer functional nerve fibers mean each must be driven harder, or signals must be repeated, increasing the metabolic cost of autonomic control. During orthostatic stress, the CNS may detect inadequate vasoconstriction (via baroreceptor feedback) despite issuing normal commands, triggering escalating compensatory signals that further drain central energy reserves.

\paragraph{Testable Predictions.}
\begin{itemize}
    \item Intraepidermal nerve fiber density (IENFD) should inversely correlate with brainstem and hypothalamic glucose uptake (FDG-PET) during autonomic challenges such as tilt testing ($r < -0.5$ expected).
    \item ME/CFS patients with confirmed SFN should demonstrate worse cognitive fatigue and brain fog than those without SFN, controlling for pain severity and autonomic dysfunction magnitude.
    \item Treatment of autoimmune SFN with IVIG should reduce CNS metabolic burden measurable by PET or MR spectroscopy, with corresponding improvements in cognitive symptoms.
    \item Corneal nerve fiber tortuosity (measured non-invasively via corneal confocal microscopy) should correlate with CNS lactate accumulation and cognitive impairment.
    \item Cognitive load should exacerbate autonomic dysfunction more severely in SFN-positive patients, reflecting competition for limited CNS energy resources.
\end{itemize}

\paragraph{Treatment Implications.}
If this hypothesis is correct, treating SFN may reduce CNS metabolic burden and improve cognitive symptoms even without direct CNS interventions. The non-length-dependent SFN pattern documented in ME/CFS~\cite{Azcue2023sfn} suggests autoimmune etiology, potentially responsive to immunomodulation. Case series (low certainty) suggest IVIG improves pain and autonomic symptoms in autoimmune SFN~\cite{Liu2020IVIG,Oaklander2016autoimmuneSFN}, though randomized controlled trials in idiopathic SFN have shown mixed results. The distinct autoimmune pattern in ME/CFS-associated SFN may predict better immunotherapy response than idiopathic cases.

\paragraph{Limitations.}
No studies have directly measured CNS metabolic demand in relation to SFN severity in ME/CFS. SFN and cognitive dysfunction may share common causes (e.g., autoimmunity or inflammation) rather than having a causal relationship. The relative contribution of SFN to overall CNS energy constraints is unknown and may be minor compared to other factors.

\paragraph{Current Evidence.}
Azcue et al.\ documented that ME/CFS patients show prolonged heat response latencies indicating C-fiber dysfunction, with 31\% meeting POTS criteria~\cite{Azcue2023sfn}. A follow-up study using corneal confocal microscopy demonstrated increased small fiber tortuosity in ME/CFS compared to controls ($F=6.80$, $p<0.01$), with tortuosity serving as the primary discriminator between patients and controls (AUC$=0.720$)~\cite{Azcue2025sfn}. The non-length-dependent pattern (upper and lower extremities equally affected) distinguishes ME/CFS-associated SFN from metabolic neuropathies like diabetic neuropathy, suggesting immune-mediated damage targeting specific antigens on small nerve fibers. The connection between reduced parasympathetic activation and worse cognitive performance~\cite{Azcue2023sfn} provides indirect support for peripheral-CNS interface dysfunction, though directionality remains uncertain.

\textbf{Certainty:} 0.40 (SFN prevalence established; CNS metabolic mechanism speculative)
\end{speculation}
