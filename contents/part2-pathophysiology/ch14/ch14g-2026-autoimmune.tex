\section{Novel Hypotheses from 2026 Autoimmune Research}
\label{sec:2026-autoimmune-hypotheses}

The recent convergence of autoantibody research, EBV pathogenesis studies, and structural biology of receptor-targeting antibodies suggests several novel hypotheses that may explain ME/CFS pathophysiology and point toward new therapeutic strategies.

\subsection{The EBV-B Cell CNS Infiltration Hypothesis}

\begin{open_question}[Viral-Driven Autoreactive B Cell Brain Invasion]
The Pless et al.\ (2026) study~\cite{Pless2026ebv_demyelination} demonstrated that autoreactive B cells exist in healthy human blood and can cross the blood-brain barrier following viral infection. When these B cells express EBV Latent Membrane Protein 1 (LMP1), they infiltrate the brain and induce demyelinating lesions through myelin antigen capture, complement activation, and microglial activation.

What if a similar mechanism operates in ME/CFS---not necessarily causing overt demyelination, but producing subclinical neuroinflammation and autoantibody-mediated neurological dysfunction?

\textbf{Proposed mechanism:}
\begin{enumerate}
    \item EBV infection (primary or reactivation) triggers LMP1 expression in a subset of B cells
    \item LMP1-expressing B cells acquire enhanced blood-brain barrier crossing ability
    \item These B cells infiltrate the CNS and encounter neuronal antigens
    \item Unlike MS (where myelin antigens are targeted), ME/CFS B cells might target:
    \begin{itemize}
        \item Neurotransmitter receptors (explaining catecholamine/serotonin dysfunction)
        \item Ion channels (explaining autonomic symptoms)
        \item Astrocyte or microglial surface proteins (causing neuroinflammation)
    \end{itemize}
    \item Local complement activation and microglial priming create chronic neuroinflammation
    \item The neuroinflammation produces brain fog, altered effort perception, and sensory sensitivities
\end{enumerate}

This would explain why ME/CFS often follows EBV infection, why neuroinflammation is seen on PET imaging, and why CSF abnormalities are documented despite relatively normal standard testing.
\end{open_question}

\paragraph{Undocumented Phenomenon.} CSF analysis for LMP1-expressing B cells or EBV-specific B cell populations has not been performed in ME/CFS. If this hypothesis is correct:
\begin{itemize}
    \item ME/CFS patients should have elevated EBV-infected B cells in CSF compared to controls
    \item These B cells might show LMP1 expression
    \item Local complement activation products should be detectable
    \item Microglial activation markers should correlate with presence of these B cells
\end{itemize}

\paragraph{Treatment Implication.} If EBV-infected B cells are driving CNS pathology:
\begin{itemize}
    \item Antiviral therapy (valacyclovir, valganciclovir) might reduce EBV reactivation and LMP1 expression
    \item B cell depletion with rituximab might be beneficial \textit{if} the infiltrating B cells are CD20$^+$ (unlike plasma cells)
    \item Complement inhibition might reduce downstream damage
    \item EBV-specific T cell therapy (experimental) might eliminate the infected B cell population
\end{itemize}

\subsection{The GPCR Autoantibody-Monocyte Amplification Loop}

\begin{open_question}[Autoantibodies as Monocyte Programmers]
Hackel et al.\ (2025)~\cite{Hackel2025monocyte} demonstrated that GPCR autoantibodies don't just block or activate receptors---they reprogram monocyte function, causing production of specific inflammatory and neurotrophic cytokines (MIP-1$\delta$, PDGF-BB, TGF-$\beta$3).

This suggests autoantibodies may have effects far beyond simple receptor modulation. What if GPCR autoantibodies create a self-amplifying inflammatory loop through monocyte reprogramming?

\textbf{Proposed mechanism:}
\begin{enumerate}
    \item Initial infection triggers GPCR autoantibody production
    \item Autoantibodies bind to monocyte surface GPCRs
    \item Monocyte signaling pathways are reprogrammed, shifting cytokine production
    \item MIP-1$\delta$ recruits additional immune cells to tissues
    \item PDGF-BB promotes fibroblast activation and tissue remodeling
    \item TGF-$\beta$3 has complex immunomodulatory effects (potentially tolerogenic, but also fibrotic)
    \item The altered cytokine milieu:
    \begin{itemize}
        \item Maintains B cell activation (perpetuating autoantibody production)
        \item Creates tissue-level inflammation (explaining multi-system symptoms)
        \item Affects endothelial function (connecting to vascular hypothesis)
        \item Signals to the brain via vagal afferents or direct cytokine action
    \end{itemize}
    \item Unlike simple autoantibody-receptor binding (which might be compensated), monocyte reprogramming creates sustained systemic effects
\end{enumerate}
\end{open_question}

\paragraph{Undocumented Phenomenon.} The specific downstream targets of the altered cytokine profile have not been mapped in ME/CFS. Predictions:
\begin{itemize}
    \item Tissue biopsies should show increased fibroblast activation markers
    \item MIP-1$\delta$-responsive immune cell populations should be expanded
    \item TGF-$\beta$3-associated gene expression signatures should be detectable
    \item Monocyte cytokine production profiles should correlate with symptom severity
\end{itemize}

\paragraph{Treatment Implication.}
\begin{itemize}
    \item Autoantibody removal (immunoadsorption, BC007) should normalize monocyte function
    \item Targeting the downstream cytokines (anti-MIP-1, anti-PDGF) might provide symptomatic relief even if autoantibodies persist
    \item Monocyte-modulating therapies (JAK inhibitors affecting monocyte signaling) might interrupt the loop
    \item Combined autoantibody removal + monocyte modulation might be synergistic
\end{itemize}

\subsection{The Receptor Internalization Hypothesis}

\begin{open_question}[Autoantibodies Causing Functional Receptor Depletion]
The Kim et al.\ (2026) cryo-EM study~\cite{Kim2026nmdar_cryoem} of NMDA receptor autoantibodies revealed that autoantibody binding causes receptor internalization---removing functional receptors from the cell surface. This isn't receptor blocking; it's receptor elimination.

If GPCR autoantibodies in ME/CFS cause similar internalization, patients might have functional receptor depletion rather than receptor dysfunction. The receptors aren't blocked---they're gone.

\textbf{Proposed mechanism:}
\begin{enumerate}
    \item Autoantibodies bind to $\beta$-adrenergic and muscarinic receptors
    \item Rather than simply blocking or activating receptors, binding triggers receptor endocytosis
    \item Internalized receptors may be degraded rather than recycled
    \item Cells experience progressive receptor depletion
    \item With fewer receptors, normal catecholamine/acetylcholine signaling becomes ineffective
    \item This explains the autonomic dysfunction without requiring abnormal neurotransmitter levels
    \item It also explains why symptoms persist: receptor resynthesis takes time, and if autoantibodies persist, new receptors are immediately internalized
\end{enumerate}

This mechanism would create a fundamentally different pathophysiology than simple receptor blockade---one that persists as long as autoantibodies are present and requires receptor regeneration (not just antibody clearance) for recovery.
\end{open_question}

\paragraph{Undocumented Phenomenon.} Receptor density on patient cells has not been systematically measured. Predictions:
\begin{itemize}
    \item $\beta$-adrenergic receptor density on patient lymphocytes should be reduced
    \item Muscarinic receptor density on relevant tissues should be depleted
    \item Receptor density should correlate inversely with autoantibody titers
    \item After autoantibody removal (immunoadsorption), receptor density should gradually recover over weeks to months
    \item The time course of receptor recovery should parallel symptom improvement
\end{itemize}

\paragraph{Treatment Implication.}
\begin{itemize}
    \item Autoantibody removal is necessary but not sufficient---receptor regeneration takes time
    \item Receptor upregulation strategies (if they exist) might accelerate recovery
    \item The lag between autoantibody clearance and symptom improvement is explained
    \item Combined approaches: remove autoantibodies + support receptor resynthesis
\end{itemize}

\subsection{The Antigenic Hotspot Vulnerability Hypothesis}

\begin{open_question}[Structural Vulnerability to Autoimmune Attack]
The Kim et al.\ cryo-EM study~\cite{Kim2026nmdar_cryoem} identified specific ``antigenic hotspots'' on the NMDA receptor where autoantibodies preferentially bind. These aren't random locations---they're structurally exposed regions that the immune system can access.

What if certain individuals have GPCR variants with more exposed antigenic hotspots---making them structurally predisposed to autoimmune attack on these receptors?

\textbf{Proposed mechanism:}
\begin{enumerate}
    \item GPCR genes show normal polymorphic variation in the population
    \item Some variants have amino acid changes in extracellular loops
    \item These changes create more immunogenic conformations---``hotspots'' that B cells can target
    \item When an infection triggers autoantibody production (through molecular mimicry or bystander activation), individuals with hotspot-exposed receptors are more likely to develop pathogenic autoantibodies
    \item This would explain:
    \begin{itemize}
        \item Why only some people develop ME/CFS after infection
        \item Why certain families show clustering of ME/CFS
        \item Why symptom patterns vary (different receptors have different vulnerabilities)
        \item Why autoantibody titers don't perfectly correlate with symptoms (some autoantibodies target more critical hotspots than others)
    \end{itemize}
\end{enumerate}
\end{open_question}

\paragraph{Undocumented Phenomenon.} GPCR genetic variation in ME/CFS has been minimally studied. Predictions:
\begin{itemize}
    \item ME/CFS patients should show enrichment for specific GPCR variants
    \item These variants should map to extracellular domains (potential hotspots)
    \item Structural modeling should predict increased immunogenicity for these variants
    \item Autoantibody binding affinity should be higher for ``hotspot'' variants
\end{itemize}

\paragraph{Treatment Implication.}
\begin{itemize}
    \item Genetic screening might identify at-risk individuals before infection
    \item Prophylactic approaches (EBV vaccination when available) might prevent ME/CFS in susceptible individuals
    \item Personalized therapy based on which receptors are structurally vulnerable
    \item Potential for peptide-based tolerization targeting specific hotspots
\end{itemize}

\subsection{The Molecular Mimicry-Receptor Homology Hypothesis}

\begin{open_question}[Viral Proteins Mimicking Receptor Epitopes]
EBV is strongly associated with ME/CFS onset. EBV proteins share sequence homology with many human proteins (documented extensively in MS research). What if specific EBV proteins share structural homology with GPCR extracellular domains---such that anti-EBV antibodies cross-react with adrenergic and muscarinic receptors?

\textbf{Proposed mechanism:}
\begin{enumerate}
    \item EBV infection generates robust antibody response against viral proteins
    \item Certain EBV proteins (particularly those exposed on infected cell surfaces) share epitopes with human GPCRs
    \item Anti-EBV antibodies cross-react with $\beta$-adrenergic and muscarinic receptors
    \item Unlike true autoantibodies (generated by tolerance breach), these are antiviral antibodies with unfortunate cross-reactivity
    \item As long as EBV persists (which it does, lifelong), the anti-EBV response continues
    \item This maintains GPCR-targeting antibodies indefinitely
\end{enumerate}

This would explain why EBV infection so specifically triggers ME/CFS, why autoantibody titers persist, and why antiviral therapy might help (by reducing viral protein expression and thus the stimulus for cross-reactive antibodies).
\end{open_question}

\paragraph{Undocumented Phenomenon.} Structural homology between EBV proteins and GPCR extracellular domains has not been systematically analyzed. Predictions:
\begin{itemize}
    \item Computational analysis should identify EBV-GPCR homologous sequences
    \item Anti-EBV antibodies should show GPCR binding in vitro
    \item The same antibody clones should bind both EBV proteins and GPCRs
    \item Patients with higher anti-EBV titers might have higher anti-GPCR titers
    \item Reducing EBV viral load should reduce GPCR autoantibody titers
\end{itemize}

\paragraph{Treatment Implication.}
\begin{itemize}
    \item Aggressive antiviral therapy might reduce the stimulus for cross-reactive antibodies
    \item EBV vaccination (when available) might prevent ME/CFS by generating non-cross-reactive immunity
    \item Targeted B cell depletion of EBV-specific clones might eliminate the cross-reactive population
    \item Tolerization to the shared epitope might break the cycle
\end{itemize}

\subsection{The Dual-Compartment Autoantibody Hypothesis}

\begin{open_question}[Peripheral vs.\ Central Autoantibody Effects]
Bynke et al.\ (2020)~\cite{Bynke2020} found elevated GPCR autoantibodies in plasma but \textit{not} in CSF. This is usually interpreted as indicating peripheral origin. But what if it reveals something more important: different autoantibody populations in different compartments, with different effects?

\textbf{Proposed mechanism:}
\begin{enumerate}
    \item Peripheral plasma cells produce GPCR autoantibodies that cause systemic symptoms:
    \begin{itemize}
        \item Cardiovascular autonomic dysfunction (acting on vascular/cardiac receptors)
        \item GI symptoms (acting on enteric receptors)
        \item Peripheral muscle effects
    \end{itemize}
    \item Separately, EBV-infected B cells that cross the blood-brain barrier might produce \textit{different} autoantibodies locally in the CNS:
    \begin{itemize}
        \item These might target neuronal receptors (NMDA, GABA, glycine)
        \item They would cause cognitive and neurological symptoms
        \item They might not appear in lumbar puncture CSF if produced in specific brain regions
    \end{itemize}
    \item The two compartments explain the dissociation between peripheral and central symptoms
    \item Treatment targeting only peripheral autoantibodies might improve systemic symptoms but leave cognitive symptoms unchanged
\end{enumerate}
\end{open_question}

\paragraph{Undocumented Phenomenon.} Regional CNS autoantibody production has not been studied in ME/CFS. Predictions:
\begin{itemize}
    \item Post-mortem or surgical brain tissue might show local autoantibody production
    \item Advanced CSF sampling (ventricular vs.\ lumbar) might reveal different autoantibody profiles
    \item Intrathecal B cell populations might differ from peripheral B cells
    \item Patients with predominantly cognitive symptoms might have different autoantibody patterns than those with predominantly autonomic symptoms
\end{itemize}

\paragraph{Treatment Implication.}
\begin{itemize}
    \item Immunoadsorption might help peripheral but not CNS symptoms
    \item CNS-penetrant therapies might be needed for cognitive symptoms
    \item Combination approaches targeting both compartments might be necessary
    \item Biomarkers distinguishing peripheral vs.\ central autoimmunity would guide therapy
\end{itemize}

\subsection{The Autoantibody Functional Assay Discrepancy Hypothesis}

\begin{open_question}[Why Do Some Studies Fail to Replicate?]
Vernino et al.\ (2022)~\cite{POTS2022failed_replication} found no differences in GPCR autoantibodies between POTS patients and controls using standard ELISA, directly contradicting multiple positive studies. This methodological controversy has major implications.

What if both findings are correct---but measuring different things?

\textbf{Proposed mechanism:}
\begin{enumerate}
    \item ELISA detects any antibody that binds the target antigen
    \item Most humans have low-level autoantibodies against many self-proteins (natural autoantibodies)
    \item These natural autoantibodies are non-pathogenic
    \item Pathogenic autoantibodies differ in:
    \begin{itemize}
        \item Binding affinity (higher affinity = more functional effect)
        \item Epitope specificity (some epitopes are functionally important, others aren't)
        \item Effector function (some trigger internalization, others don't)
        \item Isotype (IgG1/IgG3 activate complement; IgG4 doesn't)
    \end{itemize}
    \item Standard ELISAs detect total binding antibodies, not functionally pathogenic ones
    \item Positive studies using CellTrend assays might detect a subset that correlates with pathogenicity
    \item Negative studies using different methodology might detect the non-pathogenic background
\end{enumerate}

This would mean: autoantibodies ARE involved in ME/CFS, but detecting the pathogenic subset requires functional assays, not just binding assays.
\end{open_question}

\paragraph{Undocumented Phenomenon.} Functional characterization of ME/CFS autoantibodies is minimal. Predictions:
\begin{itemize}
    \item Functional assays (receptor internalization, downstream signaling) should distinguish patients from controls better than binding assays
    \item Autoantibody affinity should correlate with symptom severity
    \item Epitope mapping should identify ``pathogenic'' vs.\ ``non-pathogenic'' binding sites
    \item Isotype profiling might reveal skewing toward complement-activating subclasses in patients
\end{itemize}

\paragraph{Treatment Implication.}
\begin{itemize}
    \item Functional autoantibody assays should be developed for patient selection
    \item Therapies might need to target specifically the high-affinity pathogenic subset
    \item Understanding functional differences could guide epitope-specific tolerization
    \item Clinical trials should stratify by functional autoantibody status, not just binding titers
\end{itemize}

\subsection{Updated Master Hypothesis Table: 2026 Autoimmune Hypotheses}

Table~\ref{tab:2026-autoimmune-hypotheses} summarizes the novel hypotheses emerging from 2026 autoimmune research.

\begin{landscape}
\tiny
\begin{longtable}{p{3.5cm}p{2cm}p{2cm}p{1.8cm}p{1.8cm}p{4.5cm}p{5cm}}
\caption{Novel hypotheses from 2026 autoimmune research} \label{tab:2026-autoimmune-hypotheses} \\
\toprule
\textbf{Hypothesis} & \textbf{Evidence Level} & \textbf{Therapeutic Potential} & \textbf{Benefit: Mild} & \textbf{Benefit: Severe} & \textbf{Explains Key Features} & \textbf{Nearest-Term Action} \\
\midrule
\endfirsthead
\multicolumn{7}{c}{\tablename\ \thetable{} -- continued from previous page} \\
\toprule
\textbf{Hypothesis} & \textbf{Evidence Level} & \textbf{Therapeutic Potential} & \textbf{Benefit: Mild} & \textbf{Benefit: Severe} & \textbf{Explains Key Features} & \textbf{Nearest-Term Action} \\
\midrule
\endhead
\midrule
\multicolumn{7}{r}{\textit{Continued on next page}} \\
\endfoot
\bottomrule
\endlastfoot
\multicolumn{7}{l}{\textit{\textbf{EBV-Related Hypotheses}}} \\
\midrule
EBV-B cell CNS infiltration & Low-Moderate & High & Moderate & Moderate-High & Post-EBV onset; neuroinflammation; brain fog; cognitive symptoms distinct from fatigue & CSF B cell analysis; EBV PCR in CSF; LMP1 expression profiling \\
\midrule
Molecular mimicry (EBV-GPCR homology) & Low & High & Moderate-High & Moderate-High & EBV trigger specificity; persistent autoantibodies; why antivirals might help & Computational homology analysis; cross-reactivity testing \\
\midrule
\multicolumn{7}{l}{\textit{\textbf{Autoantibody Mechanism Hypotheses}}} \\
\midrule
Autoantibody-monocyte amplification loop & Moderate & High & High & Moderate & Systemic inflammation; cytokine abnormalities; why symptoms persist beyond receptor effects & Monocyte functional profiling post-immunoadsorption \\
\midrule
Receptor internalization (not blockade) & Low-Moderate & Moderate-High & Moderate & Moderate & Lag between antibody removal and improvement; why symptoms persist; receptor density changes & Receptor density assays on patient lymphocytes \\
\midrule
Antigenic hotspot vulnerability & Very Low & Moderate & Moderate & Moderate & Genetic susceptibility; family clustering; why some people but not others & GPCR genetic screening; structural immunogenicity prediction \\
\midrule
Dual-compartment autoantibodies & Low & High & Moderate-High & Moderate-High & Dissociation between peripheral and cognitive symptoms; why some treatments help some symptoms & Regional CSF sampling; post-mortem tissue analysis \\
\midrule
Functional vs.\ binding assay discrepancy & Moderate & Very High & High & High & Failed replications; heterogeneous treatment response; why some high-titer patients don't respond & Develop functional autoantibody assays; stratify trials \\
\midrule
\multicolumn{7}{l}{\textit{\textbf{Combined/Integrated Hypotheses}}} \\
\midrule
EBV $\rightarrow$ LMP1 $\rightarrow$ BBB crossing $\rightarrow$ CNS autoimmunity & Low-Moderate & Very High & Moderate-High & High & Full pathway from trigger to CNS symptoms; explains post-viral onset, neuroinflammation, autoantibodies & Integrated CSF + peripheral analysis; antiviral + immunotherapy trials \\
\midrule
Plasma cell + monocyte dual targeting & Moderate & Very High & High & Moderate-High & Why single-target therapies partially work; need for combination approaches & Daratumumab + monocyte modulator (e.g., JAK inhibitor) trial \\
\end{longtable}
\normalsize
\end{landscape}

\subsection{Integration: A Unified EBV-Autoimmune Model}

Drawing together these hypotheses, a coherent model emerges:

\begin{hypothesis}[The EBV-Autoimmune Cascade Model]
ME/CFS may result from a cascade initiated by EBV infection in genetically susceptible individuals:

\begin{enumerate}
    \item \textbf{Trigger:} EBV infection (primary or reactivation) in an individual with GPCR variants containing exposed antigenic hotspots

    \item \textbf{Molecular mimicry:} Anti-EBV antibodies cross-react with homologous GPCR epitopes, or bystander activation generates true autoantibodies

    \item \textbf{Peripheral effects:} GPCR autoantibodies cause receptor internalization on cardiovascular, GI, and peripheral tissues, producing autonomic dysfunction

    \item \textbf{Monocyte reprogramming:} Autoantibody binding to monocyte GPCRs triggers altered cytokine production (MIP-1$\delta$, PDGF-BB, TGF-$\beta$3), creating systemic inflammation

    \item \textbf{CNS invasion:} EBV-infected B cells expressing LMP1 cross the blood-brain barrier and either produce local autoantibodies or trigger complement/microglial activation

    \item \textbf{Plasma cell establishment:} Some autoreactive B cells differentiate into long-lived plasma cells in bone marrow sanctuaries, ensuring persistent autoantibody production

    \item \textbf{Stable pathological state:} The combination of peripheral autoantibody effects, monocyte-driven inflammation, and CNS involvement creates a self-maintaining disease state that persists independent of ongoing EBV activity

    \item \textbf{Treatment resistance:} Single-target therapies fail because multiple mechanisms must be addressed simultaneously:
    \begin{itemize}
        \item Antivirals alone fail: plasma cells already established
        \item Rituximab alone fails: plasma cells are CD20$^-$
        \item Immunoadsorption alone fails: plasma cells regenerate antibodies
        \item Daratumumab alone partially works: addresses plasma cells but not CNS or established receptor depletion
    \end{itemize}
\end{enumerate}

\textbf{Evidence level:} Moderate overall (components individually supported; integration speculative)

\textbf{Therapeutic implication:} Comprehensive treatment might require:
\begin{itemize}
    \item Antiviral therapy (reduce ongoing EBV contribution)
    \item Immunoadsorption (clear existing autoantibodies)
    \item Daratumumab (eliminate plasma cell factories)
    \item Time for receptor regeneration (months post-antibody clearance)
    \item Possibly CNS-directed therapy for cognitive symptoms
\end{itemize}
\end{hypothesis}

\begin{warning}[Speculative Integration]
This unified model is \textbf{highly speculative}. It integrates findings from multiple studies, each with limitations, and extrapolates beyond what any single study demonstrates. The model is presented not as established fact but as a framework for generating testable predictions and guiding research priorities. Clinical application of combination therapies based on this model would require rigorous testing in appropriately designed trials.
\end{warning}


