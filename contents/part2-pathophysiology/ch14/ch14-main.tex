\section{Master Hypothesis Table: Likelihood and Therapeutic Potential}
\label{sec:master-hypothesis-table}

Table~\ref{tab:all-hypotheses} provides a comprehensive overview of all hypotheses presented in this chapter, ranked by evidence strength and therapeutic potential. This serves as a roadmap for both researchers prioritizing investigation directions and clinicians considering experimental interventions.

\begin{landscape}
\tiny
\begin{longtable}{p{3.5cm}p{2cm}p{2cm}p{1.8cm}p{1.8cm}p{4.5cm}p{5cm}}
\caption{Comprehensive ranking of all speculative hypotheses by evidence level, therapeutic potential, and impact on different severity levels} \label{tab:all-hypotheses} \\
\toprule
\textbf{Hypothesis} & \textbf{Evidence Level} & \textbf{Therapeutic Potential} & \textbf{Benefit: Mild} & \textbf{Benefit: Severe} & \textbf{Explains Key Features} & \textbf{Nearest-Term Action} \\
\midrule
\endfirsthead
\multicolumn{7}{c}{\tablename\ \thetable{} -- continued from previous page} \\
\toprule
\textbf{Hypothesis} & \textbf{Evidence Level} & \textbf{Therapeutic Potential} & \textbf{Benefit: Mild} & \textbf{Benefit: Severe} & \textbf{Explains Key Features} & \textbf{Nearest-Term Action} \\
\midrule
\endhead
\midrule
\multicolumn{7}{r}{\textit{Continued on next page}} \\
\endfoot
\bottomrule
\endlastfoot
\multicolumn{7}{l}{\textit{\textbf{CPET-Derived Hypotheses (Objective Functional Data)}}} \\
\midrule
Autonomic-mitochondrial feedback loop & Moderate & High & High & Moderate & PEM, recovery time, autonomic symptoms & Trial: tyrosine+Tetrahydrobiopterin (BH4)+antioxidants \\
\midrule
Mitochondrial turnover rate limitation & Moderate-High & High & Moderate-High & Moderate & 13-day recovery, cumulative decline, GET failure & Urolithin A + NAD+ precursor trial \\
\midrule
Exercise metabolomics-guided therapy & Moderate & Very High & High & Low & Individual variation, treatment heterogeneity & Post-CPET metabolomics study \\
\midrule
Circadian recovery gating & Low-Moderate & Moderate & Moderate & Moderate & Sleep dysfunction, non-restorative rest & Chronotherapy pilot study \\
\midrule
Vagal stimulation for recovery & Low-Moderate & Moderate & Moderate & Low-Moderate & Autonomic dysfunction, inflammation persistence & Post-exertion VNS trial \\
\midrule
\multicolumn{7}{l}{\textit{\textbf{Core Mechanistic Hypotheses}}} \\
\midrule
Metabolic ``safe mode'' lock & Moderate & High & Low-Moderate & Moderate-High & PEM, chronicity, resistance to rehabilitation & Hypothalamic modulation interventions \\
\midrule
Glymphatic clearance failure & Low-Moderate & Moderate & Moderate & Moderate-High & Brain fog, non-restorative sleep, orthostatic symptoms & CSF flow imaging; craniocervical assessment \\
\midrule
Tryptophan/kynurenine trap & Moderate & Moderate-High & Moderate & Moderate & Cognitive symptoms, depression, immune activation & IDO inhibition trials \\
\midrule
Vagal afferent danger signal loop & Low-Moderate & Moderate-High & Moderate & High & Rapid symptom onset, gut-brain connection, PEM & Vagal modulation; gut interventions \\
\midrule
Purinergic signaling dysregulation & Low-Moderate & Moderate & Moderate & Moderate & Immune dysfunction, pain, fatigue, inflammation & P2X/P2Y receptor modulators \\
\midrule
Redox compartment collapse & Moderate & Moderate & Moderate & Low-Moderate & Oxidative stress, chemical sensitivities & Glutathione/N-Acetylcysteine (NAC) optimization \\
\midrule
Metabolic memory/epigenetic lock & Moderate & Low-Moderate & Low & Low-Moderate & Chronicity, treatment resistance & Epigenetic modifiers (exploratory) \\
\midrule
Circadian-metabolic desynchronization & Moderate & Moderate & Moderate & Low-Moderate & Sleep issues, energy fluctuations & Circadian stabilization protocols \\
\midrule
\multicolumn{7}{l}{\textit{\textbf{Autoimmune/Immune Hypotheses}}} \\
\midrule
GPCR autoantibody-driven dysfunction & \textbf{Moderate-High} & \textbf{Very High} & \textbf{High} & \textbf{Moderate-High} & POTS, autonomic symptoms, 60\% daratumumab response & Autoantibody testing; immunoadsorption; daratumumab \\
\midrule
Plasma cell sanctuary hypothesis & Moderate & Very High & High & High & Rituximab failure vs daratumumab success, chronicity & Anti-CD38 therapy; combined IA+daratumumab \\
\midrule
Autoantibody-monocyte activation cascade & Low-Moderate & Moderate-High & Moderate & Moderate & Inflammatory cytokines, MIP-1$\delta$, PDGF-BB elevation & Monocyte-targeted therapy; autoantibody removal \\
\midrule
Ion channel autoimmunity & Low-Moderate & Moderate-High & Moderate-High & Moderate & Autonomic symptoms, POTS, cognitive issues & Autoantibody screening; immunoadsorption \\
\midrule
TRPM3 channelopathy & \textbf{Moderate-High} & \textbf{High} & \textbf{High} & \textbf{Moderate-High} & NK cell dysfunction, impaired immune cell calcium signaling & TRPM3 functional testing; calcium signaling studies; pregnenolone trial (speculative) \\
\midrule
Endothelial trained immunity & Low & Moderate-High & Moderate & Moderate & Multi-system symptoms, vascular dysfunction, PEM & Endothelial epigenetic profiling \\
\midrule
Receptor internalization (not blockade) & Low-Moderate & Moderate-High & Moderate & Moderate & Lag between Ab removal and improvement; receptor density changes & Receptor density assays on patient lymphocytes \\
\midrule
Functional vs.\ binding assay discrepancy & Moderate & Very High & High & High & Failed replications; heterogeneous treatment response & Develop functional autoantibody assays \\
\midrule
\multicolumn{7}{l}{\textit{\textbf{Viral/Cellular Hypotheses}}} \\
\midrule
EBV-B cell CNS infiltration & Low-Moderate & High & Moderate & Moderate-High & Post-EBV onset; neuroinflammation; brain fog & CSF B cell analysis; LMP1 profiling \\
\midrule
EBV-GPCR molecular mimicry & Low & High & Moderate-High & Moderate-High & EBV trigger specificity; persistent autoantibodies & Computational homology; cross-reactivity testing \\
\midrule
Endogenous retrovirus reactivation & Very Low & Low & Low & Low & Post-viral onset, immune activation, chronicity & HERVs expression profiling \\
\midrule
Cellular quorum sensing dysfunction & Very Low & Low & Low-Moderate & Low & Systemic coordination loss, multi-system involvement & Basic research needed \\
\midrule
\multicolumn{7}{l}{\textit{\textbf{Metabolic Compartmentalization Hypotheses}}} \\
\midrule
Lactate compartmentalization disorder & Low & Moderate & Low-Moderate & Low-Moderate & Exercise intolerance, muscle symptoms, brain lactate & MCT function studies; dietary ketones \\
\midrule
Ferroptosis susceptibility & Low & Low-Moderate & Low-Moderate & Low & Oxidative stress, lipid peroxidation, tissue damage & Ferroptosis inhibitors (research) \\
\midrule
\multicolumn{7}{l}{\textit{\textbf{Integrated/Multi-System Hypotheses}}} \\
\midrule
Selective energy dysfunction & Moderate & High & Moderate-High & Moderate-High & Preserved autonomous functions (hair, nails), impaired CNS-dependent processes; demand-response failure & Hair follicle mito assay; CSF lactate; CNS-targeted delivery (Sec.~\ref{sec:selective-dysfunction}) \\
\midrule
Multi-lock integrated trap & High conceptual & Very High & Variable & Variable & Heterogeneity, treatment resistance, chronicity & Multi-target interventions \\
\midrule
\multicolumn{7}{l}{\textit{\textbf{High-Risk/Counterintuitive Hypotheses}}} \\
\midrule
Metabolic preconditioning (hormesis) & Very Low & Low (High Risk) & Unknown & Contraindicated & Adaptation failure? & NOT RECOMMENDED clinically \\
\midrule
Blood flow restriction training & Low & Low-Moderate & Low-Moderate & Contraindicated & Oxygen delivery dysfunction & Research only; high risk \\
\end{longtable}
\normalsize
\end{landscape}

\subsection{How to Use This Table}

\subsubsection{For Researchers}

\textbf{High-priority investigations} (Moderate-High evidence, testable):
\begin{enumerate}
    \item TRPM3 channelopathy: Replication in additional cohorts; characterization of dysfunction mechanism (hypo- vs hyperfunction); correlation with symptom severity
    \item Mitochondrial turnover limitation: Urolithin A intervention with repeat two-day CPET
    \item Autonomic-mitochondrial loop: Multi-target combination trial
    \item Exercise metabolomics: Post-CPET metabolomic profiling to identify subgroups
    \item Ion channel autoimmunity: Comprehensive autoantibody screening (including anti-TRPM3)
\end{enumerate}

\textbf{Medium-priority investigations} (plausible mechanisms, need preliminary data):
\begin{enumerate}
    \item Glymphatic function: Imaging studies assessing CSF flow dynamics
    \item Tryptophan trap: IDO inhibitor safety/efficacy trials
    \item Vagal interventions: VNS for post-exertional recovery
    \item Circadian optimization: Chronotherapy protocols
\end{enumerate}

\textbf{Basic research needed} (very low evidence, high theoretical interest):
\begin{enumerate}
    \item Cellular quorum sensing mechanisms
    \item Endogenous retrovirus expression patterns
    \item Ferroptosis markers and susceptibility
\end{enumerate}

\subsubsection{For Clinicians}

\textbf{Relatively safe to trial} (assuming medical supervision and appropriate patient selection):
\begin{itemize}
    \item Autonomic-mitochondrial support (supplements, generally recognized as safe)
    \item Mitochondrial turnover acceleration (urolithin A, NAD+ precursors have human safety data)
    \item Chronotherapy/circadian stabilization (behavioral, very low risk)
    \item Vagal stimulation (non-invasive, established safety profile)
    \item Tryptophan metabolism support (within normal supplement ranges)
\end{itemize}

\textbf{Requires specialist supervision}:
\begin{itemize}
    \item Ion channel autoantibody testing and immunoadsorption
    \item IDO inhibition (investigational)
    \item Epigenetic modifiers
\end{itemize}

\textbf{Not recommended outside research protocols}:
\begin{itemize}
    \item Metabolic preconditioning/hormesis approaches (high risk of PEM)
    \item Blood flow restriction training (could worsen oxygen delivery dysfunction)
    \item Endogenous retrovirus interventions (purely theoretical)
\end{itemize}

\subsubsection{For Patients}

\textbf{Understanding evidence levels:}
\begin{itemize}
    \item \textbf{Very Low:} Purely theoretical speculation; interesting for research but no evidence
    \item \textbf{Low:} Mechanism makes sense based on other diseases; no ME/CFS-specific data
    \item \textbf{Low-Moderate:} Some indirect evidence in ME/CFS; plausible but unproven
    \item \textbf{Moderate:} Multiple ME/CFS studies support mechanism; direct intervention untested
    \item \textbf{Moderate-High:} Strong mechanistic support; similar interventions show promise
    \item \textbf{High:} Direct evidence from ME/CFS trials (rare in this chapter, as these are speculative hypotheses)
\end{itemize}

\textbf{Severity-specific guidance:}
\begin{itemize}
    \item \textbf{Mild-moderate patients:} May benefit from metabolomics-guided approaches, autonomic support, circadian optimization
    \item \textbf{Severe patients:} Prioritize hypotheses addressing core metabolic function (safe mode, mitochondrial turnover, glymphatic clearance); avoid any interventions requiring exertion
    \item \textbf{All severities:} Multi-lock hypothesis suggests combinations may work better than single interventions
\end{itemize}

\subsection{Qualification and Caveats}

\begin{warning}[Speculative Content]
ALL hypotheses in this chapter are speculative to varying degrees. The evidence levels indicate relative plausibility and existing support, but even ``Moderate-High'' evidence hypotheses remain unproven. Therapeutic approaches derived from these hypotheses should be considered experimental and discussed with knowledgeable physicians. Patient self-experimentation carries risks, especially for severe patients where any metabolic perturbation might trigger crashes.
\end{warning}

