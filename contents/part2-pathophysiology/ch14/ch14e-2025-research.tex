\section{Emerging Hypotheses from 2025 Research}
\label{sec:2025-hypotheses}

Recent multi-omics studies and clinical trials have revealed patterns that suggest several novel mechanistic hypotheses not previously considered.

\subsection{The Vascular-Immune-Energy Triad}

\begin{open_question}[Coordinated Three-System Failure]
The Heng et al.\ 2025 study~\cite{heng2025mecfs} identified a 7-biomarker diagnostic model spanning three systems: adenosine metabolism (AMP), immune markers (cDC1, LYVE1, IGHG2), and vascular factors (FN1, VWF, THBS1). This wasn't three separate findings---it was one integrated signature. What if ME/CFS fundamentally involves a coordinated failure mode across these three systems that cannot be understood or treated in isolation?

The triad might work as follows:
\begin{enumerate}
    \item \textbf{Energy failure} (elevated AMP/ADP, reduced ATP) impairs immune cell maturation and function
    \item \textbf{Immature immune cells} (elevated na\"ive B cells, reduced switched memory B cells, immature T cell subsets) fail to properly regulate vascular function and produce dysfunctional antibodies
    \item \textbf{Vascular dysfunction} (elevated VWF, fibronectin, thrombospondin) reduces tissue perfusion, causing cellular hypoxia that worsens energy production
\end{enumerate}

This creates a stable triangular trap where each vertex reinforces the others. Treating only one system fails because the other two pull it back.
\end{open_question}

\paragraph{Therapeutic Implication.} Effective treatment might require simultaneous intervention at all three vertices: NAD$^+$ precursors for energy, immunomodulation for immune maturation, and vascular-targeted therapy (anticoagulation, endothelial support) for perfusion. The daratumumab success (60\% response) might reflect cases where the autoimmune vertex was dominant---remove it, and the triad destabilizes enough to collapse.

\subsection{The Plasma Cell Sanctuary Hypothesis}

\begin{open_question}[Long-Lived Plasma Cells as Disease Reservoir]
The daratumumab trial's success---where targeting CD38$^+$ plasma cells produced sustained remission in 60\% of patients---reveals something important: rituximab (anti-CD20) failed in ME/CFS trials, yet daratumumab (anti-CD38) succeeded. Both deplete antibody-producing cells, but they target different populations.

B cells (CD20$^+$) are the precursors; plasma cells (CD38$^+$) are the factories. Crucially, long-lived plasma cells can survive for \textit{decades} in bone marrow and gut niches, continuously secreting antibodies without needing B cell replenishment. What if ME/CFS is maintained by these ``sanctuary'' plasma cells?

Under this model:
\begin{itemize}
    \item An initial trigger (infection) generates autoreactive B cells
    \item Some differentiate into long-lived plasma cells that migrate to survival niches
    \item These plasma cells produce autoantibodies (anti-GPCR, anti-ion channel) indefinitely
    \item Rituximab depletes B cells but not established plasma cells---antibody production continues
    \item By the time B cells return, the patient hasn't improved, so the trial ``fails''
    \item Daratumumab directly kills the plasma cell factories, stopping antibody production
\end{itemize}

This explains the 8--9 month delay before maximum daratumumab benefit: existing autoantibodies must decay (IgG half-life $\sim$3 weeks, but tissue-bound antibodies persist longer).
\end{open_question}

\paragraph{Undocumented Phenomenon.} If true, ME/CFS patients should have expanded populations of long-lived plasma cells in bone marrow biopsies, and these cells should be producing the pathogenic autoantibodies. This has never been directly examined.

\paragraph{Treatment Implication.} Combining daratumumab (kill factories) with immunoadsorption (remove existing antibodies) might produce faster and more complete responses than either alone.

\subsection{The Endothelial Activation Cascade}

\begin{open_question}[Chronic Endotheliopathy as Core Mechanism]
The Heng 2025 study~\cite{heng2025mecfs} found elevated plasma proteins associated with ``activation of the endothelium and remodeling of vessel walls.'' Specifically: VWF (von Willebrand factor), FN1 (fibronectin), and THBS1 (thrombospondin-1). These aren't random inflammatory markers---they suggest a specific pathology: chronic endothelial activation.

Endothelial cells line all blood vessels. When activated (by infection, inflammation, autoantibodies, or hypoxia), they:
\begin{itemize}
    \item Release VWF, promoting platelet adhesion and microclotting
    \item Deposit fibronectin, contributing to vascular remodeling
    \item Express thrombospondin, which is anti-angiogenic and pro-fibrotic
    \item Become ``leaky,'' allowing inappropriate extravasation
    \item Lose their normal anti-inflammatory and vasodilatory functions
\end{itemize}

What if ME/CFS is fundamentally an endotheliopathy---a chronic disease of blood vessel lining? This would explain:
\begin{itemize}
    \item \textbf{Exercise intolerance:} Dysfunctional endothelium cannot vasodilate properly to meet demand
    \item \textbf{Brain fog:} Cerebral microvascular dysfunction impairs cognition
    \item \textbf{Orthostatic intolerance:} Poor vascular tone regulation
    \item \textbf{PEM:} Exercise-induced endothelial stress takes days to resolve
    \item \textbf{Multi-system involvement:} Endothelium is everywhere
\end{itemize}
\end{open_question}

\paragraph{Connection to Long COVID.} This hypothesis aligns with the ``microclot'' findings in Long COVID, where amyloid-fibrin microclots persist in circulation. ME/CFS might involve the same endothelial activation without necessarily forming detectable microclots.

\paragraph{Undocumented Phenomenon.} Direct endothelial function testing (flow-mediated dilation, EndoPAT) in ME/CFS has been limited. Comprehensive endothelial biomarker panels and functional testing might reveal a consistent endotheliopathy signature.

\paragraph{Treatment Implication.} If endothelial dysfunction is central:
\begin{itemize}
    \item Endothelial-protective supplements (L-arginine, L-citrulline, beetroot/nitrates) might help
    \item Statins (pleiotropic endothelial benefits beyond cholesterol) might be beneficial
    \item Low-dose aspirin or other anti-platelet agents might reduce microclot burden
    \item ACE inhibitors (endothelial-protective independent of blood pressure) could be therapeutic
    \item HELP apheresis (removes fibrinogen and inflammatory mediators) might address both cause and consequence
\end{itemize}

\subsection{The Dendritic Cell Maturation Block}

\begin{open_question}[Stuck Immune Development]
The Heng 2025 study~\cite{heng2025mecfs} found reduced CD1c$^+$CD141$^-$ conventional dendritic cells type 2 (cDC2) and a general skewing toward ``less mature'' immune cell subsets across T cells, NK cells, and dendritic cells. This isn't random immune dysfunction---it suggests a specific developmental block.

Dendritic cells are the ``conductors'' of the immune orchestra. They:
\begin{itemize}
    \item Capture antigens and present them to T cells
    \item Determine whether immune responses are inflammatory or tolerogenic
    \item Bridge innate and adaptive immunity
    \item Mature in response to danger signals
\end{itemize}

What if ME/CFS involves a block in dendritic cell maturation? Immature DCs:
\begin{itemize}
    \item Present antigens inefficiently
    \item Fail to properly activate T cells
    \item May promote tolerance when activation is needed (chronic infection persistence)
    \item May promote inflammation when tolerance is needed (autoimmunity)
\end{itemize}

The immune system would be simultaneously ineffective (can't clear threats) and dysregulated (inappropriate responses). This dual failure could maintain chronic immune activation without resolution.
\end{open_question}

\paragraph{Why Maturation Might Be Blocked.}
\begin{itemize}
    \item \textbf{Energy deficit:} DC maturation is metabolically demanding; ATP shortage might arrest development
    \item \textbf{Chronic antigen exposure:} Persistent viral antigens or autoantibodies might cause ``exhaustion''
    \item \textbf{Cytokine milieu:} Altered cytokine patterns might signal DCs to remain immature
    \item \textbf{Epigenetic lock:} Maturation genes might be epigenetically silenced
\end{itemize}

\paragraph{Treatment Implication.} Therapies that promote DC maturation (GM-CSF, specific TLR agonists, DC-targeted vaccines) might help---but could also be dangerous if the DCs then activate against self-antigens. This is a double-edged sword requiring careful patient selection.

\subsection{The NAD$^+$ Depletion Spiral}

\begin{open_question}[NAD$^+$ as the Central Bottleneck]
Multiple findings converge on NAD$^+$:
\begin{itemize}
    \item Heng et al.~\cite{heng2025mecfs}: Abnormal NAD$^+$ metabolism in ME/CFS immune cells
    \item The tryptophan-kynurenine pathway terminates in NAD$^+$ synthesis
    \item PARP enzymes (activated by DNA damage/oxidative stress) consume NAD$^+$
    \item Sirtuins (cellular stress response) require NAD$^+$
    \item Mitochondrial Complex I requires NAD$^+$/NADH cycling
\end{itemize}

What if NAD$^+$ depletion is not just a consequence but a central driver---a bottleneck where multiple pathological processes converge?

The spiral might work as follows:
\begin{enumerate}
    \item Initial insult causes oxidative stress and DNA damage
    \item PARP enzymes activate to repair damage, consuming NAD$^+$
    \item NAD$^+$ depletion impairs mitochondrial function (Complex I requires NAD$^+$)
    \item Mitochondrial dysfunction increases oxidative stress
    \item More oxidative stress $\rightarrow$ more PARP activation $\rightarrow$ more NAD$^+$ depletion
    \item Meanwhile, inflammatory IDO activation shunts tryptophan away from serotonin toward kynurenine-NAD$^+$ pathway---but the NAD$^+$ produced may be immediately consumed by PARPs
    \item Sirtuins, starved of NAD$^+$, cannot perform their protective functions (autophagy, mitophagy, epigenetic regulation)
    \item The cell enters a stable low-NAD$^+$ state where it survives but cannot function normally
\end{enumerate}
\end{open_question}

\paragraph{Undocumented Phenomenon.} Direct measurement of NAD$^+$/NADH ratios in ME/CFS patient tissues (not just blood) has been limited. If the spiral hypothesis is correct:
\begin{itemize}
    \item Tissue NAD$^+$ should be severely depleted
    \item PARP activity should be chronically elevated
    \item Sirtuin activity should be reduced
    \item The kynurenine pathway should be active but NAD$^+$ still depleted (production consumed by PARPs)
\end{itemize}

\paragraph{Treatment Implication.} NAD$^+$ precursors (NR, NMN) alone might fail if PARPs immediately consume the new NAD$^+$. Combination with PARP inhibitors (used in cancer) might be necessary---but PARP inhibition carries risks (impaired DNA repair). A gentler approach: high-dose NAD$^+$ precursors to ``flood'' the system beyond PARP consumption capacity.

\subsection{The Effort-Preference Recalibration}

\begin{open_question}[Central Effort Computation Gone Wrong]
The Walitt 2024 NIH study made a crucial distinction: ME/CFS patients showed \textit{altered effort preference}, not physical fatigue or central fatigue. Their muscles could produce force; their brain could generate motor commands. But when given choices, they systematically avoided effortful options even when rewards were high.

This isn't laziness or depression---it's a recalibration of the brain's effort-reward computation. The brain has a system (involving the anterior cingulate cortex, insula, and dopaminergic circuits) that weighs expected effort against expected reward to decide whether actions are ``worth it.''

What if ME/CFS involves a fundamental shift in this computation, such that:
\begin{itemize}
    \item Effort is perceived as more costly than it actually is
    \item Rewards are perceived as less valuable than they would be
    \item The ``break-even'' point shifts dramatically toward rest
    \item This shift is protective (effort genuinely IS more costly due to metabolic dysfunction) but becomes miscalibrated
\end{itemize}

The CSF catecholamine deficiency found by Walitt et al.\ supports this: dopamine is central to effort-reward computation. Reduced central dopamine would systematically bias the system toward effort avoidance.
\end{open_question}

\paragraph{Why This Matters.} If effort preference is centrally altered, then:
\begin{itemize}
    \item ``Pushing through'' fights against an active brain computation, not just physical limits
    \item The system might be trainable but requires different approaches than physical reconditioning
    \item Dopaminergic interventions might help recalibrate the computation
    \item But if the recalibration is \textit{appropriate} given metabolic dysfunction, forcing change could be harmful
\end{itemize}

\paragraph{Treatment Implication.} Low-dose stimulants (methylphenidate, modafinil) might shift effort-reward computation---but could cause crashes if patients then overexert. The key might be: restore metabolic function FIRST, then (if needed) recalibrate effort perception.

\subsection{The Immune Cell Energy Crisis}

\begin{open_question}[Starving Sentinels]
The Heng 2025 finding~\cite{heng2025mecfs} of elevated AMP/ADP in white blood cells suggests immune cells specifically are energy-starved. This has profound implications because immune cells are \textit{metabolically unique}:

\begin{itemize}
    \item Na\"ive T cells are metabolically quiescent
    \item Upon activation, T cells undergo massive metabolic reprogramming (Warburg effect)
    \item This reprogramming requires abundant ATP and NAD$^+$
    \item If immune cells cannot meet energy demands, activation fails
    \item Failed activation = ineffective immune responses + potential for inappropriate responses
\end{itemize}

The pattern of ``immature'' immune cells in ME/CFS might not reflect a developmental block per se, but rather an \textit{energy crisis} that prevents cells from completing their activation/maturation programs.

Consider: a T cell encounters its antigen and begins activation. Activation requires massive ATP expenditure. But the cell is already AMP/ADP-elevated, ATP-depleted. It cannot complete activation. It either:
\begin{itemize}
    \item Dies (activation-induced cell death from energy failure)
    \item Becomes anergic (gives up on activation)
    \item Partially activates (creating dysfunctional effector cells)
\end{itemize}

Any of these outcomes would create the immune dysfunction pattern seen in ME/CFS.
\end{open_question}

\paragraph{Undocumented Phenomenon.} The metabolic competence of ME/CFS immune cells during activation has not been thoroughly studied. Prediction: ME/CFS T cells stimulated in vitro should show impaired metabolic reprogramming (measured by Seahorse assay or similar).

\paragraph{Treatment Implication.} Supporting immune cell metabolism specifically might help:
\begin{itemize}
    \item NAD$^+$ precursors might restore immune cell energy capacity
    \item Specific metabolites (pyruvate, $\alpha$-ketoglutarate) might bypass defective pathways
    \item Ketone bodies (which immune cells can use as fuel) might provide alternative energy
\end{itemize}

\subsection{The Vascular ``Memory'' Hypothesis}

\begin{open_question}[Trained Endothelial Dysfunction]
Immune cells can be ``trained''---epigenetically reprogrammed by past exposures to respond differently to future stimuli. This innate immune memory (distinct from adaptive immunity) has been demonstrated in monocytes, macrophages, and NK cells.

What if endothelial cells can also be ``trained''---and what if ME/CFS involves maladaptive endothelial training?

Endothelial cells experience the initial infection/inflammation. They activate, express adhesion molecules, become pro-thrombotic. Normally they return to quiescence. But what if severe or prolonged activation creates epigenetic changes that lock them in a partially activated state?

This ``trained endotheliopathy'' would:
\begin{itemize}
    \item Persist long after the original trigger resolves
    \item Be present throughout the vasculature (explaining multi-system symptoms)
    \item Respond excessively to normal stimuli (exercise, stress, infection)
    \item Be resistant to conventional anti-inflammatory treatment
    \item Potentially be reversible with epigenetic interventions
\end{itemize}
\end{open_question}

\paragraph{Undocumented Phenomenon.} Epigenetic profiling of endothelial cells from ME/CFS patients has not been performed. Circulating endothelial cells or endothelial progenitor cells might show characteristic epigenetic signatures.

\subsection{Speculative Treatment Approaches from 2025 Findings}

Based on the above hypotheses, several novel treatment approaches emerge:

\subsubsection{The Triple-Target Protocol}

\begin{speculation}[Simultaneous Triad Intervention]
If the vascular-immune-energy triad is the core mechanism, a protocol targeting all three simultaneously might produce synergistic effects:

\begin{enumerate}
    \item \textbf{Energy:} High-dose NAD$^+$ precursor (NR 1000--2000~mg/day) plus mitochondrial cofactors (CoQ10, PQQ, B vitamins)
    \item \textbf{Immune:} Low-dose naltrexone (immune modulation) plus vitamin D optimization (immune regulation)
    \item \textbf{Vascular:} L-arginine/citrulline (endothelial NO production) plus low-dose aspirin (anti-platelet) plus omega-3 fatty acids (endothelial protection)
\end{enumerate}

This combination is relatively safe and addresses all three triad vertices. The hypothesis predicts it should work better than any single intervention.
\end{speculation}

\subsubsection{The Plasma Cell Eradication Strategy}

\begin{speculation}[Deep Autoantibody Elimination]
For patients with evidence of autoimmunity (elevated anti-GPCR antibodies, post-infectious onset, dramatic response to immunoadsorption):

\begin{enumerate}
    \item \textbf{Phase 1:} Immunoadsorption series to remove circulating autoantibodies
    \item \textbf{Phase 2:} Daratumumab (or similar CD38-targeting agent) to eliminate plasma cell factories
    \item \textbf{Phase 3:} Monitor for autoantibody rebound; repeat if needed
    \item \textbf{Phase 4:} Once autoantibodies cleared, assess whether other ``locks'' need addressing
\end{enumerate}

This aggressive approach would only be appropriate for patients with clear autoimmune features and access to specialized centers.
\end{speculation}

\subsubsection{The Endothelial Restoration Protocol}

\begin{speculation}[Vascular Healing Focus]
If endotheliopathy is central, a vascular-focused protocol might help:

\begin{enumerate}
    \item \textbf{Reduce endothelial activation:} Statin therapy (pleiotropic endothelial effects)
    \item \textbf{Support NO production:} L-citrulline (better than L-arginine for sustained NO)
    \item \textbf{Address microclots:} Nattokinase (fibrinolytic enzyme) or low-dose anticoagulation if indicated
    \item \textbf{Protect endothelium:} Sulforaphane (Nrf2 activation), omega-3s, anthocyanins
    \item \textbf{Reduce thrombotic tendency:} Aspirin, adequate hydration, compression if tolerated
\end{enumerate}

This approach treats ME/CFS as a vascular disease, which it may fundamentally be in at least a subset of patients.
\end{speculation}


