\chapter{Controversies and Debates in ME/CFS Research}
\label{ch:controversies}

\section{Nomenclature and Definition}
\label{sec:nomenclature-controversy}

% ME vs. CFS vs. ME/CFS vs. SEID
% Impact on research and patient care
% Efforts toward consensus

\section{Diagnostic Criteria Controversies}
\label{sec:diagnostic-controversies}

% Oxford criteria problems
% Empirical definition issues
% Impact on research heterogeneity
% Path forward

\section{The PACE Trial Controversy}
\label{sec:pace-controversy}

% What the trial claimed
% Criticisms of methodology
% Reanalysis results
% Impact on GET/CBT recommendations
% Lessons for future research

\section{Psychogenic vs. Biomedical Models}
\label{sec:psychogenic-debate}

% Historical context
% Evidence for biological basis
% Role of psychological factors
% Biopsychosocial integration
% Patient advocacy concerns

\section{Exercise Therapy Debates}
\label{sec:exercise-debates}

% GET proponents' arguments
% Harms reported
% Alternative approaches
% Current recommendations

\section{Deconditioning Hypothesis}
\label{sec:deconditioning-hypothesis}

% Arguments for and against
% Evidence from exercise studies
% Distinguishing from primary pathology

\section{The NIH ``Effort Preference'' Controversy (2024--2025)}
\label{sec:effort-preference-debate}

The 2024 NIH deep phenotyping study by Walitt et al.~\cite{walitt2024deep} generated one of the most significant controversies in recent ME/CFS research history. While the study documented multiple objective biological abnormalities (catecholamine deficiency, immune dysfunction, autonomic abnormalities), its interpretive framing around ``effort preference'' sparked intense criticism from patients, clinicians, and researchers.

\subsection{The Central Claim}

The study concluded that ``effort preference, not fatigue, is the defining motor behavior'' of post-infectious ME/CFS. The authors proposed that ME/CFS patients have altered ``effort preference''---defined as ``how much effort a person subjectively wants to exert''---due to dysfunction of integrative brain regions, particularly the temporoparietal junction (TPJ).

\subsection{Why the Framing Was Controversial}

\subsubsection{Language Echoing Psychogenic Models}
The term ``preference'' implies volition and choice. Critics argued this framing echoed decades of psychogenic characterizations of ME/CFS that attributed symptoms to patients' beliefs, behaviors, or psychological states rather than biological dysfunction. The language resonated uncomfortably with PACE trial rhetoric about ``unhelpful illness beliefs.''

\subsubsection{Methodological Problems with the EEfRT}
The Effort-Expenditure for Rewards Task (EEfRT) was designed to measure motivation for rewards in psychiatric conditions, with an explicit requirement that tasks be easy enough that fatigue does not confound results. In the Walitt study:

\begin{itemize}
    \item ME/CFS patients completed only 65\% of hard trials vs.\ 96--99\% for controls
    \item Seven of 15 patients performed below any control participant
    \item Physical function scores (SF-36) were 28.7 for patients vs.\ 97.5 for controls
\end{itemize}

Kirvin-Quamme et al.'s reanalysis~\cite{kirvinquamme2025effort} demonstrated a significant correlation between task completion ability and task choice, indicating the tool measured ability, not preference. Their conclusion: patients were ``unable,'' not ``unwilling.''

\subsubsection{Failure to Document PEM}
The study used single-day CPET rather than the gold-standard two-day protocol, failing to objectively document post-exertional malaise---the defining feature of ME/CFS. PEM was mentioned only three times in the entire paper.

\subsubsection{Selection Bias}
By excluding severely affected patients (25\% of the ME/CFS population), the study could not characterize the full spectrum of disease severity.

\subsection{Published Academic Responses}

\subsubsection{Nature Communications Commentary}
Davenport et al.\ published a formal critique in \textit{Nature Communications}~\cite{effortcritique2025} stating the interpretation ``risks reinforcing skepticism about the serious biological nature of [ME] and its hallmark of post-exertional malaise (PEM), as well as its potential misclassification as a mental health condition.''

\subsubsection{Authors' Reply}
Walitt et al.\ responded~\cite{walitt2025reply} clarifying that:
\begin{itemize}
    \item Deconditioning and ME/CFS are not mutually exclusive
    \item Deconditioning is a consequence, not cause
    \item Equal maximum grip strength argues against pure deconditioning
    \item Impaired performance occurs before oxidative metabolism stress
\end{itemize}

Critics noted the reply did not address the fundamental EEfRT methodology concerns or the harm of ``preference'' language.

\subsubsection{Frontiers in Psychology Reanalysis}
The Kirvin-Quamme et al.\ reanalysis~\cite{kirvinquamme2025effort} provided detailed statistical evidence that the EEfRT data supported inability rather than altered preference, calling for proper task calibration in future studies.

\subsection{NIH Clarification}

Following criticism, NIH clarified that ``preference'' referred to ``subconscious or unconscious or pre-conscious calculations by the brain'' rather than conscious choice. Patient advocates responded that if unconscious brain dysfunction was the intended meaning, using ``preference''---a word implying choice---was misleading and potentially harmful to patients.

\subsection{Clinical and Research Implications}

\subsubsection{Potential Harms}
\begin{itemize}
    \item Reinforcement of psychogenic misconceptions among clinicians unfamiliar with ME/CFS
    \item Justification for continued use of graded exercise therapy despite harms
    \item Barriers to disability recognition if symptoms are framed as ``preference''
    \item Psychological harm to patients from invalidating language
\end{itemize}

\subsubsection{Lessons for Future Research}
\begin{itemize}
    \item Assessment tools must be appropriate for the population studied
    \item Language matters: terminology should not inadvertently pathologize or blame patients
    \item Peer review should include ME/CFS experts and patient representatives
    \item Objective measures of PEM (two-day CPET) should be standard
    \item Study designs should include severely affected patients
\end{itemize}

\subsection{Separating Data from Interpretation}

Despite the interpretive controversy, the Walitt study's biological findings---CSF catecholamine deficiency, B cell population shifts, autonomic dysfunction, chronotropic incompetence---represent valuable contributions confirmed by other research. The challenge for the field is to build on these objective findings while rejecting framings that risk harm to patients.

\begin{observation}[The Value of Controversial Studies]
The NIH deep phenotyping study illustrates how a study can simultaneously advance biological understanding and generate harmful interpretations. The catecholamine findings alone---the first CSF neurotransmitter measurements in ME/CFS---provide crucial mechanistic insight. Future citations should specify which findings are being referenced (the objective biological data vs.\ the contested ``effort preference'' interpretation) to prevent misuse while preserving scientific value.
\end{observation}

\section{Post-COVID ME/CFS}
\label{sec:post-covid-debates}

% Overlap with long COVID
% Differences in recognition and research funding
% Opportunities and challenges

\section{Research Funding Disparities}
\label{sec:funding-disparities}

% Historical underfunding
% Comparison to other diseases
% Recent improvements
% Remaining gaps
