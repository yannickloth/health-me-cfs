\chapter{Epidemiological and Outcomes Research}
\label{ch:epidemiology-outcomes}

\section{Prevalence and Incidence Studies}
\label{sec:prevalence-incidence}

% Population-based studies
% Geographic variations
% Temporal trends
% Post-COVID surge

\section{Risk Factor Studies}
\label{sec:risk-factors-studies}

\subsection{Genetic Risk Factors}
% Family and twin studies
% Genetic association studies

\subsection{Environmental Risk Factors}
% Infectious triggers
% Toxic exposures
% Stress and trauma

\subsection{Demographic Risk Factors}
% Age, sex, ethnicity
% Socioeconomic factors

\section{Natural History Studies}
\label{sec:natural-history}

% Longitudinal cohort studies
% Recovery rates
% Progression patterns
% Predictors of outcome

\section{Quality of Life and Disability Studies}
\label{sec:quality-of-life-studies}

% SF-36 findings
% Comparison to other diseases
% Economic burden studies
% Impact on caregivers

\section{Mortality Studies}
\label{sec:mortality}

Understanding mortality patterns in ME/CFS is essential for both clinical practice and actuarial assessment (life insurance underwriting). While early concerns suggested potentially elevated mortality, large population-based cohort studies have provided more nuanced evidence. This section synthesizes findings from registry studies, clinical cohorts, and memorial record analyses.

\subsection{All-Cause Mortality: Evidence from Population Cohorts}
\label{subsec:all-cause-mortality}

\subsubsection{Large Registry-Based Studies}

The most rigorous evidence comes from population-based registry studies with appropriate comparison groups:

\paragraph{Roberts et al.\ (2016) -- England and Wales National Registry.}
This landmark study published in \textit{The Lancet}~\cite{Roberts2016} analyzed mortality in 2,147 ME/CFS patients identified through English and Welsh general practice registries, with 7-year follow-up (2007--2013). The study recorded 17 deaths during follow-up.

\textbf{Key findings:}
\begin{itemize}
    \item \textbf{All-cause mortality SMR: 1.14} (95\% CI: 0.65--1.85, $p = 0.67$)
    \item No statistically significant elevation in all-cause mortality
    \item Cancer-specific SMR: 1.39 (95\% CI: 0.60--2.73, $p = 0.45$) -- not significant
    \item \textbf{Suicide-specific SMR: 6.85} (95\% CI: 2.22--15.98, $p = 0.002$) -- \textit{highly significant}
\end{itemize}

Notably, 5 of the 17 deaths were suicides, and 60\% of suicide victims had no documented depression diagnosis, suggesting that ME/CFS-specific factors (functional limitation, hopelessness about prognosis, medical gaslighting) contribute to suicide risk independent of comorbid psychiatric conditions.

\paragraph{Smith et al.\ (2006) -- US Multi-Center Cohort.}
A US study~\cite{Smith2006} followed 1,201 patients with chronic fatigue for up to 14 years, using National Death Index (NDI) linkage for mortality ascertainment.

\textbf{Key findings:}
\begin{itemize}
    \item \textbf{All-cause mortality: No elevation} above expected rates for age and sex
    \item \textbf{Suicide rate: $>$8 times higher} than US general population
    \item SMR for suicide particularly elevated in ``chronic fatigue'' not meeting full CFS criteria (SMR: 14.2) compared to CFS (SMR: 3.6)
    \item Suggests that lack of medical legitimization may increase suicide risk
\end{itemize}

\subsubsection{Conflicting Evidence: Memorial Record Studies}

Studies based on memorial records and caregiver surveys have reported more concerning findings, but these suffer from significant selection bias toward severely ill and deceased patients:

\paragraph{McManimen et al.\ (2016) -- Caregiver Survey.}
Analysis of 56 deaths reported by caregivers~\cite{McManimen2016} found:
\begin{itemize}
    \item Mean age at death: \textbf{55.9 years} vs.\ 73.5 years in general population ($p < 0.0001$)
    \item 48.2\% of deceased were bedridden before death
    \item Mean age at cardiovascular death: 58.8 years vs.\ 77.7 years ($p < 0.0001$)
\end{itemize}

\textbf{Critical limitation:} Memorial records inherently overrepresent severe cases and premature deaths (survivors do not appear in memorials). This creates profound selection bias.

\paragraph{Sirotiak \& Amro (2025) -- Updated Memorial Analysis.}
The most recent memorial record analysis~\cite{Sirotiak2025} examined 505 deaths:
\begin{itemize}
    \item Mean age at death: \textbf{52.5 years} (SD = 16.7)
    \item Most frequent causes: ME/CFS complications (28.3\%), suicide (25.4\%), cancer (23.0\%), cardiovascular disease (14.2\%)
\end{itemize}

While concerning, these findings must be interpreted cautiously given selection bias. The authors acknowledge that memorial records may capture ``the tip of the iceberg'' of severe, fatal cases rather than representing typical ME/CFS mortality patterns.

\subsection{Cause-Specific Mortality}
\label{subsec:cause-specific-mortality}

\subsubsection{Suicide: The Most Robust Finding}

Across \textit{all} study types---registry cohorts, clinical cohorts, and memorial records---suicide mortality is consistently and substantially elevated:

\begin{table}[htbp]
\centering
\caption{Suicide Mortality Across ME/CFS Studies}
\label{tab:suicide-mortality}
\begin{tabular}{lcc}
\toprule
\textbf{Study} & \textbf{SMR or Rate Ratio} & \textbf{Significance} \\
\midrule
Roberts et al.\ (2016) & 6.85 & $p = 0.002$ \\
Smith et al.\ (2006) -- CFS & 3.6 & Significant \\
Smith et al.\ (2006) -- Chronic Fatigue & 14.2 & Highly significant \\
Jason et al.\ (2006) & 2nd most common cause & --- \\
\bottomrule
\end{tabular}
\end{table}

\paragraph{Suicidal Ideation Prevalence.}
Cross-sectional surveys reveal alarming rates of suicidal thoughts:
\begin{itemize}
    \item \textbf{39--57\%} of moderately to severely ill ME/CFS patients report suicidal ideation~\cite{Chu2019}
    \item Compare to \textbf{4\%} in general US population
    \item \textbf{7.1\%} have suicidal ideation \textit{without} clinical depression~\cite{Brown2020}
\end{itemize}

\paragraph{Risk Factors for Suicide in ME/CFS.}
Research has identified ME/CFS-specific suicide risk factors distinct from typical psychiatric risk factors~\cite{Brown2020}:

\begin{itemize}
    \item \textbf{Severe functional limitations} (strongest predictor)
    \item Use of ``CFS'' diagnostic label (associated with stigma) -- 2.81$\times$ increased risk
    \item Absence of comorbidities (paradoxically increases risk, possibly due to lack of medical legitimacy) -- 3.48$\times$ increased risk
    \item Lack of social support and financial resources
    \item Hopelessness about prognosis and treatment availability
    \item Stigma and gaslighting from healthcare providers
\end{itemize}

Notably, 60\% of ME/CFS patients who died by suicide in the Roberts cohort had \textit{no documented depression diagnosis}, suggesting that ME/CFS-specific suffering---not psychiatric comorbidity---drives suicide risk.

\subsubsection{Cardiovascular Mortality: Conflicting Evidence}

\paragraph{Concerning Signals from Memorial Records.}
Memorial record studies suggest elevated cardiovascular mortality:
\begin{itemize}
    \item Mean age at cardiovascular death: 58.8 years vs.\ 77.7 years~\cite{McManimen2016}
    \item Cardiovascular disease: 14.2\% of deaths in recent memorial analysis~\cite{Sirotiak2025}
    \item Heart failure identified as most common cause of death in Jason et al.\ (2006)~\cite{Jason2006}
\end{itemize}

\paragraph{Cardiovascular Disease Prevalence.}
Epidemiological surveys show elevated cardiovascular disease prevalence in ME/CFS:
\begin{itemize}
    \item \textbf{25\%} of ME/CFS patients report history of heart disease or hypertension vs.\ \textbf{5\%} in general population
    \item Meta-analysis: \textbf{51.4\%} prevalence of any cardiac abnormalities
    \item OR 3.26 (95\% CI: 2.85--3.72, $p < 0.001$) for cardiovascular disease in 2021--2022 NHIS data
\end{itemize}

\paragraph{Mechanism Uncertainty.}
Critically, cardiovascular dysfunction in ME/CFS does \textit{not} appear to follow typical atherosclerotic pathways. Instead, it is characterized by:
\begin{itemize}
    \item Reduced stroke volume and cardiac output
    \item Impaired cerebral blood flow
    \item Small heart size (``athlete's heart'' in reverse)
    \item These abnormalities are \textit{not} influenced by deconditioning
\end{itemize}

\textbf{Implication:} Standard cardiovascular risk models developed for atherosclerotic disease may not apply to ME/CFS. Whether this translates to elevated mortality risk remains unclear, and large registry studies have not confirmed elevated cardiovascular mortality.

\subsubsection{Cancer Mortality: No Evidence of Elevation}

Despite cancer appearing in memorial records (23.0\% of deaths~\cite{Sirotiak2025}), population-based studies find no significant elevation:
\begin{itemize}
    \item Roberts et al.: Cancer-specific SMR 1.39 (95\% CI: 0.60--2.73, $p = 0.45$)~\cite{Roberts2016}
    \item Age at cancer death in memorial records not significantly different from general population
\end{itemize}

\subsection{Actuarial and Insurance Industry Perspective}
\label{subsec:actuarial-perspective}

\subsubsection{Mortality Risk Assessment}

The insurance industry has begun evaluating ME/CFS mortality risk. Gen Re, a major global reinsurer, published an analysis in 2023~\cite{GenRe2023} concluding:

\begin{quote}
``There seems to be no significant difference between all-cause mortality rates of ME/CFS patients and the general population.''
\end{quote}

However, the report notes that patients with ``very severe fatigue'' may have elevated cardiovascular mortality, though evidence remains limited.

\subsubsection{Disability vs.\ Mortality: The Primary Actuarial Concern}

Critically, \textbf{disability---not mortality---represents the primary actuarial risk} in ME/CFS:

\begin{itemize}
    \item Recovery rates: $<$10\%
    \item Unemployment: 35--69\%
    \item Annual US economic costs: \$17--24 billion
    \item Functional impairment drives actuarial risk more than mortality
\end{itemize}

This distinction matters: life insurance underwriters assess \textit{mortality} risk, while disability insurers assess \textit{functional capacity}. ME/CFS poses greater risk to the latter.

\subsection{Methodological Challenges and Evidence Quality}
\label{subsec:mortality-methodology}

\subsubsection{Why Study Results Differ}

The stark discrepancy between registry studies (no elevated all-cause mortality) and memorial records (mean age at death 52--56 years) reflects methodological differences:

\begin{enumerate}
    \item \textbf{Selection bias:} Memorial records capture only deaths, inherently overrepresenting severe and fatal cases. Registry studies capture all diagnosed patients regardless of outcome.

    \item \textbf{Case definition:} Broader ``chronic fatigue'' definitions (Smith et al.) vs.\ strict ME/CFS criteria (Roberts et al.) vs.\ patient-identified cases (memorial records) represent different populations.

    \item \textbf{Cohort source:} Clinical cohorts (treatment-seeking patients) vs.\ population-based registries (all diagnosed patients) vs.\ memorial nominations (deceased patients) have fundamentally different selection mechanisms.

    \item \textbf{Follow-up duration:} Longer studies (Smith: 14 years) may capture delayed mortality effects better than shorter studies (Roberts: 7 years).
\end{enumerate}

\subsubsection{Highest-Quality Evidence}

The most methodologically rigorous studies are:
\begin{itemize}
    \item \textbf{Roberts et al.\ (2016):} $n = 2{,}147$, 7-year follow-up, registry-based, appropriate comparison group~\cite{Roberts2016}
    \item \textbf{Smith et al.\ (2006):} $n = 1{,}201$, up to 14-year follow-up, clinic-based with NDI linkage~\cite{Smith2006}
\end{itemize}

Both studies found \textit{no elevation in all-cause mortality} but \textit{substantial elevation in suicide mortality}.

\subsection{Summary: Evidence-Based Conclusions}
\label{subsec:mortality-summary}

\begin{enumerate}
    \item \textbf{All-cause mortality:} No consistent evidence of elevation in large, well-designed cohort studies with appropriate comparison groups. Memorial record studies showing early death are subject to severe selection bias.

    \item \textbf{Suicide mortality:} \textit{Consistently and substantially elevated} (6--8$\times$ general population) across all study types. This represents a legitimate and well-documented mortality risk.

    \item \textbf{Cardiovascular mortality:} Conflicting evidence. Memorial records suggest elevation, but mechanism differs from atherosclerotic disease and large registry studies have not confirmed excess mortality. Requires further research.

    \item \textbf{Cancer mortality:} No evidence of elevation in population-based studies.

    \item \textbf{Life expectancy:} Cannot be reliably estimated due to methodological limitations. Best available evidence suggests normal life expectancy for all-cause mortality, with elevated suicide risk as the primary exception.

    \item \textbf{Actuarial implications:} \textit{Disability} represents greater actuarial risk than \textit{mortality} in ME/CFS. Life insurance underwriters may focus on suicide risk (well-documented) but have weak evidence for blanket mortality risk assessment.
\end{enumerate}

\begin{observation}[Suicide Prevention as Clinical Priority]
The robust evidence of elevated suicide mortality---particularly among patients without comorbid depression---highlights suicide prevention as a critical clinical priority in ME/CFS care. Risk factors include severe functional limitation, lack of social support, medical dismissal, and hopelessness about prognosis. Clinical interventions should address ME/CFS-specific suffering (energy limitations, loss of identity and purpose, medical gaslighting) rather than treating suicide risk as a purely psychiatric issue.
\end{observation}

\section{Comorbidity Studies}
\label{sec:comorbidity-studies}

% Patterns of comorbid conditions
% Temporal relationships
% Shared mechanisms
