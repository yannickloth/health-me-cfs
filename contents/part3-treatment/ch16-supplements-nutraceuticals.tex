\chapter{Supplements and Nutraceuticals}
\label{ch:supplements}

\begin{flushright}
\textit{``First, do no harm. Second, be honest about what we don't know.''}
\end{flushright}

\vspace{1em}

This chapter reviews supplements and nutraceuticals commonly used in ME/CFS. We emphasize accessible, over-the-counter options that patients can try without prescription, while being honest about the limited evidence base. Most supplements have not been rigorously tested in ME/CFS-specific trials; recommendations are based on mechanism, related-condition evidence, and patient-reported outcomes.

\begin{observation}[Evidence Limitations]
Very few supplements have been tested in randomized controlled trials specifically for ME/CFS. Most evidence comes from: (1) mechanistic plausibility based on documented ME/CFS abnormalities, (2) trials in related conditions (fibromyalgia, Long COVID, chronic fatigue), (3) small open-label studies, and (4) patient surveys. We note the evidence level for each supplement. ``Widely used'' does not mean ``proven effective.''
\end{observation}


\section{Foundational Supplements: Electrolytes and Hydration}
\label{sec:electrolytes}

For patients with autonomic dysfunction (the majority of ME/CFS patients), electrolyte and fluid management is often the single most impactful intervention.

\subsection{Why Electrolytes Matter in ME/CFS}

Autonomic dysfunction in ME/CFS frequently manifests as:
\begin{itemize}
    \item Reduced blood volume (hypovolemia)
    \item Impaired vasoconstriction
    \item Excessive venous pooling
    \item Orthostatic intolerance
\end{itemize}

Adequate sodium, potassium, and fluid intake help maintain blood volume and support cardiovascular compensation.

\subsection{Sodium}

\paragraph{Rationale.} Sodium increases blood volume by promoting water retention. POTS guidelines recommend 10--12~g of salt daily (versus 2--3~g typical intake).

\paragraph{Evidence.} Moderate for POTS; limited ME/CFS-specific data, but widely reported helpful.

\paragraph{Forms.}
\begin{itemize}
    \item \textbf{Table salt:} Cheapest; 2.3~g sodium per teaspoon
    \item \textbf{Electrolyte drinks:} LMNT, Liquid IV, Drip Drop, Nuun---convenient but expensive
    \item \textbf{Salt tablets:} Precise dosing; some find easier than drinking salty fluids
    \item \textbf{Oral rehydration salts (ORS):} WHO formula; includes glucose for sodium-glucose cotransport
\end{itemize}

\paragraph{Practical Protocol.}
\begin{enumerate}
    \item Add 1/4--1/2 teaspoon salt to each liter of water
    \item Drink before standing or activity
    \item Morning loading: 16--24~oz with salt before getting up
    \item Target: 2--3~L fluid plus 8--12~g sodium daily
\end{enumerate}

\paragraph{Cautions.}
\begin{itemize}
    \item Check with physician if hypertension, heart failure, or kidney disease
    \item Some patients with hyperadrenergic POTS may worsen with excess sodium
    \item Monitor for edema; some is expected and indicates effectiveness
\end{itemize}

\subsection{Potassium}

\paragraph{Rationale.} Intracellular potassium is essential for nerve and muscle function. Some ME/CFS patients show functional potassium deficiency even with normal serum levels.

\paragraph{Evidence.} Theoretical; no ME/CFS-specific trials.

\paragraph{Forms.}
\begin{itemize}
    \item \textbf{Potassium chloride:} Most common; Nu-Salt, Morton Lite Salt
    \item \textbf{Potassium citrate:} Better tolerated GI-wise
    \item \textbf{Coconut water:} Natural source ($\sim$600~mg per cup)
\end{itemize}

\paragraph{Dosing.} 2,000--4,700~mg daily (food + supplements); start low.

\paragraph{Cautions.} Excess potassium can cause cardiac arrhythmias. Do not exceed 99~mg per dose in supplement form without medical supervision. Those on ACE inhibitors, ARBs, or with kidney disease should be especially cautious.

\subsection{Magnesium}

\paragraph{Rationale.} Magnesium is a cofactor for $>$300 enzymes, including ATP synthesis. Deficiency is common and underdiagnosed (serum magnesium poorly reflects tissue status). Relevant to ME/CFS because:
\begin{itemize}
    \item Required for mitochondrial ATP production
    \item Modulates NMDA receptors (relevant to central sensitization)
    \item Supports autonomic function
    \item Promotes sleep (GABA-A receptor modulation)
\end{itemize}

\paragraph{Evidence.} Low--Moderate for ME/CFS; one small trial showed benefit with IM magnesium sulfate.

\paragraph{Forms.}
\begin{itemize}
    \item \textbf{Magnesium glycinate:} Well-absorbed; calming; good for sleep; less GI upset
    \item \textbf{Magnesium malate:} Malic acid may support TCA cycle; often recommended for fibromyalgia
    \item \textbf{Magnesium L-threonate:} Crosses blood-brain barrier; may help cognition; expensive
    \item \textbf{Magnesium citrate:} Well-absorbed; can cause loose stools (useful if constipated)
    \item \textbf{Magnesium oxide:} Poorly absorbed; cheap; mainly useful as laxative
    \item \textbf{Magnesium taurate:} Combined with taurine; may benefit cardiovascular system
\end{itemize}

\paragraph{Dosing.} 200--600~mg elemental magnesium daily; split doses. Start low (100--200~mg) and increase gradually. Bowel tolerance is the limiting factor for oral forms.

\paragraph{Practical Tip.} Topical magnesium (Epsom salt baths, magnesium oil) provides modest absorption and may help with muscle symptoms, though evidence is limited.

\subsection{Complete Electrolyte Formulas}

Many patients find pre-mixed electrolyte formulas convenient. Key ingredients to look for:
\begin{itemize}
    \item Sodium: 500--1000~mg per serving
    \item Potassium: 200--400~mg per serving
    \item Magnesium: 50--100~mg per serving
    \item Minimal or no sugar (some glucose aids sodium absorption; excessive sugar is counterproductive)
\end{itemize}

\paragraph{DIY Oral Rehydration Solution.}
\begin{quote}
1~L water + 1/2~tsp salt + 1/4~tsp potassium chloride (Nu-Salt) + 2~tbsp sugar or honey + optional: squeeze of citrus
\end{quote}
Cost: pennies per liter versus \$1--3 for commercial products.


\section{Mitochondrial and Energy Support}
\label{sec:mito-support}

Given the evidence for energy metabolism dysfunction in ME/CFS (Chapter~\ref{ch:energy-metabolism}), supplements supporting mitochondrial function are among the most commonly used.

\subsection{Coenzyme Q10 (CoQ10/Ubiquinone/Ubiquinol)}

\paragraph{Rationale.} CoQ10 is essential for electron transport chain function (Complex III) and is a potent lipid-soluble antioxidant.

\paragraph{Evidence.} Moderate. Multiple small studies show benefit in ME/CFS and fibromyalgia. A 2021 systematic review found CoQ10 reduced fatigue in several chronic conditions.

\paragraph{Forms.}
\begin{itemize}
    \item \textbf{Ubiquinone:} Oxidized form; must be converted to ubiquinol
    \item \textbf{Ubiquinol:} Reduced (active) form; better absorbed, especially over age 40; more expensive
\end{itemize}

\paragraph{Dosing.}
\begin{itemize}
    \item Typical: 100--300~mg daily
    \item Higher doses in studies: 400--600~mg daily
    \item Take with fat-containing meal for absorption
    \item Split doses if $>$200~mg
\end{itemize}

\paragraph{Response Timeline.} Benefits may take 4--12 weeks to manifest.

\paragraph{Cautions.} Generally well-tolerated. May reduce warfarin effectiveness. Can cause insomnia if taken late in day.

\subsection{NAD$^+$ Precursors: Nicotinamide Riboside (NR) and NMN}

\paragraph{Rationale.} NAD$^+$ is essential for mitochondrial function, DNA repair, and cellular signaling. The Heng 2025 study~\cite{heng2025mecfs} documented NAD$^+$ metabolism abnormalities in ME/CFS. NAD$^+$ cannot be directly supplemented (poor absorption), but precursors can raise levels.

\paragraph{Evidence.} Preliminary. A 2025 RCT in Long COVID showed NR 2000~mg/day increased NAD$^+$ levels 2.6--3.1 fold. Cognitive benefits were variable but some individuals showed substantial improvement after $\geq$10 weeks.

\paragraph{Forms.}
\begin{itemize}
    \item \textbf{Nicotinamide Riboside (NR):} Tru Niagen is the most studied brand
    \item \textbf{Nicotinamide Mononucleotide (NMN):} One step closer to NAD$^+$; theoretically more direct but less clinical data
    \item \textbf{Niacin (B3):} Cheapest NAD$^+$ precursor but causes flushing; extended-release reduces flushing but has liver toxicity concerns
    \item \textbf{Nicotinamide:} No flushing; may inhibit sirtuins at high doses
\end{itemize}

\paragraph{Dosing.}
\begin{itemize}
    \item NR: 300--1000~mg daily typical; research doses up to 2000~mg
    \item NMN: 250--1000~mg daily
    \item Niacin: 500--1500~mg daily (with caution)
\end{itemize}

\paragraph{Response Timeline.} May require 10+ weeks for noticeable benefit.

\paragraph{Cost Consideration.} NR/NMN are expensive (\$50--150/month at therapeutic doses). Niacin is cheap but has tolerability issues.

\subsection{D-Ribose}

\paragraph{Rationale.} D-ribose is the sugar backbone of ATP. Supplementation may accelerate ATP resynthesis after depletion.

\paragraph{Evidence.} Low--Moderate. Two small studies in ME/CFS showed benefit; widely used based on patient reports.

\paragraph{Dosing.} 5~g three times daily (15~g/day total); can be reduced to 5--10~g daily for maintenance.

\paragraph{Practical Tips.}
\begin{itemize}
    \item Take with meals (can lower blood sugar)
    \item Sweet taste; dissolves in beverages
    \item Some patients report energy improvement within days
\end{itemize}

\paragraph{Cautions.} May lower blood sugar; diabetics should monitor carefully.

\subsection{Acetyl-L-Carnitine (ALCAR) and L-Carnitine}

\paragraph{Rationale.} Carnitine transports fatty acids into mitochondria for oxidation. Deficiency impairs fat-based energy production. Acetyl-L-carnitine crosses the blood-brain barrier and may support cognitive function.

\paragraph{Evidence.} Moderate. Multiple studies show benefit in chronic fatigue and ME/CFS.

\paragraph{Forms.}
\begin{itemize}
    \item \textbf{L-Carnitine:} General mitochondrial support
    \item \textbf{Acetyl-L-Carnitine (ALCAR):} Better for cognitive symptoms; crosses BBB
    \item \textbf{Propionyl-L-Carnitine:} May be better for cardiovascular symptoms
\end{itemize}

\paragraph{Dosing.} 500--2000~mg daily; split doses.

\paragraph{Response Timeline.} 2--8 weeks.

\paragraph{Cautions.} Can increase TMAO (cardiovascular risk marker) with chronic use; some experience overstimulation or insomnia.

\subsection{Creatine}

\paragraph{Rationale.} Creatine buffers ATP, providing rapid energy during high-demand situations. Well-studied for muscle function; emerging evidence for cognitive benefits.

\paragraph{Evidence.} Theoretical for ME/CFS; strong for muscle fatigue in general populations.

\paragraph{Dosing.}
\begin{itemize}
    \item Loading (optional): 5~g four times daily for 5--7 days
    \item Maintenance: 3--5~g daily
\end{itemize}

\paragraph{Cautions.} Requires adequate hydration. May cause water retention. Concerns about kidney stress are largely unfounded in healthy individuals at normal doses.

\subsection{PQQ (Pyrroloquinoline Quinone)}

\paragraph{Rationale.} PQQ stimulates mitochondrial biogenesis (creation of new mitochondria) and has antioxidant properties.

\paragraph{Evidence.} Preliminary. Small studies suggest cognitive benefits; no ME/CFS-specific trials.

\paragraph{Dosing.} 10--20~mg daily.


\section{Antioxidant and Anti-inflammatory Supplements}
\label{sec:antioxidants}

Oxidative stress is documented in ME/CFS. Antioxidants may help, though evidence for specific supplements is limited.

\subsection{N-Acetylcysteine (NAC)}

NAC is one of the most versatile and evidence-supported supplements relevant to ME/CFS.

\paragraph{Rationale.}
\begin{itemize}
    \item \textbf{Glutathione precursor:} NAC provides cysteine, the rate-limiting amino acid for glutathione synthesis
    \item \textbf{Direct antioxidant:} Scavenges free radicals
    \item \textbf{Anti-inflammatory:} Reduces NF-$\kappa$B activation
    \item \textbf{Mucolytic:} Thins mucus (originally developed for this)
    \item \textbf{Supports liver detoxification:} Used clinically for acetaminophen overdose
    \item \textbf{May reduce viral replication:} Some evidence for various viruses
\end{itemize}

\paragraph{Evidence.} Moderate for general antioxidant/anti-inflammatory effects; preliminary for ME/CFS specifically. Widely used with generally positive patient reports.

\paragraph{Dosing.}
\begin{itemize}
    \item Typical: 600--1200~mg daily
    \item Higher doses: 1800--2400~mg daily (used in psychiatric applications)
    \item Take on empty stomach for best absorption
    \item Divide doses if $>$600~mg
\end{itemize}

\paragraph{Response Timeline.} Antioxidant effects within days; systemic benefits may take 4--8 weeks.

\paragraph{Cautions.}
\begin{itemize}
    \item Can cause GI upset; start low
    \item Sulfur smell (normal)
    \item Theoretical concern about reducing beneficial ROS signaling; probably not clinically significant at normal doses
    \item May thin mucus excessively in some (actually beneficial for most)
\end{itemize}

\paragraph{Synergy.} NAC works synergistically with:
\begin{itemize}
    \item Glycine: Another glutathione precursor
    \item Selenium: Required for glutathione peroxidase function
    \item Vitamin C: Regenerates oxidized glutathione
\end{itemize}

\subsection{Alpha-Lipoic Acid (ALA)}

\paragraph{Rationale.} ALA is both water- and fat-soluble, allowing it to work in all cellular compartments. Regenerates other antioxidants (vitamins C and E, glutathione). Supports mitochondrial function.

\paragraph{Evidence.} Moderate for diabetic neuropathy; theoretical for ME/CFS.

\paragraph{Dosing.} 300--600~mg daily; R-lipoic acid is the more bioactive form.

\paragraph{Cautions.} Can lower blood sugar. May chelate minerals (take separately from mineral supplements).

\subsection{Omega-3 Fatty Acids (EPA/DHA)}

\paragraph{Rationale.}
\begin{itemize}
    \item Anti-inflammatory (compete with omega-6 for inflammatory mediator synthesis)
    \item Support cell membrane fluidity
    \item Neuroprotective
    \item May support endothelial function (relevant to vascular hypothesis)
\end{itemize}

\paragraph{Evidence.} Moderate for general anti-inflammatory effects; limited ME/CFS-specific data.

\paragraph{Dosing.}
\begin{itemize}
    \item General health: 1--2~g combined EPA/DHA daily
    \item Anti-inflammatory: 2--4~g daily
    \item Higher EPA ratio may be more anti-inflammatory
\end{itemize}

\paragraph{Quality Matters.} Fish oil can oxidize; look for third-party tested products (IFOS certification). Triglyceride form is better absorbed than ethyl ester.

\subsection{Curcumin}

\paragraph{Rationale.} Potent anti-inflammatory; inhibits NF-$\kappa$B; antioxidant.

\paragraph{Evidence.} Strong for inflammation generally; no ME/CFS-specific trials.

\paragraph{Bioavailability Challenge.} Standard curcumin is poorly absorbed ($<$1\%). Enhanced formulations necessary:
\begin{itemize}
    \item Curcumin + piperine (black pepper extract): 20$\times$ absorption increase
    \item Phytosome forms (Meriva): Lipid-bound for better absorption
    \item Nano-curcumin, micellar curcumin: Various enhanced delivery systems
\end{itemize}

\paragraph{Dosing.} Depends on formulation; typically 500--2000~mg of enhanced curcumin daily.

\paragraph{Cautions.} May thin blood; caution with anticoagulants. Can cause GI upset. May interact with some medications.

\subsection{Quercetin}

\paragraph{Rationale.}
\begin{itemize}
    \item Mast cell stabilizer (relevant if MCAS component)
    \item Antioxidant
    \item Anti-inflammatory
    \item May have antiviral properties
\end{itemize}

\paragraph{Evidence.} Theoretical for ME/CFS; moderate for mast cell conditions.

\paragraph{Dosing.} 500--1000~mg daily; enhanced absorption forms (quercetin phytosome) preferred.

\paragraph{Cautions.} Generally well-tolerated. May interact with some antibiotics.


\section{B Vitamins}
\label{sec:vitamins}

B vitamins are essential cofactors for energy metabolism and neurological function.

\subsection{Thiamine (B1)}

\paragraph{Rationale.} Essential for pyruvate dehydrogenase (PDH)---the enzyme that feeds pyruvate into the TCA cycle. PDH dysfunction is documented in ME/CFS.

\paragraph{Evidence.} Preliminary. Case reports and small studies suggest high-dose thiamine may help a subset of ME/CFS patients. One Italian study used 600--1800~mg daily with significant benefit in chronic fatigue.

\paragraph{Forms.}
\begin{itemize}
    \item \textbf{Thiamine HCl:} Standard form; limited absorption
    \item \textbf{Benfotiamine:} Fat-soluble; better absorbed; doesn't cross BBB well
    \item \textbf{Thiamine TTFD (Allithiamine):} Lipid-soluble; crosses BBB; may be most relevant for ME/CFS
\end{itemize}

\paragraph{Dosing.} Standard: 50--100~mg. High-dose protocols: 300--1800~mg daily (under medical supervision).

\subsection{Riboflavin (B2)}

\paragraph{Rationale.} Precursor to FAD, essential for Complex II (succinate dehydrogenase) and fatty acid oxidation.

\paragraph{Evidence.} Theoretical; studied in migraine prevention (400~mg daily).

\paragraph{Dosing.} 25--400~mg daily. Harmless neon yellow urine at higher doses.

\subsection{Niacin/Niacinamide (B3)}

See NAD$^+$ precursors above.

\subsection{Pyridoxine/P5P (B6)}

\paragraph{Rationale.} Cofactor for neurotransmitter synthesis (serotonin, dopamine, GABA).

\paragraph{Dosing.} 25--100~mg daily. P5P (pyridoxal-5-phosphate) is the active form and may be better for those with conversion issues.

\paragraph{Cautions.} High doses ($>$200~mg/day chronically) can cause peripheral neuropathy.

\subsection{Folate (B9)}

\paragraph{Rationale.} Essential for methylation and DNA synthesis.

\paragraph{Forms.}
\begin{itemize}
    \item \textbf{Folic acid:} Synthetic; requires conversion; some people have MTHFR variants impairing conversion
    \item \textbf{Methylfolate (5-MTHF):} Active form; bypasses MTHFR; often preferred
    \item \textbf{Folinic acid:} Intermediate; doesn't require MTHFR
\end{itemize}

\paragraph{Dosing.} 400--1000~mcg daily; higher doses (up to 15~mg) used for specific indications.

\paragraph{Cautions.} Must be balanced with B12; folate alone can mask B12 deficiency.

\subsection{Cobalamin (B12)}

\paragraph{Rationale.} Essential for methylation, nerve function, and energy metabolism.

\paragraph{Evidence.} Low--Moderate. Some ME/CFS patients respond dramatically to B12, especially sublingual or injectable forms; others show no benefit.

\paragraph{Forms.}
\begin{itemize}
    \item \textbf{Cyanocobalamin:} Cheapest; requires conversion; contains cyanide moiety (trivial amount)
    \item \textbf{Methylcobalamin:} Active methylated form; supports methylation
    \item \textbf{Adenosylcobalamin:} Active form used in mitochondria
    \item \textbf{Hydroxocobalamin:} Well-retained; often used in injections
\end{itemize}

\paragraph{Dosing.} Oral: 1000--5000~mcg sublingual daily. Injections: 1000~mcg weekly to monthly (requires prescription in most countries).

\paragraph{Note on Testing.} Serum B12 is a poor marker of tissue status. Methylmalonic acid (MMA) and homocysteine are more sensitive.


\section{Vitamin D}
\label{sec:vitamin-d}

\paragraph{Rationale.} Vitamin D is actually a hormone with effects on:
\begin{itemize}
    \item Immune regulation (relevant to ME/CFS immune dysfunction)
    \item Muscle function
    \item Mood
    \item Bone health
\end{itemize}

Deficiency is common in ME/CFS patients (often housebound with limited sun exposure).

\paragraph{Evidence.} Moderate for general health; limited ME/CFS-specific data.

\paragraph{Target Levels.} Controversy exists:
\begin{itemize}
    \item Conventional: 30--50~ng/mL (75--125~nmol/L)
    \item Some ME/CFS practitioners target: 50--80~ng/mL
    \item Toxicity typically $>$150~ng/mL
\end{itemize}

\paragraph{Dosing.}
\begin{itemize}
    \item Maintenance: 1000--2000~IU daily
    \item Deficiency correction: 5000--10,000~IU daily for 8--12 weeks, then retest
    \item Take with fat-containing meal
\end{itemize}

\paragraph{Cofactors.} Vitamin D requires cofactors for proper function:
\begin{itemize}
    \item \textbf{Magnesium:} Required for vitamin D activation
    \item \textbf{Vitamin K2:} Directs calcium to bones (away from arteries)
    \item \textbf{Vitamin A:} Balances vitamin D effects
\end{itemize}


\section{Amino Acids}
\label{sec:amino-acids}

\subsection{Taurine}

\paragraph{Rationale.}
\begin{itemize}
    \item Mitochondrial membrane stabilization
    \item Antioxidant
    \item Supports bile acid conjugation
    \item May support cardiac and nervous system function
    \item Autonomic support
\end{itemize}

\paragraph{Evidence.} Theoretical for ME/CFS; widely used.

\paragraph{Dosing.} 500--3000~mg daily.

\subsection{Glycine}

\paragraph{Rationale.}
\begin{itemize}
    \item Glutathione precursor (with NAC)
    \item Inhibitory neurotransmitter (calming)
    \item Supports collagen synthesis
    \item May improve sleep quality
\end{itemize}

\paragraph{Dosing.} 1--3~g daily; 3~g before bed for sleep.

\subsection{L-Glutamine}

\paragraph{Rationale.}
\begin{itemize}
    \item Gut barrier support
    \item Immune cell fuel
    \item Glutathione precursor
\end{itemize}

\paragraph{Dosing.} 5--15~g daily for gut support.

\paragraph{Cautions.} Some patients with neurological sensitivity may not tolerate glutamine (converts to glutamate).


\section{Practical Supplement Protocols}
\label{sec:protocols}

Given the complexity, here are evidence-informed starting points organized by symptom cluster and budget.

\subsection{Minimal Cost Protocol (Under \$30/month)}

For patients with limited resources:

\begin{enumerate}
    \item \textbf{Electrolytes:} Salt + potassium salt (Nu-Salt) + DIY rehydration (\$5/month)
    \item \textbf{Magnesium glycinate:} 200--400~mg at bedtime (\$10/month)
    \item \textbf{B-complex:} Basic B-complex with methylated B12/folate (\$10/month)
    \item \textbf{Vitamin D3:} 2000--5000~IU daily (\$5/month)
\end{enumerate}

This addresses the most common deficiencies and supports autonomic function.

\subsection{Moderate Protocol (\$50--100/month)}

Adding mitochondrial and antioxidant support:

\begin{enumerate}
    \item Everything in minimal protocol, plus:
    \item \textbf{CoQ10 (ubiquinol):} 100--200~mg daily (\$20--30/month)
    \item \textbf{NAC:} 600--1200~mg daily (\$10--15/month)
    \item \textbf{Omega-3:} 2~g EPA/DHA daily (\$15--20/month)
    \item \textbf{D-ribose:} 5--10~g daily (\$15--20/month)
\end{enumerate}

\subsection{Comprehensive Protocol (\$100--200/month)}

For those who can afford broader support:

\begin{enumerate}
    \item Everything above, plus:
    \item \textbf{NR or NMN:} 300--500~mg daily (\$40--60/month)
    \item \textbf{Acetyl-L-carnitine:} 1000~mg daily (\$15/month)
    \item \textbf{Alpha-lipoic acid:} 300~mg daily (\$10/month)
    \item \textbf{Curcumin (enhanced):} 500~mg daily (\$15--20/month)
\end{enumerate}

\subsection{By Symptom Cluster}

\paragraph{Predominant Orthostatic Symptoms.}
\begin{itemize}
    \item Electrolytes (priority)
    \item Magnesium
    \item Taurine
    \item Licorice root (caution: raises BP)
\end{itemize}

\paragraph{Predominant Cognitive Symptoms.}
\begin{itemize}
    \item Magnesium L-threonate
    \item Acetyl-L-carnitine
    \item Omega-3 (high DHA)
    \item NR/NMN
    \item Creatine
\end{itemize}

\paragraph{Predominant Immune/Inflammatory Symptoms.}
\begin{itemize}
    \item NAC
    \item Omega-3
    \item Curcumin
    \item Quercetin (especially if mast cell component)
    \item Vitamin D (optimize)
\end{itemize}

\paragraph{Predominant Muscle/Fatigue Symptoms.}
\begin{itemize}
    \item CoQ10
    \item D-ribose
    \item L-carnitine
    \item Creatine
    \item Magnesium malate
\end{itemize}

\subsection{Introduction Strategy}

\begin{observation}[One at a Time]
ME/CFS patients often have multiple sensitivities. Introducing multiple supplements simultaneously makes it impossible to identify what helps or harms. Start one new supplement at a time, at low dose, and wait 1--2 weeks before adding another. Keep a symptom diary.
\end{observation}

\paragraph{Suggested Order.}
\begin{enumerate}
    \item Electrolytes and magnesium (foundational; rarely cause problems)
    \item B vitamins (essential cofactors)
    \item CoQ10 (well-tolerated; core mitochondrial support)
    \item NAC (antioxidant; watch for sulfur sensitivity)
    \item Additional mitochondrial support based on response
\end{enumerate}

\section{Additional Supplements}
\label{sec:additional-supplements}

\subsection{Medium-Chain Triglycerides (MCT)}

\paragraph{Rationale.} MCTs bypass normal fat digestion and are converted directly to ketones by the liver. Ketones provide alternative brain and muscle fuel, potentially bypassing impaired glucose metabolism.

\paragraph{Evidence.} Theoretical for ME/CFS; moderate for cognitive support in other conditions.

\paragraph{Dosing.} Start with 1 teaspoon and increase slowly to 1--2 tablespoons daily. Rapid introduction causes GI distress.

\subsection{Resveratrol}

\paragraph{Rationale.} Activates sirtuins and AMPK; promotes mitochondrial biogenesis; antioxidant.

\paragraph{Evidence.} Preliminary; animal studies promising but human data limited.

\paragraph{Dosing.} 150--500~mg daily; trans-resveratrol is the active form.

\subsection{Melatonin}

\paragraph{Rationale.}
\begin{itemize}
    \item Sleep initiation and circadian rhythm regulation
    \item Potent antioxidant (especially mitochondrial)
    \item Immune modulation
    \item Anti-inflammatory
\end{itemize}

\paragraph{Evidence.} Moderate for sleep; theoretical for other effects in ME/CFS.

\paragraph{Dosing.}
\begin{itemize}
    \item Sleep: 0.5--3~mg, 30--60 minutes before bed
    \item Some protocols use higher doses (5--20~mg) for antioxidant effects
    \item Extended-release forms for sleep maintenance issues
\end{itemize}

\paragraph{Note.} ``Less is more'' for sleep---higher doses can paradoxically worsen sleep quality. Start at 0.5~mg.


\section{Probiotics and Gut Health}
\label{sec:probiotics}

Given the documented gut microbiome abnormalities in ME/CFS (Chapter~\ref{ch:gut-microbiome}), gut-targeted interventions may help.

\subsection{Probiotics}

\paragraph{Rationale.} May restore beneficial bacteria, reduce gut permeability, modulate immune function, and reduce systemic inflammation.

\paragraph{Evidence.} Low--Moderate. Two small RCTs in ME/CFS showed modest benefit.

\paragraph{Strain Selection.}
\begin{itemize}
    \item \textbf{\textit{Lactobacillus} and \textit{Bifidobacterium} species:} Most studied; generally safe
    \item \textbf{\textit{Saccharomyces boulardii}:} Yeast-based; may help after antibiotics
    \item \textbf{Soil-based organisms (SBOs):} More controversial; some find helpful
\end{itemize}

\paragraph{Practical Notes.}
\begin{itemize}
    \item Start low; die-off reactions possible
    \item May take 4--8 weeks to assess
    \item Quality varies enormously between brands
    \item Refrigerated products generally more viable
\end{itemize}

\subsection{Prebiotics}

Prebiotics feed beneficial bacteria. Options include:
\begin{itemize}
    \item Partially hydrolyzed guar gum (PHGG)
    \item Resistant starch (cooked and cooled potatoes/rice)
    \item Inulin and FOS (can cause bloating)
    \item Acacia fiber (generally well-tolerated)
\end{itemize}

\paragraph{Caution.} Some ME/CFS patients have SIBO or IBS and may not tolerate prebiotics initially.


\section{Supplement Quality and Safety}
\label{sec:supplement-quality}

\subsection{Quality Considerations}

Supplements are minimally regulated. Quality varies enormously:
\begin{itemize}
    \item \textbf{Third-party testing:} Look for NSF, USP, ConsumerLab, or IFOS certification
    \item \textbf{GMP compliance:} Minimum standard; not sufficient alone
    \item \textbf{Bioavailability:} Cheap forms may not be absorbed
    \item \textbf{Contaminants:} Heavy metals, pesticides, adulterants possible
\end{itemize}

\subsection{Drug Interactions}

Common interactions to be aware of:
\begin{itemize}
    \item \textbf{Blood thinners (warfarin):} CoQ10, omega-3, vitamin E, curcumin can affect clotting
    \item \textbf{Blood pressure medications:} Potassium, magnesium, licorice can interact
    \item \textbf{Thyroid medications:} Take separately from calcium, magnesium, iron
    \item \textbf{Immunosuppressants:} Immune-modulating supplements may interfere
\end{itemize}

\subsection{When to Stop}

Discontinue a supplement if:
\begin{itemize}
    \item Clear worsening of symptoms
    \item No benefit after adequate trial (typically 8--12 weeks)
    \item Financial burden outweighs uncertain benefit
    \item Interactions with new medications
\end{itemize}

\begin{observation}[The Supplement Trap]
Many ME/CFS patients accumulate large, expensive supplement regimens over time without systematically evaluating benefit. Periodically reassess: stop everything non-essential for 2--4 weeks, then reintroduce one at a time. You may discover that many supplements you've been taking for years provide no discernible benefit.
\end{observation}


\section{Conclusion}

Supplements can play a supportive role in ME/CFS management, but expectations should be realistic:
\begin{itemize}
    \item \textbf{No supplement cures ME/CFS}
    \item \textbf{Effects are typically modest}---10--20\% symptom improvement is a good outcome
    \item \textbf{Response varies enormously} between individuals
    \item \textbf{Cost adds up}---prioritize based on evidence and your specific symptoms
    \item \textbf{Foundation first:} Electrolytes, magnesium, B vitamins, and vitamin D before exotic interventions
\end{itemize}

The supplements most likely to help, based on current evidence and mechanistic plausibility, are:
\begin{enumerate}
    \item Electrolytes (especially if orthostatic symptoms)
    \item CoQ10 (mitochondrial support)
    \item NAC (antioxidant, glutathione support)
    \item Magnesium (ubiquitous cofactor, often deficient)
    \item NAD$^+$ precursors (emerging evidence, but expensive)
\end{enumerate}

Work with a knowledgeable healthcare provider when possible, especially for higher-dose protocols or if taking multiple medications.
