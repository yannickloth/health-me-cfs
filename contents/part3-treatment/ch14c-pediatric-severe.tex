% FILE: Pediatric ME/CFS treatment — severe and housebound pediatric cases, home-based care, developmental considerations
\chapter{Pediatric ME/CFS: Severe and Housebound Cases}
\label{ch:pediatric-severe}

Children and adolescents with severe ME/CFS represent a particularly vulnerable population requiring specialized management approaches. While pediatric ME/CFS overall carries a substantially better prognosis than adult disease (54--94\% improvement or recovery versus $\leq$22\% in adults; see Chapter~\ref{ch:disease-course}), severe cases present unique challenges that demand urgent, developmentally appropriate intervention. This chapter addresses the approximately 5--10\% of pediatric ME/CFS patients who are housebound or bedbound, providing evidence-based guidance for home-based care, medical management with pediatric dosing, and strategies to preserve developmental progress during this critical period~\cite{Rowe2017pediatric, CDC2024pediatric}.

Unlike adult severe ME/CFS (Chapter~\ref{ch:urgent-action-severe}), pediatric severe disease retains meaningful potential for improvement or recovery, making appropriate early intervention particularly critical. The interventions described here are designed to reduce suffering, prevent complications of prolonged bedrest, maintain developmental trajectory, and preserve the window of opportunity for recovery.

\begin{practicalwarning}[Physician Collaboration Required]
Pediatric medication dosing, particularly for off-label uses, requires close collaboration with physicians experienced in pediatric medicine. The dosing recommendations in this chapter reflect published literature and clinical practice guidelines but must be individualized based on the child's weight, age, comorbidities, and response. The Centers for Disease Control and Prevention explicitly recommends ``extra caution when prescribing medicines for children with ME/CFS'' and ``starting medications at the smallest possible doses''~\cite{CDC2024pediatric}. All pharmacological interventions should be initiated and monitored by qualified healthcare providers.
\end{practicalwarning}

\section{Defining Severe Pediatric ME/CFS}
\label{sec:ped-severe-definition}

Severe pediatric ME/CFS is defined by functional impairment that prevents normal school attendance and requires substantial caregiver support for daily activities. Unlike adult severity metrics that emphasize work capacity, pediatric severity must be assessed through the lens of age-appropriate functioning, including school participation, extracurricular activities, and peer socialization~\cite{Rowe2017pediatric}.

\subsection{Functional Criteria}

Severe pediatric ME/CFS is characterized by:

\begin{itemize}
    \item \textbf{Housebound or bedbound status}: Unable to leave home for school or most activities; may be confined to bed for substantial portions of the day
    \item \textbf{School attendance impossibility}: Cannot attend school even with accommodations; requires homebound instruction or complete withdrawal
    \item \textbf{Dependence on caregivers}: Requires assistance with basic activities of daily living (bathing, dressing, meal preparation, mobility within home)
    \item \textbf{Severe post-exertional malaise}: Minimal activities (brief conversations, short walks within home) trigger prolonged symptom exacerbation
    \item \textbf{Multiple severe symptoms simultaneously}: Profound fatigue, cognitive dysfunction, orthostatic intolerance, pain, and sleep dysfunction occurring concurrently
\end{itemize}

Approximately 5--10\% of pediatric ME/CFS cases fall into the severe category, though this may be underestimated because severely affected children are often unable to participate in medical visits or research studies~\cite{Rowe2020severeME}.

\subsection{Distinguishing from Adult Severity Metrics}

While adult ME/CFS severity is often measured by work capacity (reduced hours, part-time work, inability to work), pediatric assessment requires different metrics:

\begin{itemize}
    \item \textbf{School attendance percentage}: Days per week able to attend, hours per day tolerated
    \item \textbf{Academic performance trajectory}: Compared to pre-illness baseline, not peers
    \item \textbf{Extracurricular participation}: Sports, clubs, social activities—often the first casualties of ME/CFS
    \item \textbf{Self-care independence}: Age-appropriate comparison (a 15-year-old requiring help with bathing represents greater impairment than a 7-year-old needing the same assistance)
    \item \textbf{Peer interaction capacity}: Ability to maintain friendships through any medium (in-person, phone, online)
\end{itemize}

A child who cannot attend school at all, requires caregiver assistance for self-care, and has minimal capacity for peer interaction meets criteria for severe pediatric ME/CFS regardless of whether they can occasionally walk short distances within the home.

\section{Home-Based Care Considerations}
\label{sec:ped-home-care}

Severely affected pediatric patients cannot travel to clinics, making home-based and telehealth care essential. Families of housebound children face unique challenges that require proactive planning.

\subsection{Telehealth Protocols}

Telehealth should be the primary mode of medical follow-up for severe pediatric cases. Effective telehealth for this population requires:

\begin{itemize}
    \item \textbf{Flexible scheduling}: Appointments during the child's best hours (which may vary unpredictably); willingness to reschedule if child is crashing
    \item \textbf{Caregiver as intermediary}: Parents may need to relay information if child cannot tolerate screen time or conversation
    \item \textbf{Brief, focused encounters}: 10--15 minute appointments with follow-up by message rather than extended video sessions
    \item \textbf{Written summaries}: Provide brief written summaries after appointments; cognitive dysfunction may prevent retention of verbal information
    \item \textbf{Symptom tracking between visits}: Use simple diaries or apps (completed by parent if needed) to capture patterns
\end{itemize}

\subsection{Home Visit Considerations}

When in-person assessment is essential (physical examination, procedures), home visits may be necessary. Considerations include:

\begin{itemize}
    \item \textbf{Advance notice}: 24--48 hours minimum so family can prepare and child can rest beforehand
    \item \textbf{Minimal stimulation}: Visitors should speak quietly, minimize movements, avoid fragrances
    \item \textbf{Examination in place}: Examine child in bed if unable to sit; avoid requiring position changes that trigger orthostatic symptoms
    \item \textbf{Brief duration}: Complete essential examination in minimum time; defer non-urgent components
    \item \textbf{Recovery time}: Expect child to need 1--3 days recovery from home visit; schedule accordingly
\end{itemize}

\subsection{Equipment Needs for Bedbound Children}

Families may need guidance on equipment to improve care of bedbound children:

\begin{itemize}
    \item \textbf{Hospital bed or adjustable bed}: Allows position changes without requiring child's effort; elevating head helps orthostatic symptoms
    \item \textbf{Bedside commode}: Reduces energy expenditure of bathroom trips
    \item \textbf{Shower chair or bath seat}: For children who can bathe but cannot stand
    \item \textbf{Wheelchair}: For medical appointments or rare outings; should be available even if rarely used
    \item \textbf{Communication aids}: Bell, monitor, or walkie-talkie to summon caregivers without shouting
    \item \textbf{Blue-light filtering}: Glasses or screen settings for any screen time; reduces sensory burden
    \item \textbf{Noise-canceling headphones}: For noise-sensitive children
\end{itemize}

\subsection{Environmental Modifications}

The home environment may require modifications to support a bedbound child:

\begin{itemize}
    \item \textbf{Light control}: Blackout curtains or eye masks if light-sensitive
    \item \textbf{Noise control}: White noise machines, weatherstripping on doors, identifying and eliminating noise sources
    \item \textbf{Temperature regulation}: Space heater or fan near bed; layered blankets for temperature fluctuations
    \item \textbf{Air quality}: HEPA filter if chemical or odor sensitivities; fragrance-free household products
    \item \textbf{Accessibility}: Move child's bedroom to main floor if bathroom access is problematic on stairs
\end{itemize}

\section{Medical Management}
\label{sec:ped-severe-medical}

Medical management of severe pediatric ME/CFS follows similar principles to adult management (Chapter~\ref{ch:urgent-action-severe}) but requires pediatric-specific dosing and heightened attention to developmental effects. The CDC recommends extra caution with pediatric ME/CFS medications, starting at lowest possible doses~\cite{CDC2024pediatric}.

\subsection{Orthostatic Intolerance (Priority)}
\label{subsec:ped-severe-oi}

Orthostatic intolerance (OI) affects 70--90\% of pediatric ME/CFS patients, compared to approximately 40--70\% of adults~\cite{Rowe2017pediatric}. This higher prevalence makes OI management the single most important intervention in pediatric ME/CFS. Treatment of orthostatic intolerance improves not only cardiovascular symptoms but also fatigue, cognitive function, and overall wellbeing~\cite{Ojha2024pediatricPOTS}.

\subsubsection{Non-Pharmacological Interventions}

Non-pharmacological measures should be implemented first and continued even if medications are added:

\begin{protocol}[Pediatric OI Non-Pharmacological Protocol]
\label{protocol:ped-oi-nonpharm}

\textbf{Hydration Targets (age-adjusted):}
\begin{itemize}
    \item Ages 4--8 years: 1.5--2 liters/day
    \item Ages 9--13 years: 2--2.5 liters/day
    \item Ages 14--18 years: 2.5--3 liters/day
    \item \textbf{Note}: These targets exceed standard pediatric recommendations because ME/CFS patients have reduced plasma volume; adequate hydration is therapeutic, not merely maintenance
    \item \textbf{Timing}: Spread throughout day; concentrated fluid intake before upright activities
    \item \textbf{Type}: Water, oral rehydration solutions, dilute juice; avoid excessive caffeine
\end{itemize}

\textbf{Salt Supplementation (pediatric dosing):}
\begin{itemize}
    \item Children $<$30 kg: 2--3 grams sodium/day (from dietary sources plus salt tablets if needed)
    \item Children 30--50 kg: 3--5 grams sodium/day
    \item Adolescents $>$50 kg: 5--8 grams sodium/day
    \item \textbf{Implementation}: Salt tablets (0.5--1 g each) with meals and snacks; salted foods; oral rehydration solutions
    \item \textbf{Monitoring}: Blood pressure should be monitored; reduce salt if sustained hypertension develops
    \item \textbf{Contraindications}: Reduce dose in renal or cardiac disease
\end{itemize}

\textbf{Compression Garments (sizing for children):}
\begin{itemize}
    \item \textbf{Waist-high compression}: Most effective; 20--30 mmHg compression
    \item \textbf{Pediatric sizing}: Measure carefully; adult sizes often too large
    \item \textbf{Alternative}: Abdominal binder if full-leg compression not tolerated
    \item \textbf{Tolerance}: Start with shorter wear times (1--2 hours); increase gradually
    \item \textbf{Application}: Put on while lying down, before rising
\end{itemize}

\textbf{Positioning Strategies:}
\begin{itemize}
    \item \textbf{Elevate head of bed}: 10--15 degrees (blocks under head of bed, not pillows) to maintain blood volume training
    \item \textbf{Avoid prolonged standing}: Sit or lie whenever possible
    \item \textbf{Lower extremity movement}: Fidgeting, ankle pumping, crossing legs when must stand
    \item \textbf{Squat when symptomatic}: If feeling faint, squat immediately (raises blood pressure)
    \item \textbf{Avoid warm environments}: Heat worsens vasodilation and OI symptoms
\end{itemize}
\end{protocol}

\subsubsection{Pharmacological Interventions}

When non-pharmacological measures are insufficient, medications may be needed. Pediatric POTS studies demonstrate high response rates to standard OI medications~\cite{Ojha2024pediatricPOTS}.

\begin{protocol}[Pediatric OI Pharmacological Protocol]
\label{protocol:ped-oi-pharm}

\textbf{First-Line: Fludrocortisone (Florinef)}

Mineralocorticoid that expands plasma volume by promoting sodium and water retention.

\textbf{Pediatric Dosing:}
\begin{itemize}
    \item \textbf{Starting dose}: 0.05 mg (50 mcg) once daily (half of the standard 0.1 mg tablet)
    \item \textbf{Titration}: Increase by 0.05 mg weekly if tolerated and symptoms persist
    \item \textbf{Target dose}: 0.1--0.2 mg daily (rarely exceeds 0.2 mg in pediatrics)
    \item \textbf{Timing}: Morning, with breakfast
    \item \textbf{Response timeline}: 1--2 weeks for initial effect; full effect may take 4--6 weeks
\end{itemize}

\textbf{Monitoring Requirements:}
\begin{itemize}
    \item \textbf{Blood pressure}: Weekly initially; must catch hypertension early
    \item \textbf{Potassium}: Baseline and at 2--4 weeks; fludrocortisone can cause hypokalemia
    \item \textbf{Weight}: Weekly; excessive weight gain suggests fluid retention beyond therapeutic
    \item \textbf{Edema}: Ankle swelling indicates dose may be too high
    \item \textbf{Growth}: Long-term corticosteroid effects on growth are theoretical but monitor height velocity
\end{itemize}

\textbf{Side Effects and Management:}
\begin{itemize}
    \item \textbf{Hypokalemia}: Supplement with potassium-rich foods or potassium chloride if levels fall
    \item \textbf{Hypertension}: Reduce dose or discontinue
    \item \textbf{Headache}: Often transient; reduce dose if persistent
    \item \textbf{Edema}: Mild ankle edema may be acceptable; reduce dose if significant
\end{itemize}

\textbf{Second-Line: Midodrine (ProAmatine)}

Alpha-1 agonist that causes vasoconstriction, increasing blood pressure and reducing venous pooling. Pediatric POTS studies report 78\% response rate~\cite{Ojha2024pediatricPOTS}.

\textbf{Pediatric Dosing:}
\begin{itemize}
    \item \textbf{Starting dose}: 2.5 mg (half tablet) once or twice daily
    \item \textbf{Titration}: Increase by 2.5 mg every 3--7 days as tolerated
    \item \textbf{Target dose}: 5--10 mg three times daily (maximum 40 mg/day in adolescents)
    \item \textbf{Timing}: Upon waking, midday, mid-afternoon; \textbf{do not give within 4 hours of bedtime} (supine hypertension risk)
    \item \textbf{Response timeline}: Hours to days; relatively rapid onset
\end{itemize}

\textbf{Critical Warnings:}
\begin{itemize}
    \item \textbf{Supine hypertension}: Most important side effect; patient must be upright when taking medication and should not lie flat for 4 hours after dose
    \item \textbf{Last dose timing}: No later than 4--6 PM to avoid nocturnal hypertension
    \item \textbf{Scalp tingling}: Common (piloerection); not dangerous but can be bothersome
    \item \textbf{Urinary retention}: Rare in pediatrics; more common in older patients
\end{itemize}

\textbf{Monitoring Requirements:}
\begin{itemize}
    \item \textbf{Supine blood pressure}: Check BP lying down periodically; if systolic $>$150 while lying, reduce dose
    \item \textbf{Standing blood pressure}: Should improve with treatment; document response
\end{itemize}

\textbf{Third-Line: Pyridostigmine (Mestinon)}

Acetylcholinesterase inhibitor that enhances autonomic function and may improve orthostatic tolerance.

\textbf{Pediatric Dosing:}
\begin{itemize}
    \item \textbf{Starting dose}: 15--30 mg twice to three times daily
    \item \textbf{Titration}: Increase gradually based on response
    \item \textbf{Target dose}: 30--60 mg three times daily
    \item \textbf{Timing}: With meals to reduce GI side effects
\end{itemize}

\textbf{Side Effects:}
\begin{itemize}
    \item \textbf{GI symptoms}: Nausea, diarrhea, abdominal cramping (most common; often dose-limiting)
    \item \textbf{Increased salivation/sweating}: Cholinergic effects
    \item \textbf{Bradycardia}: Monitor heart rate, especially in combination with beta-blockers
\end{itemize}

\textbf{Fourth-Line: IV Fluids}

For severe OI unresponsive to oral measures and medications.

\textbf{Indications:}
\begin{itemize}
    \item Syncope or near-syncope despite adequate oral hydration, salt, and medications
    \item Unable to maintain adequate oral intake due to nausea or gastroparesis
    \item Severe dehydration from illness exacerbation
    \item Acute severe crashes requiring rapid support
\end{itemize}

\textbf{Implementation:}
\begin{itemize}
    \item \textbf{Typical regimen}: Normal saline 1--2 liters, 1--3 times weekly
    \item \textbf{Access}: Peripheral IV for intermittent use; PICC line or port for frequent/regular infusions
    \item \textbf{Setting}: May be given at infusion center, at home by visiting nurse, or by trained parent
    \item \textbf{Risks}: Line infection (PICC/port), volume overload, electrolyte disturbance
\end{itemize}

\textbf{Note on IV access}: Decisions about central access (PICC, port) in pediatric ME/CFS require careful risk-benefit analysis. Line infections are serious complications. IV fluids should be reserved for patients with clearly documented benefit from fluid loading who cannot achieve equivalent benefit orally.
\end{protocol}

\subsection{Pain Management}
\label{subsec:ped-severe-pain}

Pain is common in pediatric ME/CFS, including widespread musculoskeletal pain, headaches, and abdominal pain. Pain management in developing nervous systems requires particular caution.

\subsubsection{Non-Pharmacological Approaches}

Non-pharmacological strategies should be first-line:

\begin{itemize}
    \item \textbf{Positioning}: Supportive pillows, position changes to relieve pressure
    \item \textbf{Heat/cold}: Heating pads for muscle pain; cold packs for acute inflammation or headaches
    \item \textbf{Gentle massage}: Light massage by caregiver (not deep tissue; avoid triggering PEM)
    \item \textbf{TENS units}: Transcutaneous electrical nerve stimulation for localized pain
    \item \textbf{Distraction}: Audio content (audiobooks, music, podcasts) at tolerable volumes
\end{itemize}

\subsubsection{Analgesic Medications}

\begin{protocol}[Pediatric Pain Management Protocol]
\label{protocol:ped-pain}

\textbf{Tier 1: Acetaminophen and NSAIDs}

\textbf{Acetaminophen (Tylenol):}
\begin{itemize}
    \item \textbf{Dose}: 10--15 mg/kg every 4--6 hours as needed
    \item \textbf{Maximum}: 75 mg/kg/day (up to 4 grams/day in adolescents)
    \item \textbf{Advantages}: No anti-inflammatory effects that might mask fever; low GI risk
    \item \textbf{Caution}: Hepatotoxicity at high doses; avoid in liver disease
\end{itemize}

\textbf{Ibuprofen (Advil, Motrin):}
\begin{itemize}
    \item \textbf{Dose}: 5--10 mg/kg every 6--8 hours as needed
    \item \textbf{Maximum}: 40 mg/kg/day (up to 2.4 grams/day in adolescents)
    \item \textbf{Advantages}: Anti-inflammatory effects helpful for musculoskeletal pain
    \item \textbf{Caution}: GI irritation, renal effects with chronic use; take with food
\end{itemize}

\textbf{Naproxen (Aleve):}
\begin{itemize}
    \item \textbf{Dose}: 5--7 mg/kg every 12 hours
    \item \textbf{Maximum}: 1 gram/day in adolescents
    \item \textbf{Advantages}: Longer duration; twice-daily dosing
    \item \textbf{Caution}: Same NSAID precautions as ibuprofen
\end{itemize}

\textbf{Tier 2: Neuropathic Pain Agents}

For neuropathic pain characteristics (burning, tingling, shooting, allodynia), consider agents targeting neuropathic pathways. These require specialist consultation in pediatrics.

\textbf{Gabapentin:}
\begin{itemize}
    \item \textbf{Starting dose}: 5 mg/kg/day divided into 3 doses (e.g., 100 mg TID for 60 kg adolescent)
    \item \textbf{Titration}: Increase by 5 mg/kg/day every 5--7 days
    \item \textbf{Target}: 15--35 mg/kg/day divided TID (typical adolescent dose 900--1800 mg/day)
    \item \textbf{Side effects}: Sedation, dizziness, peripheral edema
    \item \textbf{Advantages}: Generally well-tolerated; no significant drug interactions
\end{itemize}

\textbf{Amitriptyline (low-dose):}
\begin{itemize}
    \item \textbf{Starting dose}: 0.1 mg/kg at bedtime (typically 5--10 mg for child, 10--25 mg for adolescent)
    \item \textbf{Titration}: Increase by 0.1 mg/kg weekly if tolerated
    \item \textbf{Target}: 0.5--1 mg/kg at bedtime (rarely exceeds 50 mg in pediatrics)
    \item \textbf{Side effects}: Anticholinergic effects (dry mouth, constipation, urinary retention), sedation, QTc prolongation
    \item \textbf{Dual benefit}: May help pain AND sleep
\end{itemize}

\begin{practicalwarning}[Black Box Warning: Pediatric Antidepressants]
Tricyclic antidepressants (amitriptyline) and SSRIs/SNRIs carry FDA black box warnings for increased suicidal thinking and behavior in children and adolescents. Low-dose amitriptyline for pain is typically below antidepressant doses, but families should be counseled about warning signs and patients should be monitored for mood changes, particularly during dose initiation and adjustments.
\end{practicalwarning}

\textbf{Tier 3: Specialist Pain Management}

Referral to pediatric pain specialist is indicated for:
\begin{itemize}
    \item Pain refractory to Tier 1--2 interventions
    \item Pain significantly limiting function beyond ME/CFS baseline
    \item Complex regional pain syndrome (CRPS) features
    \item Need for opioid consideration (rarely appropriate in pediatric ME/CFS)
\end{itemize}
\end{protocol}

\subsection{Sleep Optimization}
\label{subsec:ped-severe-sleep}

Sleep dysfunction is nearly universal in ME/CFS. Children with ME/CFS experience unrefreshing sleep regardless of duration, along with difficulty falling asleep, maintaining sleep, or both. The severely ill child may sleep 12--16 hours yet feel exhausted.

\subsubsection{Sleep Hygiene for Bedbound Children}

Standard sleep hygiene recommendations require modification for bedbound patients who cannot leave their beds:

\begin{itemize}
    \item \textbf{Bed association}: When child is in bed 20+ hours/day, the bed becomes associated with wakefulness. Mitigation: different positions, blankets, or pillow arrangements for ``sleep time'' versus ``awake time''; different lighting (lamp on for awake, off for sleep)
    \item \textbf{Light exposure}: Attempt some natural light during ``daytime'' hours, even if from window while lying down; complete darkness for sleep periods
    \item \textbf{Screen timing}: Reduce screens 1--2 hours before intended sleep; blue-light filtering if screens are used
    \item \textbf{Consistent schedule}: Maintain regular sleep and wake times even when homebound; circadian rhythm disorders are common in ME/CFS
    \item \textbf{Temperature}: Cooler room temperatures (65--68{\textdegree}F) facilitate sleep
\end{itemize}

\subsubsection{Sleep Medications}

\begin{protocol}[Pediatric Sleep Protocol]
\label{protocol:ped-sleep}

\textbf{First-Line: Melatonin}

Melatonin is the preferred first-line sleep aid for pediatric ME/CFS due to safety profile and documented efficacy~\cite{castromarrero2021melatonin}.

\textbf{Dosing:}
\begin{itemize}
    \item \textbf{Starting dose}: 0.5--1 mg, 30--60 minutes before desired sleep time
    \item \textbf{Titration}: Increase by 0.5--1 mg every 3--7 days if ineffective
    \item \textbf{Target dose}: 1--5 mg for most children; some adolescents may need up to 10 mg
    \item \textbf{Timing is critical}: Must be taken at consistent time; efficacy depends on circadian timing, not just sedation
    \item \textbf{Extended-release formulations}: May help with sleep maintenance (waking in night)
\end{itemize}

\textbf{Important Considerations:}
\begin{itemize}
    \item \textbf{Quality matters}: Pharmaceutical-grade melatonin is preferable; supplement quality varies widely
    \item \textbf{Not purely sedating}: Melatonin works by signaling circadian timing; some children feel more alert initially before sleep onset
    \item \textbf{Long-term safety}: Generally regarded as safe for extended use in pediatrics, though long-term studies are limited
\end{itemize}

\textbf{Second-Line: Low-Dose Trazodone}

Serotonin antagonist and reuptake inhibitor with sedating properties; commonly used off-label for pediatric insomnia.

\textbf{Dosing:}
\begin{itemize}
    \item \textbf{Starting dose}: 12.5--25 mg at bedtime
    \item \textbf{Titration}: Increase by 12.5--25 mg weekly if needed
    \item \textbf{Target dose}: 25--100 mg at bedtime (rarely exceeds 100 mg for sleep in pediatrics)
\end{itemize}

\textbf{Side Effects:}
\begin{itemize}
    \item \textbf{Morning sedation}: Most common; may need dose reduction
    \item \textbf{Orthostatic hypotension}: Caution in patients with OI (may worsen or improve---varies by patient)
    \item \textbf{Priapism}: Rare but serious; educate male adolescents to seek immediate care
\end{itemize}

\textbf{Third-Line: Low-Dose Amitriptyline}

As above for pain, low-dose amitriptyline at bedtime can improve sleep quality. Dosing: 5--25 mg at bedtime. Dual benefit for patients with pain plus sleep dysfunction.

\textbf{Other Options (Specialist Consultation Recommended):}
\begin{itemize}
    \item \textbf{Clonidine}: 0.05--0.1 mg at bedtime; alpha-2 agonist with sedating effects
    \item \textbf{Hydroxyzine}: 12.5--50 mg at bedtime; antihistamine with sedating effects
    \item \textbf{Mirtazapine}: 7.5--15 mg at bedtime; antidepressant with sedating effects at low doses
\end{itemize}

\begin{practicalwarning}[Avoid in Pediatrics]
\begin{itemize}
    \item \textbf{Benzodiazepines}: Risk of dependence, cognitive effects, respiratory depression; avoid except for acute crisis management under specialist supervision
    \item \textbf{Z-drugs (zolpidem, eszopiclone)}: Not FDA-approved for pediatrics; limited safety data; complex sleep behaviors reported
\end{itemize}
\end{practicalwarning}
\end{protocol}

\subsubsection{Circadian Rhythm Disorders}

Delayed sleep phase disorder (DSPS) is common in adolescents with ME/CFS---the natural sleep-wake cycle shifts later, making it difficult to fall asleep before 2--4 AM and wake before noon. This overlaps with but is distinct from ME/CFS sleep dysfunction.

Management of DSPS:
\begin{itemize}
    \item \textbf{Chronotherapy}: Gradual advance of sleep time (15--30 minutes earlier every few days)
    \item \textbf{Light therapy}: Bright light exposure (10,000 lux light box) immediately upon waking for 20--30 minutes; helps advance circadian phase
    \item \textbf{Melatonin timing}: Low-dose melatonin (0.5--1 mg) 4--6 hours before desired sleep time (not at bedtime) can help advance circadian phase
    \item \textbf{Evening light restriction}: Minimize bright light exposure, especially blue light, in evening hours
\end{itemize}

\subsection{Cognitive Support}
\label{subsec:ped-severe-cognitive}

Cognitive dysfunction (``brain fog'') significantly impacts educational progress. In severe cases, children may be unable to read, follow conversations, or retain information.

\subsubsection{Cognitive Function Assessment}

Formal neuropsychological testing is often not feasible for severely ill children (the testing itself may trigger PEM). Instead, functional assessment through observation:

\begin{itemize}
    \item \textbf{Reading tolerance}: How long can child read or be read to before cognitive fatigue?
    \item \textbf{Conversation tolerance}: How long can child participate in conversation before losing comprehension?
    \item \textbf{Information retention}: Can child recall content from previous day? Previous hour?
    \item \textbf{Word-finding}: Does child frequently lose words or use wrong words?
    \item \textbf{Processing speed}: Is there noticeable delay between question and response?
\end{itemize}

\subsubsection{Maintaining Developmental Progress}

Even during severe illness, maintaining some cognitive engagement is important for developmental trajectory:

\begin{itemize}
    \item \textbf{Match modality to capacity}: Audiobooks if reading is impossible; dictated responses if writing is impossible
    \item \textbf{Micro-learning}: 5--10 minute educational activities with breaks, rather than extended lessons
    \item \textbf{Interest-driven}: Children may tolerate more cognitive effort for topics of personal interest
    \item \textbf{No pressure}: Removing academic pressure paradoxically often improves cognitive function
    \item \textbf{Social connection}: Even brief social interaction (texting friends, short video calls) maintains developmental skills
\end{itemize}

\subsubsection{When Cognitive Symptoms May Improve}

Cognitive function often improves with OI treatment. If orthostatic intolerance is inadequately treated, cerebral hypoperfusion (reduced blood flow to brain) contributes significantly to brain fog (see Chapter~\ref{ch:cardiovascular}, Section~\ref{sec:orthostatic-hypotension}). Many children experience noticeable cognitive improvement within 1--2 weeks of effective OI management. Always optimize OI treatment before assuming cognitive dysfunction is irreversible.

\section{Preventing Complications of Prolonged Bedrest}
\label{sec:ped-bedrest-complications}

Prolonged bedrest, while necessary for severe ME/CFS, carries its own complications. These must be proactively prevented without triggering PEM through excessive activity.

\begin{practicalwarning}[This Is NOT Graded Exercise Therapy]
The interventions in this section are passive range of motion, positioning, and deconditioning prevention---NOT graded exercise therapy (GET). GET involves progressive increases in exercise with the goal of reconditioning. The interventions here maintain baseline physical function and prevent complications while respecting the energy envelope. They should never cause PEM. If any intervention triggers symptoms, it should be reduced or discontinued.
\end{practicalwarning}

\subsection{Contracture Prevention}

Prolonged immobility can lead to joint contractures (permanent shortening of muscles and tendons). Prevention:

\begin{itemize}
    \item \textbf{Passive range of motion}: Caregiver gently moves each major joint through full range of motion daily
    \item \textbf{Duration}: 5--10 minutes total; each joint moved 5--10 times
    \item \textbf{Key joints}: Ankles (prevent foot drop), knees, hips, shoulders, elbows, wrists, fingers
    \item \textbf{Technique}: Slow, gentle movements; never force past comfortable range; stop if painful
    \item \textbf{Timing}: During lower-symptom periods; not during acute crashes
\end{itemize}

\subsection{Osteoporosis Risk}

Bedbound children and adolescents are at risk for bone loss, which is particularly concerning during growth periods.

Prevention:
\begin{itemize}
    \item \textbf{Calcium}: Ensure adequate dietary calcium or supplement (1000--1300 mg/day depending on age)
    \item \textbf{Vitamin D}: 1000--2000 IU/day; higher doses (4000 IU/day) if deficient; monitor serum 25-OH vitamin D
    \item \textbf{Weight-bearing when possible}: Standing transfers (bed to commode), even briefly, provide some bone loading
    \item \textbf{Whole body vibration}: Some evidence supports vibration platforms for bone health in immobilized populations (requires equipment; minimal energy expenditure)
\end{itemize}

\subsection{Growth Considerations}

Severe illness during adolescence can affect growth. Monitoring:

\begin{itemize}
    \item \textbf{Height and weight}: Track on growth curves at each medical contact
    \item \textbf{Nutritional status}: Ensure adequate calories and protein despite reduced appetite
    \item \textbf{Pubertal development}: Chronic illness can delay puberty; document Tanner staging
    \item \textbf{Growth velocity}: Slowing growth velocity may warrant endocrine evaluation
\end{itemize}

\subsection{Skin Integrity}

Pressure ulcers are rare in pediatric ME/CFS but can occur with prolonged immobility:

\begin{itemize}
    \item \textbf{Position changes}: Change position every 2--4 hours (can be done during wakefulness without disrupting sleep)
    \item \textbf{Pressure redistribution}: Specialized mattress (foam, alternating pressure) for prolonged bedrest
    \item \textbf{Skin inspection}: Check pressure points (heels, sacrum, scapulae) regularly
    \item \textbf{Nutrition}: Adequate protein and vitamin C support skin integrity
    \item \textbf{Moisture management}: Address incontinence promptly if present
\end{itemize}

\subsection{Cardiovascular Deconditioning}

Prolonged bedrest causes cardiovascular deconditioning (reduced stroke volume, reduced exercise capacity), which can worsen orthostatic intolerance.

Mitigation:
\begin{itemize}
    \item \textbf{Head-up tilt}: Elevating head of bed 10--15 degrees provides mild orthostatic challenge even while lying down
    \item \textbf{Reclined exercises}: If tolerated, very gentle exercises while reclined (ankle pumps, leg slides) may maintain some conditioning without triggering PEM
    \item \textbf{Gradual mobilization}: When improvement allows, very gradual increase in upright time (5 minutes sitting, building slowly over weeks)
    \item \textbf{Monitor for overexertion}: Any intervention causing symptom worsening should be reduced
\end{itemize}

\section{Educational Continuity}
\label{sec:ped-severe-education}

Loss of education is one of the most distressing aspects of severe pediatric ME/CFS. Children lose not only academic progress but also peer connections, sense of identity, and future opportunities. Maintaining educational continuity to the extent possible is important.

\subsection{Legal Rights to Education}

In the United States, severely ill children retain legal rights to education:

\begin{itemize}
    \item \textbf{Individuals with Disabilities Education Act (IDEA)}: Entitles children with disabilities to Free Appropriate Public Education (FAPE) through Individualized Education Program (IEP)
    \item \textbf{Section 504 of Rehabilitation Act}: Prohibits discrimination against students with disabilities; provides for 504 Plans with accommodations
    \item \textbf{Homebound instruction}: Most states require school districts to provide instruction at home for students unable to attend school due to illness
\end{itemize}

\subsection{Homebound Instruction Protocols}

Homebound instruction is typically provided when a child is expected to be out of school for more than 2--4 weeks (varies by state). Effective homebound instruction for ME/CFS requires:

\begin{itemize}
    \item \textbf{Physician certification}: Letter documenting medical necessity, expected duration, and any limitations on instruction
    \item \textbf{Flexible scheduling}: Instruction during child's best hours; willingness to reschedule for crashes
    \item \textbf{Reduced hours}: Standard homebound instruction (often 5--10 hours/week) may be excessive for severely ill children; request reduction as needed
    \item \textbf{Modified curriculum}: Reduced course load; prioritize core subjects
    \item \textbf{Alternative assessments}: Portfolio assessment, oral exams, or modified testing as needed
    \item \textbf{Communication with teacher}: Teacher should understand ME/CFS and not interpret illness as laziness or school avoidance
\end{itemize}

\subsection{Maintaining Academic Trajectory}

For children who cannot tolerate any formal instruction, maintaining some academic trajectory:

\begin{itemize}
    \item \textbf{Priority subjects}: Focus on reading/language arts and math as foundations
    \item \textbf{Interest-based learning}: Learning driven by child's interests may be better tolerated than required curriculum
    \item \textbf{Audiobooks}: Maintain exposure to literature and learning even when reading is impossible
    \item \textbf{Educational podcasts and videos}: Low-demand learning modalities
    \item \textbf{Credit accumulation}: Work with school to ensure any completed work counts toward credits
\end{itemize}

\subsection{Social Connection Strategies}

Social isolation compounds the impact of severe illness. Strategies to maintain peer connection:

\begin{itemize}
    \item \textbf{Low-demand communication}: Text messaging, voice messages allow asynchronous connection
    \item \textbf{Brief video calls}: 5--15 minute video calls with friends during good periods
    \item \textbf{Online communities}: Age-appropriate online groups for chronic illness (finding peers who understand)
    \item \textbf{Creative projects}: Collaborative online projects with friends (writing, gaming, art)
    \item \textbf{Card/letter exchange}: Receiving mail from friends; dictating responses if writing impossible
\end{itemize}

\subsection{Long-Term Planning}

For children with prolonged severe illness:

\begin{itemize}
    \item \textbf{Gap years}: Consider formal gap year(s) rather than falling behind; reenroll when improved
    \item \textbf{Graduation timeline flexibility}: Work with school on extended timeline for graduation requirements
    \item \textbf{GED as option}: High school equivalency may be appropriate if traditional graduation impossible
    \item \textbf{Community college start}: When ready, community college allows part-time enrollment with transfer to four-year institution later
    \item \textbf{Focus on recovery}: Academic recovery can follow physical recovery; pressure to maintain academic pace often impedes physical improvement
\end{itemize}

\section{Family Support and Caregiver Coordination}
\label{sec:ped-severe-family}

Severe pediatric ME/CFS affects the entire family unit. Parents become full-time caregivers while managing their own grief, frustration, and exhaustion. Siblings may feel neglected or resentful. Family-centered care must address these dynamics.

\subsection{Parent Education}

Parents need accurate information to provide appropriate care and advocate effectively:

\begin{itemize}
    \item \textbf{Understanding severity}: Severe ME/CFS is a serious, disabling illness; the child's limitations are real, not behavioral
    \item \textbf{Realistic expectations}: Improvement is likely in children but may take years; recovery cannot be rushed
    \item \textbf{Crash recognition}: Parents must learn to recognize early signs of PEM and enforce rest
    \item \textbf{Pacing enforcement}: Parents may need to limit activity even when child feels well (``good day'' overexertion triggers crashes)
    \item \textbf{Medical advocacy}: Parents often must educate healthcare providers about ME/CFS
\end{itemize}

\subsection{Caregiver Burden and Support}

Caregiving for a severely ill child is exhausting and isolating. Parents are at high risk for burnout, depression, and relationship stress.

Support strategies:
\begin{itemize}
    \item \textbf{Respite care}: Regular breaks from caregiving (family members, hired help, or respite programs)
    \item \textbf{Support groups}: Online or in-person groups for parents of children with ME/CFS
    \item \textbf{Mental health support}: Individual or couples therapy for parents
    \item \textbf{Practical help}: Accept help with meals, errands, household tasks
    \item \textbf{Caregiver health}: Parents must maintain their own medical care and self-care
\end{itemize}

\subsection{Sibling Considerations}

Healthy siblings of severely ill children face their own challenges:

\begin{itemize}
    \item \textbf{Reduced parental attention}: Parents' energy goes to the ill child
    \item \textbf{Household disruption}: Quiet house, restricted activities to avoid triggering sibling's symptoms
    \item \textbf{Emotional responses}: Worry about ill sibling, resentment of attention imbalance, guilt about negative feelings
    \item \textbf{Social impact}: May be reluctant to bring friends home; may feel embarrassed explaining sibling's illness
\end{itemize}

Supporting siblings:
\begin{itemize}
    \item \textbf{One-on-one time}: Ensure each healthy sibling gets individual attention from each parent
    \item \textbf{Age-appropriate explanation}: Help siblings understand the illness at their developmental level
    \item \textbf{Validate feelings}: All feelings (worry, resentment, sadness) are normal and allowed
    \item \textbf{Maintain normalcy}: Healthy siblings should continue their activities and friendships
    \item \textbf{Role limits}: Siblings should not become caregivers (brief help appropriate, but not primary care role)
\end{itemize}

\subsection{Financial Navigation}

Severe pediatric ME/CFS creates financial strain:

\begin{itemize}
    \item \textbf{Lost parental work}: One or both parents may need to reduce work or stop working to provide care
    \item \textbf{Medical expenses}: Specialty care, medications, supplements, equipment often poorly covered by insurance
    \item \textbf{Home modifications}: May require investment in equipment or home changes
    \item \textbf{Lost child income}: Adolescents lose ability to work part-time jobs
\end{itemize}

Resources:
\begin{itemize}
    \item \textbf{Family Medical Leave Act (FMLA)}: Provides job-protected leave for care of ill child (unpaid, but protects employment)
    \item \textbf{State disability programs}: Some states have family caregiver support programs
    \item \textbf{SSI for child}: Supplemental Security Income may be available for severely disabled children in low-income families
    \item \textbf{Charitable assistance}: ME/CFS organizations and general chronic illness charities may provide grants
    \item \textbf{Medical bill negotiation}: Many hospitals offer charity care or payment plans
\end{itemize}

\subsection{Mental Health Support for Family}

Depression and anxiety are common in both patients and family members. Warning signs requiring professional intervention:

\begin{itemize}
    \item \textbf{In patient}: Statements of hopelessness, suicidal ideation, complete disengagement, refusing all care
    \item \textbf{In parents}: Inability to function, severe depression, relationship breakdown, child neglect
    \item \textbf{In siblings}: Academic decline, behavioral changes, social withdrawal, expressed resentment toward ill sibling
\end{itemize}

Seek mental health support proactively rather than waiting for crisis. Family therapy addressing the systemic impact of chronic illness can benefit all members.

\section{Red Flags and Emergency Protocols}
\label{sec:ped-severe-red-flags}

Severely ill children require vigilant monitoring for complications requiring urgent intervention. The threshold for escalation should be lower in pediatrics than in adults.

\subsection{Malnutrition and Feeding Concerns}

Severe ME/CFS can impair ability to eat adequately:

\begin{itemize}
    \item \textbf{Contributing factors}: Nausea, gastroparesis, food sensitivities, energy cost of eating, cognitive difficulty with meal planning
    \item \textbf{Warning signs}: Weight loss $>$10\% body weight, BMI declining below healthy range, dehydration
\end{itemize}

Interventions:
\begin{itemize}
    \item \textbf{Calorie-dense foods}: Prioritize high-calorie, high-protein foods that require minimal eating effort
    \item \textbf{Liquid nutrition}: Ensure shakes, smoothies if solid food difficult
    \item \textbf{Small frequent meals}: 6--8 small meals/snacks rather than 3 large meals
    \item \textbf{Appetite stimulants}: Consider mirtazapine (appetite-stimulating side effect) if severely underweight
    \item \textbf{Tube feeding consideration}: If unable to maintain adequate nutrition orally and weight loss continues, nasogastric or PEG tube feeding may be necessary---this is a serious decision requiring specialist involvement
\end{itemize}

\subsection{Dehydration}

Dehydration is particularly dangerous given the importance of blood volume for orthostatic tolerance.

Warning signs:
\begin{itemize}
    \item Decreased urine output (less than 3--4 voids/day)
    \item Dark urine
    \item Dry mucous membranes
    \item Worsening orthostatic symptoms beyond baseline
    \item Tachycardia at rest
\end{itemize}

Intervention:
\begin{itemize}
    \item Increase oral fluids if tolerated
    \item Oral rehydration solutions
    \item IV fluids if oral rehydration inadequate or not tolerated
\end{itemize}

\subsection{Severe Orthostatic Symptoms}

OI symptoms requiring urgent evaluation:
\begin{itemize}
    \item Syncope (fainting)
    \item Near-syncope with falls or injuries
    \item Chest pain with activity
    \item Palpitations suggestive of arrhythmia (irregular, rapid, or pounding)
\end{itemize}

These may indicate inadequately treated OI, cardiac arrhythmia, or other cardiovascular pathology requiring workup.

\subsection{Mental Health Crisis}

Adolescents with severe ME/CFS are at increased risk for depression and suicidal ideation due to:
\begin{itemize}
    \item Loss of functional life, identity, and social connections
    \item Chronic pain and suffering
    \item Hopelessness about recovery
    \item Feeling disbelieved by healthcare providers
\end{itemize}

\begin{practicalwarning}[Suicide Risk Assessment]
Take ALL expressions of suicidal thoughts seriously. Warning signs:
\begin{itemize}
    \item Statements about wanting to die, being a burden, having no future
    \item Giving away possessions
    \item Withdrawal from remaining connections
    \item Sudden calm after period of distress (may indicate decision made)
    \item Seeking means (medications, access to weapons)
\end{itemize}

If suicidal ideation is present:
\begin{itemize}
    \item Do not leave child alone
    \item Remove access to means (medications, sharp objects)
    \item Contact mental health crisis services or emergency department
    \item National Suicide Prevention Lifeline (US): 988
    \item Crisis Text Line: Text HOME to 741741
\end{itemize}

Mental health care for suicidal adolescents with ME/CFS requires providers who understand the illness. Psychiatric hospitalization, with its stimulation and forced activity, can worsen ME/CFS and should be avoided if possible through intensive outpatient support.
\end{practicalwarning}

\subsection{Distinguishing Severity from Concurrent Illness}

Severely ill ME/CFS patients may have difficulty recognizing new acute illness (infection, appendicitis, other medical emergency) because chronic symptoms mask new symptoms.

Concerning changes from baseline:
\begin{itemize}
    \item New fever (ME/CFS typically does not cause fever)
    \item New localized pain (abdominal, chest, flank)
    \item Acute worsening of specific symptoms beyond usual fluctuation
    \item New neurological symptoms (weakness, numbness, vision changes)
\end{itemize}

Maintain low threshold for medical evaluation of new symptoms. Do not assume all symptoms are ME/CFS.

\section{Evidence That Severe Pediatric Disease CAN Reverse}
\label{sec:ped-severe-prognosis}

Despite the severity of current illness, there is strong evidence that severe pediatric ME/CFS can reverse---a critical source of hope for families.

\subsection{Prognosis Data}

Long-term follow-up studies demonstrate substantially better outcomes in pediatric ME/CFS than adult disease~\cite{Rowe2019pediatric}:

\begin{itemize}
    \item \textbf{Recovery at 5 years}: 38\%
    \item \textbf{Recovery at 10 years}: 68\%
    \item \textbf{Overall improvement or recovery}: 54--94\% across studies
    \item \textbf{Mean illness duration}: 5 years (range 1--15 years)
    \item \textbf{Functional status at 10-year follow-up}: Mean 8/10
    \item \textbf{Proportion still very unwell}: Only 5\% with function $<$6/10 at 10 years
    \item \textbf{Working or studying full-time}: 63\% at follow-up
\end{itemize}

These figures include children who were severely affected. The data demonstrate that even severe pediatric ME/CFS is not necessarily permanent.

\subsection{Time to Improvement}

Realistic timelines for improvement:
\begin{itemize}
    \item \textbf{Months 1--6}: Stabilization with appropriate treatment; symptom reduction possible but functional improvement limited
    \item \textbf{Months 6--18}: Gradual improvement in many; may transition from severe to moderate
    \item \textbf{Years 2--5}: Continued slow improvement; many achieve significant recovery
    \item \textbf{Years 5--10}: Most who will recover have recovered by this point
\end{itemize}

Improvement is typically slow and non-linear. Good weeks are followed by setbacks. The trajectory is overall positive even if daily experience fluctuates.

\subsection{Factors Associated with Better Outcomes}

Research suggests better outcomes are associated with:
\begin{itemize}
    \item \textbf{Shorter diagnostic delay}: Earlier diagnosis and appropriate management
    \item \textbf{Younger age at onset}: Pre-adolescent onset may have slightly better prognosis
    \item \textbf{Absence of comorbidities}: Children without additional chronic conditions fare better
    \item \textbf{Adequate rest and accommodations}: Avoiding repeated severe crashes
    \item \textbf{Family support}: Stable, supportive family environment
\end{itemize}

\subsection{Contrast with Adult Prognosis}

Adult ME/CFS has dramatically worse outcomes (see Chapter~\ref{ch:disease-course}):
\begin{itemize}
    \item Recovery: Only 5\% (median across studies)
    \item Improvement: $\leq$22\%
    \item Most adults with ME/CFS have lifelong illness
\end{itemize}

The stark difference between pediatric and adult outcomes suggests that biological factors related to developmental plasticity, or barriers to recovery present in adults but absent in children (continued work demands, financial pressures), may significantly influence trajectory.

\subsection{Hope Maintenance with Realistic Expectations}

Families need both hope and realism:

\begin{itemize}
    \item \textbf{Hope}: The majority of children with ME/CFS, including severe cases, improve significantly or recover. This is well-documented.
    \item \textbf{Realism}: Improvement takes years, not weeks. Recovery cannot be rushed. Some children do not fully recover and require ongoing accommodations into adulthood.
    \item \textbf{Focus}: The goal during severe illness is symptom management, preventing complications, and maintaining developmental trajectory---not forcing recovery
    \item \textbf{Success metrics}: Recovery is not the only success. A child who improves from bedbound to attending school part-time has had an excellent outcome, even if not fully ``recovered.''
\end{itemize}

\subsection{Critical Window for Early Intervention}

The better prognosis in pediatric ME/CFS suggests a critical intervention window that may close as patients age. This underscores the urgency of appropriate treatment in severe pediatric cases:

\begin{itemize}
    \item Do not delay treatment waiting for recovery
    \item Aggressive symptom management (OI treatment, sleep optimization) may facilitate natural recovery processes
    \item Avoiding severe crashes (strict pacing) may preserve recovery potential
    \item The developing nervous system and immune system may have plasticity that allows recovery if not damaged by repeated overexertion
\end{itemize}

\begin{keypoint}[The Pediatric Advantage]
Children with ME/CFS have a window of opportunity for recovery that appears to narrow with age and illness duration. Appropriate early intervention---aggressive symptom management, strict pacing, accommodations, and avoidance of harmful treatments like GET---may maximize the likelihood of utilizing this window. Every year matters. Every severe crash may reduce recovery potential. The urgency of correct management cannot be overstated.
\end{keypoint}

\section{Summary of Key Recommendations}
\label{sec:ped-severe-summary}

\begin{enumerate}
    \item \textbf{Prioritize OI treatment}: Orthostatic intolerance affects 70--90\% of pediatric ME/CFS patients and is often the most treatable symptom. Start with non-pharmacological measures (hydration, salt, compression, positioning) and add medications (fludrocortisone, midodrine) if needed.

    \item \textbf{Optimize sleep}: Melatonin is first-line for pediatric sleep dysfunction. Low-dose trazodone or amitriptyline are second-line options. Maintain consistent sleep-wake schedule even when homebound.

    \item \textbf{Manage pain appropriately}: Use acetaminophen and NSAIDs first-line. Add gabapentin or low-dose amitriptyline for neuropathic pain. Avoid routine opioid use. Referral to pediatric pain specialist for refractory cases.

    \item \textbf{Prevent bedrest complications}: Daily passive range of motion prevents contractures. Ensure adequate calcium, vitamin D, and occasional weight-bearing to prevent bone loss. Monitor growth.

    \item \textbf{Maintain educational continuity}: Secure homebound instruction through school district. Reduce academic demands to match capacity. Focus on core subjects and interest-based learning. Plan flexibly for graduation timeline.

    \item \textbf{Support the whole family}: Parents need respite and support. Siblings need individual attention and validation. Financial and mental health resources should be mobilized proactively.

    \item \textbf{Monitor for emergencies}: Maintain low threshold for evaluating new symptoms. Watch for malnutrition, dehydration, severe OI complications, and mental health crises.

    \item \textbf{Maintain hope}: Pediatric ME/CFS prognosis is substantially better than adult disease. With appropriate management, most children improve significantly or recover over time. Recovery cannot be rushed, but it is achievable.
\end{enumerate}

\begin{continuation}[Related Content]
For ambulatory pediatric patients who are still attending school, see Chapter~\ref{ch:pediatric-ambulatory} for school accommodation frameworks, IEP/504 planning, and pacing strategies for school-attending children. For adult severe ME/CFS management, see Chapter~\ref{ch:urgent-action-severe}. For detailed discussion of pathophysiology underlying these interventions, see Chapter~\ref{ch:cardiovascular} (orthostatic mechanisms), Chapter~\ref{ch:energy-metabolism} (metabolic dysfunction), and Chapter~\ref{ch:immune-dysfunction} (immune abnormalities).
\end{continuation}
