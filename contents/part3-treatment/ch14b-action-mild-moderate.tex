% FILE: Mild-moderate management — initial approach, step-wise escalation, foundational interventions, standard protocols
\chapter{Action Plans for Mild to Moderate Cases}
\label{ch:action-mild-moderate}

This chapter addresses patients with mild to moderate ME/CFS who retain some functional capacity but experience significant symptom burden that impairs quality of life. The goal is to maximize function, prevent progression to severe disease, and pursue recovery.

\section{Defining Mild to Moderate ME/CFS}
\label{sec:defining-mild-moderate}

\subsection{Functional Categories}

\begin{description}
    \item[Mild ME/CFS] Mobile, can care for self, able to work/study (often reduced hours or difficulty maintaining), symptoms significantly impact quality of life but not completely disabling. May appear healthy to outsiders. Represents approximately 50\% of ME/CFS patients.

    \item[Moderate ME/CFS] Reduced mobility, restricted in activities of daily living, usually unable to work/study full-time, requires frequent rest periods, homebound 2--4 days per week. Represents approximately 25\% of ME/CFS patients.
\end{description}

\subsection{Why Action is Urgent for Non-Severe Cases}

\begin{enumerate}
    \item \textbf{Prevention of progression}: 25\% of ME/CFS patients are severe/very severe. Many started as mild-moderate and progressed due to continued overexertion.
    \item \textbf{Window of opportunity}: Earlier intervention may prevent immune exhaustion phase (Section~\ref{ach:cytokine-duration}).
    \item \textbf{Quality of life}: Even mild ME/CFS significantly impairs function and well-being; deserves treatment.
    \item \textbf{Biomarker evidence}: Cytokine dysregulation, immune abnormalities present even in mild cases.
\end{enumerate}

\section{Immediate Action Plan (Mild-Moderate Cases)}
\label{sec:immediate-mild-moderate}

\subsection{Core Principles}

\begin{enumerate}
    \item \textbf{Prevent progression}: Primary goal is to avoid worsening to severe ME/CFS
    \item \textbf{Optimize function}: Maximize sustainable activity within energy envelope
    \item \textbf{Symptom control}: Address limiting symptoms to improve quality of life
    \item \textbf{Root causes}: Pursue disease-modifying treatments early, before exhaustion phase
\end{enumerate}

\subsection{Foundation: Energy Envelope Management}
\label{sec:energy-envelope}

\paragraph{Critical Importance}

Pacing is \emph{more important} for mild-moderate cases than for severe cases, paradoxically. Severe patients are forced to rest by their symptoms. Mild-moderate patients can push through, leading to progressive worsening and eventual severity. The post-exertional malaise mechanism (Section~\ref{sec:energy-consequences}) documents that repeated energy envelope violations cause cumulative mitochondrial damage and progressive decline.

\paragraph{The Energy Envelope Concept}

\begin{itemize}
    \item \textbf{Available energy}: Fixed daily energy budget (lower than healthy individuals)
    \item \textbf{Energy expenditure}: All activities (physical, cognitive, emotional) cost energy
    \item \textbf{Energy envelope}: Staying within available energy prevents PEM and progression
    \item \textbf{Exceeding envelope}: Triggers PEM, depletes reserves, leads to progressive decline
\end{itemize}

\paragraph{Quantifying Your Envelope}

\begin{enumerate}
    \item \textbf{Activity tracking} (2-week baseline):
    \begin{itemize}
        \item Record all activities with duration and intensity
        \item Rate symptoms at end of each day (0--10 scale)
        \item Note PEM episodes (typically 24--72 hours post-exertion)
        \item Identify threshold: Maximum activity level that does NOT trigger PEM
    \end{itemize}

    \item \textbf{Heart rate monitoring}:
    \begin{itemize}
        \item Wear continuous HR monitor
        \item Calculate anaerobic threshold (AT): $(220 - \text{age}) \times 0.60$ for mild cases
        \item Optimal: Get CPET to measure actual AT
        \item Stay below AT for all activities
    \end{itemize}

    \item \textbf{Symptom-based pacing}:
    \begin{itemize}
        \item Stop activity BEFORE symptoms worsen
        \item If mild increase in fatigue/pain/brain fog → rest immediately
        \item Do not ``push through''—this depletes reserves
    \end{itemize}
\end{enumerate}

\paragraph{Conservative Baseline Establishment During Interventions}

\begin{warning}[Graded Exercise Therapy is Harmful]
\label{warn:get-harmful}
Graded exercise therapy (GET) has been heavily criticized for causing patient deterioration and is no longer recommended by major health organizations. The PACE trial, which originally promoted GET for ME/CFS, was subsequently discredited following reanalysis revealing unscientific methodology. Multiple patient reports document severe crashes, prolonged recovery periods, and permanent functional decline following GET protocols. Exercise "pushing through" symptoms violates the fundamental principle of energy envelope management and can trigger the post-exertional malaise mechanism. One elite athlete reported that "trying to run through mild symptoms" resulted in 5-month long COVID episode with 1--2 month recovery time. The "crash limit rule" from patient communities suggests individuals should not experience more than 5 total severe crashes, as recovery time increases by months with each subsequent crash, potentially leading to irreversible worsening.
\end{warning}

\begin{warning}[Do Not Test PEM During Early Intervention Phase]
When starting new interventions (electrolytes, supplements, medications), resist the urge to ``test'' whether you can now do more activity. Initial improvements may reflect temporary metabolic support rather than restored capacity.

\textbf{Critical principles:}
\begin{itemize}
    \item \textbf{Establish baseline stability first}: Minimum 2--4 weeks of consistent symptom improvement before considering activity increase
    \item \textbf{PEM can occur without identifiable trigger}: Even ``normal'' daily activities (childcare, sitting at computer) may trigger crashes when operating near threshold
    \item \textbf{Afternoon crash patterns persist}: Metabolic improvements may reduce crash severity but vulnerability windows remain
    \item \textbf{Joint pain as inflammatory marker}: Severe joint pain during crashes indicates cytokine/inflammatory component; pain resolution with magnesium does not eliminate crash risk
\end{itemize}

\textbf{Why this matters:}
\begin{itemize}
    \item Electrolyte/supplement improvements address \emph{symptoms} and metabolic bottlenecks
    \item Underlying PEM mechanism (Section~\ref{sec:energy-consequences}) remains active
    \item Testing limits during early intervention phase can trigger severe crashes that erase weeks of progress
    \item Example: Patient improving on day 3 of electrolyte protocol wisely stated \emph{``PEM: not tested yet, I don't dare''} --- this caution prevented potential severe relapse
\end{itemize}

\textbf{Appropriate timeline for activity testing:}
\begin{enumerate}
    \item \textbf{Weeks 1--4}: Establish intervention (electrolytes, supplements, medications); maintain current activity level
    \item \textbf{Weeks 4--8}: If stable improvement sustained, very gradually test small increases (5--10\% activity increase)
    \item \textbf{Months 2--3}: If no PEM episodes, consider slightly larger envelope expansion
    \item \textbf{Always}: If any PEM episode occurs, immediately return to prior safe activity level
\end{enumerate}
\end{warning}

\paragraph{50\% Rule for Mild-Moderate Cases}

\begin{itemize}
    \item \textbf{Conservative estimate}: Do 50\% of what you think you can do
    \item Example: If you feel you can walk 30 minutes, walk 15 minutes
    \item Example: If you feel you can work 8 hours, work 4 hours
    \item \textbf{Rationale}: Most patients overestimate capacity; 50\% rule provides safety margin
    \item \textbf{Adjustment}: If no PEM after 2 weeks at 50\%, increase to 60\%; iterate until you find sustainable level
\end{itemize}

\paragraph{Preventing Boom-Bust Cycles}

\begin{itemize}
    \item \textbf{Boom phase}: Feel better → do too much → crash
    \item \textbf{Bust phase}: Severe PEM → bedbound → recover slowly → repeat
    \item \textbf{Solution}: Consistent daily activity within envelope, even on ``good days''
    \item \textbf{Good days}: Do NOT increase activity; bank energy for inevitable bad days
\end{itemize}

\subsection{Symptom Management for Mild-Moderate Cases}
\label{sec:symptom-management-mild-moderate}

\subsubsection{Cognitive Dysfunction (Brain Fog)}

\paragraph{Rationale}
Cognitive dysfunction results from multiple mechanisms: catecholamine deficiency (Section~\ref{sec:catecholamine-metabolism}), cerebral hypoperfusion (Section~\ref{sec:cerebral-blood-flow}), and reduced ATP availability in the brain (Section~\ref{sec:energy-overview}). Targeting neurotransmitter precursors and optimizing cerebral blood flow can improve function.

\paragraph{Non-Pharmaceutical}
\begin{itemize}
    \item \textbf{Cognitive pacing}:
    \begin{itemize}
        \item Work in 25-minute blocks (Pomodoro technique), then 10-minute rest
        \item Schedule cognitively demanding tasks for peak energy times (usually morning)
        \item Minimize multitasking (switching costs energy)
        \item Reduce decision-making load (meal planning, outfit planning in advance)
    \end{itemize}

    \item \textbf{Environmental optimization}:
    \begin{itemize}
        \item Reduce sensory overload (quiet workspace, minimal visual clutter)
        \item Close unnecessary browser tabs/apps
        \item Use noise-canceling headphones if sound-sensitive
    \end{itemize}
\end{itemize}

\paragraph{Pharmaceutical/Supplement}
\begin{itemize}
    \item \textbf{Tier 1} (try first):
    \begin{itemize}
        \item Caffeine + L-theanine (100 mg + 200 mg, 1--2 times daily)
        \item Alpha-GPC 300 mg BID (choline support for acetylcholine)
        \item Rhodiola rosea 200--400 mg morning (adaptogen, focus)
    \end{itemize}

    \item \textbf{Tier 2} (add if Tier 1 helps):
    \begin{itemize}
        \item Bacopa monnieri 300 mg daily (memory consolidation)
        \item Lion's Mane mushroom 500--1000 mg BID (nerve growth factor)
        \item Citicoline 250 mg BID (neuroprotection)
    \end{itemize}

    \item \textbf{Tier 3} (prescription if severe cognitive impairment):
    \begin{itemize}
        \item Modafinil 50--100 mg morning (wakefulness, often prescribed off-label)
        \item Or: Methylphenidate 5 mg BID (stimulant, use cautiously)
    \end{itemize}
\end{itemize}

\subsubsection{Sleep Dysfunction}

\paragraph{Rationale}
Non-restorative sleep is a core ME/CFS symptom (Section~\ref{sec:sleep}). Sleep dysfunction amplifies all other symptoms through effects on immune function (Section~\ref{sec:chronic-activation}), pain sensitization, and cognitive impairment. Optimizing sleep is foundational to symptom control.

\paragraph{Sleep Hygiene (Non-Negotiable Foundation)}
\begin{itemize}
    \item Same sleep/wake time every day (weekends included)
    \item 7--9 hour sleep opportunity (in bed, dark, quiet)
    \item Room: 65--68°F, completely dark, quiet
    \item No screens 2 hours before bed (or blue blockers)
    \item No caffeine after 2pm
    \item No large meals 3 hours before bed
    \item Wind-down routine: 30 minutes relaxing activity before bed (reading, gentle stretching, meditation)
\end{itemize}

\paragraph{Supplements (Mild Cases Can Start Here)}
\begin{itemize}
    \item Melatonin 0.5--3 mg (2 hours before target sleep time; start low)
    \item \textbf{Magnesium glycinate 400 mg evening} - NOTE: At upper end of RDA (320 mg women, 420 mg men). Provides 400 mg elemental magnesium for muscle relaxation and calming. Very safe, well-tolerated. May cause loose stools if exceed tolerance (reduce dose if occurs).
    \item L-theanine 200 mg before bed (anxiolytic)
    \item \textbf{Glycine 3 g before bed} - NOTE: Exceeds typical supplement dose (1--2 g) by 1.5--3$\times$. Clinical studies for sleep quality improvement use 3 g. Mechanism: Glycine acts as inhibitory neurotransmitter, lowers core body temperature promoting sleep onset. Extremely safe (used as food additive), no upper limit established. Sweet taste can be mixed in water.
\end{itemize}

\paragraph{Prescription (If Supplements Insufficient)}
\begin{itemize}
    \item Trazodone 25--50 mg (lower dose than severe cases; increase if needed)
    \item Mirtazapine 7.5 mg (also helps appetite)
    \item Doxepin 3--6 mg (low-dose, histamine antagonist, improves sleep maintenance)
\end{itemize}

\subsubsection{Pain}

\paragraph{Rationale}
Pain in ME/CFS involves inflammatory mediators (Section~\ref{sec:pro-inflammatory}), small fiber neuropathy (Section~\ref{sec:sfn}), and central sensitization. Addressing inflammation and neuropathic pathways reduces pain burden.

\paragraph{Mild-Moderate Pain Management}
\begin{itemize}
    \item \textbf{First-line}:
    \begin{itemize}
        \item Ibuprofen 400 mg PRN or BID (with food)
        \item Or: Naproxen 220--500 mg BID
        \item Topical: Diclofenac gel (Voltaren) to painful areas
    \end{itemize}

    \item \textbf{Add if insufficient}:
    \begin{itemize}
        \item Low-dose naltrexone (LDN) 1.5--4.5 mg nightly (immune modulation + pain)
        \item Turmeric/curcumin 500--1000 mg BID (natural anti-inflammatory)
        \item Magnesium glycinate 400 mg daily (muscle relaxation)
    \end{itemize}

    \item \textbf{Neuropathic pain component}:
    \begin{itemize}
        \item Gabapentin 100 mg at bedtime, increase slowly to 300--600 mg BID if needed
        \item Or: Duloxetine 30--60 mg daily (also helps mood)
    \end{itemize}
\end{itemize}

\subsubsection{Orthostatic Intolerance (POTS)}

\paragraph{Rationale}
Orthostatic intolerance affects 70--90\% of ME/CFS patients (Section~\ref{sec:orthostatic-mechanisms}). Reduced blood volume (Section~\ref{sec:blood-volume}), autonomic dysfunction (Section~\ref{sec:ans-pathophysiology}), and impaired vascular regulation contribute. Blood volume expansion and compression improve tolerance.

\paragraph{Mild-Moderate Interventions}
\begin{itemize}
    \item \textbf{Compression}: Waist-high stockings 20--30 mmHg (lower compression than severe cases)

    \item \textbf{Salt: 6--8 g sodium daily} - \textbf{NOTE - DRAMATICALLY EXCEEDS STANDARD RECOMMENDATION}: Standard guideline is $<$2300 mg (2.3 g) daily. We recommend 6000--8000 mg (6--8 g) sodium daily, which is 2.6--3.5$\times$ standard. See Chapter~\ref{ch:urgent-action-severe} for complete justification (blood volume expansion for orthostatic intolerance, standard POTS treatment). Electrolyte drinks make compliance easier. CONTRAINDICATIONS: Hypertension, heart failure, kidney disease. Monitor BP weekly.

    \item \textbf{Oral rehydration solution (ORS) - dual benefit}: Beyond simple blood volume expansion, properly formulated electrolyte solutions address the chronic metabolic stress state documented in Section~\ref{sec:catecholamine-metabolism}. ME/CFS patients exist in a continuous state of lactate accumulation and reliance on anaerobic metabolism similar to post-exercise metabolic stress in athletes (see Chapter~\ref{ch:energy-metabolism}). Strategic electrolyte replacement serves multiple purposes:
    \begin{itemize}
        \item \textbf{Blood volume expansion}: Maintains preload for cardiac output; reduces orthostatic intolerance
        \item \textbf{Lactate clearance}: Helps clear accumulated lactic acid from impaired oxidative metabolism
        \item \textbf{Glucose availability}: Provides immediate energy when fat-burning is impaired
        \item \textbf{Electrolyte balance}: Supports muscle function and reduces cramping from ATP depletion
    \end{itemize}

    \textbf{Recommended formulation} (sports medicine-derived):
    \begin{itemize}
        \item Dry mix: 100 g sugar + 15 g low-sodium salt (KCl) + 15 g table salt (NaCl)
        \item Dosing: 7 g dry mix in 250 mL water, twice daily
        \item Flavoring optional (e.g., 10 mL grenadine for palatability)
        \item Cost: $<$\euro{}5 for months of supply
    \end{itemize}

    This formulation provides sodium, potassium, chloride, and glucose in ratios optimized for absorption and metabolic support. See Appendix~\ref{subsubsubsec:sports-medicine-parallel} for the clinical insight that led to this protocol development.

    \item \textbf{Fluids}: 2.5--3 L daily
    \item \textbf{Positional changes}: Rise slowly (sit 30 seconds before standing)
    \item \textbf{Counter-maneuvers}: Leg crossing, muscle tensing when standing
    \item \textbf{Exercise}: Recumbent bike or rowing (horizontal position) within energy envelope
\end{itemize}

\paragraph{Prescription (If Above Insufficient)}
\begin{itemize}
    \item Fludrocortisone 0.05--0.1 mg daily (increases blood volume)
    \item Midodrine 2.5--10 mg TID (peripheral vasoconstrictor)
    \item Beta-blockers (propranolol, metoprolol) - use cautiously, can worsen fatigue in some
\end{itemize}

\subsection{Mast Cell Activation Syndrome (MCAS) Management}
\label{sec:mcas-mild-moderate}

\subsubsection{Evidence and Rationale}

Mast cell activation affects 30--50\% of ME/CFS patients~\cite{Wirth2023}. Recent research demonstrates measurable mast cell phenotype abnormalities with significant increases in naïve mast cells and elevated activation markers~\cite{Hardcastle2016}. MCAS may worsen orthostatic intolerance, brain fog, and fatigue through excessive histamine and vasoactive mediator release~\cite{Wirth2023}.

\textbf{Critical finding}: H1 antihistamine alone showed NO benefit in double-blind RCT~\cite{Steinberg1996}. However, \textbf{H1+H2 combination} showed dramatic improvement in Long COVID case meeting ME/CFS criteria, with symptom worsening upon discontinuation~\cite{Davis2023}.

\subsubsection{Trial Indications}

Consider MCAS trial if ANY present:
\begin{itemize}
    \item Food sensitivities/intolerances (especially new-onset)
    \item Documented allergies (elevated IgE to foods, pollens, environmental allergens)
    \item Flushing, hives, itching
    \item Reactive to fragrances, chemicals
    \item GI symptoms (post-meal nausea, bloating)
    \item Unexplained anxiety/panic-like episodes
    \item Fluctuating brain fog (worse after eating or exposure to triggers)
\end{itemize}

\subsubsection{Treatment Options (Evidence-Based Hierarchy)}

\paragraph{Option 1: Standard H1+H2 Combination}
Based on Long COVID case evidence~\cite{Davis2023}:
\begin{itemize}
    \item \textbf{H1}: Loratadine 10 mg OR fexofenadine 180 mg (morning)
    \item \textbf{H2}: Famotidine 20 mg twice daily
    \item \textbf{Expected benefits}: Energy, cognitive function, orthostatic tolerance
\end{itemize}

\paragraph{Option 2: Rupatadine (Superior H1 Choice)}
\textbf{Rupatadine offers unique advantages}~\cite{Pinero-Gonzalez2024,Mullol2008}:
\begin{itemize}
    \item \textbf{Triple mechanism}: H1 antagonist + PAF antagonist + mast cell stabilizer
    \item \textbf{Superior efficacy}: Network meta-analysis ranks rupatadine 20 mg highest (SUCRA 99.7\%) vs loratadine (lowest rank)~\cite{Mullol2008}
    \item \textbf{PAF antagonism}: 31$\times$ more potent than loratadine at blocking PAF; addresses vascular dysfunction in ME/CFS~\cite{Pinero-Gonzalez2024}
    \item \textbf{Mast cell stabilization}: Inhibits IL-8 (80\%), VEGF (73\%), histamine (88\%)~\cite{Pinero-Gonzalez2024}
\end{itemize}

\textbf{Recommended protocol}:
\begin{itemize}
    \item Rupatadine 10 mg morning (increase to 20 mg after 1--2 weeks if insufficient benefit)
    \item Add famotidine 20 mg BID for complete histamine receptor coverage
    \item Optional: Add quercetin 500--1000 mg daily (see below)
\end{itemize}

\paragraph{Option 3: Quercetin (Natural Mast Cell Stabilizer)}
Evidence shows quercetin MORE effective than prescription cromolyn~\cite{Theoharides2012}:
\begin{itemize}
    \item \textbf{Dose}: 500--1000 mg daily (clinical trials used up to 2 g/day)
    \item \textbf{Evidence}: Reduced contact dermatitis reactions $>$50\% in 8 of 10 patients; outperformed cromolyn for substance P-induced mast cell activation~\cite{Theoharides2012}
    \item \textbf{Advantages}: Over-the-counter, well-tolerated, additional antioxidant benefits
    \item Can combine with H1+H2 antihistamines for comprehensive mast cell targeting
\end{itemize}

\subsubsection{4-Week Trial Protocol}

\textbf{Week 1--2}: Start H1 antihistamine
\begin{itemize}
    \item Rupatadine 10 mg morning (preferred), OR
    \item Fexofenadine 180 mg OR loratadine 10 mg morning
    \item Monitor for sedation (rare with rupatadine/fexofenadine)
\end{itemize}

\textbf{Week 2--4}: Add H2 blocker
\begin{itemize}
    \item Famotidine 20 mg twice daily (morning and evening)
    \item Note: May reduce stomach acid; take iron supplements 2 hours apart
\end{itemize}

\textbf{Optional Enhancement}:
\begin{itemize}
    \item Add quercetin 500--1000 mg daily for additional mast cell stabilization
\end{itemize}

\textbf{Low-histamine diet} (adjunct):
\begin{itemize}
    \item Avoid: Aged/fermented foods, alcohol, cured meats, leftovers $>$24 hours
    \item Duration: Strict 2-week trial, then gradual reintroduction
\end{itemize}

\textbf{Assessment at Week 4}:
\begin{itemize}
    \item \textbf{Discontinuation test}: Stop antihistamines for 2--3 days
    \item If symptoms worsen $\to$ mast cell component confirmed $\to$ continue therapy
    \item If no change $\to$ discontinue (not MCAS-driven)
\end{itemize}

\paragraph{Expected Response}

\textbf{May improve} (if MCAS-related):
\begin{itemize}
    \item Brain fog and cognitive clarity
    \item Energy levels (especially post-meal fatigue)
    \item GI symptoms (bloating, nausea, diarrhea)
    \item Orthostatic tolerance
    \item Flushing and allergic symptoms
    \item Anxiety/panic-like episodes
\end{itemize}

\textbf{Will NOT improve} (metabolic/mitochondrial):
\begin{itemize}
    \item Core fatigue (``running on empty'') --- requires mitochondrial support
    \item Muscle cramps --- requires carnitine, magnesium
    \item PEM from overexertion --- requires pacing
    \item Progressive vision/hearing loss --- different mechanisms
\end{itemize}

\paragraph{Special Note: Amitriptyline for Dual Benefit}

If pain and/or sleep issues coexist with MCAS features, amitriptyline provides dual benefit~\cite{Clemons2011}:
\begin{itemize}
    \item \textbf{Dose}: 10--50 mg at bedtime
    \item \textbf{Mechanisms}: Mast cell inhibition (reduces IL-8, VEGF, IL-6, histamine)~\cite{Clemons2011} + pain relief + sleep improvement
    \item \textbf{Specificity}: This mast cell effect is unique to amitriptyline; other antidepressants (bupropion, citalopram, atomoxetine) do NOT inhibit mast cells~\cite{Clemons2011}
    \item Can combine with rupatadine + famotidine for comprehensive mast cell targeting
\end{itemize}

\section{Systematic Comorbidity Screening: The Septad Framework}
\label{sec:septad-screening-mild-moderate}

ME/CFS patients frequently present with a cluster of interrelated comorbidities that require distinct treatment approaches. The ``Septad'' framework (Section~\ref{sec:septad}) organizes seven conditions that commonly co-occur. Systematic screening can identify treatable contributors to symptom burden.

\subsection{The Seven Septad Components}

\begin{enumerate}
    \item \textbf{Mast Cell Activation Syndrome (MCAS)}: See Section~\ref{sec:mcas-mild-moderate} for screening and treatment
    \item \textbf{Ehlers-Danlos Syndrome (EDS) / Hypermobility}: Joint hypermobility, subluxations, chronic pain
    \item \textbf{Dysautonomia / POTS}: Orthostatic intolerance (Section~\ref{sec:symptom-management-mild-moderate})
    \item \textbf{Autoimmunity}: Subclinical or overt autoimmune markers
    \item \textbf{Chronic Infection}: Viral reactivation (EBV, HHV-6), tick-borne infections
    \item \textbf{Small Fiber Neuropathy (SFN)}: Pain, paresthesias, autonomic symptoms
    \item \textbf{GI Dysmotility}: Gastroparesis, SIBO, malabsorption
\end{enumerate}

\subsection{Screening Recommendations for Mild-Moderate Cases}

\paragraph{EDS / Hypermobility Screening.}
Screen all ME/CFS patients for hypermobility using the Beighton score. If Beighton score $\geq$5/9 or clinical features suggest EDS:

\begin{itemize}
    \item \textbf{Physical therapy referral}: Hypermobility-aware PT for joint stabilization
    \item \textbf{Avoid overextension}: Joints at risk for subluxation and chronic instability
    \item \textbf{Consider genetics referral}: For formal EDS typing if features suggest vascular or classical type
    \item \textbf{Monitor for progression}: Hypermobile patients may develop additional complications over time
\end{itemize}

\paragraph{Craniocervical Instability (CCI) Awareness.}
CCI is not part of the original Septad but occurs in hypermobile patients and can cause ME/CFS-like symptoms (fatigue, cognitive dysfunction, autonomic dysfunction). A specialized clinic study found high prevalence of structural abnormalities (80\% with craniocervical obstructions) in ME/CFS patients, predominantly hypermobile~\cite{Bragee2020}; however, these findings require replication in unselected populations (see Section~\ref{sec:septad} for detailed evidence and caveats). Consider CCI evaluation if:

\begin{itemize}
    \item Confirmed EDS/hypermobility PLUS
    \item Symptoms worse with neck position changes, or
    \item Occipital headaches, or
    \item Symptoms suggestive of brainstem compression (dysphagia, facial numbness, gait instability)
\end{itemize}

\textbf{Evaluation}: Upright MRI preferred over supine (dynamic instability may not appear supine); reference ranges for measurements are available~\cite{Nicholson2023}. Specialist referral (neurosurgeon with CCI expertise) if clinical suspicion high. See Lohkamp et al.~\cite{Lohkamp2022} for diagnostic criteria review.

\begin{warning}[CCI Is Rare But Treatable]
CCI is uncommon even in hypermobile ME/CFS patients. However, it represents a \emph{structural}, potentially \emph{treatable} cause of symptoms. Conservative management (physical therapy~\cite{Russek2023}, cervical collar) is first-line; surgery shows 60--80\% improvement in properly selected cases but carries significant complication rates (19\%)~\cite{Henderson2024}. Do not pursue CCI workup unless hypermobility is present and symptoms are positionally related.
\end{warning}

\paragraph{Small Fiber Neuropathy Screening.}
Consider SFN testing if:
\begin{itemize}
    \item Burning pain, paresthesias, or allodynia
    \item Symptoms in stocking-glove distribution
    \item Autonomic symptoms (sweating abnormalities, GI dysmotility, orthostatic intolerance)
\end{itemize}

\textbf{Evaluation}: Skin punch biopsy (intraepidermal nerve fiber density) is gold standard. Sudomotor function testing also useful.

\paragraph{Autoimmune Screening.}
Consider autoimmune workup if:
\begin{itemize}
    \item Family history of autoimmune disease
    \item Symptoms suggesting specific autoimmune conditions
    \item Unexplained inflammatory markers
\end{itemize}

\textbf{Basic panel}: ANA, ENA panel, RF, anti-CCP, TPO antibodies, anti-gliadin/tTG.

\paragraph{Chronic Infection Evaluation.}
Consider viral reactivation workup if post-infectious onset or ongoing immune activation:
\begin{itemize}
    \item EBV: VCA IgG, EBNA IgG, EA IgG (EA elevation suggests reactivation)
    \item CMV, HHV-6: IgG levels
    \item Tick-borne: Lyme and co-infections if exposure history
\end{itemize}

\paragraph{GI Dysmotility Screening.}
Screen for SIBO and gastroparesis if:
\begin{itemize}
    \item Bloating, early satiety, nausea, constipation alternating with diarrhea
    \item Food intolerances or malabsorption symptoms
\end{itemize}

\textbf{Testing}: Hydrogen/methane breath test for SIBO; gastric emptying study if gastroparesis suspected.

\subsection{Treatment Sequencing}

Based on clinical experience (not validated research), Kaufman suggests addressing MCAS first, then systematically working through other Septad components. Rationale: mast cell stabilization may improve other conditions due to interconnections.

\begin{warning}[Framework Limitations]
The Septad is a \emph{clinical framework} based on expert observation, not a validated research model. Systematic prevalence data for each component in ME/CFS populations is lacking. Use for organizing comorbidity screening, not as diagnostic criteria for ME/CFS itself. PEM remains the hallmark diagnostic feature (Section~\ref{sec:septad}).
\end{warning}

\section{Disease-Modifying Strategies for Mild-Moderate Cases}
\label{sec:disease-modifying-mild-moderate}

\subsection{Early Intervention Advantage}

Mild-moderate patients have a critical advantage: potential to intervene before immune exhaustion phase (Section~\ref{ach:cytokine-duration}). This provides opportunity for disease modification rather than pure symptom management.

\subsection{Immune Profiling and Targeted Intervention}

\paragraph{Recommended Testing}
\begin{itemize}
    \item \textbf{Basic panel}:
    \begin{itemize}
        \item CBC with differential
        \item Comprehensive metabolic panel
        \item Thyroid function (TSH, free T4, free T3)
        \item Iron studies (ferritin, iron, TIBC)
        \item Vitamin D, B12, folate
    \end{itemize}

    \item \textbf{Immune panel} (if accessible):
    \begin{itemize}
        \item Lymphocyte subsets (CD4, CD8, NK cells)
        \item Immunoglobulins (IgG, IgA, IgM)
        \item ANA, ENA panel (screening for autoimmunity)
        \item Inflammatory markers (CRP, ESR)
    \end{itemize}

    \item \textbf{Advanced panel} (if pursuing aggressive treatment):
    \begin{itemize}
        \item Cytokine panel (IL-6, IL-1$\beta$, TNF-$\alpha$, IL-10)
        \item GPCR autoantibodies (CellTrend - Germany)
        \item NK cell function assay
        \item Viral reactivation markers (EBV EA, VCA IgG, CMV IgG)
    \end{itemize}
\end{itemize}

\subsection{Early-Disease Anti-Cytokine Strategy}

\begin{tcolorbox}[colback=green!5!white,colframe=green!75!black,title=Novel Preventive Framework]
\textbf{Original Contribution}: This section applies the novel ``Immune Exhaustion Timeline'' hypothesis to mild-moderate cases. The insight: \textbf{early aggressive intervention in the first 3 years may prevent progression to severe disease and immune exhaustion}. While anti-inflammatory approaches exist, \textbf{stratifying by duration to create a preventive window is original}. This represents a paradigm shift from waiting for severity to worsen before intervening, to aggressive early treatment to prevent deterioration. Applicable to mild-moderate patients diagnosed within 3 years of onset.
\end{tcolorbox}

\paragraph{Rationale}
If illness duration $<$3 years and cytokines elevated (particularly IL-6 $>$3--5 pg/mL), consider anti-inflammatory intervention to prevent progression to exhaustion phase. Section~\ref{ach:cytokine-duration} documents duration-dependent cytokine patterns, and Section~\ref{sec:tier1-research} presents the ``Immune Exhaustion Timeline'' hypothesis.

\paragraph{Conservative Approach (Before Biologics)}
\begin{enumerate}
    \item \textbf{Aggressive anti-inflammatory supplementation}:
    \begin{itemize}
        \item \textbf{Omega-3 fatty acids (EPA+DHA) 2--4 g daily}
        \begin{itemize}
            \item \textbf{NOTE - EXCEEDS TYPICAL SUPPLEMENT DOSE}: Standard fish oil supplements provide 1000 mg (1 g) combined EPA+DHA daily. We recommend 2--4 g daily, which is 2--4$\times$ typical supplementation.
            \item \textbf{Justification}: Omega-3 fatty acids (EPA/DHA) reduce pro-inflammatory cytokine production (IL-1, IL-6, TNF-$\alpha$) via inhibition of arachidonic acid metabolism and NF-$\kappa$B signaling. Therapeutic anti-inflammatory effects require EPA+DHA doses of 2--4 g/day based on cardiovascular and rheumatologic studies. Lower doses provide general health benefits but insufficient cytokine modulation.
            \item \textbf{Safety margin}: Doses up to 5 g/day are considered safe by FDA. Our recommendation of 2--4 g/day is well within this limit.
            \item \textbf{Side effects}: Fishy aftertaste (take with meals), mild GI upset, loose stools at higher doses. Mild blood-thinning effect.
            \item \textbf{Drug interactions}: May potentiate anticoagulants (warfarin). Monitor INR if on blood thinners.
            \item \textbf{Monitoring}: None required for most patients. If on warfarin, monitor INR.
        \end{itemize}
        \item \textbf{Turmeric/curcumin 1000--2000 mg BID} (see Chapter~\ref{ch:urgent-action-severe} for complete dosing rationale - 2--4$\times$ typical supplement dose, well-tolerated, anti-inflammatory via NF-$\kappa$B inhibition)
        \item \textbf{Resveratrol 500 mg BID}
        \begin{itemize}
            \item \textbf{NOTE - DRAMATICALLY EXCEEDS TYPICAL DOSE}: Typical resveratrol supplements provide 100--250 mg once daily. We recommend 500 mg twice daily (1000 mg/day total), which is 4--10$\times$ typical supplementation.
            \item \textbf{Justification}: Resveratrol activates sirtuins (SIRT1) and inhibits NF-$\kappa$B, providing anti-inflammatory and potential mitochondrial benefits. Therapeutic doses for metabolic and inflammatory conditions in research studies use 500--1000 mg/day or higher. Lower doses may not achieve sufficient tissue concentrations for anti-inflammatory effects.
            \item \textbf{Bioavailability note}: Resveratrol has poor bioavailability ($<$1\%). This necessitates higher oral doses to achieve therapeutic levels. Micronized or liposomal formulations may improve absorption.
            \item \textbf{Safety margin}: Clinical trials have used up to 2000--5000 mg/day without serious adverse effects. Our recommendation of 1000 mg/day is moderate.
            \item \textbf{Side effects}: Generally well-tolerated. Occasional GI upset (nausea, diarrhea) at high doses. Take with food.
            \item \textbf{Drug interactions}: May potentiate anticoagulants. Theoretical interaction with immunosuppressants.
            \item \textbf{Monitoring}: None required.
        \end{itemize}
        \item \textbf{Green tea extract (EGCG) 400 mg BID}
        \begin{itemize}
            \item \textbf{NOTE - EXCEEDS TYPICAL SUPPLEMENT DOSE}: Typical green tea extract supplements provide 200--300 mg EGCG once daily. We recommend 400 mg twice daily (800 mg/day total), which is 2.5--4$\times$ typical supplementation.
            \item \textbf{Justification}: Epigallocatechin gallate (EGCG) is the primary catechin in green tea with anti-inflammatory and antioxidant properties. Therapeutic doses for metabolic and inflammatory benefits in studies use 400--800 mg/day EGCG. Lower doses provide antioxidant effects but may be insufficient for immune modulation.
            \item \textbf{Safety margin}: Doses up to 800--1200 mg/day have been studied. Our recommendation of 800 mg/day is at the upper studied range.
            \item \textbf{CRITICAL WARNING - HEPATOTOXICITY RISK}: High-dose green tea extract ($>$800 mg EGCG/day) on empty stomach has been associated with rare cases of liver injury. ALWAYS take with food. If ALT/AST elevation occurs, discontinue immediately.
            \item \textbf{Side effects}: Nausea, GI upset (take with food), jitteriness (contains some caffeine unless decaffeinated).
            \item \textbf{Drug interactions}: May interact with beta-blockers, blood thinners. Contains caffeine (unless decaffeinated).
            \item \textbf{Monitoring}: Consider baseline and 3-month liver function tests (ALT/AST) if using high-dose chronically.
        \end{itemize}
    \end{itemize}

    \item \textbf{Low-dose naltrexone (LDN)}:
    \begin{itemize}
        \item 1.5--4.5 mg nightly
        \item Immune modulation (reduces pro-inflammatory cytokines)
        \item Safe, well-tolerated
        \item Takes 2--4 weeks for benefit
    \end{itemize}

    \item \textbf{Dietary anti-inflammatory approach}:
    \begin{itemize}
        \item Mediterranean diet (vegetables, fruits, olive oil, fish)
        \item Eliminate processed foods, refined sugars
        \item Consider anti-inflammatory elimination diet trial
    \end{itemize}
\end{enumerate}

\paragraph{Aggressive Approach (If Mild Conservative Fails)}
\begin{itemize}
    \item Discuss anti-cytokine biologics with rheumatologist (tocilizumab, etanercept)
    \item More justifiable in early disease ($<$3 years) with documented high cytokines
    \item May prevent progression to severe disease and immune exhaustion
    \item Requires close monitoring due to infection risk
\end{itemize}

\subsection{Hormonal Optimization}

\paragraph{For All Patients}
\begin{itemize}
    \item \textbf{Thyroid}: Optimize thyroid replacement if hypothyroid (many need T3 supplementation, not just T4)
    \item \textbf{Vitamin D}: Target 50--80 ng/mL (higher than standard; immune function benefit)
    \item \textbf{Iron}: Ferritin $>$50 ng/mL; some patients need higher for symptom improvement
\end{itemize}

\paragraph{Sex-Specific}
\begin{itemize}
    \item \textbf{Pre-menopausal women with cycle-linked crashes}:
    \begin{itemize}
        \item Track symptoms across menstrual cycle
        \item If consistent luteal-phase worsening (days 14--28): Consider continuous oral contraceptives (eliminate hormone fluctuations)
        \item Or: Progesterone supplementation luteal phase
    \end{itemize}

    \item \textbf{Post-menopausal women}:
    \begin{itemize}
        \item Check estradiol
        \item If low ($<$30 pg/mL) → trial HRT (Section~\ref{sec:hormonal-modulation})
        \item Particularly if high IL-6 or prominent immune symptoms
    \end{itemize}

    \item \textbf{Men with fatigue + cognitive dysfunction}:
    \begin{itemize}
        \item Check testosterone (total and free)
        \item If low → testosterone replacement (immune and energy benefits)
    \end{itemize}
\end{itemize}

\subsection{Microbiome Restoration}

\paragraph{Gut-Immune Axis}

\begin{tcolorbox}[colback=green!5!white,colframe=green!75!black,title=Novel Mechanistic Hypothesis]
\textbf{Original Contribution}: The ``Dysbiotic Priming'' hypothesis (Section~\ref{sec:tier2-research}) is a \textbf{novel synthesis} connecting Che et al.'s finding~\cite{Che2025} of exaggerated immune responses to Candida stimulation with gut barrier dysfunction and microbiome alterations. The hypothesis: gut dysbiosis with fungal overgrowth provides constant low-level antigenic exposure, priming immune cells to overreact. This explains both baseline immune activation and post-exertional malaise (exertion worsens gut barrier). The estrogen-microbiome-immune connection explaining sex differences is also original. \textbf{No prior framework explicitly connects these findings into a unified therapeutic rationale.}
\end{tcolorbox}

Section~\ref{sec:tier2-research} presents the ``Dysbiotic Priming'' hypothesis: gut dysbiosis (Section~\ref{sec:microbiome}) may maintain immune hyperactivation (Section~\ref{sec:chronic-activation}). Addressing gut health may reduce systemic inflammation.

\paragraph{Stepwise Approach}
\begin{enumerate}
    \item \textbf{Assess GI involvement}:
    \begin{itemize}
        \item Do you have GI symptoms (bloating, diarrhea, constipation, pain)?
        \item Stool testing for dysbiosis (consider: GI-MAP, organic acids test, or similar)
    \end{itemize}

    \item \textbf{Dietary intervention}:
    \begin{itemize}
        \item Eliminate processed foods, added sugars
        \item Increase fiber (vegetables, fruits - unless FODMAP-sensitive)
        \item Consider elimination diet if food sensitivities (low-FODMAP, AIP, etc.)
        \item Probiotic-rich foods (if tolerated): yogurt, kefir, sauerkraut
    \end{itemize}

    \item \textbf{Targeted supplementation}:
    \begin{itemize}
        \item Probiotics: Multi-strain (Lactobacillus, Bifidobacterium), 25--50 billion CFU
        \item Saccharomyces boulardii 250 mg BID (anti-Candida, immune modulation)
        \item \textbf{Gut barrier support}:
        \begin{itemize}
            \item \textbf{L-glutamine 5 g daily}: NOTE - Exceeds typical supplement dose (1--2 g). See Chapter~\ref{ch:urgent-action-severe} for complete dosing rationale. Therapeutic dose for gut barrier repair is 5--10 g/day (5--10$\times$ typical supplement dose). Extremely safe, well-tolerated.
            \item \textbf{Zinc carnosine 75 mg BID} (150 mg/day total): NOTE - 2$\times$ typical supplement dose (75 mg once daily). See Chapter~\ref{ch:urgent-action-severe} for complete dosing rationale. Clinical mucosal healing studies use 75--150 mg BID. Provides ~32 mg elemental zinc, below UL of 40 mg/day.
        \end{itemize}
        \item Prebiotics: Inulin, partially hydrolyzed guar gum (feed beneficial bacteria)
    \end{itemize}

    \item \textbf{Antifungal trial if indicated}:
    \begin{itemize}
        \item If stool testing shows yeast overgrowth or strong clinical suspicion
        \item Fluconazole 100--200 mg daily for 4 weeks (prescription)
        \item Or: \textbf{Berberine 500 mg TID} (1500 mg/day total, natural antimicrobial) - NOTE: Exceeds typical supplement dose (500--1000 mg/day) by 1.5--3$\times$. See Chapter~\ref{ch:urgent-action-severe} for complete dosing rationale. CRITICAL WARNING: May cause hypoglycemia if taking diabetes medications - physician supervision required.
        \item Concurrent probiotics and gut support
    \end{itemize}
\end{enumerate}

\section{Work and Study Accommodations}
\label{sec:work-study}

\subsection{Critical Reality}

Most mild-moderate patients attempt to maintain work/study. This often leads to progressive worsening because energy spent on work leaves none for social life, self-care, or recovery. \textbf{Accommodations are essential}, not optional.

\subsection{Formal Accommodations}

\paragraph{Request These Accommodations}
\begin{itemize}
    \item \textbf{Reduced hours}: 50--75\% time if full-time unsustainable
    \item \textbf{Flexible schedule}: Work during peak energy times
    \item \textbf{Remote work}: Eliminate commute energy cost, enable rest breaks
    \item \textbf{Rest breaks}: Formal 15-minute horizontal rest every 2 hours
    \item \textbf{Quiet workspace}: Reduce sensory overload
    \item \textbf{Reduced meetings}: Cognitive load of meetings often underestimated
    \item \textbf{Deadline flexibility}: Accommodate fluctuating capacity
    \item \textbf{Parking accommodation}: Close parking to reduce walking
\end{itemize}

\paragraph{Legal Protections (Varies by Country)}
\begin{itemize}
    \item \textbf{US}: Americans with Disabilities Act (ADA) - ME/CFS qualifies; employer must provide reasonable accommodations
    \item \textbf{UK}: Equality Act - ME/CFS is protected disability
    \item \textbf{EU}: National disability discrimination laws vary by country
    \item \textbf{Documentation}: Physician letter documenting diagnosis and functional limitations
\end{itemize}

\subsection{Self-Imposed Boundaries}

\begin{itemize}
    \item \textbf{Do not work through lunch}: Use for horizontal rest
    \item \textbf{Do not work evenings/weekends}: Reserve all non-work time for recovery
    \item \textbf{Say no to optional tasks}: Decline extra projects, social work events
    \item \textbf{Communicate limitations}: Better to set expectations than to fail to deliver
\end{itemize}

\subsection{When to Stop Working}

\begin{warning}[Work Cessation Criteria]
If despite accommodations you are:
\begin{itemize}
    \item Bedbound on weekends recovering from work week
    \item Progressively worsening (more frequent/severe PEM)
    \item Unable to maintain basic self-care (cooking, hygiene, errands)
    \item Developing new symptoms or severity increase
\end{itemize}

Then working is \textbf{causing progression} to severe disease. Apply for disability. Your health is more important than employment. Working yourself into severe ME/CFS leaves you unable to work \emph{and} severely disabled.
\end{warning}

\section{Graded Exercise Therapy (GET): Why to Avoid}
\label{sec:get-avoidance}

\subsection{Critical Warning}

Graded Exercise Therapy (GET) remains recommended in some countries despite evidence of harm. \textbf{GET is contraindicated in ME/CFS and can cause severe, lasting worsening.}

\subsection{Why GET Fails}

\begin{enumerate}
    \item \textbf{Fundamental misunderstanding}: GET assumes deconditioning causes symptoms; increasing exercise reconditions. This is false. PEM is pathological response to exertion (Section~\ref{sec:energy-consequences}), not deconditioning.

    \item \textbf{Ignores PEM}: GET protocols ignore delayed symptom exacerbation, attributing it to ``expected discomfort'' rather than disease mechanism (Section~\ref{sec:energy-consequences}).

    \item \textbf{Biomarker evidence}: Chapters 6--7 document that exertion triggers immune activation (Section~\ref{sec:immune-activation}), oxidative stress (Section~\ref{sec:oxidative-stress}), and metabolic dysfunction (Section~\ref{sec:mitochondrial-dysfunction}) - not adaptation.

    \item \textbf{Patient harm surveys}:
    \begin{itemize}
        \item 50--70\% of patients report worsening from GET
        \item Some become severe/bedbound after GET programs
        \item UK NICE guidelines (2021) removed GET recommendation due to harm
    \end{itemize}
\end{enumerate}

\subsection{If Pressured by Physician}

\begin{itemize}
    \item Cite NICE 2021 guidelines (UK), recent reviews documenting harm
    \item Request pacing/energy envelope management instead
    \item Seek second opinion from ME/CFS-knowledgeable physician
    \item If insurance requires ``exercise program,'' document that standard GET worsens ME/CFS; request adaptive pacing therapy (APT) instead
\end{itemize}

\subsection{Safe Activity Increase (If Appropriate)}

\textbf{Only} if:
\begin{itemize}
    \item Baseline symptom stability for 6+ months
    \item No PEM episodes for 3+ months
    \item Energy envelope well-established
    \item Under guidance of ME/CFS-knowledgeable professional
\end{itemize}

\textbf{Principles:}
\begin{itemize}
    \item Increase activity 5--10\% every 4--6 weeks (very gradual)
    \item If any PEM → immediately reduce to prior level
    \item Horizontal/recumbent exercise (recumbent bike, rowing)
    \item Never exceed anaerobic threshold
    \item Prioritize activities of daily living over formal exercise
\end{itemize}

\section{Long-Term Strategy for Mild-Moderate Cases}
\label{sec:long-term-mild-moderate}

\subsection{Goals}

\begin{enumerate}
    \item \textbf{Primary}: Prevent progression to severe disease
    \item \textbf{Secondary}: Improve function within energy envelope
    \item \textbf{Tertiary}: Achieve remission or substantial recovery (ambitious but possible in some)
\end{enumerate}

\subsection{Timeline}

\begin{itemize}
    \item \textbf{Months 1--6}: Establish pacing, optimize symptom management, identify triggers
    \item \textbf{Months 6--12}: Implement disease-modifying strategies (immune modulation, hormones, microbiome)
    \item \textbf{Year 1--2}: Assess trajectory - stable? improving? worsening?
    \item \textbf{Year 2--5}: Continued optimization; some patients achieve significant recovery or remission
\end{itemize}

\subsection{Realistic Expectations}

\begin{itemize}
    \item \textbf{Remission}: 5--10\% of patients achieve sustained remission (symptom-free $>$1 year)
    \item \textbf{Substantial improvement}: 20--30\% improve significantly (mild symptoms, near-normal function)
    \item \textbf{Stable mild-moderate}: 40--50\% remain stable with good management
    \item \textbf{Progression}: 10--20\% worsen despite intervention (often due to continued overexertion)
\end{itemize}

The goal is to maximize your chances of being in the improvement categories through aggressive early intervention and strict pacing.

\section{Summary: Preventing the Descent}
\label{sec:summary-mild-moderate}

\subsection{Key Principles}

\begin{enumerate}
    \item \textbf{Pacing is paramount}: More important than any medication or supplement
    \item \textbf{Early intervention}: Treating mild disease aggressively may prevent severe disease
    \item \textbf{Accommodations are essential}: Reduce work/study load to sustainable level
    \item \textbf{Avoid GET}: Do not be pressured into graded exercise programs
    \item \textbf{Target root causes}: Immune dysregulation, hormonal imbalance, microbiome - not just symptoms
    \item \textbf{Hope with realism}: Some improve significantly; not all recover; pacing prevents worsening for most
\end{enumerate}

\subsection{Action Checklist}

\begin{itemize}
    \item[$\square$] Establish energy envelope (2-week activity tracking + HR monitoring)
    \item[$\square$] Implement 50\% rule (do half of perceived capacity)
    \item[$\square$] Optimize sleep (hygiene + supplements or medication if needed)
    \item[$\square$] Address dominant symptoms (brain fog, pain, POTS, GI)
    \item[$\square$] Trial MCAS protocol if indicated (2-week cetirizine + famotidine + diet)
    \item[$\square$] Obtain basic labs (CBC, CMP, thyroid, iron, vitamin D, B12)
    \item[$\square$] Request work/study accommodations (reduced hours, flexible schedule, remote work)
    \item[$\square$] Avoid GET programs; seek pacing-based approach
    \item[$\square$] If early disease ($<$3 years), consider immune profiling and anti-inflammatory strategy
    \item[$\square$] If post-menopausal woman or low testosterone, check hormone levels
    \item[$\square$] Address microbiome if GI symptoms present
    \item[$\square$] Reassess every 3--6 months: Stable? improving? worsening? Adjust accordingly.
\end{itemize}

Mild-moderate ME/CFS is not mild suffering. It is life-altering, disabling, and deserves aggressive management. You are not being lazy. You are not deconditioned. You have a biological illness. Protect your energy envelope. Advocate for accommodations. Pursue treatments. Prevent progression.

Your future self will thank you for the boundaries you set today.
