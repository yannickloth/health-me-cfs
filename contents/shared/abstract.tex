\chapter*{Abstract}
\addcontentsline{toc}{chapter}{Abstract}

Myalgic encephalomyelitis/chronic fatigue syndrome (ME/CFS) is a severe, chronic, multi-system neuroimmune disease affecting an estimated 0.89\% to 2.5\% of the global population. Characterized by profound post-exertional malaise, unrefreshing sleep, cognitive dysfunction, and autonomic dysregulation, ME/CFS represents one of the most disabling chronic conditions in modern medicine. Despite affecting millions worldwide, the disease has historically suffered from underfunding, dismissal by medical professionals, and classification as a syndrome rather than a disease with identifiable pathophysiology.

The February 2024 NIH deep phenotyping study fundamentally transformed this landscape by demonstrating specific biological abnormalities: decreased brain activity in effort-related neural circuits, exhausted T-cell populations, chronic B-cell activation deficits, and depleted catecholamine levels in cerebrospinal fluid. These findings conclusively established ME/CFS as a systemic biological disease with measurable immune, neurological, and metabolic dysfunction.

This comprehensive documentation synthesizes current research across clinical presentation, pathophysiological mechanisms, treatment approaches, epidemiological evidence, and mathematical modeling frameworks. The work integrates findings from hundreds of peer-reviewed literature sources spanning energy metabolism dysfunction, immune exhaustion, neuroinflammation, endocrine dysregulation, cardiovascular abnormalities, gut-brain axis disruption, and genetic-epigenetic factors.

Part I provides detailed clinical characterization of core symptoms, diagnostic criteria evolution, and disease course variations from mild to very severe presentations. Part II examines established and hypothetical pathophysiological mechanisms, including mitochondrial dysfunction, chronic immune activation, autonomic nervous system failure, and integrative systems models. Part III documents evidence-based treatment strategies, medication protocols, supplement regimens, and emerging therapeutic approaches including immune modulation, metabolic support, and neurological interventions. Part IV synthesizes biomarker research, clinical trial outcomes, mechanistic studies, and epidemiological patterns. Part V presents mathematical and computational modeling approaches to understanding disease dynamics and predicting treatment responses.

The appendices include comprehensive terminology guides, diagnostic tool summaries, supplement protocols, research synthesis frameworks, an extensively annotated bibliography of key papers.

Methodologically, this work distinguishes between established findings (marked as achievements with high-certainty evidence from replicated studies with $n>100$), working hypotheses (unproven theories requiring validation), predictions (testable claims for future research), and warnings (critical limitations and contraindications). Evidence quality is systematically classified as high, medium, or low certainty based on sample size, peer-review status, replication, and methodological rigor.

This documentation serves multiple audiences: researchers seeking comprehensive mechanistic understanding and modeling frameworks, clinicians requiring evidence-based treatment protocols with dosing guidance and contraindication awareness, patients and caregivers needing accessible explanations of symptoms and management strategies, and advocates working toward recognition, funding, and medical education reform. The work is released under the Creative Commons Attribution 4.0 International License to maximize accessibility and enable derivative works.

Written by a software architect and patient-researcher with degrees in industrial engineering and management sciences, this documentation applies systems thinking, computational analysis, and first-principles reasoning to ME/CFS pathophysiology while maintaining epistemic humility about the substantial uncertainties remaining in the field. The author explicitly disclaims medical expertise and emphasizes that all content represents literature synthesis and personal experience documentation, not clinical advice. All treatment decisions must be made in consultation with qualified healthcare providers.

ME/CFS research is at a critical inflection point. The biological validation provided by recent NIH and international studies, combined with shared research agendas driven by Long COVID parallels, offers unprecedented opportunity for mechanistic discovery and therapeutic development. This document aims to accelerate progress by organizing scattered findings into an accessible, comprehensive reference while identifying critical knowledge gaps requiring focused investigation.
