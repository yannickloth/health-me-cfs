\chapter*{About the Author}
\addcontentsline{toc}{chapter}{About the Author}

Yannick Loth is a software architect and independent patient-researcher with degrees in Industrial Engineering (Information Systems, University of Luxembourg) and Management Sciences (General Management, HEC Liège), with prior completion of first-cycle civil engineering studies (University of Liège). With nearly two decades of professional software engineering experience and iSAQB CPSA-F certification (2015), he brings computational thinking and systems analysis to medical research.

Having experienced symptoms consistent with ME/CFS since childhood, with progressive worsening over the past decade and marked acceleration in recent years, Yannick has applied his background in discrete mathematics, information systems architecture, and analytical research to understanding this complex multisystem disease (though formal diagnostic documentation has not yet been received). This work-in-progress represents an ongoing effort to synthesize the current state of ME/CFS research into a comprehensive, accessible reference while documenting his own case with scientific rigor.

Based in Messancy, Belgium, he has published research on software architecture principles (notably the Independent Variation Principle) and fundamental physics (Causal Graph Theory), now applying similar first-principles thinking to understanding chronic illness mechanisms.

\section*{Contact Information}

\begin{itemize}
    \item Email: \href{mailto:yl@infolead.eu}{yl@infolead.eu}
    \item ORCID: \href{https://orcid.org/0009-0003-5754-827X}{0009-0003-5754-827X}
\end{itemize}

\section*{Motivation}

This documentation project is driven by both intellectual rigor and existential urgency. Witnessing the devastating reality of severe ME/CFS through videos and posts from bedbound patients has crystallized a stark question: how to prevent deterioration into severe illness before all desire to live vanishes? With a family, children, and friends depending on me, the stakes extend beyond personal survival---I cannot risk descending into a state where I lose the capacity to be present for those I love. This work aims to bridge the gap between scattered research findings and accessible, comprehensive information about ME/CFS---organizing current knowledge about symptoms, mechanisms, and treatments while rigorously documenting a progressive case with quantitative data. By systematically analyzing what is known and identifying what must be discovered, this project seeks to serve researchers, clinicians, patients, and advocates while racing against time to find pathways out of progressive decline.
