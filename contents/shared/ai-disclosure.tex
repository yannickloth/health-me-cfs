\chapter*{AI Disclosure Statement}
\addcontentsline{toc}{chapter}{AI Disclosure Statement}

This manuscript was developed through extensive collaboration between a human author and AI language models. In the interest of scientific transparency, this statement describes the nature and extent of each party's contributions.

\section*{Author's Contributions}

The author (Yannick Loth) contributed:

\begin{itemize}
    \item \textbf{Lived experience}: Direct, first-person experience of severe illness with symptoms consistent with ME/CFS (formal diagnostic documentation pending) providing the phenomenological foundation and motivation for this work
    \item \textbf{Research direction}: Identifying research gaps, selecting topics for investigation, and determining which mechanisms and treatments warranted detailed exploration
    \item \textbf{Literature selection}: Choosing which papers to prioritize, which findings were most significant, and how to organize the overwhelming volume of ME/CFS research into a coherent framework
    \item \textbf{Critical evaluation}: Assessing study quality, identifying methodological limitations, distinguishing high-certainty findings from speculative claims, and evaluating evidence strength
    \item \textbf{Clinical data}: All personal case data in Appendix I, including symptom tracking, medication trials, and functional capacity measurements collected through lived experience with the disease
    \item \textbf{Structural decisions}: Choosing theorem-like environments (hypothesis, achievement, warning, etc.) to make epistemic status of claims immediately clear---a structural choice designed to facilitate critical review
    \item \textbf{Systems analysis}: Applying software architecture thinking to understand ME/CFS as a complex multisystem disease, identifying potential mechanistic relationships and integration points across physiological systems
    \item \textbf{Quality control}: Conducting extensive review cycles to verify medical accuracy, logical consistency, and appropriate citation of sources
    \item \textbf{Organizational framework}: Deciding document structure, what content to include, how to balance comprehensiveness with accessibility, and how to serve multiple audiences (patients, clinicians, researchers)
\end{itemize}

\section*{AI Contributions}

AI language models (primarily Claude Sonnet 4.5 and Opus 4.5, Anthropic Inc.) performed:

\begin{itemize}
    \item \textbf{Literature synthesis}: Processing and summarizing large volumes of research papers, extracting key findings, and organizing information thematically
    \item \textbf{Technical exposition}: Drafting explanatory text for complex biological mechanisms, translating technical research into accessible language
    \item \textbf{Citation management}: Identifying relevant studies, formatting references, managing bibliography, and ensuring proper attribution
    \item \textbf{LaTeX preparation}: Writing and formatting LaTeX source code, creating document structure, managing cross-references and environments
    \item \textbf{Consistency checking}: Identifying contradictions, checking internal consistency, and verifying that claims match cited sources
\end{itemize}

\section*{Nature of the Collaboration}

This work represents a new mode of medical documentation in which the author's lived experience, clinical judgment, and research direction combined with AI's information processing capabilities. The author provided the conceptual framework, selected research priorities, and maintained continuous quality control throughout development.

The manuscript exceeds \RoundedPageCount{} pages spanning clinical symptomatology, multisystem pathophysiology, treatment protocols, and research synthesis. The author has spent extensive time reviewing this material to ensure medical accuracy and appropriate epistemic calibration---distinguishing established findings from preliminary results and clearly marking speculative content.

AI frequently required redirection when synthesizing research, occasionally missing nuances in study design, overstating certainty, or losing focus on ME/CFS-specific findings. A substantial portion of the author's effort involved recognizing when outputs missed important caveats, diagnosing misunderstandings of research context, and redirecting toward more accurate representations.

This collaboration enabled processing a volume of literature that would be extremely challenging for a single individual, particularly one disabled by the disease being studied. The combination of AI's processing capabilities with the author's continuous strategic direction, clinical insight from lived experience, and exhaustive quality control made this comprehensive documentation possible.

\section*{Author Responsibility}

Despite the substantial AI contribution, the author takes full responsibility for:

\begin{itemize}
    \item The decision to publish and disseminate this work
    \item All medical claims, treatment discussions, and mechanistic explanations
    \item Any errors, misconceptions, or unjustified conclusions
    \item The interpretation and implications of research findings
    \item Personal case data and clinical observations
\end{itemize}

\textbf{Critical disclaimer}: The author is \textbf{not a medical doctor} and has no formal medical training. While he holds engineering degrees and received formal training in mathematics and analytical methods, he is not qualified to provide medical advice. This work represents independent patient research and systematic literature review, not clinical guidance.

The medical and scientific communities are invited to review this work critically. Such scrutiny is essential before any claims or treatment approaches discussed here can be considered validated for clinical use.

\section*{Transparency and Reproducibility}

\begin{itemize}
    \item This disclosure is made voluntarily and in good faith.
    \item The manuscript was developed using Claude Sonnet 4.5 and Claude Opus 4.5 (Anthropic).
    \item The manuscript source is available at: \href{https://github.com/yannickloth/health-me-cfs}{github.com/yannickloth/health-me-cfs}
    \item Correspondence regarding the content should be directed to the author.
\end{itemize}

\vspace{1em}
\noindent\textbf{Author:} Ing. Yannick Loth, M.Sc. (Management)\\
\textbf{Date:} \today\\
\textbf{ORCID:} \href{https://orcid.org/0009-0003-5754-827X}{0009-0003-5754-827X}
