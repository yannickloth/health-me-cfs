% APPENDIX G ADDITIONS: Mechanistic Hypotheses Literature
% To be integrated into contents/appendices/appendix-g-research-synthesis.tex

\subsection{Developmental and Age-Dependent Mechanisms}
\label{subsec:developmental-mechanisms-studies}

\begin{table}[htbp]
\centering
\caption{Developmental Mechanisms and Pediatric Recovery}
\label{tab:developmental-mechanisms}
\scriptsize
\begin{tabular}{p{2cm}p{1.8cm}p{2.2cm}p{4cm}p{2.5cm}p{1.8cm}}
\toprule
\textbf{Study} & \textbf{Design} & \textbf{Sample} & \textbf{Key Findings} & \textbf{Implications for ME/CFS} & \textbf{Certainty} \\
\midrule
Nakatomi 2014~\cite{Nakatomi2014} & PET imaging; case-control & n=9 ME/CFS, n=10 controls & Microglial activation 45--199\% higher in multiple brain regions; correlated with cognitive impairment, pain, depression & Neuroinflammation documented; potential target for treatment & MODERATE (small n; rigorous methodology; contradicted by later study) \\
\midrule
Weinhard 2018~\cite{Weinhard2018} & Mouse model; developmental study & Adolescent mice (P39 peak) & Transient increase in microglial synaptic pruning during adolescence in prefrontal cortex; C1q-C3-CR3 pathways & Developmental microglial reprogramming may reset inflammatory states in children & MODERATE (animal model; mechanistic insight) \\
\midrule
Yamashita \& Passegué 2020~\cite{Yamashita2020} & Review; synthesis & Multiple studies & HSCs undergo exhaustion under chronic stress (infection, inflammation, BMT); mechanisms include DNA damage, epigenetic changes, telomere shortening & ME/CFS may involve accelerated HSC exhaustion; children have greater HSC reserves & MODERATE-HIGH (established HSC biology; application to ME/CFS hypothetical) \\
\midrule
Kovtonyuk 2022~\cite{Kovtonyuk2022} & Review; mechanistic & Multiple studies & Aging-associated HSC dysfunction involves cell-intrinsic (DNA damage, epigenetics) and cell-extrinsic (niche, inflammation) mechanisms; "inflamm-aging" (2--4x ↑ IL-1, IL-6, TNF) impairs HSC function & Chronic inflammation in ME/CFS may accelerate HSC aging; explains immune dysfunction & MODERATE-HIGH (established aging biology) \\
\midrule
Cashman \& Jenks 2022~\cite{Cashman2022} & Review; tolerance mechanisms & Multiple studies & Peripheral B cell tolerance via anergy, deletion, Breg suppression; ~20\% of peripheral B cells retain self-reactivity & Developmental tolerance may clear autoreactive B cells in children but not adults & MODERATE (established immunology; developmental timing speculative) \\
\bottomrule
\end{tabular}
\end{table}

\subsection{Epstein-Barr Virus and Autoimmunity}
\label{subsec:ebv-autoimmunity-studies}

\begin{table}[htbp]
\centering
\caption{EBV, Infectious Mononucleosis, and Autoantibody Generation}
\label{tab:ebv-autoimmunity}
\scriptsize
\begin{tabular}{p{2cm}p{1.8cm}p{2.2cm}p{4cm}p{2.5cm}p{1.8cm}}
\toprule
\textbf{Study} & \textbf{Design} & \textbf{Sample} & \textbf{Key Findings} & \textbf{Implications for ME/CFS} & \textbf{Certainty} \\
\midrule
Katz 2009~\cite{Katz2009} & Prospective cohort & n=301 adolescents with IM & 13\% developed CFS at 6 months; 7\% at 12 months; 4\% at 24 months & EBV-triggered IM is established ME/CFS trigger in adolescents & HIGH (prospective; well-characterized) \\
\midrule
Dunmire 2022~\cite{Dunmire2022} & Review; EBV biology & Comprehensive & EBV causes IM in 35--50\% of adolescent/young adult infections; latency in memory B cells via LMP1/LMP2A signaling & Age-dependent presentation suggests developmental interaction & HIGH (established virology) \\
\midrule
Frontera \& Kaltsas 2020~\cite{Frontera2020} & Review; EBV autoimmunity & Multiple studies & IM stimulates multiple autoantibodies: ANA, RF, anti-i, cryoglobulins; molecular mimicry between EBV and host proteins & EBV infection creates autoreactive B cell populations & MODERATE-HIGH (well-documented phenomenon) \\
\midrule
Fluge 2015~\cite{Fluge2015} & Open-label trial; rituximab & n=29 ME/CFS & B cell depletion with rituximab showed benefit in subgroup; delayed response (months) & B cells may contribute to pathology; treatment timing critical & MODERATE (open-label; positive but requires RCT confirmation) \\
\midrule
Chalmers 2022~\cite{Chalmers2022} & RCT; rituximab & n=TBD ME/CFS & No overall benefit from rituximab in UK trial & Negative result; possible explanations: patient selection, timing, plasma cell resistance & MODERATE (contradicts earlier study; highlights heterogeneity) \\
\bottomrule
\end{tabular}
\end{table}

\subsection{Orthostatic Intolerance and Autonomic Dysfunction}
\label{subsec:oi-treatment-studies}

\begin{table}[htbp]
\centering
\caption{Orthostatic Intolerance Treatment Outcomes}
\label{tab:oi-treatment}
\scriptsize
\begin{tabular}{p{2cm}p{1.8cm}p{2.2cm}p{4cm}p{2.5cm}p{1.8cm}}
\toprule
\textbf{Study} & \textbf{Design} & \textbf{Sample} & \textbf{Key Findings} & \textbf{Implications for ME/CFS} & \textbf{Certainty} \\
\midrule
Kalupahana 2022~\cite{Kalupahana2022} & Retrospective & n=27 adolescents (ages 12--17) with POTS & Ivabradine significantly reduced heart rate, palpitations, fatigue, syncope, lightheadedness & OI treatment improves multi-system symptoms beyond cardiovascular & MODERATE (retrospective; no control; pediatric-specific) \\
\midrule
Stewart 2018~\cite{Stewart2018} & Guideline; consensus & Pediatric OI & Comprehensive OI management (salt/fluids, compression, medications) improves cognition, exercise tolerance, wellbeing in addition to orthostatic symptoms & OI may be upstream driver of multi-system dysfunction & MODERATE-HIGH (expert consensus; clinical observations) \\
\midrule
Li 2015~\cite{Li2015} & Case-control; autoantibody & n=55 POTS, controls & Adrenergic and muscarinic receptor autoantibodies in POTS patients; correlate with symptom severity & Autoimmune basis for some OI; links to broader autoimmunity in ME/CFS & MODERATE (replicated finding; mechanism plausible) \\
\midrule
Newton 2007~\cite{Newton2007} & Cross-sectional; prevalence & n=96 ME/CFS & 60--70\% of ME/CFS patients have measurable orthostatic dysfunction & OI highly prevalent but not universal; suggests subgroups & MODERATE (well-characterized cohort) \\
\bottomrule
\end{tabular}
\end{table}

\subsection{Supporting Mechanistic Studies}
\label{subsec:supporting-mechanisms}

\begin{table}[htbp]
\centering
\caption{Supporting Studies for Mechanistic Hypotheses}
\label{tab:supporting-mechanisms}
\scriptsize
\begin{tabular}{p{2cm}p{1.8cm}p{2.2cm}p{4cm}p{2.5cm}p{1.8cm}}
\toprule
\textbf{Study} & \textbf{Design} & \textbf{Sample} & \textbf{Key Findings} & \textbf{Implications} & \textbf{Certainty} \\
\midrule
Elmore 2014~\cite{Elmore2014} & Mouse model; CSF-1R inhibition & Adult mice & CSF-1R inhibitors (PLX5622) deplete >99\% of microglia; repopulation from progenitors within 2 weeks & Forced microglial turnover possible; potential therapeutic approach & HIGH (animal model; mechanistic) \\
\midrule
Cheng 2014~\cite{Cheng2014} & Mouse + human; fasting & Mice + n=19 chemotherapy patients & Prolonged fasting promotes HSC-based regeneration; reverses immunosuppression via IGF-1/PKA pathway & Fasting-mimicking diets may regenerate HSC pools & MODERATE-HIGH (animal + pilot human data) \\
\midrule
Geiger \& Rudolph 2018~\cite{Geiger2018} & Review; HSC aging & Multiple studies & HSCs decline with age: reduced self-renewal, myeloid skewing, clonal dominance, telomere shortening, immunosenescence & Adult vs. pediatric HSC reserves explain differential recovery & HIGH (established gerontology) \\
\midrule
Pelanda \& Torres 2012~\cite{Pelanda2012} & Review; B cell tolerance & Multiple studies & Central B cell tolerance via receptor editing and clonal deletion in bone marrow; ~20\% escape to periphery with self-reactivity & Developmental checkpoints may be more active in children & HIGH (established immunology) \\
\midrule
Montoya 2017~\cite{Montoya2017} & Case-control; cytokines & n=192 ME/CFS, n=392 controls & Cytokine signatures correlate with disease severity; IL-1, IL-6, TNF elevated in subset & Chronic inflammation documented; supports HSC exhaustion model & MODERATE-HIGH (large cohort; replicated finding) \\
\midrule
Roerink 2021~\cite{Roerink2021} & PET imaging; negative result & n=15 women ME/CFS & No significant difference in TSPO binding (microglial activation) compared to controls using ¹¹C-PK11195 & Contradicts Nakatomi 2014; suggests heterogeneity or methodological issues & MODERATE (contradictory finding; important for balanced view) \\
\bottomrule
\end{tabular}
\end{table}

% Integration notes:
% - Insert these tables into appendix-g-research-synthesis.tex after existing mechanism tables
% - Cross-reference from main text hypotheses to these table entries
% - Update table of contents if needed
% - Verify all citations exist in references.bib
