% Quick Reference Card - MajesticSpinach2909
% 2 pages (recto-verso) for daily use
% Compile with: pdflatex quickref-daily.tex

\documentclass[9pt,a4paper]{article}

\usepackage[utf8]{inputenc}
\usepackage[T1]{fontenc}
\usepackage[ngerman]{babel}
\usepackage{booktabs}
\usepackage{array}
\usepackage{xcolor}
\usepackage{colortbl}
\usepackage{geometry}
\usepackage{enumitem}
\usepackage{tcolorbox}
\usepackage{fancyhdr}
\usepackage{amssymb}

\geometry{margin=0.8cm, top=2cm, bottom=0.8cm, headheight=14pt, headsep=0.5cm}

% Colors
\definecolor{morning}{RGB}{255,248,220}
\definecolor{midday}{RGB}{255,255,224}
\definecolor{evening}{RGB}{230,230,250}
\definecolor{asneeded}{RGB}{245,245,245}
\definecolor{headerblue}{RGB}{41,128,185}
\definecolor{newitem}{RGB}{46,204,113}
\definecolor{resumeitem}{RGB}{255,165,0}

% Header
\pagestyle{fancy}
\fancyhf{}
\fancyhead[L]{\textbf{MajesticSpinach2909 -- Tagesprotokoll}}
\fancyhead[R]{\colorbox{red!20}{\textbf{CRASH-PHASE}}}
\fancyfoot[C]{\thepage}
\renewcommand{\headrulewidth}{1pt}

% Compact lists
\setlist{noitemsep, topsep=2pt, parsep=0pt, partopsep=0pt}

\begin{document}

% =============================================================================
% CRASH PROTOCOL - PRIORITY
% =============================================================================

% CRITICAL INSIGHT BOX
\begin{tcolorbox}[colback=blue!15, colframe=blue!80!black, title={\textbf{KRITISCHE EINSICHT -- Januar 2026}}, fonttitle=\bfseries, boxsep=1pt, left=2pt, right=2pt, top=1pt, bottom=1pt]
{\scriptsize
\textbf{Cimetidin + Aminosäuren} haben früher funktioniert (``aus dem Bett gebracht''). Aber: \textbf{temporär}, nicht dauerhaft. Rückfall nach Infekt zeigt: \textbf{Ursache nicht adressiert}.

\colorbox{yellow!50}{\textbf{HÖCHSTE PRIORITÄT:}} \textbf{EBV/HHV-6 Diagnostik SOFORT} (auch während Crash möglich = nur Blutabnahme)

\textbf{Hypothese:} Cimetidin verstärkt Immunantwort gegen Virus, eliminiert es aber nicht. $\rightarrow$ Bei positivem Befund: \textbf{Antiviral + Cimetidin = Synergie}.
}
\end{tcolorbox}

\vspace{0.1em}

\begin{tcolorbox}[colback=red!15, colframe=red!80!black, title={\textbf{CRASH-PROTOKOLL}}, fonttitle=\bfseries, boxsep=1pt, left=2pt, right=2pt, top=1pt, bottom=1pt]
{\scriptsize
\textbf{1. Ruhe:} Körperlich + kognitiv (kein Bildschirm >10min) + sozial (Gespräche minimieren)

\textbf{2. ORS:} 3$\times$/Tag, liegend \quad
\textbf{3. OI/POTS:} Langsam aufstehen! BP sackt ab trotz HR-Kontrolle.

\textbf{4. Extra:} D-Ribose +5g (3$\times$/Tag), Mg +200mg abends

\textbf{5. Schlaf/Schmerz:} Ketotifen beibehalten; Reserve: Quviviq 25mg; Schmerz: Teufelskralle, PEA

\textbf{6. Zyklus:} Perimenstuell ($-$3d bis $+$2d): extra Ruhe, Mg +100mg, Ferritin prüfen (Ziel >50)

\textbf{7. KEINE neuen Supplemente!} \textit{Erst nach 1 Woche Stabilisierung.}

\textbf{8. DIAGNOSTIK:} EBV-VCA-IgM, EBV-EA-IgG, EBV-PCR, HHV-6-IgG -- \textbf{KANN JETZT GEMACHT WERDEN!}
}
\end{tcolorbox}

\vspace{0.1em}

% =============================================================================
% PAGE 1: MEDICATION SCHEDULE
% =============================================================================

\section*{Tagesplan -- CRASH-PHASE (aktuell)}

\vspace{-0.8em}
{\footnotesize\textit{\textbf{NUR Basis-Medikamente + aktuelle Supplemente.} \colorbox{newitem!30}{Neu} und \colorbox{resumeitem!30}{Wiederaufnehmen} = NACH Crash-Erholung (1 Woche stabil).}}
\vspace{0.1em}

% MORNING
\begin{tcolorbox}[colback=morning, colframe=orange!70!black, title={\textbf{MORGENS}}, fonttitle=\bfseries, boxsep=1pt, left=2pt, right=2pt, top=1pt, bottom=1pt]
{\footnotesize
\begin{tabular}{@{}p{3.5cm}p{1.6cm}p{7.5cm}@{}}
\textbf{Substanz} & \textbf{Dosis} & \textbf{Hinweise} \\
\midrule
\multicolumn{3}{l}{\textit{Medikamente:}} \\
Levocetirizin & 5~mg & H1-Blocker (nüchtern) \\
\rowcolor{resumeitem!30} \textbf{Cimetidin} & \textbf{200~mg} & \textbf{H2-Blocker -- WIEDERAUFNEHMEN} \\
LDA (Aripiprazol) & 1,5~mg & Niedrigdosis \\
Ivabradin & 2,5--5~mg & POTS \\
Mestinon & 30~mg & Mit Essen \\
\midrule
\multicolumn{3}{l}{\textit{Mitochondrien:}} \\
NAC & 600~mg & Mit Essen \\
\rowcolor{newitem!30} ALCAR & 1000~mg & Kognition \\
\rowcolor{newitem!30} NR oder NMN & 300--500~mg & NAD$^+$ \\
\rowcolor{newitem!30} Alpha-Liponsäure & 300--600~mg & Antioxidans \\
CoQ10 (Ubiquinol) & 200--300~mg & Mit Fett \\
D-Ribose & 5~g & ATP \\
\rowcolor{newitem!30} Kreatin & 3--5~g & ATP-Puffer \\
\rowcolor{newitem!30} Taurin & 1--2~g & Autonom \\
\midrule
\multicolumn{3}{l}{\textit{Sonstige:}} \\
Cerebokan (Ginkgo) & 80~mg & Kognition \\
PQQ & 20~mg & Mito-Biogenese \\
Pregnenolon & 30~mg & Neurosteroid \\
\rowcolor{newitem!30} Quercetin & 500--1000~mg & Mastzell \\
Vit C & 1000--2000~mg & -- \\
Vit D3 & 4000~IE & -- \\
Zink & 15--25~mg & Nicht nüchtern \\
B-Komplex (methyl.) & 1~Kps. & -- \\
\rowcolor{newitem!30} Omega-3 (EPA/DHA) & 2--4~g & Mit Fett \\
\rowcolor{newitem!30} ORS-Lösung & 250~mL & Elektrolyte \\
\end{tabular}}
\end{tcolorbox}

% MIDDAY
\begin{tcolorbox}[colback=midday, colframe=yellow!70!black, title={\textbf{MITTAGS}}, fonttitle=\bfseries, boxsep=1pt, left=2pt, right=2pt, top=1pt, bottom=1pt]
{\footnotesize
\begin{tabular}{@{}p{3.5cm}p{1.6cm}p{7.5cm}@{}}
\textbf{Substanz} & \textbf{Dosis} & \textbf{Hinweise} \\
\midrule
Mestinon & 30~mg & Mit Essen \\
NAC & 600~mg & Mit Essen \\
D-Ribose & 5~g & -- \\
L-Citrullin-Malat & 3~g & NO-Synthese \\
PEA & 600~mg & Entzündung \\
\rowcolor{newitem!30} ORS-Lösung & 250~mL & Elektrolyte \\
\end{tabular}}
\end{tcolorbox}

% EVENING
\begin{tcolorbox}[colback=evening, colframe=purple!70!black, title={\textbf{ABENDS}}, fonttitle=\bfseries, boxsep=1pt, left=2pt, right=2pt, top=1pt, bottom=1pt]
{\footnotesize
\begin{tabular}{@{}p{3.5cm}p{1.6cm}p{7.5cm}@{}}
\textbf{Substanz} & \textbf{Dosis} & \textbf{Hinweise} \\
\midrule
\rowcolor{resumeitem!30} \textbf{Cimetidin} & \textbf{200~mg} & \textbf{H2-Blocker} \\
Ivabradin & 2,5--5~mg & -- \\
Mestinon & 30~mg & Optional \\
NAC & 600~mg & Mit Essen \\
D-Ribose & 5~g & -- \\
L-Citrullin-Malat & 3~g & -- \\
PEA & 600~mg & Schmerz \\
Magnesiumglycinat & 400~mg & Entspannung \\
\rowcolor{newitem!30} Glycin & 3~g & Schlaf \\
\midrule
\textit{Vor dem Schlaf:} & & \\
LDN (Naltrexon) & 3,0~mg & Zieldosis \\
Ketotifen & 1~mg & Macht müde \\
\end{tabular}}
\end{tcolorbox}

% AS NEEDED
\begin{tcolorbox}[colback=asneeded, colframe=gray!70!black, title={\textbf{BEI BEDARF / BEDINGT}}, fonttitle=\bfseries, boxsep=1pt, left=2pt, right=2pt, top=1pt, bottom=1pt]
{\footnotesize
\begin{tabular}{@{}p{3.5cm}p{1.6cm}p{7.5cm}@{}}
\rowcolor{newitem!30} Valacyclovir & 1000~mg 2$\times$ & \textbf{Nur bei positivem EBV/HHV-6!} \\
L-Lysin & 1000~mg 2$\times$ & Bei viraler Reaktivierung \\
Quviviq (Daridorexant) & 25~mg & Schlaf-Reserve \\
Teufelskralle & 20~Trpf. & Schmerz \\
Mometason & nach Bedarf & Nasal \\
\end{tabular}}
\end{tcolorbox}

% =============================================================================
\newpage
% PAGE 2: tVNS, TITRATION STATUS, WARNINGS
% =============================================================================

\section*{Neuromodulation \& Einführungssequenz}

% tVNS
\begin{tcolorbox}[colback=blue!5, colframe=headerblue, title={\textbf{tVNS -- Tägliche Anwendung}}, fonttitle=\bfseries, boxsep=1pt, left=2pt, right=2pt, top=1pt, bottom=1pt]
{\footnotesize
\begin{tabular}{@{}p{3.2cm}p{3.2cm}p{6cm}@{}}
\textbf{Parameter} & \textbf{Einstellung} & \textbf{Notiz} \\
\midrule
Lokalisation & Cymba conchae (li.) & Oder Tragus \\
Frequenz & 25~Hz & Nicht ändern \\
Pulsbreite & 250~$\mu$s & Nicht ändern \\
Stromstärke & 0,5--2,0~mA & Leicht spürbar \\
\textbf{Dauer} & \textbf{60~min/Tag} & Minimum 35~min \\
\end{tabular}}
\end{tcolorbox}

\vspace{0.2em}

% INTRODUCTION SEQUENCE
\begin{tcolorbox}[colback=gray!15, colframe=gray!70!black, title={\textbf{Einführungssequenz -- PAUSIERT (Crash-Phase)}}, fonttitle=\bfseries, boxsep=1pt, left=2pt, right=2pt, top=1pt, bottom=1pt]
{\scriptsize\textbf{Wiederaufnahme erst nach 1 Woche stabiler Baseline ohne Verschlechterung.}}

{\small
\begin{tabular}{|c|l|l|c|}
\hline
\textbf{Woche} & \textbf{Intervention} & \textbf{Dosis} & \textbf{$\checkmark$} \\
\hline
1--2 & Cimetidin wiederaufnehmen & 200~mg 2$\times$/Tag & $\square$ \\
\hline
3--4 & NAC + Quercetin & 1800~mg + 500~mg & $\square$ \\
\hline
5--6 & ALCAR & 1000~mg morgens & $\square$ \\
\hline
7--8 & Alpha-Liponsäure & 300~mg morgens & $\square$ \\
\hline
9--10 & NR/NMN + Kreatin & 300~mg + 3~g & $\square$ \\
\hline
11--12 & Elektrolyt-Protokoll & ORS 2$\times$/Tag & $\square$ \\
\hline
\multicolumn{4}{|l|}{\textit{Nach positiver EBV/HHV-6 Diagnostik:}} \\
\hline
13+ & Valacyclovir-Trial & 1000~mg 2$\times$/Tag & $\square$ \\
\hline
\end{tabular}
}

\vspace{0.3em}
\textbf{Parallel laufend (PAUSIERT -- nicht steigern während Crash):}

{\small
\begin{tabular}{|l|c|c|c|c|c|c|}
\hline
\textbf{LDN-Titration} & 0,5~mg & 1,0~mg & 1,5~mg & 2,0~mg & 2,5~mg & 3,0~mg \\
\hline
Woche & 1--2 & 3--4 & 5--6 & 7--8 & 9--10 & 11--12 \\
\hline
Status & $\square$ & $\square$ & $\square$ & $\square$ & $\square$ & $\square$ \\
\hline
\end{tabular}

\vspace{0.2em}

\begin{tabular}{|l|c|c|c|c|}
\hline
\textbf{Mestinon} & 20~mg (1$\times$) & 30~mg (1$\times$) & 60~mg (2$\times$) & 90~mg (3$\times$) \\
\hline
Status & $\square$ & $\square$ & $\square$ & $\square$ \\
\hline
\end{tabular}

\vspace{0.2em}

\begin{tabular}{|l|c|c|c|}
\hline
\textbf{PEA} & 400~mg & 800~mg & 1200~mg \\
\hline
Status & $\square$ & $\square$ & $\square$ \\
\hline
\end{tabular}
}

\end{tcolorbox}

\vspace{0.2em}

% WARNINGS
\begin{tcolorbox}[colback=red!5, colframe=red!70!black, title={\textbf{WICHTIG -- Nicht vergessen!}}, fonttitle=\bfseries, boxsep=1pt, left=2pt, right=2pt, top=1pt, bottom=1pt]
\begin{itemize}[leftmargin=*]
    \item \textbf{Myokarditis:} Keine Stimulanzien, kardiale Belastung minimieren!
    \item \textbf{OI/POTS:} LANGSAM aufstehen -- BP sackt ab trotz Ivabradine!
    \item \textbf{Cimetidin:} CYP450-Interaktionen prüfen. Eisen 2h getrennt.
    \item \textbf{LDN:} Nicht mit Opioiden! Abends 21:00 Uhr.
    \item \textbf{Mestinon:} Immer mit Essen. Bei Durchfall Dosis reduzieren.
    \item \textbf{Ivabradin:} Keine Grapefruit!
    \item \textbf{Valacyclovir:} NUR bei positivem Virusnachweis starten!
\end{itemize}
\end{tcolorbox}

% ORS RECIPE
\begin{tcolorbox}[colback=yellow!10, colframe=yellow!50!black, title={\textbf{ORS-Rezept (Elektrolyt-Lösung)}}, fonttitle=\bfseries, boxsep=1pt, left=2pt, right=2pt, top=1pt, bottom=1pt]
\textbf{Trockenmischung:} 100~g Zucker + 15~g KCl (Low-Sodium-Salz) + 15~g NaCl (Tafelsalz)

\textbf{Anwendung:} 7~g der Mischung in 250~mL Wasser lösen. \textbf{Crash: 3$\times$/Tag} | Normal: 2$\times$/Tag
\end{tcolorbox}

\vspace{0.2em}

% PROLONGED CRASH WARNING
\begin{tcolorbox}[colback=orange!10, colframe=orange!70!black, title={\textbf{Crash >4 Wochen?}}, fonttitle=\bfseries, boxsep=1pt, left=2pt, right=2pt, top=1pt, bottom=1pt]
{\scriptsize
\textbf{Erholung:} Weniger Schlaf nötig, mehr Toleranz, klarere Kognition \quad
\textbf{Arzt bei:} Verschlechterung >2Wo, Fieber, Gewichtsverlust >5\%, Suizidgedanken

\textbf{>6 Wochen:} Labor (Ferritin, D, B12, SD), Infektabklärung (EBV/HHV-6), psychol. Unterstützung, \textbf{AKH Wien Severe-Programm}

\textbf{Hoffnung:} Frühere Erholung durch Aminosäuren + Cimetidin zeigt: Besserung ist möglich!
}
\end{tcolorbox}

\vspace{0.1em}

% QUICK SYMPTOM LOG
\begin{tcolorbox}[colback=white, colframe=gray, title={\textbf{Tages-Kurzprotokoll}}, fonttitle=\bfseries, boxsep=1pt, left=2pt, right=2pt, top=1pt, bottom=1pt]
\begin{tabular}{@{}p{2.5cm}p{1.2cm}p{2.5cm}p{1.2cm}p{2.5cm}p{1.2cm}p{2cm}p{1.2cm}@{}}
Energie (0--10): & \rule{0.8cm}{0.4pt} & Schlaf (0--10): & \rule{0.8cm}{0.4pt} & Schmerz (0--10): & \rule{0.8cm}{0.4pt} & Kognition: & \rule{0.8cm}{0.4pt} \\
\end{tabular}
\vspace{0.2em}

PEM? $\square$ Ja $\square$ Nein \quad Auslöser: \rule{3cm}{0.4pt} \quad Neue Intervention vertragen? $\square$ Ja $\square$ Nein
\end{tcolorbox}

\vspace{0.5em}
\hrule
\vspace{0.2em}
{\footnotesize\textit{Alle Änderungen erfordern ärztliche Genehmigung. Nur eine neue Intervention alle 1--2 Wochen einführen.}}

\end{document}
