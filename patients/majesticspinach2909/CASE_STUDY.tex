% Case Study: Viral-Immune-Metabolic ME/CFS Phenotype
% Patient: Exemplar Case 2909 (MajesticSpinach2909)
% Created: 2026-01-30
% Privacy: Anonymized with patient consent

\documentclass[11pt,a4paper]{article}

\usepackage[utf8]{inputenc}
\usepackage[T1]{fontenc}
\usepackage{booktabs}
\usepackage{hyperref}
\usepackage[table]{xcolor}
\usepackage{tcolorbox}
\usepackage{enumitem}
\usepackage{geometry}
\usepackage{float}  % For [H] placement specifier
\geometry{margin=2.5cm}

% Color definitions
\definecolor{keyfindingbg}{RGB}{212,237,218}
\definecolor{keyfindingborder}{RGB}{40,167,69}
\definecolor{hypothesisbg}{RGB}{217,237,247}
\definecolor{hypothesisborder}{RGB}{23,162,184}
\definecolor{cautionbg}{RGB}{248,215,218}
\definecolor{cautionborder}{RGB}{220,53,69}
\definecolor{mechanismbg}{RGB}{232,232,232}
\definecolor{mechanismborder}{RGB}{108,117,125}

\newtcolorbox{keyfinding}[1][]{
  colback=keyfindingbg, colframe=keyfindingborder,
  title={\textbf{Key Finding: #1}}, fonttitle=\bfseries, sharp corners, boxrule=1pt
}

\newtcolorbox{hypothesis}[1][]{
  colback=hypothesisbg, colframe=hypothesisborder,
  title={\textbf{Hypothesis: #1}}, fonttitle=\bfseries, sharp corners, boxrule=1pt
}

\newtcolorbox{caution}[1][]{
  colback=cautionbg, colframe=cautionborder,
  title={\textbf{Caution: #1}}, fonttitle=\bfseries, sharp corners, boxrule=1pt
}

\newtcolorbox{mechanism}[1][]{
  colback=mechanismbg, colframe=mechanismborder,
  title={\textbf{#1}}, fonttitle=\bfseries, sharp corners, boxrule=1pt
}

\title{Case Study: Viral-Immune-Metabolic ME/CFS Phenotype\\[0.5em]
\large Exemplar Case 2909 --- ``Cimetidine-Responder'' Subtype}
\author{ME/CFS Documentation Project}
\date{January 30, 2026}

\begin{document}

\maketitle

\begin{caution}{Privacy and Consent}
This case study is published with explicit patient consent. The patient (pseudonym: ``Exemplar Case 2909'') has agreed to serve as a reference case for the documentation project. Identifying details have been anonymized while preserving clinically relevant information.
\end{caution}

\tableofcontents
\newpage

%=============================================================================
\section{Clinical Presentation}
\label{sec:presentation}
%=============================================================================

\subsection{Demographics and History}

\begin{table}[H]
\centering
\begin{tabular}{ll}
\toprule
\textbf{Parameter} & \textbf{Value} \\
\midrule
Sex & Female \\
Age & 50 \\
Menopause status & Induced 2021 (post-vaccination) \\
Location & Austria (Wien region) \\
Illness duration & 12+ years (fatigue symptoms since puberty) \\
ME/CFS diagnosis & 2022 (prior misdiagnoses included fibromyalgia, iron deficiency) \\
Suspected trigger & EBV infection (patient-reported) \\
Social history & Single mother with young children during mild/moderate phase \\
\bottomrule
\end{tabular}
\caption{Demographic summary}
\end{table}

\subsection{Vaccine Adverse Events}

\begin{caution}{Significant Neurological Vaccine Reactions}
The patient experienced severe adverse events following COVID-19 vaccination in 2021:

\textbf{COVID Vaccination \#1 (2021):}
\begin{itemize}
    \item Bilateral facial paralysis (Fascialisparese)
\end{itemize}

\textbf{COVID Vaccination \#2 (2021):}
\begin{itemize}
    \item Bilateral facial paralysis (recurrence)
    \item Induced menopause
    \item Vision loss (-3 diopters)
    \item Left thigh numbness with severe nerve pain
\end{itemize}

\textbf{Clinical interpretation:} This pattern suggests molecular mimicry targeting neural tissue. The constellation of neurological symptoms (bilateral facial nerve paralysis, peripheral neuropathy, vision changes) alongside endocrine disruption (premature menopause) indicates a systemic autoimmune or inflammatory process triggered by vaccination.

\textbf{Relevance to ME/CFS phenotype:} These events may represent early manifestations of the immune dysregulation and small fiber neuropathy now documented in this patient's clinical presentation.
\end{caution}

\subsection{Severity Trajectory}

The patient's severity has fluctuated significantly over the illness course:

\begin{enumerate}
    \item \textbf{Since puberty}: Chronic fatigue
    \begin{itemize}
        \item Long-term fatigue misattributed to iron deficiency (common in women)
        \item Iron supplementation never resolved symptoms
    \end{itemize}

    \item \textbf{2014--2024 (10 years)}: Mild/Moderate severity
    \begin{itemize}
        \item Functional limitations but able to maintain activities
        \item Managed as single mother with young children during this phase
        \item Multiple healthcare encounters, diagnosis delayed until 2022
    \end{itemize}

    \item \textbf{Autumn 2024}: COVID-19 infection $\rightarrow$ Myocarditis
    \begin{itemize}
        \item Acute deterioration from moderate to severe
        \item Documented myocarditis (cardiologist-confirmed)
        \item Became largely bedbound
    \end{itemize}

    \item \textbf{2025}: Treatment response phase
    \begin{itemize}
        \item Introduction of cimetidine + amino acid supplementation
        \item \textbf{Significant improvement}: able to leave bed, short walks possible
        \item Demonstrates reversibility of severe state with appropriate intervention
    \end{itemize}

    \item \textbf{Early 2025}: Influenza infection $\rightarrow$ Further deterioration
    \begin{itemize}
        \item Significant worsening after influenza
        \item Underscores critical importance of infection prevention
    \end{itemize}

    \item \textbf{Late 2025--February 2026}: Current state
    \begin{itemize}
        \item Severity: Severe/Moderate fluctuating
        \item Currently uses \textbf{wheelchair}
        \item Unable to work for years
        \item Can only do very little with many breaks
        \item Recent sleep lab completed, awaiting results
    \end{itemize}

    \item \textbf{May 1, 2025--Present}: Severe cognitive decline
    \begin{itemize}
        \item Patient describes symptoms as ``partly like dementia''
        \item Concentration severely impaired
        \item Currently taking Cerebokan (Ginkgo) and LDA (Aripiprazole) for cognitive support
    \end{itemize}
\end{enumerate}

\begin{keyfinding}{Treatment-Responsive Severe ME/CFS}
This patient demonstrates that severe ME/CFS can show dramatic improvement with targeted intervention (cimetidine + amino acids), suggesting a treatable subtype. The relapse after infection indicates vulnerability but not irreversibility.
\end{keyfinding}

\subsection{PEM Evolution Pattern}

A striking feature of this case is the documented progression of post-exertional malaise (PEM) severity over time:

\begin{table}[H]
\centering
\begin{tabular}{lll}
\toprule
\textbf{Phase} & \textbf{PEM Recovery Time} & \textbf{Notes} \\
\midrule
Mild/Moderate (years) & 2--3 days typical & Longest: 3--4 weeks after major exertion \\
Severe (post-COVID 2024) & 2--3 weeks minimum & Often longer \\
Current (2026) & Extended, unpredictable & Requires wheelchair, cannot work \\
\bottomrule
\end{tabular}
\caption{PEM recovery timeline evolution}
\end{table}

\begin{caution}{Progressive PEM Worsening}
This pattern of progressively lengthening PEM recovery periods is a concerning trajectory that underscores:
\begin{itemize}
    \item The critical importance of \textbf{infection prevention} (COVID, influenza, other respiratory viruses)
    \item Each major infection appears to cause a \textbf{step-down in baseline function}
    \item Recovery capacity diminishes with each crash cycle
    \item Current functional status: wheelchair-dependent, unable to work, requires frequent rest breaks
\end{itemize}

\textbf{Clinical implication:} Aggressive infection prevention strategies (masking, vaccination timing considerations given prior adverse events, antivirals at first signs of infection) should be a cornerstone of management.
\end{caution}

%=============================================================================
\section{Baseline Phenotype}
\label{sec:phenotype}
%=============================================================================

\subsection{Confirmed Diagnoses}

\begin{table}[H]
\centering
\begin{tabular}{p{4.5cm}p{3.5cm}p{5.5cm}}
\toprule
\textbf{Condition} & \textbf{Status} & \textbf{Notes} \\
\midrule
ME/CFS & Confirmed (2022) & Canadian Consensus Criteria \\
POTS & Confirmed & Cardiologist-diagnosed; medically controlled \\
Histamine Intolerance (HIT) & Confirmed & Elaborate dietary management \\
Myocarditis & Confirmed (2024) & Post-COVID complication \\
MCAS & Suspected & Clinical features present; formal workup pending \\
Prediabetes & Confirmed (2025) & New diagnosis; likely LDA-related \\
Asthma & Confirmed & Managed with Ketotifen \\
Sleep disorder & Under investigation & Sleep lab completed Jan 2026; severe daytime sleepiness \\
Interstitial Cystitis (IC) & Confirmed (2012) & Precedes ME/CFS onset \\
Small Fiber Neuropathy (SFN) & Confirmed (clinical) & Biopsy refused by AKH Wien \\
L5/S1 Prolapse & Confirmed (2014) & With osteochondrosis, spondylitis \\
HWS/LWS Protrusion & Confirmed & Cervical spine straightening \\
\bottomrule
\end{tabular}
\caption{Diagnostic status}
\end{table}

\subsection{POTS Characterization}

\begin{itemize}
    \item \textbf{Pre-treatment}: Heart rate spikes to 150+ bpm upon standing
    \item \textbf{With Ivabradine 5 mg/day}: Heart rate controlled
    \item \textbf{Persistent orthostatic intolerance}: Blood pressure drops continue despite HR control
    \item \textbf{Management}: ORS solution, salt loading, compression, slow position changes
    \item \textbf{Monitoring}: Tracking watch and/or Visible app for self-monitoring
\end{itemize}

\subsection{Histamine Intolerance Features}

\begin{itemize}
    \item Multiple food intolerances and true allergies
    \item Elaborate low-histamine dietary protocol
    \item Symptoms fluctuate with dietary triggers
    \item Responds to H1/H2 blockade
    \item Butyrate-supporting diet (green bananas, resistant starch)
\end{itemize}

\subsection{Septad Component Assessment}

\begin{table}[H]
\centering
\begin{tabular}{lcc}
\toprule
\textbf{Septad Component} & \textbf{Present} & \textbf{Certainty} \\
\midrule
ME/CFS & Yes & Confirmed \\
POTS/Dysautonomia & Yes & Confirmed \\
MCAS & Probable & Clinical features, formal workup pending \\
HIT (related to MCAS) & Yes & Confirmed \\
hEDS/Hypermobility & Unknown & Not assessed \\
GI Dysmotility & Probable & Dietary management suggests involvement \\
Chronic Infection & Suspected & EBV trigger; testing pending \\
Small Fiber Neuropathy & Yes & Confirmed (clinical diagnosis; biopsy refused by AKH Wien) \\
Autoimmunity & Unknown & Not assessed \\
\bottomrule
\end{tabular}
\caption{Septad component status}
\end{table}

\subsection{Allergies and Intolerances}

The patient has an extensive allergy and intolerance profile that significantly impacts daily life and treatment options:

\textbf{Contact Allergies (Kontaktallergien):}
\begin{itemize}
    \item Tolubalsam (with cross-reactions)
    \item Potassium dichromate (Kaliumdichromat)
    \item Fragrance mix (Duftstoff-Mix)
    \item Propolis
    \item Sorbitan sesquioleate
\end{itemize}

\textbf{Note for clinicians:} Patient is an artist and must avoid certain art materials containing these allergens.

\textbf{Inhalant Allergies:}
\begin{itemize}
    \item Tree pollen: Birch
    \item Weeds: Mugwort, Ragweed
    \item Grasses: Timothy grass, Rye
    \item Molds: Cladosporium herbarum
    \item Other: House dust mite, Elderberry
\end{itemize}

\textbf{Food Allergies (True IgE-mediated):}
\begin{itemize}
    \item Wheat
    \item Peanut
    \item Soy
    \item Gluten
\end{itemize}

\textbf{Food Intolerances (Non-IgE):}
\begin{itemize}
    \item \textbf{Histamine intolerance (HEREDITARY)}: Inherited reduced DAO enzyme activity
    \item Lactose intolerance
    \item Fructose intolerance
    \item Sorbitol intolerance
\end{itemize}

\begin{keyfinding}{Hereditary Histamine Intolerance}
The patient reports that histamine intolerance is \textbf{hereditary} in her family, with reduced DAO (diamine oxidase) enzyme activity. This is distinct from acquired histamine intolerance secondary to MCAS and may represent a primary genetic predisposition that contributes to her ME/CFS phenotype.
\end{keyfinding}

%=============================================================================
\section{Treatment Response Timeline}
\label{sec:treatment}
%=============================================================================

\subsection{Breakthrough Response: Cimetidine + Amino Acids}

\begin{keyfinding}{``Got Me Out of Bed''}
Patient reports that \textbf{cimetidine combined with amino acid supplementation} produced a dramatic improvement: ``The amino acids and cimetidine got me out of bed.'' This response was reproducible and consistent with published literature on both interventions.
\end{keyfinding}

\subsubsection{Cimetidine (H2 Receptor Antagonist)}

\begin{itemize}
    \item \textbf{Dose}: 200 mg twice daily
    \item \textbf{Response}: Significant energy improvement
    \item \textbf{Mechanism hypothesis}: Immunomodulation against herpesviruses (EBV, HHV-6)
    \item \textbf{Status}: Currently paused (replaced by Ketotifen for sleep); planned to resume
\end{itemize}

\subsubsection{Amino Acid Supplementation}

Core amino acids showing benefit:
\begin{itemize}
    \item \textbf{L-Citrulline-Malate}: NO synthesis, TCA cycle support
    \item \textbf{L-Arginine}: NO precursor
    \item \textbf{L-Glutathione / NAC}: Antioxidant support
    \item \textbf{D-Ribose}: ATP precursor (15 g/day in divided doses)
\end{itemize}

\subsection{Treatment Philosophy: ``Brain First''}

The patient follows a treatment sequence prioritizing cognitive function:

\begin{keyfinding}{Standard Therapy Sequence}
\textbf{``Brain First!''} --- The recommended treatment introduction order:
\begin{enumerate}
    \item \textbf{Low-Dose Abilify (LDA)} --- Dopamine modulation for cognitive symptoms
    \item \textbf{Low-Dose Naltrexone (LDN)} --- For fatigue and pain (patient reports: ``Gamechanger!'')
    \item \textbf{Mestinon} --- For mast cells and muscle weakness
\end{enumerate}

\textbf{H1 + H2 antihistamines:} Always combine, never H1 alone.

\textbf{Patient experience with LDN:} Started October 2025, describes effect as transformative for fatigue and pain management.

\textbf{Patient experience with Mestinon:} Started recently (January 2026), already noticing improvement in muscle function.
\end{keyfinding}

\subsection{Current Protocol (February 2026)}

\subsubsection{Medications}

\begin{table}[H]
\centering
\small
\begin{tabular}{llll}
\toprule
\textbf{Medication} & \textbf{Dose} & \textbf{Timing} & \textbf{Indication} \\
\midrule
Low-dose Aripiprazole (LDA) & 1.5 mg & Morning & Dopamine modulation, cognition \\
Levocetirizine & 5 mg & Morning & H1 antihistamine \\
Ivabradine & 2.5 mg & BID & Heart rate control (POTS) \\
Ketotifen & 1 mg & Evening & Mast cell stabilizer, sleep, asthma \\
LDN (Naltrexone) & titrating & Evening & Fatigue, pain (``Gamechanger'') \\
Mestinon (Pyridostigmine) & 20 mg & With meals & Mast cells, muscle weakness \\
Cerebokan (Ginkgo) & 80 mg & Morning & Cognitive support \\
Sultanol spray & PRN & As needed & Bronchodilator, off-label for muscles \\
\midrule
\multicolumn{4}{l}{\textit{Amino acids for NO/muscle support:}} \\
L-Arginine & varies & Daily & NO precursor \\
L-Citrulline & varies & Daily & NO synthesis, muscle nitrogen \\
\midrule
\multicolumn{4}{l}{\textit{Planned/pending:}} \\
Pregnenolone & 30 mg & Morning & Neurosteroid support \\
\bottomrule
\end{tabular}
\caption{Current medication regimen (February 2026)}
\end{table}

\subsubsection{Supplements}

Key supplements targeting mitochondrial and metabolic function:
\begin{itemize}
    \item Mitochondrial support: CoQ10, PQQ, D-Ribose, Magnesium
    \item Amino acids: Citrulline-Malate, Arginine, Glutathione
    \item Anti-inflammatory: PEA (palmitoylethanolamide), Teufelskralle
    \item Cognitive: Ginkgo biloba (Cerebokan)
    \item Vitamins: C, D3, B-complex
\end{itemize}

\subsubsection{Neuromodulation}

\begin{itemize}
    \item \textbf{tVNS (transcutaneous auricular vagus nerve stimulation)}: In use, parameters being optimized
    \item Target: 60 minutes/day, 25 Hz, cymba conchae placement
\end{itemize}

\subsection{Response Pattern}

\begin{mechanism}{Treatment Response Pattern}
\textbf{Responsive interventions:}
\begin{itemize}
    \item Cimetidine: Strong energy response (``out of bed'' effect)
    \item Amino acids: Synergistic with cimetidine
    \item Ivabradine: Effective HR control for POTS
    \item Ketotifen: Sleep improvement, mast cell stabilization
\end{itemize}

\textbf{Partial responses:}
\begin{itemize}
    \item LDN: Titrating; response assessment ongoing
    \item Low-dose Aripiprazole: Helps with fatigue; metabolic side effects noted
\end{itemize}

\textbf{Not yet trialed:}
\begin{itemize}
    \item Antivirals (Valacyclovir): Planned if EBV/HHV-6 confirmed
    \item Mestinon: Scheduled for introduction
\end{itemize}
\end{mechanism}

%=============================================================================
\section{Laboratory Findings}
\label{sec:laboratory}
%=============================================================================

\subsection{January 2025 (Post-COVID Infection)}

\begin{table}[H]
\centering
\small
\begin{tabular}{llll}
\toprule
\textbf{Parameter} & \textbf{Value} & \textbf{Reference} & \textbf{Interpretation} \\
\midrule
\multicolumn{4}{l}{\textit{Lymphocyte Subsets:}} \\
Lymphocytes & 19\% & 25--40\% & Low \\
\textbf{B-cells (CD19+) abs.} & \textbf{0.05 G/l} & 0.10--0.50 & \textbf{CRITICAL: 10\% of lower limit} \\
B-cells (CD19+) rel. & 3.66\% & 6--19\% & Severely depleted \\
T-cells (CD3+) & 85\% & 55--83\% & Elevated (compensatory) \\
\midrule
\multicolumn{4}{l}{\textit{Viral Serology:}} \\
\textbf{EBV IgG (quant.)} & \textbf{596 E/ml} & <20 & \textbf{30$\times$ upper limit} \\
EBV EBNA IgG & 213 E/ml & <20 & 10$\times$ upper limit \\
\midrule
\multicolumn{4}{l}{\textit{Inflammation/Coagulation:}} \\
Fibrinogen & 3.85 g/l & 1.88--3.54 & Elevated (acute phase) \\
\midrule
\multicolumn{4}{l}{\textit{Lipids:}} \\
Total Cholesterol & 286 mg/dl & <200 & Elevated \\
HDL Cholesterol & 48 mg/dl & >50 & Low \\
Triglycerides & 278 mg/dl & <150 & Nearly 2$\times$ upper limit \\
\midrule
\multicolumn{4}{l}{\textit{Other:}} \\
Gamma-GT & 54 U/l & <40 & Elevated \\
Total IgE & 188 kU/l & <100 & Elevated (MCAS-consistent) \\
Gamma-globulin & 11.0\% & 11.1--18.8\% & Borderline low \\
\bottomrule
\end{tabular}
\caption{Laboratory findings --- January 2025 (post-COVID)}
\end{table}

\subsection{May 2025 (05.05.2025)}

\begin{table}[H]
\centering
\small
\begin{tabular}{llll}
\toprule
\textbf{Parameter} & \textbf{Value} & \textbf{Reference} & \textbf{Interpretation} \\
\midrule
Eosinophilic Cationic Protein & 30.20 µg/l & <15 & 2$\times$ upper limit \\
\bottomrule
\end{tabular}
\caption{Laboratory findings --- 05.05.2025 (eosinophilic activation post-COVID)}
\end{table}

\subsection{November 2025 (24.11.2025)}

\begin{table}[H]
\centering
\small
\begin{tabular}{lllll}
\toprule
\textbf{Parameter} & \textbf{Value} & \textbf{Reference} & \textbf{Trend} & \textbf{Interpretation} \\
\midrule
\multicolumn{5}{l}{\textit{Lymphocyte Subsets:}} \\
Lymphocytes & 21\% & 25--40\% & $\uparrow$ (was 19\%) & Slight improvement \\
T-cells (CD3+) & 84\% & 55--83\% & $\approx$ (was 85\%) & Stable elevated \\
\textbf{NK cells (CD56+CD3-)} & \textbf{6.89\%} & 7--31\% & \textit{New} & \textbf{Low --- impaired surveillance} \\
\midrule
\multicolumn{5}{l}{\textit{Viral Serology:}} \\
EBV IgG (quant.) & 514 E/ml & <20 & $\downarrow$ 14\% & Still 25$\times$ upper limit \\
EBV EBNA IgG & 156 E/ml & <20 & $\downarrow$ 27\% & Still 8$\times$ upper limit \\
\midrule
\multicolumn{5}{l}{\textit{Iron Status:}} \\
Iron & 181 µg/dl & 37--145 & -- & 125\% upper limit \\
Ferritin & ``strongly elevated'' & -- & -- & Inflammation marker? \\
\midrule
\multicolumn{5}{l}{\textit{Lipids:}} \\
Total Cholesterol & 258 mg/dl & <200 & $\downarrow$ (was 286) & Improved \\
Triglycerides & 247 mg/dl & <150 & $\downarrow$ (was 278) & Improved \\
\midrule
\multicolumn{5}{l}{\textit{Gastrointestinal:}} \\
Candida non-albicans & Positive & Negative & \textit{New} & May contribute to intestinal barrier dysfunction \\
\bottomrule
\end{tabular}
\caption{Laboratory findings --- 24.11.2025 (follow-up)}
\end{table}

\subsection{Laboratory Interpretation}

\begin{keyfinding}{Exhausted Immune Surveillance Pattern}
The laboratory constellation reveals a characteristic ``exhausted immune surveillance'' phenotype:
\begin{enumerate}
    \item \textbf{Extreme B-cell depletion} (0.05 G/l = 10\% of lower reference limit)
    \item \textbf{Massively elevated EBV titers} (30$\times$ upper limit) despite antibody production
    \item \textbf{Low NK cells} (6.89\% --- just below reference)
    \item \textbf{Compensatory T-cell elevation} (84--85\%)
\end{enumerate}

\textbf{Proposed mechanism}: Chronic EBV stimulation drives continuous B-cell differentiation into antibody-producing plasma cells, depleting the CD19+ B-cell pool. However, antibodies alone cannot clear intracellular viruses --- this requires NK cells and cytotoxic T cells. With NK cells also low, viral control fails despite high antibody production. This creates a vicious cycle.

\textbf{Why cimetidine works}: Cimetidine blocks H2 receptors on suppressor T cells, enhancing cellular immunity (T cells, NK cells). This directly addresses the cellular immunity deficit while the patient's humoral immunity (antibodies) is actually overactive but ineffective.
\end{keyfinding}

\begin{mechanism}{Trend Analysis (January $\rightarrow$ November 2025)}
\begin{itemize}
    \item EBV IgG: 596 $\rightarrow$ 514 E/ml (14\% decrease)
    \item EBV EBNA: 213 $\rightarrow$ 156 E/ml (27\% decrease)
    \item Cholesterol: 286 $\rightarrow$ 258 mg/dl (10\% decrease)
    \item Triglycerides: 278 $\rightarrow$ 247 mg/dl (11\% decrease)
    \item Lymphocytes: 19\% $\rightarrow$ 21\% (slight improvement)
\end{itemize}

The viral titers are decreasing but remain extremely elevated. The November measurement revealing low NK cells explains why viral control remains inadequate despite high antibody levels.
\end{mechanism}

\subsection{Sleep Investigation}

\begin{observation}[Sleep Lab -- January 2026]
The patient has completed a sleep study (Schlaflabor) approximately 2 weeks prior to February 2026. Results are pending.

\textbf{Presenting sleep complaints:}
\begin{itemize}
    \item Severe sleep problems (schlimme Schlafprobleme)
    \item Mandatory daytime sleeping (muss auch tagsüber oft schlafen)
    \item Non-restorative sleep pattern
\end{itemize}

\textbf{Potential findings to watch for:}
\begin{itemize}
    \item Sleep apnea (central or obstructive)
    \item Periodic limb movement disorder
    \item REM sleep behavior disorder
    \item Disrupted sleep architecture typical of ME/CFS
    \item Reduced slow-wave sleep
\end{itemize}

\textbf{Clinical relevance:} Sleep disorders are common comorbidities in ME/CFS and can significantly worsen fatigue and cognitive symptoms. Treatment of underlying sleep pathology may provide meaningful symptom relief.
\end{observation}

\subsection{Outstanding Diagnostic Workup}

\begin{table}[H]
\centering
\begin{tabular}{lll}
\toprule
\textbf{Test} & \textbf{Purpose} & \textbf{Status} \\
\midrule
EBV VCA-IgM & Active reactivation? & Pending \\
EBV EA-IgG (Early Antigen) & Recent reactivation? & Pending \\
EBV PCR (whole blood) & Active viral replication? & Pending \\
HHV-6 IgG & Co-infection? & Pending \\
Transferrin saturation, TIBC & Clarify iron status & Pending \\
B-cell absolute count (repeat) & Track recovery? & Pending \\
\bottomrule
\end{tabular}
\caption{Priority diagnostic tests --- can be performed during crash phase}
\end{table}

%=============================================================================
\section{Mechanistic Hypotheses}
\label{sec:mechanisms}
%=============================================================================

The treatment response pattern and laboratory findings suggest specific pathophysiological mechanisms that may define this patient's phenotype.

\subsection{Hypothesis 1: Viral-Immune Driver}

\begin{hypothesis}{EBV/HHV-6 Reactivation as Primary Driver}
\textbf{Supporting evidence:}
\begin{itemize}
    \item Patient suspects EBV as original trigger
    \item Cimetidine produced dramatic improvement
    \item Cimetidine has documented immunomodulatory effects against herpesviruses
    \item COVID-19 (2024) may have triggered EBV/HHV-6 reactivation (known phenomenon)
\end{itemize}

\textbf{Proposed mechanism:}
\begin{enumerate}
    \item H2 receptors on suppressor T cells inhibit cellular immunity
    \item Cimetidine blocks H2 $\rightarrow$ removes immunosuppression
    \item Enhanced T cell and NK cell activity against latent viruses
    \item Reduced viral load $\rightarrow$ reduced immune activation $\rightarrow$ symptom improvement
\end{enumerate}

\textbf{Validation needed:}
\begin{itemize}
    \item EBV Early Antigen IgG (EA-IgG)
    \item EBV VCA IgM (active reactivation)
    \item HHV-6 IgG titers
    \item EBV/HHV-6 PCR from whole blood
\end{itemize}
\end{hypothesis}

\subsection{Hypothesis 2: Amino Acid Malabsorption}

\begin{hypothesis}{MCAS/HIT $\rightarrow$ Intestinal Barrier $\rightarrow$ Malabsorption}
\textbf{Supporting evidence:}
\begin{itemize}
    \item Confirmed HIT with elaborate dietary management
    \item Strong response to exogenous amino acid supplementation
    \item If amino acids were adequately absorbed from diet, supplementation should have minimal effect
\end{itemize}

\textbf{Proposed mechanism:}
\begin{enumerate}
    \item Mast cell activation in intestinal mucosa
    \item Histamine increases intestinal permeability
    \item Mast cell proteases damage tight junctions
    \item Impaired amino acid transport/absorption
    \item Deficiency in citrulline, arginine, glycine, cysteine
    \item Secondary effects on NO synthesis and glutathione
\end{enumerate}

\textbf{Validation needed:}
\begin{itemize}
    \item Baseline serum amino acid panel
    \item Zonulin (intestinal permeability marker)
    \item LPS antibodies (bacterial translocation)
    \item Fecal calprotectin (intestinal inflammation)
\end{itemize}
\end{hypothesis}

\subsection{Hypothesis 3: NO/Urea Cycle Dysfunction}

\begin{hypothesis}{Arginine-Citrulline-NO Pathway Impairment}
\textbf{Supporting evidence:}
\begin{itemize}
    \item Strong response to L-Citrulline-Malate
    \item POTS with persistent orthostatic hypotension (NO affects vascular tone)
    \item Citrulline converts to arginine, which produces NO
\end{itemize}

\textbf{Proposed mechanism:}
\begin{enumerate}
    \item Malabsorption $\rightarrow$ low citrulline/arginine
    \item Low substrate $\rightarrow$ impaired NO synthase activity
    \item Reduced NO $\rightarrow$ endothelial dysfunction
    \item Endothelial dysfunction $\rightarrow$ orthostatic intolerance, vascular dysregulation
    \item Malate component supports TCA cycle
\end{enumerate}

\textbf{Validation needed:}
\begin{itemize}
    \item ADMA (asymmetric dimethylarginine --- NO inhibitor)
    \item Flow-mediated dilation (endothelial function)
    \item Arginine/ADMA ratio
\end{itemize}
\end{hypothesis}

\subsection{Hypothesis 4: Exhausted Immune Surveillance Pattern}

\begin{hypothesis}{B-Cell Depletion with Ineffective Humoral Response}
\textbf{Laboratory pattern (January 2025):}
\begin{itemize}
    \item B-cells (CD19+): 0.05 G/l (10\% of lower reference limit)
    \item EBV IgG: 596 E/ml (30$\times$ upper limit)
    \item NK cells: 6.89\% (below reference)
    \item T-cells (CD3+): 84--85\% (compensatory elevation)
\end{itemize}

\textbf{Interpretation:} Despite producing massive amounts of EBV antibodies, the patient cannot control viral replication. The antibody-producing arm (humoral immunity) is overactive but ineffective. The cell-killing arm (cellular immunity) is depleted.

\textbf{Proposed mechanism:}
\begin{enumerate}
    \item Chronic EBV stimulation drives continuous B-cell differentiation to plasma cells
    \item Terminal differentiation depletes the CD19+ B-cell pool (one-way process)
    \item Antibodies cannot enter infected cells --- viral clearance requires NK/T-cells
    \item NK cells also low $\rightarrow$ surveillance failure
    \item Virus persists $\rightarrow$ more B-cell stimulation $\rightarrow$ vicious cycle
\end{enumerate}

\textbf{Why cimetidine works:}
\begin{itemize}
    \item H2 receptors on suppressor T cells inhibit cellular immunity
    \item Cimetidine blocks H2 $\rightarrow$ removes suppression $\rightarrow$ enhanced T/NK cytotoxicity
    \item Better killing of EBV-infected cells
    \item Reduced viral load $\rightarrow$ reduced B-cell stimulation $\rightarrow$ potential recovery
\end{itemize}

\textbf{Testable prediction:} Patients matching this pattern (CD19+ <0.10 G/l + EBV IgG >10$\times$ ULN) should have higher cimetidine response rates than the general ME/CFS population.
\end{hypothesis}

\subsection{Hypothesis 5: Cimetidine-Antiviral Synergy}

\begin{hypothesis}{Combined Immunomodulation and Antiviral Therapy}
\textbf{Rationale:} In patients with exhausted immune surveillance, combining cimetidine with valacyclovir may be more effective than either alone.

\textbf{Proposed mechanism:}
\begin{itemize}
    \item Cimetidine: Enhances cellular immunity (killing capacity)
    \item Valacyclovir: Reduces viral replication (reduces target count)
    \item Combined: More efficient viral clearance through complementary mechanisms
\end{itemize}

\textbf{Clinical implications:}
\begin{enumerate}
    \item Resume cimetidine immediately during crash (documented response)
    \item Perform EBV diagnostic workup (VCA-IgM, EA-IgG, PCR)
    \item If reactivation confirmed: Add valacyclovir 1g BID
    \item Monitor: EBV titers monthly, B-cell count every 3 months
    \item Target: EBV IgG <200 E/ml (currently 514), B-cells >0.10 G/l (currently 0.05)
\end{enumerate}

\textbf{Testable prediction:} EBV titers should decline faster with combination therapy than with either alone. B-cell counts should begin recovering as chronic stimulation decreases.
\end{hypothesis}

\subsection{Hypothesis 6: Paradoxical Reactor as BBB Dysfunction}

\begin{hypothesis}{Blood-Brain Barrier Dysfunction Phenotype}
\textbf{Documented paradoxical/severe reactions:}
\begin{itemize}
    \item LDN (naltrexone): Depression, suicidal ideation
    \item Famotidine: Depression, suicidal ideation
    \item Low-dose prednisolone: Hypermania, psychotic states
    \item Mestinon 60 mg: Severe prostration (``scheppernd im Bett'')
\end{itemize}

\textbf{Crucially:} Cimetidine is tolerated despite famotidine reaction (same H2 blocker class). This rules out class-wide intolerance.

\textbf{Proposed mechanism:}
\begin{enumerate}
    \item Blood-brain barrier dysfunction allows increased CNS penetration
    \item Neuroinflammation alters receptor sensitivity
    \item MCAS-related medication sensitivity amplifies effects
    \item Result: CNS effects at doses that typically don't cross BBB significantly
\end{enumerate}

\textbf{Supporting evidence:}
\begin{itemize}
    \item Psychiatric reactions to medications not typically penetrating BBB well
    \item Multiple neurological adverse events documented (facial paralysis, vision loss post-vaccination)
    \item Paradoxical corticosteroid reaction suggests altered CNS pharmacodynamics
\end{itemize}

\textbf{Clinical implications:}
\begin{itemize}
    \item Start ALL new medications at 1/4 to 1/10 standard dose
    \item Daily mood monitoring with new neuroactive medications
    \item Caregiver involvement for behavioral observation
    \item Immediate discontinuation protocol if psychiatric symptoms emerge
\end{itemize}

\textbf{Validation needed:}
\begin{itemize}
    \item S100B or other BBB integrity markers
    \item MRI for subtle white matter changes
    \item Comparison of CNS-penetrant vs non-penetrant medication tolerance
\end{itemize}
\end{hypothesis}

\subsection{Integrated Cascade Model}

\begin{mechanism}{Proposed Pathophysiological Cascade}
\begin{enumerate}
    \item \textbf{Trigger}: EBV infection (12 years ago)
    \item \textbf{Persistence}: Viral latency with periodic reactivation
    \item \textbf{Immune exhaustion}: T cell dysfunction, reduced viral control
    \item \textbf{MCAS/HIT development}: Mast cell dysregulation (possibly viral-triggered)
    \item \textbf{Intestinal barrier dysfunction}: Mast cell mediators damage gut lining
    \item \textbf{Amino acid malabsorption}: Impaired nutrient absorption
    \item \textbf{Metabolic consequences}:
    \begin{itemize}
        \item NO synthesis impairment $\rightarrow$ endothelial dysfunction $\rightarrow$ POTS
        \item Glutathione depletion $\rightarrow$ oxidative stress $\rightarrow$ neuroinflammation
        \item TCA cycle impairment $\rightarrow$ mitochondrial dysfunction $\rightarrow$ fatigue
    \end{itemize}
    \item \textbf{ME/CFS phenotype}: Final common pathway of energy failure
\end{enumerate}

\textbf{Treatment targets this cascade at multiple points:}
\begin{itemize}
    \item Cimetidine: Enhances immune control of virus + reduces mast cell activation
    \item Amino acids: Bypass malabsorption by direct supplementation
    \item H1/H2 blockade + mast cell stabilizers: Reduce intestinal inflammation
    \item Antivirals (if confirmed): Address root cause
\end{itemize}
\end{mechanism}

%=============================================================================
\section{Remaining Uncertainties}
\label{sec:uncertainties}
%=============================================================================

\subsection{Tests Not Yet Performed}

\begin{table}[H]
\centering
\begin{tabular}{llll}
\toprule
\textbf{Priority} & \textbf{Test} & \textbf{Hypothesis Tested} & \textbf{Status} \\
\midrule
1 & EBV/HHV-6 serology + PCR & Viral reactivation & Pending \\
1 & Serum amino acid panel & Malabsorption & Pending \\
1 & Tryptase, plasma histamine & MCAS & Pending \\
\midrule
2 & Zonulin & Intestinal permeability & Pending \\
2 & Organic acids (urine) & Mitochondrial function & Pending \\
2 & Lactate/pyruvate ratio & Mitochondrial function & Pending \\
\midrule
3 & ADMA, FMD & Endothelial dysfunction & Pending \\
3 & Skin biopsy & Small fiber neuropathy & Pending \\
3 & T cell immunophenotyping & Immune exhaustion & Pending \\
\bottomrule
\end{tabular}
\caption{Prioritized diagnostic workup}
\end{table}

\subsection{Alternative Hypotheses}

While the viral-immune-metabolic cascade is the leading hypothesis, alternatives should be considered:

\begin{enumerate}
    \item \textbf{Primary mitochondrial disorder}: Amino acid response could reflect primary mitochondrial dysfunction rather than secondary to malabsorption

    \item \textbf{Autoimmune autonomic neuropathy}: POTS and dysautonomia could be autoimmune-mediated (anti-ganglionic receptor antibodies)

    \item \textbf{HPA axis dysfunction}: Hypocortisolism could explain some features; cortisol testing not documented

    \item \textbf{Craniocervical instability}: Not assessed; could contribute to autonomic dysfunction if hypermobility present

    \item \textbf{Primary DAO enzyme deficiency}: HIT could be primary rather than secondary to MCAS
\end{enumerate}

\subsection{Confounding Factors}

\begin{itemize}
    \item \textbf{COVID-19 + Myocarditis (2024)}: Complicates interpretation of cardiac symptoms
    \item \textbf{Multiple concurrent interventions}: Difficult to isolate individual treatment effects
    \item \textbf{Crash state}: Current crash phase limits ability to assess baseline
    \item \textbf{Medication interactions}: LDA (Aripiprazole) may have metabolic effects
\end{itemize}

%=============================================================================
\section{Implications for Phenotype Classification}
\label{sec:implications}
%=============================================================================

\subsection{Exemplar for Viral-Immune-Metabolic Cluster}

This patient exemplifies a proposed ME/CFS subtype characterized by:

\begin{enumerate}
    \item \textbf{Post-infectious onset}: Suspected viral trigger (EBV)
    \item \textbf{Immune-mast cell component}: HIT/MCAS with strong antihistamine response
    \item \textbf{Metabolic component}: Dramatic amino acid response suggesting malabsorption
    \item \textbf{Autonomic component}: POTS responsive to targeted treatment
    \item \textbf{Treatment-responsive phenotype}: Improvement with cimetidine + amino acids
\end{enumerate}

\subsection{Diagnostic Markers for This Phenotype}

Based on this case, patients with similar phenotypes might be identified by:

\begin{table}[H]
\centering
\begin{tabular}{ll}
\toprule
\textbf{Screening Question} & \textbf{Suggesting This Phenotype} \\
\midrule
Post-infectious onset? & Yes (especially EBV, COVID) \\
HIT/MCAS features? & Yes (dietary triggers, antihistamine response) \\
POTS/dysautonomia? & Yes \\
Response to H2 blockade? & Dramatic improvement \\
Response to amino acids? & Dramatic improvement \\
\bottomrule
\end{tabular}
\caption{Clinical screening for viral-immune-metabolic phenotype}
\end{table}

\subsection{Treatment Protocol Template}

For patients matching this phenotype, the following treatment sequence is suggested:

\begin{enumerate}
    \item \textbf{Phase 1 --- Confirmatory (2--4 weeks)}:
    \begin{itemize}
        \item Trial cimetidine 200 mg BID
        \item Baseline amino acid panel
        \item If positive response: proceed to Phase 2
    \end{itemize}

    \item \textbf{Phase 2 --- Foundation (Weeks 1--12)}:
    \begin{itemize}
        \item H1 + H2 + mast cell stabilizer triple therapy
        \item Amino acid supplementation (NAC, Citrulline-Malate, D-Ribose)
        \item Optimize POTS management
    \end{itemize}

    \item \textbf{Phase 3 --- Optimization (Weeks 4--24)}:
    \begin{itemize}
        \item LDN titration
        \item Amino acid dose optimization
        \item Neuromodulation (tVNS)
    \end{itemize}

    \item \textbf{Phase 4 --- Diagnostic Confirmation}:
    \begin{itemize}
        \item EBV/HHV-6 PCR
        \item If positive: antiviral trial (Valacyclovir)
        \item Intestinal permeability markers
        \item Repeat amino acid panel to assess response
    \end{itemize}
\end{enumerate}

%=============================================================================
\section{Creative Therapeutic Directions}
\label{sec:creative-directions}
%=============================================================================

Based on the patient's message and current understanding, the following novel hypotheses and therapeutic directions warrant consideration:

\subsection{Hypothesis 7: Sleep Architecture as Driver}

\begin{hypothesis}{Sleep Disorder Contributing to Disease Progression}
The patient reports severe sleep problems requiring daytime sleep, with a pending sleep lab evaluation.

\textbf{Potential mechanisms:}
\begin{itemize}
    \item Sleep apnea (if present) $\rightarrow$ intermittent hypoxia $\rightarrow$ oxidative stress $\rightarrow$ inflammation
    \item Disrupted slow-wave sleep $\rightarrow$ impaired glymphatic clearance $\rightarrow$ neuroinflammation accumulation
    \item Sleep fragmentation $\rightarrow$ HPA axis dysregulation $\rightarrow$ cortisol abnormalities
    \item Non-restorative sleep $\rightarrow$ impaired tissue repair $\rightarrow$ prolonged PEM recovery
\end{itemize}

\textbf{Clinical implication:} If sleep study reveals treatable pathology (apnea, PLMD), addressing this could reduce PEM severity and improve baseline function. This represents a potentially modifiable factor independent of the viral-immune cascade.

\textbf{Action:} Await sleep study results; consider sleep-focused interventions based on findings.
\end{hypothesis}

\subsection{Hypothesis 8: Prediabetes--ME/CFS Bidirectional Interaction}

\begin{hypothesis}{Metabolic--Immune Feedback Loop}
New diagnosis of prediabetes creates potential bidirectional interactions:

\textbf{Prediabetes $\rightarrow$ ME/CFS worsening:}
\begin{itemize}
    \item Insulin resistance $\rightarrow$ impaired glucose uptake in neurons $\rightarrow$ brain fog
    \item Hyperglycemia $\rightarrow$ advanced glycation end-products $\rightarrow$ vascular dysfunction
    \item Metabolic inflammation $\rightarrow$ amplified immune dysregulation
\end{itemize}

\textbf{ME/CFS $\rightarrow$ Prediabetes worsening:}
\begin{itemize}
    \item Forced inactivity $\rightarrow$ reduced insulin sensitivity
    \item LDA (Aripiprazole) side effect $\rightarrow$ metabolic syndrome
    \item Stress hormones $\rightarrow$ gluconeogenesis $\rightarrow$ elevated fasting glucose
\end{itemize}

\textbf{Therapeutic implications:}
\begin{itemize}
    \item Consider metformin trial (also has potential anti-inflammatory effects)
    \item Berberine as supplement alternative (blood glucose + antimicrobial)
    \item Time-restricted eating (if tolerated with medication schedule)
    \item Monitor HbA1c every 3 months
\end{itemize}
\end{hypothesis}

\subsection{Hypothesis 9: PEM Progression as Central Sensitization}

\begin{hypothesis}{Progressive PEM as CNS Sensitization Phenomenon}
The documented progression of PEM recovery times (2--3 days $\rightarrow$ 2--3 weeks) suggests possible central sensitization:

\textbf{Proposed mechanism:}
\begin{enumerate}
    \item Repeated PEM episodes $\rightarrow$ cumulative microglial activation
    \item Neuroinflammation $\rightarrow$ lowered PEM threshold
    \item Each crash $\rightarrow$ further sensitization $\rightarrow$ easier to trigger next crash
    \item Vicious cycle similar to chronic pain sensitization
\end{enumerate}

\textbf{Therapeutic implications:}
\begin{itemize}
    \item LDN (already helping) --- microglial modulation
    \item Consider low-dose ketamine nasal spray trial (NMDA antagonism, anti-sensitization)
    \item Palmitoylethanolamide (PEA) --- endocannabinoid support
    \item Aggressive pacing to prevent triggering sensitization cycles
\end{itemize}

\textbf{Testable prediction:} If this hypothesis is correct, neuroimaging (PET with microglial tracers) would show increased activation in ME/CFS patients with progressive PEM compared to stable PEM.
\end{hypothesis}

\subsection{Novel Therapeutic Directions to Explore}

\begin{keyfinding}{Emerging Therapeutic Opportunities}
Based on this case analysis, the following novel directions warrant consideration:

\textbf{1. Infection Prevention Protocol:}
\begin{itemize}
    \item N95/FFP2 masking in crowded settings (each infection causes step-down)
    \item Prophylactic antivirals during flu season (if tolerated)
    \item Immediate antiviral initiation at first symptom of respiratory infection
    \item COVID booster timing carefully considered given prior vaccine reactions
\end{itemize}

\textbf{2. ``Brain First'' Cognitive Protocol Optimization:}
\begin{itemize}
    \item Current: LDA + Ginkgo + LDN
    \item Consider adding: Lion's mane mushroom (nerve growth factor)
    \item Consider: Phosphatidylserine for membrane support
    \item Consider: Methylene blue (low-dose) for mitochondrial electron transport
\end{itemize}

\textbf{3. Mast Cell Protocol Enhancement:}
\begin{itemize}
    \item Current: H1 (Levocetirizine) + H2 (Cimetidine) + stabilizer (Ketotifen)
    \item Add: Quercetin (natural stabilizer)
    \item Consider: Cromolyn sodium (if GI symptoms prominent)
    \item Consider: Luteolin (crosses BBB, neuroinflammation)
\end{itemize}

\textbf{4. Muscle/NO Support Enhancement:}
\begin{itemize}
    \item Current: Arginine + Citrulline + Sultanol
    \item Consider: Beetroot juice/powder (nitrate loading)
    \item Consider: Carnitine (already in protocol consideration)
    \item Consider: Ribose increase during PEM recovery
\end{itemize}
\end{keyfinding}

\subsection{Conditions to Rule Out}

\begin{caution}{Differential Diagnoses Requiring Investigation}
Given the clinical picture, the following conditions should be actively ruled out:

\textbf{1. Primary Sleep Disorders:}
\begin{itemize}
    \item Sleep apnea (central or obstructive) --- sleep study pending
    \item Narcolepsy type 2 (hypersomnia, need for daytime sleep)
    \item Idiopathic hypersomnia
\end{itemize}

\textbf{2. Autoimmune Conditions:}
\begin{itemize}
    \item Autoimmune autonomic ganglionopathy (anti-ganglionic antibodies)
    \item Small fiber neuropathy (already suspected, biopsy refused)
    \item Sjögren's syndrome (fatigue, sicca symptoms, autonomic dysfunction)
\end{itemize}

\textbf{3. Endocrine Disorders:}
\begin{itemize}
    \item Adrenal insufficiency (morning cortisol, ACTH stimulation test)
    \item Thyroid dysfunction (TSH, free T3/T4, antibodies)
    \item Growth hormone deficiency (can cause fatigue, cognitive issues)
\end{itemize}

\textbf{4. Hematologic/Oncologic:}
\begin{itemize}
    \item Chronic EBV lymphoproliferative disorder (given extreme EBV titers)
    \item Smoldering myeloma (fatigue, elevated IgE)
    \item Monitor for lymphoma given chronic B-cell stimulation
\end{itemize}
\end{caution}

%=============================================================================
\section{Conclusion}
\label{sec:conclusion}
%=============================================================================

Exemplar Case 2909 represents a clinically significant ME/CFS subtype with the following key features:

\begin{keyfinding}{Case Summary}
\begin{enumerate}
    \item \textbf{Treatable phenotype}: Severe ME/CFS that responded dramatically to cimetidine + amino acids
    \item \textbf{Proposed mechanism}: Viral-driven immune dysfunction $\rightarrow$ MCAS/HIT $\rightarrow$ intestinal barrier dysfunction $\rightarrow$ amino acid malabsorption $\rightarrow$ secondary mitochondrial failure
    \item \textbf{Validation pending}: EBV/HHV-6 testing, amino acid panel, intestinal permeability markers
    \item \textbf{Treatment template}: This case provides a protocol template for similar patients
    \item \textbf{Relapse vulnerability}: Infections can trigger crashes even in treatment-responsive patients
\end{enumerate}
\end{keyfinding}

\begin{caution}{Limitations}
This is a single case report, not a controlled study. The treatment responses are patient-reported and not objectively verified. The proposed mechanisms are hypothetical and require validation through appropriate testing. This case should not be generalized without further research, but may guide hypothesis development and treatment trials in similar patients.
\end{caution}

\vspace{1em}

\textbf{Follow-up planned}: Diagnostic workup to validate mechanistic hypotheses; treatment optimization after crash recovery; longitudinal tracking of response patterns.

\end{document}
