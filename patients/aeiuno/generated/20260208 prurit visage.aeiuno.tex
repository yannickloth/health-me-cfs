% Étude de cas : Évaluation diagnostique différentielle
% Patiente : aeiuno
% Créé : 2026-02-08
% Confidentialité : Anonymisé avec consentement de la patiente
% Statut : Évaluation initiale --- diagnostic de Gibert à reconsidérer

\documentclass[11pt,a4paper]{article}

\usepackage[utf8]{inputenc}
\usepackage[T1]{fontenc}
\usepackage[french]{babel}
\usepackage{newpxtext}         % Palatino (texte)
\usepackage{newpxmath}         % Palatino (maths)
\usepackage{booktabs}
\usepackage{hyperref}
\usepackage[table]{xcolor}
\usepackage{tcolorbox}
\tcbuselibrary{breakable}
\usepackage{enumitem}
\usepackage{geometry}
\usepackage{float}
\usepackage{tabularx}
\geometry{margin=2.5cm}

% Color definitions
\definecolor{keyfindingbg}{RGB}{212,237,218}
\definecolor{keyfindingborder}{RGB}{40,167,69}
\definecolor{hypothesisbg}{RGB}{217,237,247}
\definecolor{hypothesisborder}{RGB}{23,162,184}
\definecolor{cautionbg}{RGB}{248,215,218}
\definecolor{cautionborder}{RGB}{220,53,69}
\definecolor{mechanismbg}{RGB}{232,232,232}
\definecolor{mechanismborder}{RGB}{108,117,125}
\definecolor{questionbg}{RGB}{255,243,205}
\definecolor{questionborder}{RGB}{255,193,7}

\newtcolorbox{keyfinding}[1][]{
  colback=keyfindingbg, colframe=keyfindingborder,
  adjusted title={\textbf{Constat clé~: #1}}, fonttitle=\bfseries, sharp corners, boxrule=1pt,
  breakable
}

\newtcolorbox{hypothesis}[1][]{
  colback=hypothesisbg, colframe=hypothesisborder,
  adjusted title={\textbf{Hypothèse~: #1}}, fonttitle=\bfseries, sharp corners, boxrule=1pt,
  breakable
}

\newtcolorbox{caution}[1][]{
  colback=cautionbg, colframe=cautionborder,
  adjusted title={\textbf{Attention~: #1}}, fonttitle=\bfseries, sharp corners, boxrule=1pt,
  breakable
}

\newtcolorbox{mechanism}[1][]{
  colback=mechanismbg, colframe=mechanismborder,
  adjusted title={\textbf{#1}}, fonttitle=\bfseries, sharp corners, boxrule=1pt,
  breakable
}

\newtcolorbox{question}[1][]{
  colback=questionbg, colframe=questionborder,
  adjusted title={\textbf{À clarifier~: #1}}, fonttitle=\bfseries, sharp corners, boxrule=1pt,
  breakable
}

\title{Étude de cas~: Évaluation diagnostique différentielle\\[0.5em]
\large Aeiuno --- Dermatose récidivante chronique avec atteinte muqueuse nasale\\[0.3em]
\normalsize Diagnostic initial de maladie de Gibert à reconsidérer}
\author{Document de travail --- À discuter avec un dermatologue}
\date{8 février 2026}

\begin{document}

\maketitle

\begin{caution}{Document préliminaire}
Cette étude de cas repose sur des informations \textbf{limitées} recueillies lors d'échanges informels. Aucun diagnostic définitif n'est posé. Les hypothèses ci-dessous orientent la démarche diagnostique et ne doivent pas être interprétées comme des conclusions. \textbf{Une biopsie cutanée est indispensable} pour établir un diagnostic histologique.
\end{caution}

\tableofcontents
\newpage

%=============================================================================
\section{Présentation clinique}
\label{sec:presentation}
%=============================================================================

\subsection{Données démographiques}

\begin{table}[H]
\centering
\begin{tabularx}{\textwidth}{l>{\raggedright\arraybackslash}X}
\toprule
\textbf{Paramètre} & \textbf{Valeur} \\
\midrule
Pseudonyme & Aeiuno \\
Sexe & Femme \\
Date de naissance & 8 septembre 1988 \\
Âge & 37 ans \\
Âge de début & $\sim$20 ans (soit $\sim$17 ans d'évolution) \\
Diagnostic initial & Maladie de Gibert (pityriasis rosea) \\
Durée de la maladie & Récidives annuelles depuis \textbf{$\sim$17 ans} \\
Épisodes habituels & Légers, auto-résolutifs, environ 1 par an \\
Épisode actuel & $>$ 3 mois, persistant (contrairement aux épisodes précédents) \\
Atteinte muqueuse & Oui --- extension à l'intérieur du nez \\
Parcours médical & Médecins généralistes consultés, sans orientation diagnostique claire \\
\bottomrule
\end{tabularx}
\caption{Résumé démographique}
\end{table}

\subsection{Symptômes et signes rapportés}

\begin{table}[H]
\centering
\begin{tabular}{p{5cm}p{8cm}}
\toprule
\textbf{Signe / Symptôme} & \textbf{Détails} \\
\midrule
Dermatose récidivante & Épisodes annuels depuis l'âge de $\sim$20 ans ($\sim$17 ans d'évolution). Épisodes habituellement légers et auto-résolutifs \\
Épisode actuel prolongé & $>$ 3 mois de durée, \textbf{persistant} contrairement aux épisodes précédents \\
Atteinte muqueuse nasale & Extension des lésions à l'intérieur du nez (muqueuse nasale) \\
Mode de propagation & \textbf{Centrifuge séquentiel}~: apparition d'un petit bouton très localisé, extrêmement prurigineux $\rightarrow$ le bouton disparaît $\rightarrow$ la zone \textbf{autour} est atteinte, avec un prurit moindre. Nouveaux boutons toujours à proximité de la zone déjà touchée (extension par contiguïté) \\
Prurit & Intense au stade initial (bouton localisé), diminuant lors de l'extension périphérique \\
Facteurs déclenchants & Aucun facteur déclenchant identifié \\
\bottomrule
\end{tabular}
\caption{Symptômes et signes rapportés}
\end{table}

\subsection{Chronologie}

\begin{mechanism}{Évolution temporelle}
\begin{itemize}
    \item \textbf{Vers l'âge de 20 ans ($\sim$2009)}~: Première manifestation d'une dermatose, diagnostiquée comme maladie de Gibert (pityriasis rosea) par un médecin généraliste
    \item \textbf{De 20 à 37 ans ($\sim$17 ans)}~: Épisodes récidivants \textbf{chaque année}, habituellement \textbf{légers} et se résolvant spontanément. Le diagnostic de Gibert est maintenu malgré les récidives annuelles
    \item \textbf{Épisode actuel (depuis $>$ 3 mois)}~: Poussée \textbf{persistante}, ne se résolvant pas spontanément contrairement aux épisodes précédents. Rupture du schéma habituel
    \item \textbf{Extension récente}~: Les lésions s'étendent à l'\textbf{intérieur du nez} (atteinte muqueuse --- jamais observée auparavant~?)
    \item \textbf{Consultation de médecins généralistes}~: Aucune orientation diagnostique claire proposée
\end{itemize}

\textbf{Éléments notables}~:
\begin{enumerate}[nosep]
    \item La \textbf{durée de 17 ans} avec récidives annuelles exclut définitivement la maladie de Gibert
    \item Le \textbf{changement de comportement} de l'épisode actuel (persistance $>$ 3 mois, atteinte muqueuse) suggère soit une progression de la maladie sous-jacente, soit un facteur aggravant nouveau
    \item L'absence de facteur déclenchant identifié est compatible avec une maladie endogène (auto-immune, lymphoproliférative) plutôt qu'exogène (infectieuse, allergique)
\end{enumerate}
\end{mechanism}

\subsection{Mode de propagation des lésions}

\begin{keyfinding}{Séquence de propagation --- indice diagnostique majeur}
La patiente décrit un mode de propagation \textbf{centrifuge séquentiel} très caractéristique~:

\begin{enumerate}
    \item \textbf{Phase 1 --- Lésion initiale}~: Apparition d'un petit bouton très localisé, \textbf{extrêmement prurigineux}
    \item \textbf{Phase 2 --- Résolution centrale}~: Le bouton initial disparaît
    \item \textbf{Phase 3 --- Extension périphérique}~: La zone \textbf{autour} du bouton initial est atteinte, avec un prurit \textbf{moindre} que la lésion initiale
\end{enumerate}

De plus, les nouveaux boutons apparaissent \textbf{toujours à proximité} de la zone déjà touchée, ce qui indique une \textbf{extension par contiguïté} plutôt qu'une dissémination à distance.

Ce mode de propagation centrifuge avec résolution centrale et extension par contiguïté est un \textbf{signe sémiologique important}. Il oriente vers des diagnostics spécifiques~:

\begin{itemize}
    \item \textbf{Érythème annulaire centrifuge (EAC)}~: Propagation centrifuge avec résolution centrale --- c'est la \textbf{définition même} de l'EAC. Hypothèse à ajouter en priorité (voir section~\ref{sec:differentials})
    \item \textbf{Dermatophytose (tinea)}~: Extension annulaire centrifuge classique, mais exclue si pas de réponse aux antifongiques et si atteinte muqueuse
    \item \textbf{Psoriasis annulaire}~: Variante annulaire du psoriasis avec extension centrifuge
    \item \textbf{Granulome annulaire}~: Expansion centrifuge avec résolution centrale, mais habituellement non prurigineux
    \item \textbf{Mycosis fongoïde}~: Peut montrer une extension centrifuge dans certaines formes
\end{itemize}

\textbf{Le gradient de prurit} (intense au centre/initial $\rightarrow$ moindre en périphérie) suggère que le \textbf{front actif} de la maladie est au point d'apparition, et que l'extension périphérique représente une réaction inflammatoire secondaire ou résiduelle.
\end{keyfinding}

%=============================================================================
\section{Pourquoi le diagnostic de maladie de Gibert est inadéquat}
\label{sec:gibert-inadequate}
%=============================================================================

\begin{keyfinding}{Incompatibilité avec la maladie de Gibert}
Le diagnostic de maladie de Gibert (pityriasis rosea) ne peut \textbf{pas} expliquer le tableau clinique d'Aeiuno. Trois critères majeurs sont violés~:

\begin{table}[H]
\centering
\begin{tabular}{p{4cm}p{4.5cm}p{4.5cm}}
\toprule
\textbf{Critère} & \textbf{Gibert classique} & \textbf{Cas d'Aeiuno} \\
\midrule
Durée & 6--12 semaines, auto-limitante & $>$ 3 mois (épisode actuel) \\
Récidive & Quasi jamais (immunité acquise) & Récidives régulières depuis des années \\
Atteinte muqueuse & Absente & Extension intranasale \\
\bottomrule
\end{tabular}
\end{table}

\textbf{Conclusion}~: Le diagnostic de maladie de Gibert doit être \textbf{abandonné}. Une nouvelle évaluation diagnostique est nécessaire.
\end{keyfinding}

\begin{mechanism}{Rappel --- Maladie de Gibert}
La maladie de Gibert (pityriasis rosea) est~:
\begin{itemize}[nosep]
    \item Une dermatose virale bénigne (HHV-6 / HHV-7 suspectés)
    \item \textbf{Auto-limitante} en 6--12 semaines (rarement plus)
    \item Caractérisée par un médaillon initial (herald patch) suivi d'une éruption secondaire en «~sapin de Noël~»
    \item Ne récidivant \textbf{quasi jamais} ($<$2\% des cas) grâce à l'immunité acquise
    \item N'atteignant \textbf{jamais} les muqueuses
\end{itemize}

\textbf{Tout} tableau dépassant ces caractéristiques impose de reconsidérer le diagnostic.
\end{mechanism}

%=============================================================================
\section{Diagnostics différentiels}
\label{sec:differentials}
%=============================================================================

Les diagnostics suivants sont classés par probabilité estimée, en tenant compte de~: chronicité ($\sim$17 ans), récidives annuelles, atteinte muqueuse nasale, et mode de propagation centrifuge séquentiel avec extension par contiguïté.

\subsection{Priorité 1~: Hypothèses principales}

\begin{hypothesis}{Érythème annulaire centrifuge (EAC)}
\textbf{Probabilité estimée}~: \textbf{Élevée} --- le mode de propagation décrit correspond à la définition même de l'EAC

L'érythème annulaire centrifuge est une dermatose réactionnelle caractérisée par~:
\begin{itemize}
    \item \textbf{Propagation centrifuge}~: lésion initiale qui s'étend en anneau, avec résolution centrale --- \textbf{exactement le schéma décrit par la patiente}
    \item \textbf{Récidivant}~: évolue par poussées récurrentes sur des mois ou des années
    \item \textbf{Prurit variable}~: souvent prurigineux au stade initial, diminuant avec l'extension (compatible avec le gradient de prurit décrit)
    \item \textbf{Extension par contiguïté}~: les nouvelles lésions apparaissent à proximité des zones déjà atteintes
\end{itemize}

\textbf{Arguments en faveur}~:
\begin{enumerate}[nosep]
    \item Mode de propagation centrifuge avec résolution centrale (pathognomonique)
    \item Gradient de prurit (intense au centre $\rightarrow$ moindre en périphérie)
    \item Récidives annuelles depuis $\sim$17 ans
    \item Nouvelles lésions apparaissant toujours à proximité de la zone déjà atteinte (extension par contiguïté)
\end{enumerate}

\textbf{Point important --- l'EAC est souvent un \textit{marqueur} d'une pathologie sous-jacente}~:
\begin{itemize}[nosep]
    \item Infections fongiques (dermatophytoses à distance --- «~dermatophytide~»)
    \item Néoplasies (lymphomes, tumeurs solides) --- l'EAC peut être \textbf{paranéoplasique}
    \item Médicaments
    \item Maladies auto-immunes (lupus, Sjögren)
    \item Infections chroniques (hépatite, EBV)
    \item Idiopathique dans $\sim$70\% des cas
\end{itemize}

\textbf{La persistance de l'épisode actuel ($>$ 3 mois) et l'atteinte muqueuse nasale} pourraient signaler l'apparition ou l'aggravation d'un facteur sous-jacent. Un bilan étiologique est indiqué.

\textbf{Bilan recommandé}~:
\begin{itemize}[nosep]
    \item \textbf{Biopsie cutanée}~: infiltrat lymphocytaire péri-vasculaire superficiel et profond en «~manchon~» (coat-sleeve pattern)
    \item Recherche de dermatophytose à distance (pieds, ongles, aines)
    \item Bilan infectieux (hépatites, EBV)
    \item Bilan auto-immun (ANA, anti-SSA)
    \item \textbf{Bilan néoplasique} si EAC confirmé~: NFS, LDH, électrophorèse des protéines, scanner thoraco-abdomino-pelvien (surtout si épisode actuel réfractaire)
\end{itemize}

\textbf{Traitement si confirmé}~:
\begin{itemize}[nosep]
    \item \textit{Symptomatique}~: dermocorticoïdes topiques, antihistaminiques oraux
    \item \textit{Étiologique}~: traitement de la cause sous-jacente si identifiée (résolution de l'EAC attendue)
    \item \textit{Formes réfractaires}~: méthotrexate, photothérapie UVB
\end{itemize}
\end{hypothesis}

\begin{hypothesis}{Psoriasis (en gouttes ou en plaques)}
\textbf{Probabilité estimée}~: \textbf{Élevée}

Le psoriasis est le diagnostic différentiel le plus probable~:
\begin{itemize}
    \item \textbf{Chronique par définition}~: évolue par poussées et rémissions sur des années, voire toute la vie
    \item \textbf{Souvent confondu avec un Gibert}~: les formes en gouttes (psoriasis guttata) produisent de petites plaques ovales, squameuses, disséminées sur le tronc --- morphologie très similaire au Gibert
    \item \textbf{Atteinte muqueuse possible}~: le psoriasis peut toucher les muqueuses nasales (psoriasis muqueux), bien que ce soit moins fréquent que les atteintes cutanées
    \item \textbf{Récidives régulières}~: caractéristique du psoriasis
    \item \textbf{Facteurs déclenchants typiques}~: stress, infections ORL (streptocoques), médicaments (bêtabloquants, lithium, AINS), traumatisme cutané (phénomène de Koebner)
\end{itemize}

\textbf{Arguments en faveur}~:
\begin{enumerate}[nosep]
    \item Chronicité et récidives sur des années
    \item Morphologie possiblement similaire au Gibert (d'où la confusion initiale)
    \item Atteinte intranasale compatible
    \item Très haute prévalence (2--3\% de la population)
\end{enumerate}

\textbf{Bilan recommandé}~:
\begin{itemize}[nosep]
    \item Examen dermatologique complet (recherche de plaques typiques aux coudes, genoux, cuir chevelu, plis interfessiers, ongles)
    \item \textbf{Biopsie cutanée}~: histologie caractéristique (parakératose, acanthose, micro-abcès de Munro)
    \item Recherche d'atteinte unguéale (pitting, onycholyse, taches d'huile)
    \item Recherche de signes articulaires (raideur matinale, gonflement des doigts)
\end{itemize}

\textbf{Traitement si confirmé}~:
\begin{itemize}[nosep]
    \item \textit{Formes légères à modérées}~: dermocorticoïdes, analogues de la vitamine D (calcipotriol), association calcipotriol/bétaméthasone
    \item \textit{Atteinte muqueuse nasale}~: émollients nasaux, corticoïdes nasaux doux (sous supervision dermatologique)
    \item \textit{Formes modérées à sévères}~: photothérapie UVB, méthotrexate, ciclosporine
    \item \textit{Formes réfractaires}~: biothérapies (anti-TNF, anti-IL-17, anti-IL-23)
\end{itemize}
\end{hypothesis}

\begin{hypothesis}{Lichen plan}
\textbf{Probabilité estimée}~: \textbf{Modérée à élevée}

Le lichen plan est une dermatose inflammatoire chronique qui mérite une attention particulière~:
\begin{itemize}
    \item \textbf{Chronique et récidivant}~: peut évoluer sur des mois, voire des années, avec des poussées récurrentes
    \item \textbf{Atteinte muqueuse fréquente}~: c'est l'un des diagnostics où l'atteinte muqueuse est \textbf{caractéristique} (bouche, nez, organes génitaux). Le lichen plan muqueux est souvent plus tenace que la forme cutanée
    \item \textbf{Morphologie variable}~: plaques violacées polygonales (forme classique), mais aussi formes annulaires, linéaires, ou érosives
    \item \textbf{Peut mimer un Gibert}~: certaines formes disséminées peuvent être confondues avec un pityriasis rosea
\end{itemize}

\textbf{Arguments en faveur}~:
\begin{enumerate}[nosep]
    \item Atteinte muqueuse nasale (très compatible)
    \item Chronicité et récidives
    \item Association connue avec l'hépatite C (à rechercher)
\end{enumerate}

\textbf{Bilan recommandé}~:
\begin{itemize}[nosep]
    \item \textbf{Biopsie cutanée}~: histologie caractéristique (infiltrat lichénoïde en bande, dégénérescence vacuolaire de la basale, corps de Civatte)
    \item Examen de la muqueuse buccale (réseau de Wickham~?)
    \item Sérologie hépatite C (association connue)
    \item Biopsie de la muqueuse nasale si lésions accessibles
\end{itemize}

\textbf{Traitement si confirmé}~:
\begin{itemize}[nosep]
    \item \textit{Cutané}~: dermocorticoïdes puissants, photothérapie UVB
    \item \textit{Muqueux}~: corticoïdes topiques, tacrolimus topique (hors AMM mais efficace)
    \item \textit{Formes sévères}~: corticothérapie orale courte, acitrétine, méthotrexate
\end{itemize}
\end{hypothesis}

\begin{hypothesis}{Mycosis fongoïde (lymphome cutané T)}
\textbf{Probabilité estimée}~: \textbf{Modérée} --- mais ne doit \textbf{pas} être manqué

Le mycosis fongoïde est un lymphome T cutané de bas grade qui doit être évoqué~:
\begin{itemize}
    \item \textbf{Évolution typiquement sur des années}~: c'est une maladie lente, souvent précédée d'une phase de «~pré-mycosis~» avec des plaques non spécifiques pendant 5 à 20 ans
    \item \textbf{Souvent diagnostiqué tardivement}~: confondu avec eczéma, psoriasis, Gibert, ou dermatite pendant des années
    \item \textbf{Rémissions et récidives}~: les plaques peuvent s'améliorer spontanément puis revenir, simulant une maladie bénigne récidivante
    \item \textbf{Atteinte muqueuse possible} dans les formes avancées
    \item \textbf{Extension progressive}~: les lésions tendent à s'étendre au fil du temps
\end{itemize}

\textbf{Arguments en faveur}~:
\begin{enumerate}[nosep]
    \item Récidives sur \textbf{des années} (parcours classique du mycosis fongoïde)
    \item Diagnostic initial erroné maintenu pendant longtemps (situation typique)
    \item Extension progressive aux muqueuses
\end{enumerate}

\textbf{Signaux d'alerte}~:
\begin{itemize}[nosep]
    \item Plaques fixes (reviennent toujours aux mêmes endroits)
    \item Plaques d'aspects variables (certaines plus infiltrées que d'autres)
    \item Prurit persistant
    \item Réponse incomplète aux traitements classiques
\end{itemize}

\textbf{Bilan recommandé}~:
\begin{itemize}[nosep]
    \item \textbf{Biopsie cutanée avec immunohistochimie}~: recherche de lymphocytes T atypiques, épidermotropisme, micro-abcès de Pautrier
    \item \textbf{Plusieurs biopsies} peuvent être nécessaires (les premières biopsies sont souvent non concluantes)
    \item NFS avec formule (recherche de cellules de Sézary circulantes)
    \item LDH
    \item Scanner thoraco-abdomino-pelvien si biopsie positive
\end{itemize}

\begin{caution}{Importance du diagnostic précoce}
Le mycosis fongoïde au stade précoce (patches/plaques) a un \textbf{excellent pronostic}~: survie comparable à la population générale. Mais le diagnostic tardif (stade tumoral) aggrave considérablement le pronostic. C'est pourquoi une biopsie est urgente même si les lésions semblent banales.
\end{caution}
\end{hypothesis}

\subsection{Priorité 2~: Hypothèses à explorer activement}

\begin{hypothesis}{Dermatite séborrhéique chronique}
\textbf{Probabilité estimée}~: Modérée

\begin{itemize}
    \item Chronique et récidivante par nature
    \item Touche classiquement le visage, le cuir chevelu, les ailes du nez
    \item L'extension intranasale est atypique mais possible
    \item La forme disséminée peut mimer un pityriasis rosea
\end{itemize}

\textbf{Arguments contre}~: L'extension intranasale franche est inhabituelle. La dermatite séborrhéique est généralement facilement reconnaissable par un dermatologue.

\textbf{Bilan}~: Examen clinique dermatologique (localisation typique~: sillons nasogéniens, sourcils, cuir chevelu). Biopsie si doute.
\end{hypothesis}

\begin{hypothesis}{Syphilis secondaire}
\textbf{Probabilité estimée}~: Faible à modérée --- mais doit être \textbf{systématiquement exclue}

La syphilis secondaire est surnommée «~la grande simulatrice~»~:
\begin{itemize}
    \item Peut mimer \textbf{parfaitement} un pityriasis rosea (éruption papulosquameuse disséminée)
    \item Atteinte muqueuse fréquente (plaques muqueuses, «~plaques fauchées~»)
    \item Peut évoluer par poussées si non traitée (syphilis tertiaire)
    \item La récidive sur des années pourrait correspondre à une syphilis latente avec poussées secondaires
\end{itemize}

\textbf{Bilan}~: Sérologie syphilitique (TPHA + VDRL ou RPR) --- \textbf{simple prise de sang, résultat rapide, à faire en priorité}
\end{hypothesis}

\begin{hypothesis}{Lupus cutané (lupus discoïde ou subaigu)}
\textbf{Probabilité estimée}~: Faible à modérée

\begin{itemize}
    \item Chronique et récidivant
    \item Le lupus cutané subaigu peut produire des lésions annulaires ou papulosquameuses
    \item Atteinte muqueuse nasale possible (ulcérations, plaques)
    \item Peut être isolé (lupus cutané pur) ou associé à un lupus systémique
\end{itemize}

\textbf{Bilan}~: ANA, anti-ADN natif, anti-SSA/Ro (associé au lupus subaigu), complément C3/C4, biopsie cutanée avec immunofluorescence directe
\end{hypothesis}

\begin{hypothesis}{Pityriasis lichénoïde chronique (PLC) / Pityriasis lichénoïde et varioliforme aigu (PLEVA)}
\textbf{Probabilité estimée}~: Modérée

Le pityriasis lichénoïde est une dermatose souvent méconnue qui peut mimer un Gibert~:
\begin{itemize}
    \item \textbf{Évolue par poussées récurrentes} sur des mois, voire des années
    \item Les lésions papulosquameuses peuvent être confondues avec un pityriasis rosea
    \item Deux formes~: PLC (chronique, plaques brunâtres) et PLEVA (aigu, lésions nécrotiques)
    \item L'atteinte muqueuse est rare mais décrite
    \item \textbf{Considéré comme un spectre pré-lymphomateux} par certains auteurs (surveillance nécessaire)
\end{itemize}

\textbf{Bilan}~: Biopsie cutanée (infiltrat lymphocytaire péri-vasculaire, parakératose). Immunohistochimie si atypies.
\end{hypothesis}

\subsection{Priorité 3~: Diagnostics moins probables à garder en tête}

\begin{hypothesis}{Dermatophytose (tinea corporis) disséminée}
\textbf{Probabilité estimée}~: Faible

\begin{itemize}
    \item Les dermatophytoses annulaires (tinea corporis) peuvent être confondues avec un Gibert
    \item Les récidives sont possibles en cas de recontamination ou d'immunodépression
    \item L'atteinte intranasale serait très inhabituelle
\end{itemize}

\textbf{Bilan}~: Examen mycologique (grattage cutané avec examen direct et culture)
\end{hypothesis}

\begin{hypothesis}{Eczéma nummulaire chronique}
\textbf{Probabilité estimée}~: Faible

\begin{itemize}
    \item Plaques arrondies pouvant mimer un Gibert
    \item Chronique et récidivant
    \item L'atteinte intranasale serait atypique
\end{itemize}

\textbf{Bilan}~: Examen clinique, biopsie si doute (spongiose caractéristique)
\end{hypothesis}

%=============================================================================
\section{Synthèse diagnostique}
\label{sec:synthesis}
%=============================================================================

\begin{mechanism}{Hiérarchie des diagnostics par probabilité}
Tenant compte de l'ensemble des éléments~: récidives annuelles depuis $\sim$17 ans, épisode actuel persistant $>$ 3 mois, atteinte muqueuse nasale, propagation centrifuge séquentielle avec résolution centrale, extension par contiguïté, et gradient de prurit.

\begin{enumerate}
    \item \textbf{Érythème annulaire centrifuge (EAC)}~: \textbf{Probabilité élevée} --- le mode de propagation décrit (bouton initial prurigineux $\rightarrow$ résolution $\rightarrow$ extension périphérique moins prurigineuse, avec lésions toujours à proximité de la zone atteinte) correspond à la \textbf{définition sémiologique} de l'EAC. Récidivant sur des années. \textbf{Attention}~: l'EAC est souvent un marqueur de pathologie sous-jacente (dermatophytose à distance, néoplasie, auto-immunité) --- un bilan étiologique est indispensable.

    \item \textbf{Psoriasis}~: \textbf{Probabilité élevée} --- chronique, récidivant, morphologie pouvant mimer un Gibert, atteinte muqueuse possible. La forme annulaire du psoriasis peut montrer une extension centrifuge. Diagnostic très fréquent.

    \item \textbf{Lichen plan}~: \textbf{Probabilité modérée à élevée} --- l'atteinte muqueuse nasale est un argument fort. Le lichen plan muqueux est souvent plus tenace que la forme cutanée.

    \item \textbf{Mycosis fongoïde}~: \textbf{Probabilité modérée} --- à ne pas sous-estimer. L'évolution sur 17 ans avec un diagnostic initial erroné est un parcours classique du mycosis fongoïde. Les conséquences d'un diagnostic manqué justifient une biopsie avec immunohistochimie.

    \item \textbf{Pityriasis lichénoïde chronique}~: \textbf{Probabilité modérée} --- souvent méconnu, évolue par poussées récurrentes. Spectre pré-lymphomateux possible.

    \item \textbf{Syphilis}~: \textbf{Probabilité faible} --- récidives sur 17 ans (depuis l'âge de 20 ans) rendent une syphilis non traitée peu probable, mais une sérologie reste justifiée pour l'exclure formellement.

    \item \textbf{Lupus cutané}~: \textbf{Probabilité faible à modérée} --- à évoquer si ANA positifs ou photosensibilité.
\end{enumerate}

\textbf{Point clé}~: L'EAC et le psoriasis sont en tête de la hiérarchie. Tous les diagnostics nécessitent un \textbf{dermatologue} et une \textbf{biopsie cutanée}.
\end{mechanism}

%=============================================================================
\section{Bilan recommandé}
\label{sec:workup}
%=============================================================================

\subsection{Examen clinique dermatologique (priorité absolue)}

\begin{keyfinding}{Éléments à documenter lors de l'examen dermatologique}
\begin{enumerate}
    \item \textbf{Morphologie précise des lésions}~:
    \begin{itemize}[nosep]
        \item Papules, plaques, macules~?
        \item Squames~? Type (fines, épaisses, argentées, grasses)~?
        \item Couleur (roses, rouges, violacées, brunes)~?
        \item Bords (nets, flous, surélevés)~?
        \item Disposition (annulaire, linéaire, en sapin de Noël)~?
    \end{itemize}

    \item \textbf{Distribution}~:
    \begin{itemize}[nosep]
        \item Topographie exacte (tronc, membres, visage, cuir chevelu)~?
        \item Symétrique ou asymétrique~?
        \item Respect ou atteinte des plis~?
        \item Atteinte des paumes / plantes~?
    \end{itemize}

    \item \textbf{Atteinte muqueuse nasale}~:
    \begin{itemize}[nosep]
        \item Aspect des lésions intranasales (plaques, érosions, croûtes)~?
        \item Unilatéral ou bilatéral~?
        \item Extension vers d'autres muqueuses (bouche, organes génitaux)~?
    \end{itemize}

    \item \textbf{Signes associés à rechercher}~:
    \begin{itemize}[nosep]
        \item Atteinte unguéale (pitting, dystrophie) $\rightarrow$ psoriasis
        \item Atteinte du cuir chevelu $\rightarrow$ psoriasis, lupus
        \item Réseau de Wickham sur les muqueuses $\rightarrow$ lichen plan
        \item Photosensibilité $\rightarrow$ lupus
        \item Adénopathies $\rightarrow$ syphilis, lymphome
        \item Phénomène de Koebner (lésions sur zones de traumatisme) $\rightarrow$ psoriasis, lichen plan
    \end{itemize}
\end{enumerate}
\end{keyfinding}

\subsection{Biopsie cutanée (indispensable)}

\begin{caution}{La biopsie est l'examen clé}
Après des années de récidives sans diagnostic clair, la biopsie cutanée est l'\textbf{examen le plus important}. C'est un geste simple~:
\begin{itemize}[nosep]
    \item Anesthésie locale
    \item Punch biopsie (3--4 mm) ou biopsie au bistouri
    \item Durée~: 5--10 minutes
    \item Quelques points de suture
    \item Résultat histologique en 1--2 semaines
\end{itemize}

\textbf{Recommandations}~:
\begin{itemize}[nosep]
    \item Biopsier une lésion \textbf{récente et active} (pas une lésion ancienne en voie de résolution)
    \item Demander une \textbf{immunohistochimie} (permet de distinguer les infiltrats lymphocytaires bénins des lymphomes)
    \item Demander une \textbf{immunofluorescence directe} (IFD) si lupus ou lichen plan suspectés
    \item Envisager \textbf{deux biopsies} sur des lésions d'aspects différents si la morphologie est hétérogène
    \item Si la première biopsie est non concluante, une \textbf{seconde biopsie} est justifiée (le mycosis fongoïde nécessite souvent plusieurs biopsies avant confirmation)
\end{itemize}
\end{caution}

\subsection{Bilan biologique de première intention}

\begin{table}[H]
\centering
\small
\begin{tabular}{p{4.5cm}p{4.5cm}p{4cm}}
\toprule
\textbf{Examen} & \textbf{Ce qu'on cherche} & \textbf{Hypothèse testée} \\
\midrule
\multicolumn{3}{l}{\textit{Sérologie (prioritaire)~:}} \\
TPHA + VDRL & Syphilis & Syphilis secondaire \\
\midrule
\multicolumn{3}{l}{\textit{Hématologie~:}} \\
NFS avec formule & Lymphocytose, cellules atypiques & Lymphome cutané \\
VS, CRP & Syndrome inflammatoire & Orientation globale \\
LDH & Marqueur tumoral & Lymphome \\
\midrule
\multicolumn{3}{l}{\textit{Immunologie~:}} \\
ANA & Auto-immunité systémique & Lupus \\
Anti-SSA/Ro & Lupus subaigu & Lupus cutané \\
\midrule
\multicolumn{3}{l}{\textit{Sérologie virale~:}} \\
Sérologie hépatite C & Hépatite chronique & Lichen plan (association) \\
\midrule
\multicolumn{3}{l}{\textit{Mycologie~:}} \\
Grattage cutané & Dermatophytes & Tinea corporis \\
\bottomrule
\end{tabular}
\caption{Bilan biologique de première intention}
\end{table}

%=============================================================================
\section{Orientation vers un dermatologue~: éléments de communication}
\label{sec:referral}
%=============================================================================

\begin{keyfinding}{Arguments pour un rendez-vous prioritaire}
Lors de la prise de rendez-vous chez un dermatologue, les éléments suivants justifient un rendez-vous \textbf{en urgence relative} (pas dans 6 mois)~:

\begin{enumerate}
    \item \textbf{Atteinte muqueuse nasale}~: signal d'alerte nécessitant une évaluation rapide
    \item \textbf{Chronicité $>$ 3 mois} de l'épisode actuel, sans résolution
    \item \textbf{Récidives depuis des années}~: exclut un diagnostic bénin auto-limitant
    \item \textbf{Nécessité d'exclure un lymphome cutané} (mycosis fongoïde)~: le pronostic dépend du stade au diagnostic
    \item \textbf{Diagnostic initial erroné} (Gibert incompatible avec l'évolution)
\end{enumerate}

\textbf{Filières d'accès~:}
\begin{itemize}[nosep]
    \item \textbf{Dermatologue libéral}~: mentionner l'atteinte muqueuse et la durée pour obtenir un créneau prioritaire
    \item \textbf{Consultation externe en CHU/hôpital}~: souvent plus rapide que le libéral pour les cas complexes
    \item \textbf{Urgences dermatologiques}~: si extension rapide ou signes systémiques
\end{itemize}
\end{keyfinding}

%=============================================================================
\section{Conseils en attendant le rendez-vous}
\label{sec:interim}
%=============================================================================

\begin{mechanism}{Mesures conservatoires}
En attendant la consultation dermatologique~:

\textbf{À faire~:}
\begin{itemize}[nosep]
    \item \textbf{Photographier} les lésions régulièrement (tous les 3--5 jours), avec un éclairage constant et une règle pour l'échelle. Documenter~: localisation, taille, couleur, évolution
    \item \textbf{Photographier} les lésions intranasales si possible (avec flash, en gros plan)
    \item \textbf{Tenir un journal}~: noter les facteurs déclenchants éventuels (stress, alimentation, médicaments, infections, exposition solaire, saison)
    \item \textbf{Hydrater} la peau avec un émollient neutre (type Dexeryl, Lipikar, ou cold cream)
    \item \textbf{Pour le nez}~: sérum physiologique pour nettoyer, vaseline ou émollient nasal pour protéger la muqueuse
\end{itemize}

\textbf{À éviter~:}
\begin{itemize}[nosep]
    \item \textbf{Dermocorticoïdes sans diagnostic}~: ils peuvent masquer un mycosis fongoïde, aggraver une dermatophytose, ou fausser les résultats d'une biopsie
    \item \textbf{Grattage ou irritation} des lésions (risque de phénomène de Koebner si psoriasis ou lichen plan)
    \item \textbf{Bains chauds prolongés} et savons irritants
    \item \textbf{Automédication} avec des crèmes non prescrites
\end{itemize}
\end{mechanism}

%=============================================================================
\section{Questions essentielles à clarifier}
\label{sec:questions}
%=============================================================================

\begin{question}{Informations critiques manquantes}
Les réponses à ces questions orienteront significativement le diagnostic~:

\textbf{1. Description des lésions~:}
\begin{itemize}[nosep]
    \item Quelle est la forme des plaques~? (Rondes, ovales, irrégulières, annulaires~?)
    \item Quelle est leur couleur~? (Roses, rouges, violacées, brunes, argentées~?)
    \item Y a-t-il des squames~? (Fines, épaisses, argentées, adhérentes~?)
    \item Les bords sont-ils nets ou flous~?
    \item Quelle est leur taille~? (Millimètres, centimètres~?)
\end{itemize}

\textbf{2. Localisation~:}
\begin{itemize}[nosep]
    \item Où sont les lésions exactement~? (Tronc, membres, visage, cuir chevelu~?)
    \item Les lésions reviennent-elles toujours aux mêmes endroits~?
    \item Y a-t-il atteinte du cuir chevelu, des ongles, des paumes, des organes génitaux~?
    \item La muqueuse buccale est-elle atteinte~?
\end{itemize}

\textbf{3. Symptômes associés~:}
\begin{itemize}[nosep]
    \item Y a-t-il du prurit (démangeaisons)~? Intensité~?
    \item Y a-t-il des douleurs ou une sensation de brûlure~?
    \item Les lésions nasales saignent-elles~? Gênent-elles la respiration~?
    \item Y a-t-il des signes généraux (fièvre, fatigue, perte de poids, sueurs nocturnes)~?
    \item Y a-t-il une sécheresse oculaire ou buccale~? $\rightarrow$ Sjögren
\end{itemize}

\textbf{4. Facteurs déclenchants et modulateurs~:}
\begin{itemize}[nosep]
    \item[\textcolor{keyfindingborder}{$\checkmark$}] \textbf{Répondu}~: Pas de facteurs déclenchants identifiés
    \item À préciser~: Les poussées sont-elles saisonnières~?
    \item À préciser~: Le soleil améliore-t-il ou aggrave-t-il les lésions~? $\rightarrow$ Lupus si aggravation
    \item À préciser~: Y a-t-il un lien temporel même vague avec des infections (angine, rhume)~? $\rightarrow$ Psoriasis guttata
\end{itemize}

\textbf{5. Antécédents~:}
\begin{itemize}[nosep]
    \item Âge~?
    \item Antécédents familiaux de psoriasis, eczéma, maladies auto-immunes~?
    \item Antécédents personnels (maladies chroniques, allergies)~?
    \item Médicaments actuels~?
    \item Tabagisme~? (Facteur aggravant du psoriasis palmoplantaire et du lichen plan muqueux)
\end{itemize}

\textbf{6. Traitements antérieurs~:}
\begin{itemize}[nosep]
    \item Quels traitements ont été essayés~? (Crèmes, pommades, comprimés~?)
    \item Y a-t-il eu une réponse aux corticoïdes topiques~? (Amélioration transitoire $\rightarrow$ psoriasis/lichen plan)
    \item Y a-t-il eu une réponse aux antifongiques~? (Amélioration $\rightarrow$ dermatophytose)
    \item Les traitements ont-ils été prescrits ou en automédication~?
\end{itemize}

\textbf{7. Parcours médical~:}
\begin{itemize}[nosep]
    \item Un dermatologue a-t-il déjà été consulté~? Si oui, quel a été le diagnostic~?
    \item Une biopsie cutanée a-t-elle déjà été réalisée~?
    \item Des examens sanguins ont-ils été faits~? Lesquels~? Résultats~?
\end{itemize}
\end{question}

%=============================================================================
\section{Conclusion préliminaire}
\label{sec:conclusion}
%=============================================================================

\begin{keyfinding}{Résumé et prochaines étapes}
\begin{enumerate}
    \item \textbf{Le diagnostic de maladie de Gibert est incompatible} avec le tableau clinique d'Aeiuno (récidives sur des années, épisode $>$ 3 mois, atteinte muqueuse nasale). Ce diagnostic doit être abandonné.

    \item \textbf{Trois diagnostics sont les plus probables}~:
    \begin{itemize}[nosep]
        \item Psoriasis (probabilité élevée)
        \item Lichen plan (probabilité modérée à élevée, surtout avec l'atteinte muqueuse)
        \item Mycosis fongoïde (probabilité modérée, mais conséquences majeures si manqué)
    \end{itemize}

    \item \textbf{L'examen indispensable est la biopsie cutanée} avec immunohistochimie, réalisée par un dermatologue.

    \item \textbf{Une sérologie syphilitique (TPHA + VDRL)} peut être demandée dès maintenant par le médecin généraliste, sans attendre le rendez-vous dermato.

    \item \textbf{La consultation dermatologique est urgente}~: l'atteinte muqueuse nasale et la chronicité justifient un rendez-vous prioritaire.
\end{enumerate}
\end{keyfinding}

\begin{caution}{Limites de cette évaluation}
\begin{itemize}
    \item Ce document est basé sur des informations \textbf{très limitées} (pas d'examen clinique, pas de description détaillée des lésions, pas de photographies)
    \item Aucun diagnostic ne peut être posé sans examen dermatologique et biopsie
    \item Les probabilités attribuées sont des estimations grossières, susceptibles de changer radicalement après un examen clinique
    \item Ce document doit être \textbf{complété} après la consultation dermatologique et les résultats de la biopsie
\end{itemize}
\end{caution}

\vspace{1em}

\textbf{Suivi prévu}~: Mise à jour de ce document après réception des réponses aux questions (section~\ref{sec:questions}) et des résultats de la biopsie cutanée.

\end{document}
