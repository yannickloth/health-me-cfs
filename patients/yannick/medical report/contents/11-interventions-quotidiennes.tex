\subsection{Cascade PEM: Points d'intervention temporels}

Basé sur modèle événementiel de malaise post-effort (EPC PEM Cascade Model, certitude 0.7), avec corrélation aux événements patient récents:

\subsubsection{Fenêtres temporelles et opportunités d'intervention}

\begin{enumerate}
\item \textbf{E1 → E2: Activité → Décalage métabolique (30min--4h)}
    \begin{itemize}
    \item \textbf{Patient Feb 12 11:15--11:45}: 30min repassage debout → faiblesse, pouls élevé (activation E1→E2)
    \item \textbf{Prévention primaire}: Surveillance FC <97 bpm (0,55 × [220-âge]); pacing basé FC
    \item \textbf{Biomarqueurs}: Lactate >2,0 mmol/L, marqueurs ROS élevés (95\% probabilité chez patients EM/SFC)
    \item \textbf{Intervention}: ARRÊT IMMÉDIAT activité si FC dépasse seuil; repos horizontal obligatoire
    \end{itemize}

\item \textbf{E2 → E3: Décalage métabolique → Activation immunitaire (4--24h)}
    \begin{itemize}
    \item \textbf{Patient Feb 12 après-midi/soir}: Sieste 1h20 non réparatrice → probablement transition E2→E3
    \item \textbf{Fenêtre critique anti-inflammatoire}: 4--24h post-activité
    \item \textbf{Biomarqueurs}: Cytokines pro-inflammatoires (IL-1$\alpha$, IL-8, IFN-$\gamma$, CXCL1)
    \item \textbf{Interventions possibles}:
        \begin{itemize}
        \item Quercétine 1000mg (stabilisateur mastocytes, anti-inflammatoire naturel)
        \item Famotidine 20mg BID (bloqueur H2, effets anti-inflammatoires)
        \item LDN dose timing optimisé (modulation immunitaire)
        \item Repos strict horizontal (prévenir progression cascade)
        \end{itemize}
    \item \textbf{Probabilité activation}: 87\% chez patients <3 ans maladie; réduite >3 ans
    \end{itemize}

\item \textbf{E3 → E4: Activation immunitaire → Pic symptomatique (12--48h)}
    \begin{itemize}
    \item \textbf{Patient Feb 13 midi}: Faiblesse après préparation déjeuner → confirmation E4 (Jour 2 post-crash)
    \item \textbf{Durée médiane jusqu'à pic}: 48h post-activité déclenchante
    \item \textbf{Manifestation symptômes}: 100\% probabilité une fois activation immunitaire établie
    \item \textbf{Gestion symptômes}:
        \begin{itemize}
        \item Repos horizontal strict (position assise NON réparatrice pour ce patient)
        \item Hydratation + électrolytes (expansion volume sanguin)
        \item Aucune activité debout (seuil <30min déjà dépassé)
        \end{itemize}
    \end{itemize}

\item \textbf{E4 → E5a/E5b: Pic → Récupération vs Chronification (7--21 jours)}
    \begin{itemize}
    \item \textbf{CRITIQUE - Patient actuellement à ce stade (Feb 13)}
    \item \textbf{Récupération complète (E5a)}: 40\% probabilité SI repos $\geq$7 jours ininterrompu
    \item \textbf{Activation chronique (E5b)}: 60\% probabilité SI repos <7j OU nouveaux déclencheurs
    \item \textbf{Impact chronicité}: Réduction baseline 5--10\% fonction; ATP baseline -5\%
    \item \textbf{RECOMMANDATION URGENTE}:
        \begin{itemize}
        \item \textbf{Repos $\geq$14 jours recommandé} (dépasse minimum 7j, augmente probabilité E5a >60\%)
        \item AUCUNE activité debout >10min
        \item Reprise activité graduelle SEULEMENT après normalisation symptômes
        \item Éviter absolument nouveaux déclencheurs pendant fenêtre récupération
        \end{itemize}
    \end{itemize}
\end{enumerate}

\subsubsection{Boucle rétroaction chronique (FL1)}

\textbf{Pattern préoccupant identifié}: Patient montre épisodes PEM récurrents (11 fév, 12 fév, 13 fév) suggérant entrée possible boucle chronique immune-métabolique.

\textbf{Caractéristiques boucle}:
\begin{itemize}
\item Chaque cycle: ATP baseline × 0,95 (perte permanente 5\%)
\item Chaque cycle: Difficulté récupération × 1,1 (10\% plus difficile récupérer)
\item Convergence: ATP baseline → minimum critique (déclin progressif)
\item \textbf{Probabilité alimentation boucle}: 60\% si repos insuffisant
\end{itemize}

\textbf{Conditions rupture boucle}:
\begin{enumerate}
\item \textbf{Repos >14 jours ininterrompu} (permet réparation complète) - PRIORITÉ ABSOLUE
\item \textbf{Intervention anti-inflammatoire} (brise étape activation immunitaire) - protocole SAMA
\item \textbf{Éducation pacing} (prévenir re-déclenchement) - surveillance FC strict
\item \textbf{Résolution spontanée} (<10\% probabilité, mécanisme unclear)
\end{enumerate}

\subsection{Support métabolisme énergétique}

\begin{enumerate}
\item \textbf{Acetyl-L-Carnitine 1000mg (matin)}
    \begin{itemize}
    \item \textbf{Fonction}: Ouvre ``navette carnitine'' pour transport graisses à longue chaîne dans mitochondries
    \item \textbf{Justification}: Aborde cause racine dysfonction métabolisme graisse (``running on empty'')
    \item \textbf{Timeline}: 4--6 semaines effet initial; 3--6 mois bénéfice maximum
    \item \textbf{Forme acétyl}: Traverse barrière hémato-encéphalique pour support cognitif
    \item \textbf{Preuves}: Correction racine vs bypass temporaire MCT oil
    \end{itemize}

\item \textbf{CoQ10 Ubiquinol 100--200mg (avec graisse alimentaire)}
    \begin{itemize}
    \item \textbf{Fonction}: ``Bougie d'allumage'' chaîne transport électrons; cofacteur essentiel synthèse ATP
    \item \textbf{Justification}: Support machinerie production énergie mitochondriale
    \item \textbf{CRITIQUE}: \textcolor{red}{Fat-soluble - DOIT prendre avec graisse alimentaire sinon absorption <10\%}
    \item \textbf{Forme ubiquinol}: Active, réduite (meilleure absorption qu'ubiquinone)
    \end{itemize}

\item \textbf{Riboflavin (B2) 400mg (dîner avec graisse)}
    \begin{itemize}
    \item \textbf{Fonction triple}:
        \begin{itemize}
        \item Précurseur FAD (flavine adénine dinucléotide) - essentiel bêta-oxydation (combustion graisses)
        \item Cofacteur critique chaîne transport électrons
        \item Prévention migraines (prouvé à 400mg/jour)
        \end{itemize}
    \item \textbf{Justification}: Support métabolisme graisses (synergie acetyl-L-carnitine) + prévention migraines déclenchées vasoconstriction stimulant
    \item \textbf{Timeline}: 4--12 semaines pour prévention migraines
    \item \textbf{CRITIQUE}: \textcolor{red}{Fat-soluble - prendre dîner contenant graisse}
    \end{itemize}

\item \textbf{MCT Oil 1 càs (matin) + 1 càc (coucher)}
    \begin{itemize}
    \item \textbf{Fonction}: Triglycérides chaîne moyenne (C8-C10) contournent navette carnitine cassée
    \item \textbf{Justification URGENCE}: \textbf{BYPASS ÉNERGÉTIQUE IMMÉDIAT} pendant réparation acetyl-L-carnitine
    \item \textbf{Mécanisme}: Va direct au foie pour production énergie; NE NÉ\-CES\-SITE PAS navette carnitine
    \item \textbf{Support absorption}: Aide absorption vitamines fat-soluble (D3, CoQ10, B2)
    \item \textbf{Timing}: 1 càc avant coucher pour support ATP nocturne (prévention crampes)
    \item \textbf{CRITIQUE}: \textcolor{red}{Commencer 1 càc, augmenter lentement sur 1--2 semaines (éviter diarrhée)}
    \item \textbf{Note}: Ceci est \textbf{PAS huile coco} - huile MCT est pure C8/C10 concentrée uniquement
    \end{itemize}

\item \textbf{D-Ribose 5g (coucher + matin pour 10g/jour total)}
    \begin{itemize}
    \item \textbf{Fonction}: Sucre simple qui est brique construction directe molécule ATP
    \item \textbf{Justification}: Reconstitue réserves ATP cellulaires rapidement; contourne voies métaboliques complexes
    \item \textbf{Ciblage}: Déplétion ATP nocturne (pendant jeûne nuit, corps devrait brûler graisse - navette bloquée → ATP s'épuise)
    \item \textbf{Effet}: ATP faible cause crampes nocturnes et sommeil non réparateur
    \item \textbf{Timeline}: Certains notent effet en jours; évaluer à 2 semaines pour réduction crampes
    \end{itemize}
\end{enumerate}

\subsection{Support malabsorption graisses (déficience chronique vitamine D suggère ceci)}

\begin{enumerate}
\item \textbf{MetaDigest TOTAL (Metagenics) - avant repas}
    \begin{itemize}
    \item \textbf{Formule enzyme complète}: lipase (décompose graisses), protéase (protéines), amylase (glucides), cellulase (fibres), lactase (laitier)
    \item \textbf{Justification}: Pancréas nécessite énergie pour produire enzymes; dysfonction mitochondriale réduit production enzyme → maldigestion/malabsorption
    \item \textbf{Évidence}: Déficience chronique vitamine D malgré supplémentation suggère fortement malabsorption graisses
    \item \textbf{Timing}: Prendre immédiatement avant ou avec première bouchée repas contenant vitamines fat-soluble
    \item \textbf{Synergy avec MCT oil}: MCT + enzymes assurent vitamines fat-soluble absorbent réellement
    \end{itemize}
\end{enumerate}

\subsection{Protocole électrolytes (pour support autonome)}

\begin{enumerate}
\item \textbf{Solution électrolyte custom 250mL, 2×/jour}
    \begin{itemize}
    \item \textbf{Sodium}: Expanse volume sanguin (effet ``éponge'' tirant eau dans circulation)
    \item \textbf{Potassium}: Permet relaxation musculaire; maintient charge électrique cellulaire
    \item \textbf{Glucose}: Améliore absorption sodium via transporteur SGLT1; fournit énergie rapide quand combustion graisses altérée
    \item \textbf{Justification EM/SFC}: Implique typiquement faible volume sanguin et intolérance orthostatique
    \item \textbf{Dose après-midi}: Nettoie acide lactique accumulé depuis activités matinales
    \item \textbf{Formule}: 7g mélange sec (sucre + sel Jozo faible sodium + sel table) dans 250mL eau
    \item \textbf{Alternative}: 4,3g par dose (version faible sucre)
    \end{itemize}
\end{enumerate}

\subsection{Optimisation timing magnésium}

\begin{enumerate}
\item \textbf{Magnésium Glycinate 300--400mg (coucher)}
    \begin{itemize}
    \item \textbf{Fonction double}:
        \begin{itemize}
        \item ``Interrupteur off'' pour contraction musculaire - permet relaxation
        \item Cofacteur critique pour 300+ réactions enzymatiques incluant synthèse ATP
        \end{itemize}
    \item \textbf{Timing coucher}: Cible crampes nocturnes quand ATP est au plus bas
    \item \textbf{Forme glycinate}: Effet pH minimal (safe coucher, 6--8h après stimulants)
    \item \textbf{CRITIQUE}: \textcolor{red}{Jamais utiliser magnésium carbonate/oxide - cause dose dumping méthylphénidate}
    \end{itemize}
\end{enumerate}

\label{app:daily-journal}

This appendix serves as a longitudinal record of symptoms, medications, and disease evolution. Regular documentation enables pattern recognition, supports clinical consultations, and provides evidence for treatment adjustments.

\subsection{Journal Entry Template}
\label{sec:journal-template}

Each daily entry should systematically capture symptoms, medications, and observations to enable pattern recognition over time. Use the severity scale in Table~\ref{tab:severity-scale} for all symptom ratings.

\subsubsection{Required Daily Elements}

\paragraph{Sleep and Energy.}
\begin{itemize}
    \item \textbf{Sleep}: Hours slept, sleep quality (refreshing/unrefreshing), interruptions
    \item \textbf{Overall energy level}: 0--10 scale (subjective assessment)
    \item \textbf{Morning state}: How you felt upon waking
\end{itemize}

\paragraph{Primary Symptoms (Rate 0--10).}
\begin{itemize}
    \item \textbf{Fatigue}: Physical exhaustion level
    \item \textbf{Brain fog}: Mental clarity/cognitive function (lower score = clearer thinking)
    \item \textbf{Headache/Migraine}: Severity (0 if absent, note location/type if present)
    \item \textbf{Air hunger}: Respiratory discomfort/dyspnea
    \item \textbf{Leg exhaustion}: Lower extremity fatigue/heaviness
    \item \textbf{Joint pain}: Specify locations (knees/shoulders/wrists/ankles) and severity
    \item \textbf{Muscle cramps}: Frequency and severity
    \item \textbf{Other symptoms}: Any additional symptoms (nausea, dizziness, sensory issues, etc.)
\end{itemize}

\paragraph{Medications and Supplements (Daily Checklist).}
\begin{itemize}
    \item \textbf{LDN}: Dose and time taken
    \item \textbf{Stimulants}: Rilatine/Provigil doses and timing (note total pill count)
    \item \textbf{Mitochondrial support}: Urolithin A, CoQ10, Riboflavin B2
    \item \textbf{Vitamins}: Vitamin D (if weekly dose day), Vitamin C, B-complex
    \item \textbf{Minerals}: Magnesium glycinate, iron
    \item \textbf{Electrolytes}: Custom solution (number of servings)
    \item \textbf{Digestive support}: MetaDigest (when started), MCT oil (when started)
    \item \textbf{Other}: Any additional supplements or medications
\end{itemize}

\paragraph{Activities and Exertion.}
\begin{itemize}
    \item \textbf{Physical activities}: Type, duration, perceived difficulty
    \item \textbf{Cognitive activities}: Mental work, screen time, concentration demands
    \item \textbf{Heart rate data}: Maximum HR, time spent above threshold, resting HR
    \item \textbf{Pacing adherence}: Did you stay within safe limits?
\end{itemize}

\paragraph{Perceived Effects and Observations.}
\begin{itemize}
    \item \textbf{Supplement effects}: Any noticeable changes after taking new supplements (positive or negative)
    \item \textbf{L-Carnitine effects} (when started): Energy changes, cognitive clarity, muscle symptoms, GI effects
    \item \textbf{Sensory function}: Vision clarity today (0--10), hearing clarity (if noticing changes)
    \item \textbf{Sensory-energy correlation}: Do vision/hearing seem worse on low-energy days?
    \item \textbf{Triggers identified}: Activities, foods, stressors that worsened symptoms
    \item \textbf{Helpful interventions}: What provided relief (rest, hydration, specific supplements)
    \item \textbf{Notable patterns}: Connections between symptoms, timing, or interventions
    \item \textbf{Questions for physician}: Observations to discuss at next appointment
\end{itemize}

\subsubsection{Severity Rating Scale}
\label{subsec:severity-scale}

\begin{table}[htbp]
\centering
\caption{Symptom Severity Scale}
\label{tab:severity-scale}
\begin{tabular}{cl}
\toprule
\textbf{Score} & \textbf{Description} \\
\midrule
0 & Absent \\
1--2 & Mild: noticeable but not limiting \\
3--4 & Moderate: affects function, manageable \\
5--6 & Significant: substantially limits activity \\
7--8 & Severe: minimal function possible \\
9--10 & Extreme: incapacitating \\
\bottomrule
\end{tabular}
\end{table}

%------------------------------------------------------------------------------
% JOURNAL ENTRIES BEGIN HERE
%------------------------------------------------------------------------------

\subsection{January 2026}
\label{sec:journal-2026-01}

\subsubsection{2026-01-20}

\begin{description}
    \item[Energy:] /10
    \item[Sleep:] hours, refreshing: Yes/No
    \item[Symptoms:]
    \begin{itemize}
        \item Fatigue: /10
        \item Brain fog: /10
        \item Air hunger: /10
        \item Leg exhaustion: /10
        \item Joint pain (knees/shoulders/wrists): /10
        \item Muscle cramps: /10
        \item Migraine: Yes/No
    \end{itemize}
    \item[Medications:]
    \begin{itemize}
        \item Usual medication: Yes
        \item Usual supplements: Yes
    \end{itemize}
    \item[Activities:]
    \item[Heart rate data:] Max HR: , time above threshold:
    \item[Observations:] Took 250\,mL water + 10\,mL grenadine + salt/sugar mixture (oral rehydration solution).
\end{description}

\subsubsection{2026-01-21}

\begin{description}
    \item[Sleep and Energy:]
    \begin{itemize}
        \item Sleep: \underline{\hspace{2cm}} hours, quality: \underline{\hspace{3cm}} (refreshing/unrefreshing)
        \item Overall energy: \underline{\hspace{1cm}}/10
        \item Morning state: \underline{\hspace{6cm}}
    \end{itemize}

    \item[Symptoms (0--10 scale):]
    \begin{itemize}
        \item Fatigue: \underline{\hspace{1cm}}/10
        \item Brain fog: \underline{\hspace{1cm}}/10
        \item Headache/Migraine: \underline{\hspace{1cm}}/10 (location/type: \underline{\hspace{3cm}})
        \item Air hunger: \underline{\hspace{1cm}}/10
        \item Leg exhaustion: \underline{\hspace{1cm}}/10
        \item Joint pain: \underline{\hspace{1cm}}/10 (locations: \underline{\hspace{4cm}})
        \item Muscle cramps: \underline{\hspace{1cm}}/10
        \item Other: \underline{\hspace{8cm}}
    \end{itemize}

    \item[Medications and Supplements:]
    \begin{itemize}
        \item LDN 3\,mg: $\square$ (time: \underline{\hspace{2cm}})
        \item Rilatine MR 30\,mg: $\square$ $\square$ (times: \underline{\hspace{3cm}})
        \item Provigil 100\,mg: $\square$ $\square$ (times: \underline{\hspace{3cm}})
        \item Total stimulant pills today: \underline{\hspace{1cm}}/3 max
        \item Urolithin A + NAD+: $\square$ (2 capsules)
        \item CoQ10 ubiquinol: $\square$ (1--2 capsules)
        \item \textbf{NEW: Riboflavin B2 400\,mg}: $\boxtimes$ \textbf{(STARTED TODAY)}
        \item Vitamin C 500\,mg: $\square$
        \item B-complex (BEFACT FORTE): $\square$
        \item \textbf{NEW: Magnesium glycinate (Metagenics)}: $\boxtimes$ \textbf{(STARTED TODAY - replacing Magnecaps)}
        \item Iron (FerroDyn FORTE): $\square$
        \item Vitamin D 25000\,U.I.: $\square$ (weekly - if applicable)
        \item Electrolyte solution: \underline{\hspace{1cm}} servings
        \item Other: \underline{\hspace{6cm}}
    \end{itemize}

    \item[Activities and Exertion:]
    \begin{itemize}
        \item Physical: \underline{\hspace{8cm}}
        \item Cognitive: \underline{\hspace{8cm}}
        \item Heart rate: Max \underline{\hspace{2cm}} bpm, time above threshold: \underline{\hspace{2cm}}
        \item Pacing adherence: $\square$ Good $\square$ Exceeded limits
    \end{itemize}

    \item[Perceived Effects and Observations:]
    \begin{itemize}
        \item New supplement effects (Riboflavin/Mg): \underline{\hspace{6cm}}
        \item Triggers identified: \underline{\hspace{6cm}}
        \item Helpful interventions: \underline{\hspace{6cm}}
        \item Notable patterns: \underline{\hspace{6cm}}
        \item Questions for physician: \underline{\hspace{6cm}}
    \end{itemize}
\end{description}

%------------------------------------------------------------------------------
% BLANK TEMPLATE FOR FUTURE DAYS
%------------------------------------------------------------------------------

\subsubsection{YYYY-MM-DD} % Copy this template for new entries

\begin{description}
    \item[Sleep and Energy:]
    \begin{itemize}
        \item Sleep: \underline{\hspace{2cm}} hours, quality: \underline{\hspace{3cm}}
        \item Overall energy: \underline{\hspace{1cm}}/10
        \item Morning state: \underline{\hspace{6cm}}
    \end{itemize}

    \item[Symptoms (0--10):]
    \begin{itemize}
        \item Fatigue: \underline{\hspace{1cm}}/10
        \item Brain fog: \underline{\hspace{1cm}}/10
        \item Headache/Migraine: \underline{\hspace{1cm}}/10 (location: \underline{\hspace{3cm}})
        \item Air hunger: \underline{\hspace{1cm}}/10
        \item Leg exhaustion: \underline{\hspace{1cm}}/10
        \item Joint pain: \underline{\hspace{1cm}}/10 (locations: \underline{\hspace{4cm}})
        \item Muscle cramps: \underline{\hspace{1cm}}/10
        \item Other: \underline{\hspace{8cm}}
    \end{itemize}

    \item[Medications/Supplements:]
    \begin{itemize}
        \item LDN 3\,mg: $\square$ | Rilatine: $\square$ $\square$ | Provigil: $\square$ $\square$ (total: \underline{\hspace{1cm}}/3)
        \item Urolithin A: $\square$ | CoQ10: $\square$ | Riboflavin B2: $\square$
        \item Vit C: $\square$ | B-complex: $\square$ | Mg glycinate: $\square$ | Iron: $\square$ | Vit D: $\square$
        \item Electrolytes: \underline{\hspace{1cm}}$\times$ | MetaDigest: $\square$ | MCT oil: $\square$
        \item Other: \underline{\hspace{6cm}}
    \end{itemize}

    \item[Activities:] \underline{\hspace{8cm}}

    \item[Heart rate:] Max \underline{\hspace{2cm}} bpm, threshold time: \underline{\hspace{2cm}}

    \item[Observations:] \underline{\hspace{10cm}}
\end{description}

% Add new months as sections:
% \subsection{February 2026}
% \label{sec:journal-2026-02}
% ...
