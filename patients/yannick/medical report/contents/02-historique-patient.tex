
\subsection{Chronologie de la maladie}

\begin{longtable}{p{3cm}p{6cm}p{5cm}}
\toprule
\textbf{Période} & \textbf{Événement} & \textbf{Signification} \\
\midrule
Enfance (1990s) & Supplémentation en fluorure (Zyma Fluor) & Effets possibles sur la glande pinéale (spéculatif) \\
\midrule
13-15 ans & Apparition progressive du brouillard mental & Premiers symptômes de type EM/SFC \\
\midrule
16 ans (c. 1997) & Tremblements des mains remarqués par d'autres & Manifestation neurologique précoce \\
\midrule
$\sim$20 ans (c. 2001) & Apparition de crampes musculaires & Implication musculo-squelettique \\
\midrule
20+ ans & Début méthylphénidate (Ritalin) & Amélioration cognitive transformative \\
\midrule
Pré-2018 & Au moins un épisode vagal & Vulnérabilité autonome établie \\
\midrule
Fin 2017 & Burnout & Stress de l'axe HPA, réserves réduites \\
\midrule
29 juin 2018 & Syncope vasovagale → chute → commotion & Syncope CAUSA chute; amnésie post-traumatique 5h; CT négatif; LAD soupçonné \\
\midrule
Post-2018 & Émergence du phénotype EM/SFC complet & Déclin fonctionnel sévère \\
\midrule
Fin 2025 & Essai d'exercice de natation (4-5 mois) & Échec: PEM cognitif constant, perte d'emploi \\
\midrule
25 jan 2026 & Infection respiratoire haute & Exacerbation autonome sévère \\
\midrule
8-13 fév 2026 & Événements autonomes récurrents & Présentation actuelle (détaillée ci-dessous) \\
\bottomrule
\end{longtable}

\subsection{Diagnostics confirmés}

\begin{itemize}
\item EM/SFC (diagnostic clinique; répond aux critères ICC 2011)
\item Perte auditive neurosensorielle bilatérale (diagnostiquée août 2024, pattern haute fréquence)
\item Presbytie avec hypermétropie (apparition progressive vers 40 ans)
\item Allergies aux noix (panel FX1 confirmé)
\item Allergies au pollen (TX5, TX6)
\end{itemize}

\subsection{Incertitudes diagnostiques clés}

\begin{enumerate}
\item \textbf{TDAH vs. déficit d'attention secondaire}: Déficits d'attention sévères toute la vie avec réponse dramatique au méthylphénidate, mais tests formels TDAH multiples: tous NÉGATIFS. Antécédents familiaux positifs (mère, 2 sœurs). Peut représenter une déficience cognitive secondaire induite par le déficit énergétique.

\item \textbf{Syndrome autonome spécifique}: Symptômes orthostatiques documentés mais non formellement caractérisés (POTS vs. hypotension orthostatique vs. autre dysautonomie).

\item \textbf{Dysfonction mitochondriale}: Présumée sur base de la présentation clinique mais non formellement testée.
\end{enumerate}

\subsection{Modèle causal multi-coups}

La voie du patient vers l'EM/SFC semble impliquer une vulnérabilité cumulative:

\begin{enumerate}
\item \textbf{Vulnérabilité développementale} (enfance): Possible dysfonction pinéale induite par le fluorure → vulnérabilité sommeil/autonome (spéculatif)
\item \textbf{Instabilité autonome établie} (adolescence-adulte): Hypersensibilité vagale documentée, hypersomnie idiopathique
\item \textbf{Dysfonction de l'axe HPA} (fin 2017): Stress neuroendocrinien lié au burnout
\item \textbf{Lésion cérébrale traumatique} (juin 2018): Syncope vasovagale → chute → commotion avec amnésie post-traumatique de 5h → lésion axonale diffuse affectant les centres autonomes du tronc cérébral
\item \textbf{Cascade EM/SFC complète} (2018-présent): Décompensation autonome suite aux blessures composées
\end{enumerate}

\textbf{Preuves à l'appui}: Bateman et al. (2024) ont trouvé que les patients EM/SFC ont 4,89 fois plus de chances d'antécédents de commotion. La dysfonction autonome post-TCC est documentée chez 40-90\% des patients TCC.

\subsection{Vue Chronologique Détaillée (30 Ans)}

Cette sous-section documente les jalons majeurs, les changements de sévérité et les événements significatifs dans le cours de la maladie.

\begin{description}
    \item[Phase Constitutionnelle (Enfance--2017):] Fatigue permanente, hypersomnie idiopathique
    \begin{itemize}
        \item \textbf{Exposition pharmaceutique durant l'enfance}: Supplémentation régulière en Zyma Fluor (fluorure de sodium)

        \begin{tcolorbox}[colback=yellow!10!white,colframe=orange!75!black,title=Facteur rétrospectif spéculatif]
        Cette exposition au fluorure est documentée pour \textbf{complétude historique du dossier patient uniquement}. La causalité individuelle ne peut être déterminée rétrospectivement à partir des patterns d'exposition durant l'enfance. Aucune action clinique n'est justifiée sur base de cette spéculation seule. De nombreux individus ont reçu une supplémentation en fluorure similaire sans développer de troubles du sommeil ou d'EM/SFC, indiquant que si le fluorure a joué un rôle, ce serait comme un des multiples facteurs contributifs chez un individu susceptible, et non une cause unique.
        \end{tcolorbox}

        \begin{itemize}
            \item Produit: Comprimés Zyma Fluor ($\sim$0.25\,mg fluorure par comprimé), prévention des caries dentaires
            \item Administration: Requis par la mère, ``plutôt régulièrement'' durant l'enfance
            \item \textbf{Pertinence mécanistique potentielle (HAUTEMENT SPÉCULATIF)}:
            \begin{itemize}
                \item La glande pinéale accumule le fluorure à des concentrations plus élevées que tout autre tissu mou
                \item Les enfants retiennent 80--90\% du fluorure absorbé (vs.\ 60\% chez les adultes)
                \item U.S.\ National Research Council (2006): ``Le fluorure est susceptible de causer une production réduite de mélatonine''
                \item Mécanisme: Inhibition des enzymes pinéales (AANAT, HIOMT) impliquées dans la synthèse de mélatonine
                \item \emph{Hypothèse spéculative}: Accumulation de fluorure durant l'enfance $\rightarrow$ dysfonction pinéale $\rightarrow$ réduction chronique de la mélatonine $\rightarrow$ vulnérabilité sommeil/autonome
            \end{itemize}
            \item \textbf{Qualité des preuves}: Accumulation pinéale HAUTE (études d'autopsie humaine), réduction de mélatonine MODÉRÉE (modèles animaux, mécanistiquement solide), pertinence pour le patient SPÉCULATIF (variation individuelle, pas de mesure directe)
        \end{itemize}

        \item Peut fournir un contexte mécanistique possible parmi d'autres pour la dysfonction du sommeil de longue date (hypersomnie idiopathique) et la réserve autonome réduite
        \item Petite enfance: Siestes requises l'après-midi jusqu'à 7--8 ans
        \item \textbf{Adolescence (âge $\sim$13--15):} Apparition du brouillard mental récurrent; fatigue constante mais performance académique maintenue
        \item \textbf{Âge $\sim$20 (circa 2001):} Apparition de crampes musculaires spontanées (nocturnes, gorge/cou, sans effort)
        \item Jeune adulte: Difficultés universitaires malgré QI élevé (>135) - déficience cognitive due au déficit énergétique, non limitation intellectuelle
        \begin{itemize}
            \item Dormait fréquemment durant les cours tout au long de la journée (pas seulement après le déjeuner)
            \item Le sommeil était une réponse involontaire à l'épuisement accablant, pas simple somnolence
            \item Les difficultés académiques reflétaient le déficit énergétique empêchant l'attention soutenue, pas un manque de capacité intellectuelle
        \end{itemize}
        \item \textbf{Années de travail:} Maintien à peine de l'emploi par des stratégies compensatoires insoutenables
        \begin{itemize}
            \item Passait les samedis entiers à dormir (matin + après-midi) pour récupérer pour les matchs de tennis de table du soir (pas pour la semaine de travail)
            \item Effondrement énergétique en milieu de match menant à une baisse de performance et des pertes
            \item Déjà trop épuisé pour un engagement professionnel approprié durant la semaine; faisait juste les gestes
            \item Difficulté progressive à maintenir même ce niveau insoutenable d'effort compensatoire
            \item L'emploi était en mode survie, pas une performance professionnelle fonctionnelle
        \end{itemize}
        \item \textbf{Tolérance historique à l'exercice:} À un certain point pouvait nager 1\,km quotidiennement
        \begin{itemize}
            \item Condition physique améliorée (meilleure performance au tennis de table)
            \item Symptômes cognitifs (brouillard, somnolence) persistaient durant la journée
            \item L'exercice fournissait un bénéfice net malgré ne pas éliminer la dysfonction sous-jacente
        \end{itemize}
        \item Statut: Sévèrement affaibli mais maintenant l'emploi par effort compensatoire extrême et insoutenable; déjà trop épuisé pour engagement social/professionnel normal
    \end{itemize}

    \item[Événement Déclencheur (Fin 2017):] Burnout sévère
    \begin{itemize}
        \item Burnout documenté fin 2017 (selon évaluation clinique du sommeil)
        \item \textbf{Incertitude causale}: Si le burnout était le déclencheur reste peu clair; cependant, ce fut indubitablement un événement profondément dépressif
        \item A probablement précipité la transition vers le phénotype EM/SFC complet
        \item Le burnout implique une dysrégulation de l'axe HPA, dysfonction du cortisol
        \item Peut avoir ``verrouillé'' le mode sécuritaire métabolique décrit dans les hypothèses spéculatives
    \end{itemize}

    \item[Phase Post-Déclencheur (2018--Présent):] EM/SFC sévère avec PEM invalidant
    \begin{itemize}
        \item \textbf{Important:} Le PEM lui-même n'est pas nouveau---il est présent depuis des décennies (cycles de crash-récupération en fin de semaine, effondrements en milieu de match)
        \item Ce qui a changé: \textbf{Escalade de sévérité} de ``gérable avec effort extrême'' à ``invalidant''
        \item \textbf{29 juin 2018:} Commotion cérébrale --- Clinique Saint-Joseph, Arlon
        \begin{itemize}
            \item \textbf{Mécanisme}: Syncope vagale en position assise sur un haut tabouret de comptoir après avoir bu du Coca à midi $\rightarrow$ chute $\rightarrow$ traumatisme crânien
            \item \textbf{Amnésie post-traumatique}: 5 heures (significative, indique commotion de sévérité modérée)
            \item \textbf{Note clinique}: ``Syncopes répétées'' --- pas un événement isolé
            \item \textbf{Pattern d'hypersensibilité vagale --- Vulnérabilité préexistante}:
            \begin{itemize}
                \item Le patient rapporte une sensibilité accrue à la stimulation du nerf vague
                \item Au moins un épisode vasovagal antérieur (moins sévère, nécessitant position assise mais pas de perte de conscience)
                \item Épisode de juin 2018: Syncope vasovagale complète avec perte complète de conscience
                \item Le pattern indique une instabilité autonome/dysautonomie de base \emph{précédant} la commotion
                \item \textbf{Interaction critique}: L'hypersensibilité vagale préexistante a probablement réduit la capacité de compensation pour l'atteinte autonome induite par le TCC
            \end{itemize}
            \item \textbf{Mécanisme de dysfonction autonome (Commotion $\rightarrow$ Dysautonomie)}:
            \begin{itemize}
                \item \textbf{Lésion axonale diffuse (LAD)}: Les forces rotationnelles durant l'impact causent un cisaillement axonal dans les centres de contrôle autonome
                \item \textbf{Régions affectées}: Noyaux autonomes du tronc cérébral (noyau moteur dorsal du vague, noyau tractus solitaire), hypothalamus, système réticulaire activateur
                \item \textbf{Effets persistants}: La LAD peut produire une dysfonction autonome durant des années post-lésion (documentée chez 40--90\% des patients TCC)
                \item \textbf{Dominance sympathique}: La dysfonction autonome post-TCC se manifeste souvent comme un surdrive sympathique (FC élevée, HRV altérée, tachycardie orthostatique)
                \item \textbf{Association EM/SFC}: Les patients EM/SFC ont 4.89$\times$ plus de chances d'antécédent de commotion
                \item \textbf{Modèle d'atteinte composée}: Hypersensibilité vagale préexistante + TCC aigu des centres autonomes = dysautonomie sévère et persistante
                \item Le TCC semble avoir été le point d'inflexion de la vulnérabilité autonome compensée au phénotype complet dysautonomie/EM/SFC
            \end{itemize}

            \item \textbf{Clarification du niveau de preuve}:
            \begin{itemize}
                \item \textbf{Association TCC-EM/SFC (Bateman 2024):} Certitude HAUTE --- grande étude rétrospective avec rapport de cotes de 4.89$\times$
                \item \textbf{Mécanisme LAD et dysfonction autonome persistante:} Certitude MOYENNE --- bien documenté dans la littérature TCC mais pas spécifique EM/SFC
                \item \textbf{Contribution spécifique à la dysautonomie de ce patient:} Certitude BASSE --- inférence clinique de l'association temporelle et correspondance phénotypique; ne peut établir définitivement la causalité d'un cas unique sans test autonome de référence pré-TCC
            \end{itemize}
            \item \textbf{Imagerie}: CT crâne + cervical: négatif pour lésions post-traumatiques
            \item \textbf{Diagnostic}: ``Commotion cérébrale très probable'' (médecin consultant d'urgence)
            \item \textbf{Suivi commandé}: EEG (2/7/2018), surveillance Holter (16/7/2018)
            \item \textbf{Traitement}: Litican (piracetam --- nootrope pour support cognitif post-TCC)
            \item \textbf{Résultats de laboratoire pertinents à l'admission}:
            \begin{itemize}
                \item Acide lactique: \textbf{3.18 mmol/L} (réf.\ 0.50--2.20) --- élevé à la baseline
                \item CPK: \textbf{254 U/L} (réf.\ 5--195) --- marqueur de dommage musculaire élevé
                \item LDH: \textbf{249 U/L} (réf.\ 135--225) --- limite supérieure
                \item Prolactine: \textbf{93.3 $\mu$g/L} (réf.\ 4.0--15.2) --- marquément élevée (post-ictale?)
                \item Glucose: 148 mg/dL (réf.\ 70--105) --- élevé (réponse au stress)
            \end{itemize}
            \item \textbf{Pertinence EM/SFC}:
            \begin{itemize}
                \item L'acide lactique élevé à la baseline supporte l'hypothèse de dysfonction métabolique
                \item Les syncopes vagales récurrentes sont cohérentes avec la dysautonomie
                \item Le pattern d'hypersensibilité vagale peut représenter une vulnérabilité autonome préexistante
                \item Le syndrome post-commotionnel partage des caractéristiques avec l'EM/SFC: dysfonction cognitive, fatigue, intolérance à l'exercice
                \item Le TCC peut déclencher ou exacerber la dysfonction neuroimmunitaire
                \item Chronologie: 6 mois après le déclencheur du burnout, durant la phase de détérioration précoce
            \end{itemize}
        \end{itemize}
        \item Transition de ``fatigué mais fonctionnel avec stratégies compensatoires'' à ``incapable de compenser''
        \item Incapable de maintenir l'emploi de manière constante
        \item \textbf{2025/2026:} Tentative de reprise d'un régime de natation (durée 4--5 mois)
        \begin{itemize}
            \item Auparavant: Natation quotidienne de 1\,km améliorait la condition physique (malgré symptômes cognitifs persistants)
            \item Tentative actuelle: A résulté en \textbf{brouillard mental constant} suffisamment sévère pour éliminer la fonction professionnelle
            \item Conséquence: Sous-performance au travail menant à la perte d'emploi
            \item Démontre la progression de la maladie: l'exercice est passé de ``bénéfice net avec symptômes'' à ``PEM cognitif invalidant surpassant tout gain de condition physique''
        \end{itemize}
        \item Statut fonctionnel actuel: Déficience fonctionnelle sévère malgré mobilité de base préservée
        \begin{itemize}
            \item \textit{Peut effectuer}: Conduire les enfants à l'école, faire les courses, s'asseoir à l'ordinateur les meilleurs jours
            \item \textit{Nécessite des stimulants}: Pour toute fonction; sans stimulants, complètement non-fonctionnel
            \item \textit{Épuisement profond}: Malgré les stimulants, trop fatigué pour engagement social, contact visuel, sourire, rire
            \item \textit{Préférence d'isolement}: L'interaction humaine nécessite une énergie qui n'existe pas; préfère la distance à l'engagement
            \item \textit{Résumé}: Peut exécuter des tâches essentielles mais pas d'énergie pour quoi que ce soit qui rende la vie significative; ``trop fatigué pour être humain''
        \end{itemize}
    \end{itemize}

    \item[Diagnostics:]
    \begin{itemize}
        \item Hypersomnie idiopathique (confirmée par étude du sommeil)
        \item Syndrome des jambes sans repos
        \item Apnée du sommeil (à un certain degré)
        \item Caractéristiques EM/SFC: PEM, dysfonction cognitive, sommeil non réparateur
    \end{itemize}

    \item[Jalons thérapeutiques:]
    \begin{itemize}
        \item Méthylphénidate (Rilatine): Efficace pour éveil/fonction
        \item Modafinil (Provigil): Efficace pour vigilance
        \item LDN: Statut actuel et effet à documenter
    \end{itemize}

    \item[Changements de statut fonctionnel:]
    \begin{itemize}
        \item Pré-2018: Maintien de l'emploi par effort insoutenable; déjà trop épuisé pour engagement professionnel approprié; nécessitait récupération extrême en fin de semaine (sommeil complet du samedi)
        \item Post-2018: Incapable de maintenir l'emploi de manière constante
        \item 2025/2026: Perte d'emploi suite au PEM cognitif induit par l'exercice (régime de natation)
        \item Actuel (2026): Déficience sévère; peut effectuer des tâches essentielles (conduire, courses, travail informatique limité) mais trop épuisé pour engagement social ou activités significatives malgré les stimulants
    \end{itemize}
\end{description}
