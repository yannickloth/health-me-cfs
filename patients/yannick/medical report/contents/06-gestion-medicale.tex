\section{Traitements en cours}

\subsection{Naltrexone à faible dose (LDN) -- 3-4mg par jour}

\textbf{Classification:} Modulateur immunitaire et anti-inflammatoire hors AMM\\
\textbf{Dosage actuel:} Alternant 3mg et 4mg (incohérent)

\textbf{Mécanisme d'action:}
À faibles doses (1-5mg), la naltrexone bloque transitoirement les récepteurs opioïdes, conduisant à une upregulation de la production d'opioïdes endogènes (endorphines) et à une modulation du récepteur Toll-like 4 (TLR4) sur la microglie, réduisant la neuroinflammation. Le LDN module également la fonction du canal ionique TRPM3 dans les cellules tueuses naturelles, qui est altérée dans l'EM/SFC.

\textbf{Base de preuves:}
\begin{itemize}
\item Polo et al. (2019): Revue rétrospective de dossiers du LDN dans l'EM/SFC a montré des améliorations de la fatigue, du sommeil et de la douleur. Limitations: pas de contrôle placebo, pas de validation RCT.
\item Bolton et al. (2020): Rapports de cas BMJ décrivant le LDN comme traitement du SFC.
\item Cabanas et al. (2021): Étude pilote (n=9 EM/SFC sous LDN, n=9 témoins) a démontré la restauration de la fonction du canal ionique TRPM3 dans les cellules tueuses naturelles.
\item Multiples RCTs en cours (2024-2026): Life Improvement Trial (OMF), essai British Columbia (n=160), essai ME Association UK (208 pré-recrutés en sept 2025).
\end{itemize}

\textbf{Qualité des preuves:} Moyenne -- preuves observationnelles positives; résultats RCT en attente (prévus 2026).

\textbf{Recommandation:} Stabiliser le dosage soit à 3mg soit à 4mg de manière cohérente. L'alternance des doses peut empêcher la pharmacocinétique à l'état stable. Envisager de discuter l'optimisation de la dose avec le médecin.

\subsection{Cétirizine -- 1 comprimé par jour (récemment ajouté)}

\textbf{Classification:} Antihistaminique H1 de deuxième génération\\
\textbf{Indication:} Gestion du syndrome d'activation mastocytaire (SAMA), contrôle des allergies

\textbf{Mécanisme d'action:}
Antagoniste des récepteurs H1 avec propriétés stabilisatrices de mastocytes supplémentaires. La cétirizine a été démontrée inhiber la libération de médiateurs mastocytaires au-delà du simple blocage H1.

\textbf{Base de preuves:}
\begin{itemize}
\item Le SAMA est de plus en plus reconnu comme comorbidité dans l'EM/SFC, avec des médiateurs dérivés des mastocytes contribuant à la fatigue, au brouillard mental et à la dysfonction autonome.
\item La cétirizine a des propriétés stabilisatrices de mastocytes documentées au-delà de ses effets antihistaminiques (recherche publiée dans Allergy journal, 2022).
\end{itemize}

\textbf{Qualité des preuves:} Moyenne pour SAMA dans EM/SFC; Élevée pour efficacité antihistaminique généralement.

\textbf{Note importante:} Le patient prend SEULEMENT cétirizine pour gestion SAMA. Un protocole SAMA complet inclurait rupatadine (triple action H1+PAF+stabilisateur mastocytes), famotidine (bloqueur H2), et quercétine (stabilisateur mastocytes naturel). Ces ajouts sont RECOMMANDÉS (voir section Recommandations protocole SAMA).

\subsection{Ritalin MR 30mg (Méthylphénidate à libération prolongée) -- Intermittent}

\textbf{Classification:} Stimulant du système nerveux central (Annexe II)\\
\textbf{Utilisation actuelle:} Intermittente, selon besoin pour fonction cognitive\\
\textbf{Historique:} 23+ ans d'utilisation (depuis environ 20 ans)

\textbf{Mécanisme d'action:}
Bloque la recapture de la dopamine et de la norépinéphrine, augmentant la disponibilité synaptique. Dans le contexte EM/SFC, compense les niveaux bas démontrés de catécholamines dans le liquide céphalorachidien (étude de phénotypage profond NIH 2024).

\textbf{Réponse clinique:}
\begin{itemize}
\item \textbf{Sans médicament:} Déficience cognitive sévère, incapacité à se concentrer, échec de compréhension en lecture
\item \textbf{1 comprimé:} Amélioration modérée, toujours limité en énergie
\item \textbf{2 comprimés:} Pleinement engagé mentalement, différence ``jour et nuit''
\item \textbf{Réponse dose-dépendante dramatique} suggère mécanisme compensatoire pour déficit énergétique plutôt que (ou en plus de) TDAH primaire
\end{itemize}

\textbf{Base de preuves:}
\begin{itemize}
\item Pas de grands RCTs spécifiquement pour EM/SFC; utilisation hors AMM
\item Étude de phénotypage profond NIH 2024 a trouvé des catécholamines anormalement basses (norépinéphrine, dopamine) dans le liquide céphalorachidien EM/SFC, supportant la justification pour supplémentation dopaminergique
\item Revue de sécurité cardiovasculaire (revue narrative 2025 dans Pharmacological Reports): augmentation de fréquence cardiaque et pression artérielle documentée; événements cardiovasculaires sérieux rares; nécessite surveillance
\end{itemize}

\textbf{Préoccupation critique:} Les stimulants masquent les vrais niveaux d'énergie, permettant une activité qui dépasse la capacité métabolique. Cet ``emprunt d'énergie'' peut contribuer au PEM. La surveillance de la fréquence cardiaque pendant l'utilisation de stimulant est essentielle. Limite FC recommandée pour le patient: 97 bpm ((220-44) × 0,55).

\textbf{Schéma de rebond (problème actuel):}
La séquence 10-11 février démontre un schéma de rebond préoccupant:
\begin{itemize}
\item Jour avec Ritalin: Énergie 6/10, cognitif 8/10 (excellente fonction)
\item Jour après sans Ritalin: Énergie 2/10, tremblements, sommeil excessif, événement autonome
\end{itemize}

\textbf{Recommandation:} Si le Ritalin doit être utilisé régulièrement, discuter dosage quotidien cohérent vs. utilisation intermittente. Le schéma de rebond suggère que l'utilisation intermittente peut être pire que soit l'utilisation cohérente soit l'abstinence.

\subsection{Provigil (Modafinil) -- Intermittent}

\textbf{Classification:} Agent favorisant l'éveil\\
\textbf{Utilisation actuelle:} Intermittente; en cours d'élimination progressive en faveur de monothérapie méthylphénidate\\
\textbf{Dose quand utilisé:} Non spécifié (standard est 100-200mg)

\textbf{Mécanisme d'action:}
Augmente la dopamine en bloquant le transporteur de dopamine; affecte également les systèmes norépinéphrine, sérotonine, histamine et orexine. Favorise l'éveil via les neurones orexine/hypocrétine hypothalamiques.

\textbf{Réponse clinique:}
\begin{itemize}
\item Efficace pour réduire la fatigue subjective
\item NE garantit PAS la clarté mentale ou l'amélioration cognitive
\item Inférieur au méthylphénidate pour ce patient (Ritalin fournit à la fois anti-fatigue ET clarté cognitive)
\item Les symptômes physiques (fatigue, faim d'air) persistent indépendamment
\end{itemize}

\textbf{Base de preuves:}
\begin{itemize}
\item Petites données d'essai dans EM/SFC: 200mg a montré des bénéfices modestes attention/planification spatiale vs. placebo; 400mg a montré des effets PIRES que placebo (réponse dose paradoxale).
\item Utilisation hors AMM pour fatigue EM/SFC; preuves insuffisantes pour recommandation générale
\item Effets autonomes: propriétés sympathomimétiques; effets d'alerte sans augmentation significative TA/FC à faibles doses
\end{itemize}

\textbf{Qualité des preuves:} Faible à Moyenne pour EM/SFC spécifiquement.

\textbf{Recommandation:} Étant donné la préférence du patient pour le méthylphénidate et les considérations de coût, l'élimination progressive du modafinil semble raisonnable. Cependant, il peut servir d'alternative les jours où le rebond de méthylphénidate est une préoccupation.

\subsection{Protocole suppléments actuels}

Basé sur le protocole médicamenteux de référence rapide (daté du 22 janvier 2026):

{\scriptsize
\begin{longtable}{p{2.5cm}p{1.3cm}p{1.8cm}p{4.8cm}}
\toprule
\textbf{Supplément} & \textbf{Dose} & \textbf{Moment} & \textbf{Justification} \\
\midrule
Acétyl-L-Carnitine & 1000mg & Matin & Support navette acides gras mito\-chondriaux; groupe acétyle traverse BHE \\
\midrule
CoQ10 (Ubiquinol) & 100mg & Matin avec gras & Cofacteur chaîne transport électrons; essentiel production ATP \\
\midrule
Riboflavine (B2) & 400mg & Déjeuner/\hspace{0pt}dîner avec gras & Précurseur FAD chaîne énergétique mito\-chondriale; prévention migraine \\
\midrule
BEFACT FORTE & 1 cp & Matin & Support complexe B \\
\midrule
Vitamine C & 500mg & Matin & Antioxydant; support absorption fer \\
\midrule
\textcolor{blue}{N-Acétylcystéine (NAC)} & \textcolor{blue}{600mg} & \textcolor{blue}{Matin} & \textcolor{blue}{Précurseur glutathion; antioxydant; anti-inflammatoire} \\
\midrule
Fer (FerroDyn FORTE) & 1 cap & Matin & Reconstitution fer (séparer Ca/Mg 2-4h) \\
\midrule
Glycinate magnésium & 300-\hspace{0pt}400mg & Coucher & Relaxation musculaire; prévention crampes; support sommeil \\
\midrule
Huile MCT & 1 c.à.c. & Coucher & Contourne navette carnitine pour substrat ATP immédiat \\
\midrule
D-Ribose & 5g & Coucher (opt.) & Précurseur ATP direct \\
\midrule
Vitamine D3 & 25000 UI & Hebdo avec gras & Modulation immunitaire \\
\midrule
Urolithin A + NAD+ & 2 caps (2000mg + 200mg) & Matin & Support mitophagie et énergie cellulaire \\
\midrule
Électrol. & {\scriptsize 2·250mL} & Mat+PM & Vse \\
\bottomrule
\end{longtable}
}

\textbf{Note importante:} Le patient ne prend PAS actuellement:
\begin{itemize}
\item Quercétine (500-1000mg) - stabilisateur mastocytes naturel
\item Rupatadine (10-20mg) - H1+PAF+stabilisateur mastocytes (SAMA)
\item Famotidine (20mg 2×/jour) - Bloqueur H2 (SAMA)
\end{itemize}

Ces trois suppléments sont listés dans le protocole médicamenteux de référence mais ne sont pas actuellement utilisés. \textbf{Ils devraient être considérés comme RECOMMANDATIONS pour gestion SAMA} (voir section Recommandations de traitement).

\textbf{Justification du protocole actuel:} Restauration énergétique en trois phases:
\begin{enumerate}
\item \textbf{Contournement} (immédiat): Huile MCT + D-Ribose fournissent substrats ATP qui contournent les voies métaboliques dysfonctionnelles
\item \textbf{Réparation} (4-6 semaines): Acétyl-L-Carnitine rouvre la navette acides gras mitochondriaux
\item \textbf{Optimisation} (en cours): CoQ10 + B2 + Mg supportent l'efficacité chaîne transport électrons
\end{enumerate}

Cette annexe documente les médicaments actuels, les protocoles de suppléments et les stratégies de gestion des symptômes de l'EM/SFC. Pour les descriptions des symptômes, voir Annexe~\ref{app:personal-symptoms}. Pour les résultats de laboratoire et les antécédents cliniques, voir Annexe~\ref{app:clinical-findings}.

\paragraph{Contexte médicamenteux actuel}
\label{sec:personal-medications}

\paragraph{Médicaments actifs}

\paragraph{Modulation immunitaire}
\begin{itemize}
    \item \textbf{Naltrexone à faible dose (LDN)}: 3\,mg par jour (débuté le 2026-01-05) pour la modulation anti-inflammatoire et immunitaire
    \begin{itemize}
        \item \textit{Horaire}: Prise matinale (note~: le protocole standard préconise une prise nocturne)
        \item \textit{Durée}: Trop tôt pour évaluer l'efficacité (réponse typique~: 4--12 semaines)
        \item \textit{Plan}: Augmenter à 4--4,5\,mg après épuisement du stock actuel
    \end{itemize}
\end{itemize}

\paragraph{Médicaments stimulants}
\begin{itemize}
    \item \textbf{Rilatine MR (méthylphénidate)}: 30\,mg par prise, 1--2 fois par jour pour le soutien cognitif et l'éveil
    \item \textbf{Provigil (modafinil)}: 100\,mg par prise, 1--2 fois par jour pour le maintien de la vigilance
\end{itemize}

\paragraph{Soutien mitochondrial}
\begin{itemize}
    \item \textbf{Urolithin A 2000\,mg + NAD+ 200\,mg (Joiavvy)}: 2 gélules par jour (1000\,mg + 100\,mg par gélule) pour la fonction mitochondriale et l'énergie cellulaire
    \item \textbf{BioActive Q10 Ubiquinol 100\,mg (Pharma Nord)}: 1--2 gélules par jour pour le soutien de la chaîne de transport des électrons
    \item \textbf{Acétyl-L-Carnitine 1000\,mg (Bandini ou équivalent)}: Débuté le 2026-01-21
    \begin{itemize}
        \item \textit{Dose}: 1000\,mg par jour (le matin, de préférence à jeun)
        \item \textit{Forme}: Toute marque réputée fournissant 1000\,mg par dose
        \item \textit{Indication}: Dysfonction de la navette carnitine~; cible à la fois les crampes musculaires et le brouillard cognitif
        \item \textit{Mécanisme}: Ouvre la navette carnitine pour transporter les acides gras à longue chaîne vers les mitochondries~; le groupe acétyle traverse la barrière hémato-encéphalique pour le soutien cognitif
        \item \textit{Délai prévu}: Effet initial en 4--6 semaines, bénéfice maximal en 3--6 mois
        \item \textit{Surveiller}: effets digestifs (nausées, diarrhée), odeur corporelle de poisson (rare), amélioration de l'énergie, clarté cognitive, réduction des crampes musculaires
        \item \textit{Effets synergiques}: Agit avec le CoQ10 et la riboflavine pour soutenir la voie complète de production d'énergie mitochondriale
    \end{itemize}
\end{itemize}

\paragraph{Vitamines et minéraux}
\begin{itemize}
    \item \textbf{D-Cure 25000\,U.I. (Cholécalciférol/Vitamine D3, Laboratoires SMB)}: 1 gélule par semaine
    \begin{itemize}
        \item \textit{Historique}: Carence chronique en vitamine D \textbf{depuis des années} malgré une supplémentation quotidienne à 3000\,U.I./jour (21000\,U.I./semaine insuffisant pour maintenir des taux normaux)
        \item \textit{Protocole actuel}: 25000\,U.I.\ hebdomadaire (dose totale légèrement supérieure seulement au schéma quotidien précédent)
        \item \textit{Statut}: Non encore vérifié par analyses biologiques si ce protocole permet d'atteindre des taux normaux de vitamine D
        \item \textit{Hypothèse}: La prise hebdomadaire peut améliorer l'absorption par rapport au protocole quotidien, peut-être en raison de~:
        \begin{itemize}
            \item Meilleure observance pour la co-ingestion de graisses (plus facile de se souvenir une fois par semaine que quotidiennement)
            \item Une concentration maximale plus élevée surmonte le déficit d'absorption
            \item La malabsorption des graisses affecte davantage les faibles doses quotidiennes que les doses hebdomadaires élevées
        \end{itemize}
        \item \textit{Critique}: \textbf{Doit être pris avec des graisses alimentaires} (vitamine liposoluble)~--- à prendre au déjeuner ou au dîner contenant des graisses~; sans graisses, la carence persistera quelle que soit la dose
        \item Le médecin recommande ce protocole à dose élevée hebdomadaire pour suspicion de malabsorption des graisses~; des analyses de suivi sont nécessaires pour confirmer l'efficacité
    \end{itemize}
    \item \textbf{BEFACT FORTE (Laboratoires SMB)}: 1 comprimé par jour pour la supplémentation en complexe B
    \item \textbf{Vitamine C (Livsane, PXG Pharma)}: 500\,mg par jour pour le soutien antioxydant et l'amélioration de l'absorption du fer
    \item \textbf{N-Acétylcystéine (NAC) 600\,mg (Lysomucil)}: Débuté le 2026-02-13
    \begin{itemize}
        \item \textit{Dose}: 600\,mg par jour (le matin avec les autres suppléments)
        \item \textit{Forme}: Lysomucil (acétylcystéine~--- médicament mucolytique contenant du NAC)
        \item \textit{Indication}: Précurseur du glutathion~; soutien antioxydant et anti-inflammatoire
        \item \textit{Mécanisme}: Fournit de la cystéine (acide aminé limitant pour la synthèse du glutathion)~; piégeage direct des radicaux libres~; réduit l'activation du NF-$\kappa$B
        \item \textit{Délai prévu}: Effets antioxydants en quelques jours~; bénéfices systémiques en 4--8 semaines
        \item \textit{Plan}: Augmenter à 1200\,mg par jour (doses fractionnées) si bien toléré après 2--3 semaines
        \item \textit{Effets synergiques}: Agit avec la Vitamine C (régénère le glutathion)~; sélénium (nécessaire à la fonction de la glutathion peroxydase)
    \end{itemize}
    \item \textbf{Magnecaps Dynatonic (ORIFARM Healthcare)}: 2 gélules par jour pour la supplémentation en magnésium et la fonction musculaire
    \begin{itemize}
        \item \textit{Note}: En cours de remplacement par du glycinate de magnésium pour éviter une interaction potentielle avec le méthylphénidate
    \end{itemize}
    \item \textbf{FerroDyn FORTE (Metagenics)}: 1 gélule par jour pour la supplémentation en fer
    \item \textbf{Vitamine A 5000\,UI (à débuter)}: Une fois par jour avec de l'huile d'olive ou d'autres graisses alimentaires
    \begin{itemize}
        \item \textit{Indication}: Soutien visuel~; soutient la fonction rétinienne et la vision nocturne
        \item \textit{Dosage}: Vitamine liposoluble~--- doit être prise avec des graisses alimentaires (huile d'olive recommandée)
        \item \textit{Sécurité}: 5000\,UI est dans la plage de supplémentation sûre à long terme ($<$10\,000\,UI/jour)
        \item \textit{Horaire}: Peut être prise au repas du matin ou du soir contenant des graisses
    \end{itemize}
\end{itemize}

\paragraph{Protocole de soutien visuel}
\label{subsubsec:vision-support}

Compte tenu de la déficience visuelle progressive avec variation dépendante de l'énergie (voir Section~\ref{subsec:personal-vision}), un protocole de soutien visuel ciblé traite à la fois les composantes structurales et métaboliques~:

\paragraph{Justification.}
La fluctuation de la qualité visuelle en fonction de l'énergie suggère une fatigue du muscle ciliaire liée à l'épuisement de l'ATP. Le soutien de la fonction rétinienne et neurale peut améliorer la stabilité visuelle et potentiellement ralentir la progression.

\paragraph{Protocole de supplémentation.}
\begin{itemize}
    \item \textbf{Lutéine} (10--20\,mg par jour)~: Caroténoïde maculaire~; filtre la lumière bleue et protège les photorécepteurs
    \item \textbf{Zéaxanthine} (2--4\,mg par jour)~: Agit en synergie avec la lutéine~; concentrée dans la macula
    \item \textbf{Taurine} (500--1000\,mg par jour)~: Soutient la fonction des cellules rétiniennes~; abondante dans les photorécepteurs~; peut protéger contre le stress oxydatif
    \item \textbf{DHA (oméga-3)} (500--1000\,mg par jour)~: Composant structural des membranes rétiniennes~; soutient la fonction des photorécepteurs
    \item \textbf{Vitamine A} (5000\,UI par jour)~: Essentielle à la régénération de la rhodopsine (vision nocturne)~; soutient la santé rétinienne générale
\end{itemize}

\paragraph{Bénéfices attendus.}
\begin{itemize}
    \item \textbf{Court terme (4--8 semaines)}~: Amélioration potentielle de la stabilité visuelle~; réduction de la variation quotidienne
    \item \textbf{Moyen terme (3--6 mois)}~: Peut ralentir la progression de la dysfonction accommodative si la composante métabolique est significative
    \item \textbf{Long terme}~: Combiné avec le soutien mitochondrial (Acétyl-L-Carnitine, CoQ10), peut partiellement améliorer la fonction du muscle ciliaire
\end{itemize}

\paragraph{Horaire et absorption.}
\begin{itemize}
    \item Lutéine, zéaxanthine et DHA sont liposolubles~: prendre avec des repas contenant des graisses alimentaires
    \item La taurine est hydrosoluble~: peut être prise avec ou sans nourriture
    \item Peut être combinée avec le régime de supplémentation existant (p.ex., prendre avec le CoQ10 au petit-déjeuner)
\end{itemize}

\paragraph{Surveillance.}
\begin{itemize}
    \item Suivre la qualité visuelle subjective quotidiennement (corréler avec les niveaux d'énergie)
    \item Noter tout changement dans la capacité d'accommodation ou le confort en lecture
    \item Envisager un examen ophtalmologique de suivi à 6 mois pour évaluer les changements objectifs de prescription
\end{itemize}

\paragraph{Gestion des électrolytes}
\begin{itemize}
    \item \textbf{Solution électrolytique personnalisée}~: Préparée à partir d'un mélange sec (100\,g de sucre, 15\,g de sel Jozo à faible teneur en sodium, 15\,g de sel de table)
    \item \textbf{Dosage}~: 7\,g de mélange sec dans 250\,mL d'eau avec 10\,mL de grenadine, deux fois par jour
    \item \textbf{Justification}~: Voir Section~\ref{sec:personal-hydration} pour le protocole détaillé et la stratégie de gestion des électrolytes
\end{itemize}

\paragraph{Protocole de dosage des stimulants.}
Le méthylphénidate et le modafinil peuvent être utilisés individuellement ou en combinaison, avec un \textbf{maximum de 3 comprimés au total par jour} pour les deux médicaments. Les schémas typiques incluent~:
\begin{itemize}
    \item Rilatine MR 30\,mg $\times$ 1--2 (matin, début d'après-midi optionnel)
    \item Provigil 100\,mg $\times$ 1--2 (matin, début d'après-midi optionnel)
    \item Combinaison~: p.ex., 1 Rilatine + 1 Provigil, ou 2 Rilatine + 1 Provigil, ou 1 Rilatine + 2 Provigil
\end{itemize}
La combinaison spécifique dépend des exigences cognitives de la journée et de la sévérité actuelle des symptômes. La dose journalière totale ne doit pas dépasser 3 comprimés pour les deux médicaments. Éviter les prises en fin de journée pour prévenir les perturbations du sommeil.

\paragraph{Considérations importantes}

\paragraph{Risque de fausse énergie.}
Le méthylphénidate et le modafinil sont tous deux des stimulants qui peuvent \textbf{masquer les véritables niveaux d'énergie}. Ils permettent d'«~emprunter~» de l'énergie sur des réserves épuisées. Cela rend la surveillance de la fréquence cardiaque essentielle~--- faites confiance au moniteur plutôt qu'au ressenti subjectif d'énergie. La combinaison des deux stimulants amplifie cet effet de masquage.

\paragraph{Interaction migraine.}
Le méthylphénidate et le modafinil provoquent tous deux une vasoconstriction, facteur déclenchant fréquent de migraine. Cela rend la riboflavine (B2) à 400\,mg/jour et une hydratation adéquate particulièrement importantes.

\paragraph{Médicaments et suppléments à l'étude}
\label{subsec:medications-under-consideration}

Sur la base des preuves cliniques dans les Chapitres~\ref{ch:medications-mechanisms}, \ref{ch:supplements} et \ref{ch:emerging-therapies}, les médicaments et suppléments suivants ont une efficacité documentée pour la gestion des symptômes de l'EM/SFC et sont à l'étude pour des essais futurs. Tous les éléments listés ci-dessous sont couverts dans le document principal.

\paragraph{Soutien autonome et cardiovasculaire}

\paragraph{Ivabradine (2,5\,mg deux fois par jour).}
\begin{itemize}
    \item \textbf{Indication}: Contrôle de la fréquence cardiaque pour POTS/intolérance orthostatique
    \item \textbf{Mécanisme}: Bloqueur sélectif du canal I$_f$~; réduit la fréquence de décharge du nœud sinusal sans affecter la contractilité
    \item \textbf{Justification patient}: Intolérance orthostatique documentée~; variabilité de la fréquence cardiaque à l'effort~; l'utilisation de stimulants complique la régulation autonome
    \item \textbf{Preuves}: Voir Annexe~\ref{app:annotated-bibliography} et Chapitre~\ref{ch:action-mild-moderate}
    \item \textbf{Considérations}: Surveiller la fréquence cardiaque de base~; nécessite une consultation cardiologique~; l'interaction potentielle avec les stimulants doit être évaluée
    \item \textbf{Priorité}: Moyenne (à envisager si les symptômes orthostatiques s'aggravent ou interfèrent avec la fonction)
\end{itemize}

\paragraph{Mestinon/Pyridostigmine (20\,mg, posologie à déterminer).}
\begin{itemize}
    \item \textbf{Indication}: Dysfonction autonome, intolérance orthostatique, soutien cognitif potentiel
    \item \textbf{Mécanisme}: Inhibiteur de l'acétylcholinestérase~; augmente la disponibilité de l'acétylcholine aux synapses parasympathiques
    \item \textbf{Justification patient}: Dysfonction autonome documentée (intolérance orthostatique, FC variable)~; bénéfices cognitifs potentiels étant donné les déficits cholinergiques dans l'EM/SFC
    \item \textbf{Preuves}: Voir Annexe~\ref{app:annotated-bibliography}, Chapitre~\ref{ch:action-mild-moderate} et Chapitre~\ref{ch:integrative-models}
    \item \textbf{Considérations}: Débuter à faible dose (20\,mg) pour évaluer la tolérance~; surveiller les effets secondaires cholinergiques (troubles digestifs, salivation)~; peut être pris avec ou sans nourriture~; peut compléter l'ivabradine pour un soutien autonome complet
    \item \textbf{Priorité}: Moyenne-haute (bénéfice bien documenté dans l'EM/SFC pour les symptômes autonomes)
\end{itemize}

\paragraph{Activation mastocytaire et modulation histaminique}

\paragraph{Lévocétirizine (5\,mg par jour).}
\begin{itemize}
    \item \textbf{Indication}: Syndrome d'activation mastocytaire (SAMA)~; intolérance à l'histamine
    \item \textbf{Mécanisme}: Antihistaminique H1 (deuxième génération, non sédatif)
    \item \textbf{Justification patient}: Antécédents de sensibilisation allergique (panel noix positif), composante mastocytaire potentielle de la fatigue/inflammation
    \item \textbf{Preuves}: Voir Annexe~\ref{app:annotated-bibliography} et Annexe~\ref{app:recommendations}
    \item \textbf{Considérations}: Non sédatif~; peut être pris le matin ou le soir~; durée d'essai 2--4 semaines pour évaluer l'effet sur la fatigue/brouillard mental
    \item \textbf{Priorité}: Moyenne (essai exploratoire)
\end{itemize}

\paragraph{Cimétidine (200\,mg par jour).}
\begin{itemize}
    \item \textbf{Indication}: Blocage du récepteur H2 pour intolérance à l'histamine/SAMA
    \item \textbf{Mécanisme}: Antihistaminique H2~; bloque les récepteurs gastriques à l'histamine
    \item \textbf{Justification patient}: Si le bloqueur H1 (lévocétirizine) montre un bénéfice partiel, le blocage dual H1/H2 peut fournir un contrôle histaminique plus complet
    \item \textbf{Preuves}: Voir Annexe~\ref{app:annotated-bibliography}, Chapitre~\ref{ch:disease-course} et Section~\ref{sec:differential}
    \item \textbf{Considérations}: Peut être combiné avec le bloqueur H1~; surveiller les interactions médicamenteuses (inhibiteur CYP450)~; prendre avec de la nourriture
    \item \textbf{Priorité}: Moyenne (secondaire à l'essai du bloqueur H1)
\end{itemize}

\paragraph{Kétotifène (1\,mg par jour).}
\begin{itemize}
    \item \textbf{Indication}: Stabilisation mastocytaire pour le SAMA
    \item \textbf{Mécanisme}: Stabilisateur de mastocytes~; prévient la dégranulation et la libération d'histamine
    \item \textbf{Justification patient}: Si les antihistaminiques seuls sont insuffisants, la stabilisation mastocytaire traite la cause en amont
    \item \textbf{Preuves}: Voir Annexe~\ref{app:recommendations}, Annexe~\ref{app:annotated-bibliography} et Chapitre~\ref{ch:gut-microbiome}
    \item \textbf{Considérations}: Peut causer une sédation initiale (prise au coucher)~; durée d'essai 4--8 semaines pour l'effet complet~; peut être combiné avec des antihistaminiques
    \item \textbf{Priorité}: Basse-moyenne (escalade si bloqueurs H1/H2 inadéquats)
\end{itemize}

\paragraph{Soutien du sommeil et circadien}

\paragraph{Quviviq/Daridorexant (25\,mg PRN).}
\begin{itemize}
    \item \textbf{Indication}: Endormissement et maintien du sommeil~; alternative aux benzodiazépines
    \item \textbf{Mécanisme}: Antagoniste dual des récepteurs à l'orexine~; favorise le sommeil en bloquant les signaux d'éveil
    \item \textbf{Justification patient}: Qualité du sommeil actuelle variable~; option non addictive pour les crises aiguës lorsque le sommeil est sévèrement perturbé
    \item \textbf{Preuves}: Voir Chapitre~\ref{ch:action-mild-moderate}, Chapitre~\ref{ch:urgent-action-severe} et Annexe~\ref{app:annotated-bibliography}
    \item \textbf{Considérations}: Utilisation PRN lors des crises ou des périodes de stress élevé~; éviter la dépendance nocturne~; sédation résiduelle minimale le lendemain signalée~; coûteux (vérifier la couverture d'assurance)
    \item \textbf{Priorité}: Basse (réserver à la gestion des crises ou aux perturbations sévères du sommeil)
\end{itemize}

\paragraph{Soutien dopaminergique et neurologique}

\paragraph{Aripiprazole à faible dose/LDA (1,5\,mg par jour).}
\begin{itemize}
    \item \textbf{Indication}: Fatigue, dysfonction cognitive, modulation immunitaire potentielle
    \item \textbf{Mécanisme}: Agoniste partiel de la dopamine à faibles doses~; peut réduire la neuroinflammation et améliorer la motivation/l'énergie
    \item \textbf{Justification patient}: Fatigue sévère et dysfonction cognitive malgré l'utilisation de stimulants~; le LDA cible une voie différente (modulation de la dopamine vs.\ inhibition de la recapture)
    \item \textbf{Preuves}: Voir Annexe~\ref{app:annotated-bibliography}, Chapitre~\ref{ch:action-mild-moderate}, Chapitre~\ref{ch:proposed-studies}, Chapitre~\ref{ch:emerging-therapies}, Chapitre~\ref{ch:medications-systems} et Chapitre~\ref{ch:clinical-trials}
    \item \textbf{Considérations}: Dose très faible (dose antipsychotique typique 10--30\,mg~; dose EM/SFC 0,5--2\,mg)~; débuter à faible dose~; surveiller l'akathisie (agitation)~; peut être pris le matin ou le soir~; nécessite une consultation psychiatrique dans de nombreuses juridictions
    \item \textbf{Priorité}: Moyenne-haute (preuves émergentes pour l'EM/SFC~; traite un mécanisme différent des stimulants actuels)
\end{itemize}

\paragraph{Ginkgo biloba/Cerebokan (80\,mg par jour).}
\begin{itemize}
    \item \textbf{Indication}: Fonction cognitive, flux sanguin cérébral, neuroprotection
    \item \textbf{Mécanisme}: Améliore la microcirculation~; antioxydant~; peut améliorer la perfusion cérébrale
    \item \textbf{Justification patient}: Brouillard mental sévère et dysfonction cognitive~; hypoperfusion cérébrale potentielle dans l'EM/SFC
    \item \textbf{Preuves}: Voir Chapitre~\ref{ch:medications-systems}, Chapitre~\ref{ch:urgent-action-severe} et Section~\ref{sec:clinical-brainstorm}
    \item \textbf{Considérations}: Extrait standardisé important (EGb 761)~; surveiller le risque hémorragique si combiné avec des anticoagulants~; durée d'essai 8--12 semaines
    \item \textbf{Priorité}: Basse-moyenne (soutien cognitif adjuvant)
\end{itemize}

\paragraph{Suppléments à l'étude}

\paragraph{Zinc (25--50\,mg par jour).}
\begin{itemize}
    \item \textbf{Indication}: Fonction immunitaire, soutien antioxydant, cofacteur mitochondrial potentiel
    \item \textbf{Mécanisme}: Oligo-élément essentiel~; cofacteur de nombreuses enzymes~; soutient la fonction immunitaire et les systèmes antioxydants
    \item \textbf{Justification patient}: Peut ne pas être adéquatement couvert par le complexe B actuel~; soutient la modulation immunitaire aux côtés du LDN
    \item \textbf{Preuves}: Voir Annexe~\ref{app:case-analysis}, Annexe~\ref{app:clinical-findings} et Chapitre~\ref{ch:action-mild-moderate}
    \item \textbf{Considérations}: Prendre séparément du fer (2--4 h)~; éviter de dépasser 50\,mg/jour à long terme (risque de déplétion en cuivre)~; surveiller les taux sériques si supplémentation > 3 mois
    \item \textbf{Priorité}: Moyenne (risque relativement faible, bénéfice immunitaire potentiel)
\end{itemize}

\paragraph{Glutathion (forme réduite, 250--500\,mg par jour ou liposomal).}
\begin{itemize}
    \item \textbf{Indication}: Stress oxydatif, soutien à la détoxification, protection mitochondriale
    \item \textbf{Mécanisme}: Antioxydant maître~; neutralise directement les radicaux libres~; soutient les voies de détoxification~; protège les mitochondries des dommages oxydatifs
    \item \textbf{Justification patient}: La dysfonction mitochondriale génère un excès de ROS~; déplétion en glutathion documentée dans l'EM/SFC~; peut compléter le CoQ10 et d'autres soutiens mitochondriaux
    \item \textbf{Preuves}: Voir Chapitre~\ref{ch:gut-microbiome}, Chapitre~\ref{ch:energy-metabolism} et Chapitre~\ref{ch:genetics-epigenetics}
    \item \textbf{Considérations}: Biodisponibilité orale faible (utiliser forme liposomale ou sublinguale)~; alternative~: N-acétylcystéine (NAC) 600--1200\,mg comme précurseur du glutathion avec meilleure absorption~; durée d'essai 6--8 semaines
    \item \textbf{Priorité}: Moyenne (soutient la pile mitochondriale~; le NAC peut être plus pratique)
\end{itemize}

\paragraph{PEA/Palmitoyléthanolamine (400\,mg deux fois par jour, micronisé ou ultramicronisé).}
\begin{itemize}
    \item \textbf{Indication}: Gestion de la douleur, modulation mastocytaire, neuroinflammation
    \item \textbf{Mécanisme}: Médiateur de type endocannabinoïde~; agoniste PPAR-$\alpha$~; réduit la dégranulation mastocytaire et la neuroinflammation
    \item \textbf{Justification patient}: Douleurs articulaires lors des crises~; composante mastocytaire potentielle~; efficacité documentée dans les conditions de douleur chronique
    \item \textbf{Preuves}: Voir Chapitre~\ref{ch:translational-findings}, Chapitre~\ref{ch:action-mild-moderate}, Chapitre~\ref{ch:urgent-action-severe}, Chapitre~\ref{ch:medications-systems} et Annexe~\ref{app:research-synthesis}
    \item \textbf{Considérations}: Forme micronisée ou ultramicronisée essentielle pour l'absorption~; prendre avec de la nourriture~; durée d'essai 4--8 semaines~; peut compléter la Griffe du Diable pour la douleur~; synergie avec les stabilisateurs de mastocytes/antihistaminiques
    \item \textbf{Priorité}: Moyenne-haute (bénéfice documenté pour la douleur et l'inflammation~; profil de sécurité favorable)
\end{itemize}

\paragraph{L-Arginine + L-Citrulline (2--3\,g arginine + 1--2\,g citrulline par jour).}
\begin{itemize}
    \item \textbf{Indication}: Production d'oxyde nitrique (NO), fonction vasculaire, tolérance à l'effort
    \item \textbf{Mécanisme}: L'arginine est précurseur du NO~; la citrulline se convertit en arginine avec une meilleure biodisponibilité~; soutient la fonction endothéliale et le flux sanguin
    \item \textbf{Justification patient}: Dysfonction vasculaire potentielle dans l'EM/SFC~; peut améliorer l'apport en oxygène et la tolérance orthostatique~; la citrulline évite le métabolisme de premier passage
    \item \textbf{Preuves}: Voir Annexe~\ref{app:annotated-bibliography}, Chapitre~\ref{ch:gut-microbiome}, Section~\ref{sec:novel-framework}, Chapitre~\ref{ch:integrative-models}, Chapitre~\ref{ch:action-mild-moderate}, Chapitre~\ref{ch:emerging-therapies} et Section~\ref{sec:2025-hypotheses}
    \item \textbf{Considérations}: La forme citrulline-malate peut être supérieure (le malate soutient le cycle de Krebs)~; prendre à jeun pour une meilleure absorption~; éviter si sujet aux boutons de fièvre (l'arginine peut déclencher la réactivation de l'herpès)~; durée d'essai 4--8 semaines
    \item \textbf{Priorité}: Basse-moyenne (soutien vasculaire adjuvant~; relativement sûr)
\end{itemize}

\paragraph{Griffe du Diable/Harpagophytum procumbens (500--1000\,mg d'extrait standardisé, 1--2 fois par jour).}
\begin{itemize}
    \item \textbf{Indication}: Gestion de la douleur, anti-inflammatoire
    \item \textbf{Mécanisme}: Teneur en harpagosides~; inhibition de la COX-2~; réduit le TNF-$\alpha$ et les cytokines inflammatoires
    \item \textbf{Justification patient}: Douleurs articulaires lors des épisodes de MPE~; l'anti-inflammatoire naturel peut réduire la sévérité des crises
    \item \textbf{Preuves}: Voir Chapitre~\ref{ch:medications-systems} et Section~\ref{sec:clinical-brainstorm}
    \item \textbf{Considérations}: Prendre avec de la nourriture~; éviter si sous anticoagulants~; surveiller les troubles digestifs~; extrait standardisé avec teneur en harpagosides précisée~; durée d'essai 4--8 semaines
    \item \textbf{Priorité}: Moyenne (anti-inflammatoire documenté~; peut réduire la douleur de MPE~; profil de sécurité favorable)
\end{itemize}

\paragraph{Stratégie de mise en œuvre}

\paragraph{Séquençage des essais.}
Ne pas initier tous les éléments simultanément. Échelonner les essais pour évaluer les effets individuels~:
\begin{enumerate}
    \item \textbf{Haute priorité} (traiter les symptômes principaux)~: LDA, Mestinon, PEA
    \item \textbf{Priorité moyenne} (spécifique aux symptômes)~: Ivabradine (si aggravation orthostatique), Griffe du Diable (si douleur persistante), Zinc, Glutathion/NAC
    \item \textbf{Basse priorité} (adjuvant)~: Ginkgo, L-Arginine/L-Citrulline, Quviviq (PRN uniquement)
    \item \textbf{Voie SAMA} (si suspectée)~: Lévocétirizine $\to$ ajouter Cimétidine $\to$ ajouter Kétotifène (escalader uniquement si l'étape précédente montre un bénéfice partiel)
\end{enumerate}

\paragraph{Exigences de documentation.}
Pour chaque essai~:
\begin{itemize}
    \item Enregistrer la date de début, la dose et l'horaire dans le journal de suivi médicamenteux (Annexe~\ref{subsec:medication-history})
    \item Documenter les symptômes de base pour comparaison
    \item Définir la durée de l'essai (typiquement 4--8 semaines pour les suppléments, 2--4 semaines pour les médicaments)
    \item Suivre les effets dans le journal quotidien des symptômes (Section~\ref{sec:daily-symptom-journal})
    \item Évaluer le résultat~: continuer, arrêter ou ajuster la dose
\end{itemize}

\paragraph{Consultation médicale requise.}
Tous les médicaments (LDA, Ivabradine, Mestinon, Lévocétirizine, Cimétidine, Kétotifène, Quviviq) nécessitent une ordonnance et l'approbation du médecin. Les suppléments peuvent être testés de manière autonome mais doivent être discutés avec le médecin, surtout si l'on en ajoute au régime médicamenteux existant.

\paragraph{Considérations de coût.}
Voir Annexe~\ref{app:case-analysis} Tableau~\ref{tab:treatment-cost-analysis} pour les coûts mensuels estimés. Prioriser les interventions à fort impact et rentables~; différer les éléments coûteux (Quviviq, alternatives à l'Urolithin A) sauf si essentiels.

\paragraph{Protocole d'horaire des suppléments et médicaments}
\label{subsec:timing-protocol}

Le respect des horaires des suppléments et médicaments est essentiel pour éviter les interactions pouvant réduire l'efficacité ou provoquer des effets indésirables. La préoccupation principale est de protéger la Rilatine MR d'une libération prématurée.

\paragraph{Séparations critiques (minimum 2--4 heures)}

\paragraph{Méthylphénidate MR $\leftrightarrow$ Magnésium.}
La Rilatine MR est une formulation à libération modifiée conçue pour se libérer progressivement sur plusieurs heures. Certaines formes de magnésium (carbonate, hydroxyde) modifient le pH gastrique et provoquent une libération prématurée («~dose dumping~»), entraînant des pics de fréquence cardiaque et une réduction de la durée d'effet.
\begin{itemize}
    \item \textbf{Séparation sûre}~: Minimum 2--4 heures~; optimal 6--8 heures
    \item \textbf{Protocole actuel}~: Stimulants le matin/après-midi~; magnésium au coucher (6--8+ heures)
    \item \textbf{La forme du magnésium est importante}~: Le glycinate a un effet minimal sur le pH~; carbonate/oxyde/hydroxyde sont à haut risque
\end{itemize}

\paragraph{Méthylphénidate MR $\leftrightarrow$ Antiacides/Composés à pH élevé.}
Tout supplément qui élève significativement le pH gastrique présente le même risque que le carbonate de magnésium~:
\begin{itemize}
    \item \textbf{Éviter près des stimulants}~: Carbonate de calcium (Tums), bicarbonate de sodium (bicarbonate alimentaire), antiacides
    \item \textbf{Sans danger}~: Solution électrolytique (NaCl + KCl ne modifie pas significativement le pH)
\end{itemize}

\paragraph{Fer $\leftrightarrow$ Calcium/Magnésium.}
Le fer et le calcium/magnésium entrent en compétition pour l'absorption intestinale. Séparer de 2--4 heures pour une absorption optimale du fer.

\paragraph{Programme journalier optimal}

\paragraph{Matin (avec ou juste avant le petit-déjeuner).}
Prendre ensemble~--- aucune séparation nécessaire~:
\begin{itemize}
    \item Rilatine MR 30\,mg
    \item Provigil 100\,mg (if taking)
    \item LDN 3\,mg
    \item Acetyl-L-carnitine 1000\,mg
    \item Urolithin A 2000\,mg + NAD+ 200\,mg (2 capsules)
    \item CoQ10 Ubiquinol 100\,mg (requires dietary fat---take with breakfast)
    \item BEFACT FORTE (1 tablet)
    \item Vitamin C 500\,mg
    \item NAC 600\,mg (Lysomucil)
    \item Electrolytes 250\,mL (7\,g dry mix)
    \item FerroDyn FORTE (1 capsule)---optional: can separate 30--60 min for better absorption
\end{itemize}

\textbf{Note on iron timing}: Iron absorbs best on an empty stomach with vitamin C but often causes troubles digestifs. Taking with breakfast reduces absorption slightly but improves tolerance. If iron deficiency is significant, consider taking 1 hour before breakfast with only vitamin C 500\,mg.

\paragraph{Afternoon.}
\begin{itemize}
    \item Electrolytes 250\,mL (7\,g dry mix)
    \item Optional second stimulant dose if needed (maintain 3-pill daily maximum)
\end{itemize}

\textbf{Rationale for afternoon electrolytes}: Helps clear accumulated lactic acid from morning activities; maintains blood volume for orthostatic tolerance; provides continued glucose availability when fat-burning is impaired.

\paragraph{Midday/Lunch (optional alternative timing for B2).}
\begin{itemize}
    \item Riboflavin (B2) 400\,mg (with lunch containing dietary fat)
\end{itemize}

\textbf{Note}: Riboflavin can be taken at lunch or dinner. Both timings work equally well as long as the meal contains fat. Choose based on which meal typically has more fat content or personal preference.

\paragraph{Evening (with dinner, 2--4 hours after last stimulant).}
\begin{itemize}
    \item Riboflavin (B2) 400\,mg (water-soluble; taken with dinner for consistency)
    \item D-Cure 25000\,U.I.\ (weekly, fat-soluble---\textbf{requires dietary fat})
\end{itemize}

\paragraph{Bedtime (minimum 2--4 hours after stimulants).}
\begin{itemize}
    \item Magnesium glycinate 300--400\,mg
\end{itemize}

\textbf{Rationale}: Bedtime dosing maximizes effect on nocturnal muscle cramps and provides sleep support. The 6--8 hour separation from morning stimulants eliminates risk of methylphenidate interaction.

\paragraph{Optimal Absorption Conditions for Each Supplement}

Understanding how each supplement is best absorbed ensures maximum effectiveness. This section details specific absorption requirements.

\begin{table}[htbp]
\centering
\caption{Supplement Absorption Optimization}
\label{tab:supplement-absorption}
\small
\begin{tabular}{lp{5cm}p{5cm}}
\toprule
\textbf{Supplement} & \textbf{Best Absorption} & \textbf{Avoid Taking With} \\
\midrule
\textbf{Rilatine MR} & With or without food; consistent timing matters most & Magnesium carbonate/hydroxide, antacids, high-pH compounds (2--4 hr separation) \\
\textbf{Provigil} & With or without food & No significant interactions \\
\textbf{LDN} & With or without food & No significant interactions \\
\midrule
\textbf{Acetyl-L-carnitine} & With food to reduce troubles digestifs; water-soluble & None significant \\
\textbf{CoQ10 Ubiquinol} & \textbf{Requires dietary fat} (fat-soluble); best with fatty meal & Minimal absorption without fat \\
\textbf{Riboflavin (B2)} & Water-soluble; can take with or without food & None significant \\
\textbf{Vitamin D3} & \textbf{Requires dietary fat} (fat-soluble); take with fatty meal & Minimal absorption without fat \\
\midrule
\textbf{Iron (FerroDyn)} & \textbf{Best: empty stomach with Vitamin C}; causes troubles digestifs for many; compromise: with food + Vitamin C & Calcium, magnesium, zinc (compete for absorption); coffee, tea, dairy (reduce absorption) \\
\textbf{Vitamin C} & With or without food; enhances iron absorption when taken together & None significant \\
\textbf{Magnesium glycinate} & Best at bedtime on empty stomach or light snack; well-tolerated form & Separate from methylphenidate by 2--4 hours minimum \\
\midrule
\textbf{Urolithin A 2000\,mg + NAD+ 200\,mg} & With or without food (check product label) & None significant \\
\textbf{BEFACT FORTE} & With food for better B-vitamin absorption & None significant \\
\textbf{Electrolytes} & Sip throughout day with water; contains glucose for quick energy & None significant \\
\bottomrule
\end{tabular}
\end{table}

\paragraph{Key Absorption Principles.}

\begin{enumerate}
    \item \textbf{Fat-soluble vitamins} (CoQ10, Vitamin D3): Require dietary fat for absorption
    \begin{itemize}
        \item Take with meals containing fats: oils, butter, cheese, nuts, avocado, fatty fish, eggs
        \item Without fat, absorption is dramatically reduced (may absorb <10\% of dose)
        \item Does not need to be a large amount of fat---a tablespoon of olive oil or a handful of nuts is sufficient
        \item \textbf{Clinical note}: History of chronic vitamin D deficiency \textbf{for years} despite 3000\,U.I.\ daily supplementation strongly suggests fat malabsorption, which is common in ME/CFS with mitochondrial dysfunction. This makes proper timing with dietary fat \textit{essential}, not optional.
        \item \textbf{Vitamin D3 dosing}: Physician recommends weekly 25000\,U.I.\ over daily lower doses for potentially superior absorption in cases of suspected malabsorption; effectiveness in this case not yet verified with laboratory testing
    \end{itemize}

    \item \textbf{Iron optimization}: Best absorbed on empty stomach with vitamin C
    \begin{itemize}
        \item \textbf{Ideal}: 1 hour before breakfast with only vitamin C 500\,mg
        \item \textbf{Practical}: With breakfast + vitamin C if troubles digestifs occurs (slightly lower absorption, much better tolerance)
        \item Avoid coffee, tea, or dairy within 1 hour (tannins and calcium inhibit absorption)
        \item Separate from calcium/magnesium supplements by 2--4 hours
    \end{itemize}

    \item \textbf{Methylphenidate protection}: Modified-release must be protected from pH changes
    \begin{itemize}
        \item Magnesium carbonate/hydroxide causes premature ``dose dumping''
        \item Antacids alter stomach pH and release kinetics
        \item Magnesium glycinate at bedtime provides 6--8 hour separation (safe)
    \end{itemize}

    \item \textbf{Mineral competition}: Iron, calcium, magnesium, and zinc compete for same transporters
    \begin{itemize}
        \item Separate these supplements by 2--4 hours for optimal absorption
        \item Current protocol achieves this: iron morning, magnesium bedtime
    \end{itemize}

    \item \textbf{Water-soluble vitamins and amino acids}: Generally well-absorbed with or without food
    \begin{itemize}
        \item Acetyl-L-carnitine, BEFACT FORTE, Vitamin C, NAD+, Urolithin A
        \item Taking with food reduces troubles digestifs for sensitive individuals
        \item No fat required for absorption
    \end{itemize}
\end{enumerate}

\paragraph{Practical Implementation.}

\textbf{Morning routine optimization}:
\begin{itemize}
    \item Ensure breakfast contains some fat (e.g., eggs, cheese, butter, nuts, or olive oil) for CoQ10 absorption
    \item Take iron with vitamin C; avoid coffee/tea for 1 hour if possible
    \item All other morning supplements well-absorbed together
\end{itemize}

\textbf{Midday/Evening meal optimization}:
\begin{itemize}
    \item Ensure lunch or dinner contains fat for Riboflavin B2 absorption
    \item Fatty fish, olive oil in salad dressing, nuts, avocado, cheese all sufficient
    \item Take B2 with whichever meal typically has more fat
\end{itemize}

\textbf{Bedtime routine}:
\begin{itemize}
    \item Magnesium glycinate can be taken on empty stomach or with light snack
    \item Primary goal is separation from methylphenidate (achieved by bedtime dosing)
\end{itemize}

\paragraph{What to Avoid Near Stimulants}

Do not take within 2--4 hours of methylphenidate:
\begin{itemize}
    \item Magnesium carbonate, oxide, or hydroxide
    \item Calcium carbonate (e.g., Tums)
    \item Sodium bicarbonate (baking soda)
    \item Antacids (Gaviscon, Rennie, etc.)
\end{itemize}

\textbf{Safe near stimulants}: Electrolyte solution (sodium chloride + potassium chloride), magnesium glycinate (at bedtime only), food.

\paragraph{Summary of Timing Rationale}

\begin{enumerate}
    \item \textbf{Stimulant protection}: Magnesium separated by 6--8+ hours to prevent premature methylphenidate release
    \item \textbf{Cramp management}: Magnesium at bedtime targets nocturnal cramps when ATP reserves are lowest
    \item \textbf{Iron absorption}: Taken with vitamin C enhances absorption; separation from calcium/magnesium prevents competition
    \item \textbf{Fat-soluble optimization}: CoQ10 and vitamin D taken with fatty meals
    \item \textbf{Lactic acid clearance}: Afternoon electrolytes support metabolic waste removal from morning activities
    \item \textbf{Sleep hygiene}: No stimulants after early afternoon; magnesium supports sleep
\end{enumerate}

\paragraph{Fat Malabsorption Management}
\label{subsec:fat-malabsorption}

\paragraph{Personal Clinical Evidence of Fat Malabsorption}

Clinical observations in this case suggest impaired fat absorption:
\begin{itemize}
    \item \textbf{Vitamin D deficiency for years} despite daily supplementation at 3000\,U.I.\ (21000\,U.I./week total)
    \item Vitamin D is fat-soluble and requires adequate fat absorption
    \item Current trial: weekly 25000\,U.I.\ (only 20\% higher total dose) to test if dosing frequency affects absorption
    \item Effectiveness not yet verified with laboratory testing
\end{itemize}

\paragraph{Hypothesized Mechanisms for Fat Malabsorption in ME/CFS}

\textit{Note: The following mechanisms are hypothesized based on known ME/CFS pathophysiology; their relative contribution in this case is unknown.}

Fat malabsorption may create a vicious cycle with mitochondrial dysfunction:

\paragraph{Primary Mechanism (Hypothesized).}
\begin{itemize}
    \item \textbf{Mitochondrial dysfunction}: Cannot efficiently process fats even when absorbed
    \item Carnitine shuttle failure blocks long-chain fatty acids from entering mitochondria
    \item This is the root cause being addressed by Acetyl-L-Carnitine supplementation
\end{itemize}

\paragraph{Secondary Contributing Factors (Hypothesized).}
\begin{enumerate}
    \item \textbf{Reduced bile acid production/secretion}: Liver requires energy to synthesize bile; impaired energy metabolism may reduce bile availability for fat emulsification
    \item \textbf{Gut dysmotility}: Autonomic dysfunction causes slow intestinal transit, potentially reducing contact time for absorption
    \item \textbf{Possible SIBO}: Slow motility creates environment for small intestinal bacterial overgrowth, which can consume bile acids before host can use them
    \item \textbf{Pancreatic enzyme insufficiency}: Pancreas requires energy to produce lipase; reduced lipase production may impair fat breakdown
\end{enumerate}

\paragraph{Clinical Consequence.}
Impaired fat absorption directly affects:
\begin{itemize}
    \item Vitamin D3 (fat-soluble)
    \item CoQ10 Ubiquinol (fat-soluble)
    \item Cellular energy availability (if dietary fats cannot be absorbed and utilized)
\end{itemize}

\paragraph{Immediate Management Strategies}

\paragraph{1. Medium-Chain Triglyceride (MCT) Oil --- Highest Priority.}

MCT oil bypasses normal fat digestion and is the single most effective intervention:
\begin{itemize}
    \item \textbf{Mechanism}: Medium-chain fatty acids (C8--C10) are absorbed directly without requiring bile acids or pancreatic lipase
    \item \textbf{Advantage}: Goes straight to liver for energy; does not require carnitine shuttle
    \item \textbf{Starting dose}: 1 teaspoon (5\,mL) daily
    \item \textbf{Target dose}: 1 tablespoon (15\,mL) daily, increase gradually over 1--2 weeks
    \item \textbf{Timing}: Take with fat-soluble vitamins (morning with CoQ10, or evening with B2/D3)
    \item \textbf{Administration}: Can add to coffee, tea, smoothies, or drizzle on food
    \item \textbf{Caution}: Increase slowly; rapid escalation can cause diarrhea
\end{itemize}

\begin{tcolorbox}[breakable,colback=blue!5!white,colframe=blue!75!black,title=Why MCT Oil Improves Fat Burning Without Causing Weight Gain]

\textbf{Understanding the two types of dietary fat:}

\textbf{Long-chain fats (14--22 carbons)} --- what is broken in ME/CFS:
\begin{itemize}
    \item Most dietary fats: butter, olive oil, meat fat, nuts, cheese
    \item Most stored body fat (including the 5--6\,kg weight gain over 3 years)
    \item \textbf{Require carnitine shuttle} to enter mitochondria for energy production
    \item \textbf{Problem}: Carnitine shuttle is blocked $\rightarrow$ cannot burn these for energy $\rightarrow$ ``running on empty'' sensation
    \item Body cannot access stored fat reserves despite having them available
\end{itemize}

\textbf{Medium-chain fats (8--10 carbons)} --- MCT oil bypasses the broken system:
\begin{itemize}
    \item \textbf{Do NOT require carnitine shuttle}
    \item Absorbed directly $\rightarrow$ go straight to liver $\rightarrow$ directly into mitochondria
    \item Provide immediate energy without needing the broken carnitine transport system
    \item \textbf{Rarely stored as body fat} --- preferentially oxidized for energy
    \item Used by athletes for quick energy WITHOUT weight gain
\end{itemize}

\textbf{The two-part metabolic strategy:}

\begin{enumerate}
    \item \textbf{MCT oil (immediate effect)}: Emergency energy bypass
    \begin{itemize}
        \item Provides fuel that mitochondria can actually USE right now
        \item Bypasses broken carnitine shuttle
        \item Also provides fat for vitamin D, CoQ10, and B2 absorption
        \item Amount is small: 1 tablespoon = 120 calories, used for energy not storage
    \end{itemize}

    \item \textbf{Acetyl-L-Carnitine (4--6 week effect)}: Repairs the main system
    \begin{itemize}
        \item Gradually opens the carnitine shuttle over weeks
        \item Allows body to burn long-chain fats again (stored body fat + dietary fats)
        \item Enables access to stored fat reserves for energy
        \item Promotes fat burning, not fat storage
    \end{itemize}
\end{enumerate}

\textbf{Why this protocol will NOT cause weight gain:}
\begin{itemize}
    \item MCT oil goes to liver for immediate energy production (not stored as body fat)
    \item Small amount added: 1 tablespoon daily = 120 calories
    \item Acetyl-L-Carnitine enables fat BURNING (unlocks stored body fat for energy)
    \item Better energy $\rightarrow$ potentially more activity $\rightarrow$ improved metabolic rate
    \item Better mitochondrial function $\rightarrow$ efficient fat utilization instead of storage
\end{itemize}

\textbf{Expected metabolic outcome:}
\begin{itemize}
    \item Week 1--2: MCT provides immediate energy; vitamins absorb better
    \item Week 4--6: Carnitine shuttle begins opening; body accesses long-chain fats
    \item Month 3--6: Full effect --- burning stored body fat + MCT energy
    \item Net result: Better energy + potential fat loss (if activity increases), NOT weight gain
\end{itemize}

\textbf{Clinical note}: The chronic vitamin D deficiency despite supplementation proves fat absorption/utilization is already impaired. This protocol fixes the broken system --- it does not add fat on top of a working system. MCT oil is a \textbf{metabolic intervention}, not simply ``adding dietary fat.''

\end{tcolorbox}

\paragraph{2. Digestive Enzymes with High Lipase.}

Supplemental enzymes compensate for inadequate pancreatic enzyme production:
\begin{itemize}
    \item \textbf{Current supplement}: Metagenics MetaDigest TOTAL (received 2026-01-22)
    \begin{itemize}
        \item Comprehensive enzyme formula containing lipase, protease, amylase, cellulase, lactase, and other enzymes
        \item Supports digestion of fats, proteins, carbohydrates, fiber, and dairy
        \item Particularly important for fat-soluble vitamin absorption (D3, CoQ10, B2)
    \end{itemize}
    \item \textbf{Timing}: Take immediately before or with first bite of meals containing fat-soluble vitamins
    \item \textbf{Frequency}: Any meal where CoQ10, B2, or D3 are taken
    \item \textbf{Alternative products}: NOW Foods Digestive Enzymes, Enzymedica Digest Gold
\end{itemize}

\paragraph{3. Strategic Dietary Fat with Fat-Soluble Vitamins.}

Ensure adequate fat co-ingestion with each fat-soluble vitamin dose:

\textbf{Morning (with CoQ10 Ubiquinol)}:
\begin{itemize}
    \item MCT oil: 1 teaspoon--1 tablespoon in coffee/tea or on food
    \item OR: Eggs cooked in butter/olive oil
    \item OR: Handful of nuts (almonds, walnuts)
    \item OR: 1 tablespoon olive oil on food
    \item \textbf{MetaDigest TOTAL}: 1 capsule immediately before or with first bite of meal
\end{itemize}

\textbf{Evening (with Riboflavin B2; weekly with Vitamin D3)}:
\begin{itemize}
    \item MCT oil: 1 teaspoon--1 tablespoon (if not taken in morning)
    \item OR: Fatty fish (salmon, mackerel, sardines) --- also provides omega-3s
    \item OR: Half an avocado
    \item OR: Cheese with meal
    \item OR: Olive oil in salad dressing (2 tablespoons)
    \item \textbf{MetaDigest TOTAL}: 1 capsule immediately before or with first bite of meal
\end{itemize}

\paragraph{4. Easier-to-Absorb Fat Types.}

Prioritize fats that require less digestive effort and support cardiovascular health:
\begin{itemize}
    \item \textbf{Best (highest priority)}:
    \begin{itemize}
        \item \textbf{MCT oil} (pure C8 or C8/C10 blend): Bypasses normal digestion; immediate energy
        \item \textbf{Olive oil}: Monounsaturated fat; heart-healthy; well-tolerated; excellent for fat-soluble vitamin absorption
    \end{itemize}
    \item \textbf{Good}: Avocado, fatty fish (salmon, mackerel---also provides omega-3s)
    \item \textbf{Moderate}: Nuts (if tolerated), eggs
    \item \textbf{Use with caution (high saturated fat/cholesterol)}:
    \begin{itemize}
        \item Butter, ghee: High in saturated fat and cholesterol; given elevated LDL (132--137 mg/dL, target $<$100), prioritize olive oil and MCT oil instead
        \item Cheese, cream: High saturated fat; use sparingly if needed for palatability
    \end{itemize}
    \item \textbf{Avoid or minimize}: Fried foods, very fatty meats, tropical oils other than MCT
\end{itemize}

\begin{tcolorbox}[breakable,colback=yellow!5!white,colframe=yellow!75!black,title=Important: Coconut Oil $\neq$ MCT Oil]
\textbf{Clarification on coconut products:}
\begin{itemize}
    \item \textbf{MCT oil}: Pure medium-chain triglycerides (C8 caprylic acid and/or C10 capric acid) extracted and concentrated from coconut or palm kernel oil
    \begin{itemize}
        \item 100\% medium-chain fats
        \item Bypasses normal fat digestion
        \item Does NOT require carnitine shuttle
        \item \textbf{This is what you need for metabolic support}
    \end{itemize}
    \item \textbf{Coconut oil}: Whole coconut oil contains only $\sim$15\% MCTs; the remaining $\sim$85\% are long-chain saturated fats
    \begin{itemize}
        \item Mostly long-chain fats (lauric acid C12, myristic acid C14, etc.)
        \item These long-chain fats \textbf{DO require the broken carnitine shuttle}
        \item High in saturated fat (raises LDL cholesterol)
        \item \textbf{Not a substitute for MCT oil}
    \end{itemize}
\end{itemize}

\textbf{Recommendation}: Use pure MCT oil (C8 or C8/C10), not coconut oil, for metabolic support. If using coconut oil for cooking, understand it will not provide the same bypass benefits.
\end{tcolorbox}

\paragraph{Optional Advanced Interventions}

Consider these if basic strategies (MCT oil + digestive enzymes + dietary fat) are insufficient:

\paragraph{Ox Bile/Bile Salts.}
Provides exogenous bile acids when endogenous production is inadequate:
\begin{itemize}
    \item Typical dose: 100--500\,mg with fatty meals
    \item Only add if digestive enzymes alone insufficient
    \item Take with meals containing fat-soluble vitamins
    \item \textbf{Not first-line}: Try MCT oil and digestive enzymes first
\end{itemize}

\paragraph{Bile Flow Support (Gentler Approach).}
Natural cholagogues (bile flow stimulants) before adding ox bile:
\begin{itemize}
    \item Beet root powder or beet juice (supports bile production)
    \item Artichoke extract (stimulates bile flow)
    \item Dandelion root tea (mild cholagogue)
\end{itemize}

\paragraph{SIBO Testing and Treatment.}
If digestive symptoms prominent or interventions ineffective:
\begin{itemize}
    \item SIBO (small intestinal bacterial overgrowth) consumes bile acids
    \item Breath test for diagnosis
    \item Treatment: Rifaximin (antibiotic) or herbal antimicrobials
    \item Not urgent; consider if other interventions fail
\end{itemize}

\paragraph{Long-Term Metabolic Correction}

\paragraph{Acetyl-L-Carnitine.}
Already starting 2026-01-21; should improve fat metabolism at cellular level:
\begin{itemize}
    \item Opens carnitine shuttle to allow long-chain fatty acids into mitochondria
    \item Does not fix absorption, but improves utilization of absorbed fats
    \item Timeline: 4--6 weeks to assess effect
    \item This addresses the \textit{root cause} of fat metabolism dysfunction
\end{itemize}

\paragraph{Implementation Protocol}

\paragraph{Week 1--2: Basic Protocol.}
\begin{enumerate}
    \item \textbf{Add MCT oil}: Start 1 teaspoon daily with CoQ10 dose
    \item \textbf{Add digestive enzymes (MetaDigest TOTAL)}: Take immediately before meals containing fat-soluble vitamins
    \item \textbf{Ensure dietary fat}: Add fat sources to meals where CoQ10, B2, or D3 are taken
    \item \textbf{Monitor tolerance}: Watch for troubles digestifs, diarrhea (indicates too much MCT oil too fast)
\end{enumerate}

\paragraph{Week 3--4: Optimize Dosing.}
\begin{enumerate}
    \item Increase MCT oil to 1 tablespoon daily if tolerated
    \item Adjust timing based on convenience (morning vs.\ evening)
    \item Continue digestive enzymes with all fat-soluble vitamin doses
\end{enumerate}

\paragraph{Week 4--6: Assess and Adjust.}
\begin{enumerate}
    \item Monitor energy levels (better fat absorption/utilization should improve energy)
    \item Note any changes in digestive symptoms
    \item Acetyl-L-Carnitine should be showing early effects by week 4--6
    \item Consider adding ox bile or bile flow support if no improvement
\end{enumerate}

\paragraph{Month 2--3: Laboratory Verification.}
\begin{enumerate}
    \item Repeat vitamin D levels to verify 25000\,U.I.\ weekly protocol effectiveness
    \item If vitamin D normalizes: fat absorption strategy is working
    \item If vitamin D remains low: consider advanced interventions (ox bile, SIBO testing)
\end{enumerate}

\paragraph{Expected Benefits if Successful}

\begin{enumerate}
    \item \textbf{Vitamin D normalization}: Levels rise to normal range on current protocol
    \item \textbf{Improved energy}: Better fat absorption and utilization provides more cellular fuel
    \item \textbf{Enhanced CoQ10 effectiveness}: Better absorption improves mitochondrial electron transport chain function
    \item \textbf{Reduced post-meal fatigue}: Improved nutrient extraction from meals
    \item \textbf{Better Acetyl-L-Carnitine synergy}: Improved fat absorption + improved fat utilization = multiplicative benefit
\end{enumerate}

\paragraph{Monitoring Checklist}

Track the following to assess effectiveness:
\begin{itemize}
    \item Vitamin D levels (retest in 2--3 months)
    \item Subjective energy levels throughout day
    \item Digestive symptoms (bloating, diarrhea, gas, etc.)
    \item Post-meal energy (do you crash after eating or feel better?)
    \item Muscle cramps frequency/severity (fat-soluble vitamin absorption affects cellular function)
\end{itemize}

\paragraph{Mitochondrial Support Protocol}
\label{sec:personal-mitoprotocol}

Based on the metabolic dysfunction described above, the following supplements address specific bottlenecks:

\begin{table}[htbp]
\centering
\caption{Mitochondrial Support Supplements}
\label{tab:mito-supplements}
\begin{tabular}{llp{6cm}}
\toprule
\textbf{Supplement} & \textbf{Dosage} & \textbf{Mechanism} \\
\midrule
Acetyl-L-carnitine & 500--2000\,mg/day & Opens the ``shuttle'' to transport fatty acids into mitochondria; crosses blood-brain barrier for cognitive support \\
CoQ10 (Ubiquinol) & 100--200\,mg/day & Acts as ``spark plug'' in electron transport chain; antioxidant for mitochondrial membranes \\
Riboflavin (B2) & 400\,mg/day & Precursor to FAD; essential for beta-oxidation; migraine prevention \\
Magnesium glycinate & 300--400\,mg at night & ``Off switch'' for muscle contraction; critical cofactor for PDH and TCA cycle \\
D-Ribose & 5\,g twice daily (10\,g total) & Building block of ATP molecule; directly replenishes cellular ATP stores; faster-acting than other mitochondrial support \\
NADH & 10--20\,mg/day & Cofactor that primes the energy cycle \\
\bottomrule
\end{tabular}
\end{table}

\paragraph{Introduction Protocol.}
Introduce one supplement every 7--10 days to monitor for paradoxical reactions (common in ME/CFS):
\begin{enumerate}
    \item Week 1: Magnesium glycinate (addresses cramps immediately)
    \item Week 2: CoQ10 (begins mitochondrial support)
    \item Week 3: Acetyl-L-carnitine (opens fat-burning pathway)
    \item Week 4: NADH (enhances ATP production)
    \item Ongoing: Riboflavin for migraine prevention (requires 4--12 weeks for effect)
\end{enumerate}

\paragraph{Hydration and Electrolyte Management}
\label{sec:personal-hydration}

\paragraph{Rationale for Electrolytes}

Plain water may be rapidly excreted, potentially diluting remaining minerals (hyponatremia). In ME/CFS with low blood volume:
\begin{itemize}
    \item \textbf{Sodium}: Acts as a ``sponge'' pulling water into blood vessels
    \item \textbf{Potassium}: Maintains cellular electrical charge
    \item \textbf{Magnesium}: Prevents muscle cell ``lock-up''
\end{itemize}

\paragraph{Protocol}
\begin{itemize}
    \item \textbf{Daytime}: Oral rehydration solution (ORS) in 500\,mL--1\,L water, sipped throughout the day
    \item \textbf{Evening}: Magnesium glycinate tablet before bed (separate from ORS by several hours)
    \item \textbf{Emergency}: For acute lactic events, may add 1/4 teaspoon sodium bicarbonate to electrolyte drink
\end{itemize}

\paragraph{Custom Rehydration Solution}
\label{subsec:custom-ors}

Two formula variants are documented: a standard formula and a reduced-sugar alternative.

\paragraph{Standard Formula (High-Both Electrolytes)}

\begin{tcolorbox}[breakable,colback=blue!5!white,colframe=blue!75!black,title=Standard Formula --- High Sodium + High Potassium]
\textbf{Dry mix preparation:}
\begin{itemize}
    \item 100\,g white sugar
    \item 15\,g Jozo low-sodium salt (approximately 66\% KCl, 33\% NaCl --- provides potassium)
    \item 15\,g table salt (provides sodium)
    \item \textbf{Total dry mix: 130\,g}
\end{itemize}

\textbf{Per-dose preparation (twice daily):}
\begin{itemize}
    \item 7\,g of dry mix dissolved in 250\,mL water
    \item 10\,g grenadine syrup (for palatability)
\end{itemize}
\end{tcolorbox}

\paragraph{Composition Analysis per 250\,mL Dose.}

\begin{table}[htbp]
\centering
\caption{Standard Formula Composition per Dose}
\label{tab:standard-ors}
\begin{tabular}{lll}
\toprule
\textbf{Component} & \textbf{Amount} & \textbf{Notes} \\
\midrule
Low-sodium salt & $\sim$0.81\,g & From 7\,g $\times$ (15/130) \\
\quad Potassium (as KCl) & $\sim$0.27\,g ($\sim$6.9\,mmol) & 66\% KCl $\times$ 0.52 K content \\
\quad Sodium (from low-Na salt) & $\sim$0.10\,g ($\sim$4.3\,mmol) & 33\% NaCl $\times$ 0.39 Na content \\
Table salt (NaCl) & $\sim$0.81\,g & From 7\,g $\times$ (15/130) \\
\quad Sodium (from table salt) & $\sim$0.32\,g ($\sim$13.9\,mmol) & NaCl $\times$ 0.39 Na content \\
\textbf{Total Sodium} & $\sim$0.42\,g ($\sim$18.2\,mmol) & \\
\textbf{Total Potassium} & $\sim$0.27\,g ($\sim$6.9\,mmol) & \\
Sugar (from mix) & $\sim$5.4\,g & From 7\,g $\times$ (100/130) \\
Sugar (from grenadine) & $\sim$7--8\,g & Typical grenadine content \\
\textbf{Total sugar} & $\sim$12--13\,g & \\
\bottomrule
\end{tabular}
\end{table}

\paragraph{Comparison to WHO ORS Standard.}

\begin{table}[htbp]
\centering
\caption{Standard Formula vs.\ WHO ORS (per liter equivalent)}
\label{tab:ors-comparison}
\begin{tabular}{lccc}
\toprule
\textbf{Component} & \textbf{Standard ($\times$4)} & \textbf{WHO ORS} & \textbf{Assessment} \\
\midrule
Sodium & $\sim$73\,mmol/L & 75\,mmol/L & Matches WHO \\
Potassium & $\sim$28\,mmol/L & 20\,mmol/L & Good for cramps \\
Glucose & $\sim$220\,mmol/L & 75\,mmol/L & High \\
Osmolarity & $\sim$260\,mOsm/L & 245\,mOsm/L & Acceptable \\
\bottomrule
\end{tabular}
\end{table}

\paragraph{Why Both Potassium AND Sodium Matter for Cramps.}

For ME/CFS muscle cramps, the instinct to maximize potassium is understandable---potassium is the ``off switch'' for muscle contraction. However, sodium serves a complementary and equally critical role:

\begin{enumerate}
    \item \textbf{Potassium}: Directly enables muscle relaxation by restoring the resting membrane potential after contraction. Without adequate potassium, muscle fibers remain in a partially contracted state.

    \item \textbf{Sodium}: Expands blood volume, which is essential for:
    \begin{itemize}
        \item Delivering oxygen to muscles (preventing the anaerobic switch)
        \item Clearing lactic acid from tissues (impaired clearance worsens cramps)
        \item Maintaining blood pressure during orthostatic stress
    \end{itemize}
\end{enumerate}

In ME/CFS with orthostatic intolerance, inadequate sodium leads to poor circulation $\rightarrow$ lactate accumulation $\rightarrow$ more cramps. The potassium addresses the \emph{contraction} side; sodium addresses the \emph{metabolic waste clearance} side.

\paragraph{Practical Considerations.}
\begin{itemize}
    \item \textbf{Taste}: The formula is noticeably salty. The grenadine helps mask this.
    \item \textbf{Hypertension}: Only a concern if you have high blood pressure. ME/CFS typically involves \emph{low} blood pressure, making high sodium intake beneficial rather than harmful.
    \item \textbf{Daily total}: With 2 doses/day, total sodium intake is $\sim$0.84\,g from ORS alone---well within safe limits and often recommended for POTS/orthostatic intolerance (some protocols recommend 3--5\,g sodium/day total).
\end{itemize}

\paragraph{Sugar Content Analysis}

The 100\,g sugar in the dry mix may seem excessive. Here is the actual daily intake:

\begin{table}[htbp]
\centering
\caption{Daily Sugar Intake from ORS}
\label{tab:sugar-analysis}
\begin{tabular}{lcc}
\toprule
\textbf{Source} & \textbf{Per Dose} & \textbf{Per Day (2 doses)} \\
\midrule
Sugar from dry mix & $\sim$5.4\,g & $\sim$10.8\,g \\
Sugar from grenadine & $\sim$7--8\,g & $\sim$14--16\,g \\
\textbf{Total} & $\sim$12--13\,g & $\sim$24--26\,g \\
\bottomrule
\end{tabular}
\end{table}

\paragraph{Context.}
\begin{itemize}
    \item WHO ORS contains $\sim$13.5\,g glucose per 500\,mL---similar to your 2-dose daily total from the mix alone
    \item A can of soda contains $\sim$35--40\,g sugar
    \item Typical daily ``added sugar'' guidance: 25--50\,g
\end{itemize}

\paragraph{ME/CFS-Specific Concerns.}
Sugar serves a functional purpose: the sodium-glucose cotransporter (SGLT1) in the intestine requires glucose to pull sodium (and water) into the bloodstream. However, excessive sugar can cause:
\begin{enumerate}
    \item Glucose spikes $\rightarrow$ insulin spikes $\rightarrow$ potential energy crashes
    \item Excess calories without nutritional benefit
    \item The grenadine adds ``empty'' sugar that doesn't improve electrolyte absorption
\end{enumerate}

\paragraph{Reduced-Sugar Alternative Formula}

\begin{tcolorbox}[breakable,colback=green!5!white,colframe=green!75!black,title=Lower-Sugar Formula]
\textbf{Dry mix preparation:}
\begin{itemize}
    \item \textbf{50\,g white sugar} (reduced from 100\,g---still sufficient for SGLT1 function)
    \item 15\,g Jozo low-sodium salt (high potassium)
    \item 15\,g table salt (high sodium)
    \item Total dry mix: \textbf{80\,g}
\end{itemize}

\textbf{Per-dose preparation:}
\begin{itemize}
    \item 4.3\,g of dry mix in 250\,mL water (maintains same electrolyte concentration)
    \item Use \textbf{sugar-free grenadine} or a squeeze of lemon for flavor
\end{itemize}

\textbf{Result:} $\sim$2.7\,g sugar per dose, $\sim$5.4\,g per day---an 80\% reduction while maintaining full electrolyte benefit.
\end{tcolorbox}

\paragraph{Recommendation.}
If glucose spikes or weight management are concerns, switch to the 50\,g sugar formula with sugar-free flavoring. The electrolyte absorption will still work adequately---the WHO formula uses glucose primarily for severe diarrhea rehydration where maximal absorption speed is critical. For daily ME/CFS maintenance, lower sugar is acceptable.

\paragraph{Long-Term Electrolyte Safety and Monitoring}
\label{subsec:electrolyte-safety}

\paragraph{Sodium Intake Analysis}

\paragraph{Current Daily Intake from Electrolyte Protocol.}

With the standard formula at 2 doses daily (500\,mL total):

\begin{table}[htbp]
\centering
\caption{Sodium Content per Dose and Daily Total}
\label{tab:sodium-content}
\begin{tabular}{lcc}
\toprule
\textbf{Source} & \textbf{Per 250\,mL Dose} & \textbf{Daily (2 doses)} \\
\midrule
Low-sodium salt (NaCl component) & 104\,mg & 208\,mg \\
Table salt (pure NaCl) & 315\,mg & 630\,mg \\
\midrule
\textbf{Total Sodium} & \textbf{419\,mg} & \textbf{838\,mg} \\
\textbf{Total Sodium (grams)} & \textbf{0.42\,g} & \textbf{0.84\,g} \\
\bottomrule
\end{tabular}
\end{table}

\paragraph{Comparison to Guidelines.}

\begin{itemize}
    \item \textbf{General population guideline}: <2300\,mg (2.3\,g) sodium daily
    \item \textbf{Your current intake}: 838\,mg (0.84\,g) from electrolytes alone
    \item \textbf{Status}: Well within safe limits; only 36\% of standard guideline maximum
    \item \textbf{Total daily intake}: 0.84\,g from electrolytes + dietary sodium (likely 1--2\,g) = approximately 2--3\,g total
\end{itemize}

\paragraph{ME/CFS/POTS Context.}

\begin{itemize}
    \item \textbf{Therapeutic target for orthostatic intolerance}: 6--10\,g sodium daily
    \item \textbf{Your current intake}: 2--3\,g total (including diet) --- actually \emph{below} therapeutic target
    \item \textbf{Could increase if needed}: If orthostatic symptoms worsen, current intake could be safely doubled or tripled
\end{itemize}

\paragraph{Duration of Use: Can This Be Taken Indefinitely?}

\paragraph{Short Answer: Yes, with Monitoring.}

At your current dose (0.84\,g/day from electrolytes), there is \textbf{no time limit} for use. This can be continued indefinitely with basic monitoring.

\paragraph{Safety Conditions for Long-Term Use.}

Electrolyte supplementation at this level is safe indefinitely if:

\begin{enumerate}
    \item \textbf{Blood pressure remains normal} (<140/90 mmHg)
    \begin{itemize}
        \item ME/CFS typically involves low blood pressure
        \item Sodium intake helps normalize BP, not raise it excessively
        \item Monitor monthly
    \end{itemize}

    \item \textbf{No kidney disease}
    \begin{itemize}
        \item Your eGFR: 81--82\,mL/min (normal range 59--137)
        \item Creatinine: 1.09--1.10\,mg/dL (normal range 0.72--1.25)
        \item Current kidney function: \textbf{Normal} --- safe for long-term sodium intake
    \end{itemize}

    \item \textbf{No heart failure}
    \begin{itemize}
        \item Not documented in your case
        \item If heart failure develops, reduce sodium immediately
    \end{itemize}

    \item \textbf{No edema (swelling)}
    \begin{itemize}
        \item Check ankles, feet, hands for swelling
        \item If edema develops, reduce sodium
    \end{itemize}
\end{enumerate}

\paragraph{Why Long-Term Use Is Safe in ME/CFS}

\paragraph{Pathophysiological Justification.}

\begin{enumerate}
    \item \textbf{Low blood volume is the underlying problem}: ME/CFS/POTS patients have reduced circulating blood volume (Section~\ref{sec:blood-volume} discusses mechanisms)

    \item \textbf{Sodium expands blood volume}: This is \emph{therapeutic}, correcting a deficit rather than adding excess

    \item \textbf{Not the same as general population}: Standard low-sodium guidelines assume normal blood volume; ME/CFS involves pathological hypovolemia

    \item \textbf{Standard medical treatment}: High sodium intake (6--10\,g/day) is prescribed indefinitely for POTS patients as first-line therapy
\end{enumerate}

\paragraph{Your Specific Advantage.}

Your current intake (0.84\,g from electrolytes) is:
\begin{itemize}
    \item Far below the therapeutic range for POTS (6--10\,g)
    \item Only 36\% of standard guideline maximum (2.3\,g)
    \item Providing cognitive benefit without orthostatic intolerance improvement (suggesting cellular/metabolic effect)
    \item Extremely conservative dose with large safety margin
\end{itemize}

\paragraph{Monitoring Protocol}

\paragraph{Monthly (Home Monitoring).}

\begin{itemize}
    \item \textbf{Blood pressure}: Check weekly initially, then monthly once stable
    \begin{itemize}
        \item Target: Maintain <140/90 (upper limit of normal)
        \item If ME/CFS baseline is low (e.g., 100/60), sodium may raise to 110/70 --- this is beneficial
        \item Action threshold: If TAconsistently >135/85, discuss with physician
    \end{itemize}

    \item \textbf{Edema check}: Inspect ankles, feet, hands for swelling
    \begin{itemize}
        \item Press thumb into skin for 5 seconds; if indentation remains, indicates edema
        \item If present, reduce sodium intake immediately
    \end{itemize}

    \item \textbf{Symptom tracking}:
    \begin{itemize}
        \item Cognitive function (primary benefit observed)
        \item Orthostatic tolerance (dizziness on standing)
        \item Overall energy level
        \item Any new symptoms (headaches, excessive thirst, etc.)
    \end{itemize}
\end{itemize}

\paragraph{Every 3--6 Months (Laboratory Testing).}

\begin{itemize}
    \item \textbf{Kidney function}:
    \begin{itemize}
        \item Creatinine, eGFR (already tracked)
        \item If eGFR declines >10\,mL/min from baseline, reduce sodium
        \item If creatinine rises >1.3\,mg/dL, reduce sodium
    \end{itemize}

    \item \textbf{Electrolytes}:
    \begin{itemize}
        \item Serum sodium (target: 135--145\,mEq/L)
        \item Serum potassium (target: 3.5--5.0\,mEq/L)
        \item If sodium >145 or potassium <3.5, adjust formulation
    \end{itemize}
\end{itemize}

\paragraph{When to Stop or Reduce}

\paragraph{Immediate Discontinuation Criteria.}

Stop electrolyte supplementation immediately if:
\begin{itemize}
    \item Blood pressure >150/95 on multiple measurements
    \item Edema (swelling) develops in ankles, feet, or hands
    \item Serum sodium >148\,mEq/L (hypernatremia)
    \item Acute kidney injury (eGFR drops suddenly)
    \item Heart failure diagnosed
\end{itemize}

\paragraph{Reduce Dose (50\% reduction) if:}
\begin{itemize}
    \item Blood pressure consistently 135--145/85--90 (borderline high)
    \item Mild ankle swelling (trace edema)
    \item Serum sodium 145--148\,mEq/L (upper normal)
    \item eGFR declines gradually but remains >60\,mL/min
\end{itemize}

\paragraph{Potassium Considerations}

\paragraph{Current Potassium Intake.}

From electrolyte solution (per dose):
\begin{itemize}
    \item Low-sodium salt (66\% KCl): 0.808\,g × 0.66 = 0.533\,g KCl
    \item Potassium content: 0.533\,g × 0.52 (K content of KCl) = 0.277\,g potassium (277\,mg)
    \item \textbf{Daily total (2 doses)}: 554\,mg potassium
\end{itemize}

\paragraph{Safety.}

\begin{itemize}
    \item \textbf{Recommended daily intake}: 2600--3400\,mg (Institute of Medicine)
    \item \textbf{Your electrolyte contribution}: 554\,mg (only 16--21\% of recommended intake)
    \item \textbf{Total with diet}: Likely 2000--3000\,mg total (adequate but not excessive)
    \item \textbf{Upper limit}: 4700\,mg/day considered safe for healthy kidneys
    \item \textbf{Your kidney function}: Normal; no concerns with current potassium intake
\end{itemize}

\paragraph{Summary: Duration and Safety}

\begin{tcolorbox}[breakable,colback=green!5!white,colframe=green!75!black,title=Can This Be Taken Indefinitely?]

\textbf{Yes, at your current dose (0.84\,g sodium/day), this protocol can be continued indefinitely.}

\textbf{Conditions for safe long-term use:}
\begin{itemize}
    \item Monitor blood pressure monthly (target <140/90)
    \item Check for edema monthly (ankle/foot swelling)
    \item Laboratory monitoring every 3--6 months (kidney function, electrolytes)
    \item Discontinue if TA>150/95, edema develops, or kidney function declines
\end{itemize}

\textbf{Your specific situation:}
\begin{itemize}
    \item Current dose is only 36\% of general population guideline maximum
    \item Far below therapeutic dose for POTS (6--10\,g)
    \item Kidney function normal (eGFR 81--82)
    \item Blood pressure likely low at baseline (ME/CFS typical)
    \item Cognitive benefit suggests addressing a real deficit
\end{itemize}

\textbf{Could even increase if needed:}
\begin{itemize}
    \item If orthostatic symptoms worsen, could safely increase to 2--3\,g sodium/day
    \item Large safety margin exists at current intake
\end{itemize}

\textbf{Bottom line:} No time limit. Continue with basic monitoring.

\end{tcolorbox}

\paragraph{Heart Rate Pacing}
\label{sec:personal-pacing}

\paragraph{The ``Safety Zone'' Strategy}

Since mitochondria struggle to burn fat efficiently and switch to anaerobic glycolysis too early, the goal is to keep heart rate below the ventilatory threshold.

\paragraph{Conservative ME/CFS Formula.}
\[
\text{Limite FC cible} = (220 - \text{age}) \times 0.55
\]

\paragraph{Application.}
\begin{itemize}
    \item Stay below this limit to remain in the ``aerobic'' zone where the body attempts to use fat and oxygen cleanly
    \item Even simple tasks (brushing teeth, standing to cook) may exceed this limit
    \item The ``training'' is learning to sit or rest the moment the heart rate monitor alerts
    \item This prevents the lactic acid accumulation that causes next-day crashes
\end{itemize}

\paragraph{Critical Warning}

\begin{tcolorbox}[breakable,colback=red!5!white,colframe=red!75!black,title=Avertissement~: médicaments stimulants]
Lors de la prise de méthylphénidate ou de modafinil, la perception subjective de l'énergie n'est pas fiable. Ces médicaments peuvent masquer les signaux d'alarme du corps. \textbf{La surveillance de la fréquence cardiaque est essentielle}~--- faites confiance aux mesures objectives plutôt qu'à votre ressenti.
\end{tcolorbox}

\paragraph{Symptom Interconnections}
\label{sec:personal-interconnections}

Understanding how symptoms relate helps with clinical reasoning:

\begin{figure}[htbp]
\centering
\begin{tikzpicture}[
    node distance=2cm,
    box/.style={rectangle, draw, rounded corners, minimum width=3cm, minimum height=1cm, align=center, font=\small},
    arrow/.style={->, >=stealth, thick}
]
    % Central node
    \node[box, fill=red!20] (mito) {Mitochondrial\\Dysfunction};

    % Symptom nodes
    \node[box, fill=blue!20, above left=of mito] (fatigue) {Fatigue /\\``Running Empty''};
    \node[box, fill=blue!20, above right=of mito] (brainfog) {Brain Fog /\\Cognitive Impairment};
    \node[box, fill=blue!20, below left=of mito] (cramps) {Muscle Cramps\\(Unexpected)};
    \node[box, fill=blue!20, below right=of mito] (airhunger) {Air Hunger /\\Breathlessness};
    \node[box, fill=orange!20, below=of mito] (lactate) {Lactic Acid\\Accumulation};
    \node[box, fill=purple!20, right=3cm of mito] (migraine) {Migraines};

    % Arrows from central dysfunction
    \draw[arrow] (mito) -- (fatigue);
    \draw[arrow] (mito) -- (brainfog);
    \draw[arrow] (mito) -- (cramps);
    \draw[arrow] (mito) -- (airhunger);
    \draw[arrow] (mito) -- (lactate);

    % Secondary connections
    \draw[arrow] (lactate) -- (cramps);
    \draw[arrow] (lactate) -- (migraine);
    \draw[arrow] (lactate) to[bend left=30] (fatigue);

\end{tikzpicture}
\caption{Interconnection of symptoms via mitochondrial dysfunction and lactic acid accumulation}
\label{fig:symptom-interconnection}
\end{figure}

\paragraph{Key Insight.}
The same ``clogged'' energy system that causes muscle cramps is a primary driver for migraines. Stopping the ``muscle burn'' events (through pacing and metabolic support) often decreases migraine frequency.

\paragraph{``Rolling Crash'' Recognition}
\label{sec:personal-rollingcrash}

When symptoms worsen gradually over months despite apparent rest, this indicates a \textbf{rolling crash}---the current ``rest'' is not actually resting the system.

\paragraph{Common Causes.}
\begin{itemize}
    \item \textbf{Invisible effort}: Cognitive activity (scrolling, reading, light exposure, sound) triggers the same metabolic failure as physical effort
    \item \textbf{Orthostatic stress}: Simply sitting upright causes ``preload failure'' where blood doesn't return adequately to the heart
    \item \textbf{Insufficient horizontal rest}: May need more hours per day completely flat
\end{itemize}

\paragraph{Advocacy Warning.}
Patient advocacy groups emphasize that when symptoms worsen despite ``refusing effort,'' the response should be \emph{more} rest, not attempts to ``push through.'' The 2024 NIH study's ``effort preference'' terminology was criticized precisely because it could be misinterpreted as suggesting patients should override their protective pacing.

\paragraph{Nocturnal ATP Depletion Management}
\label{sec:nocturnal-atp}

\paragraph{The Overnight Energy Crisis}

Nocturnal muscle cramps and morning exhaustion result from ATP depletion during sleep:

\paragraph{Why ATP Depletes Overnight.}
\begin{itemize}
    \item During 8+ hour overnight fast, no food glucose coming in
    \item Body \textbf{should} switch to fat oxidation (burning stored fat for ATP production)
    \item \textbf{Problem}: Carnitine shuttle blocked $\rightarrow$ cannot access fat stores for energy
    \item ATP reserves progressively drop through the night
    \item Muscles require ATP to relax; low ATP $\rightarrow$ muscles ``lock up'' $\rightarrow$ cramps
    \item Wake up exhausted despite sleeping because cells were starving overnight
\end{itemize}

\paragraph{Clinical Consequence.}
\begin{itemize}
    \item Nocturnal cramps (throat, neck, legs, spontaneous locations)
    \item Unrefreshing sleep
    \item Morning exhaustion worse than evening exhaustion
    \item Feeling ``more tired after sleep than before''
\end{itemize}

\paragraph{Immediate Management Strategies}

\paragraph{1. Bedtime MCT Oil (Highest Priority).}

Provides fat-based energy that bypasses the blocked carnitine shuttle:
\begin{itemize}
    \item \textbf{Dose}: 1 teaspoon (5\,mL) MCT oil
    \item \textbf{Timing}: 30--60 minutes before bed
    \item \textbf{Mechanism}: Medium-chain fats do NOT require carnitine shuttle; go straight to liver for energy production
    \item \textbf{Benefit}: Provides fuel overnight that mitochondria can actually use
    \item \textbf{Expected effect}: Reduced nocturnal cramps, less severe morning exhaustion
\end{itemize}

\paragraph{2. D-Ribose Before Bed (Direct ATP Replenishment).}

Provides building blocks to maintain ATP overnight:
\begin{itemize}
    \item \textbf{Dose}: 5\,g D-Ribose powder dissolved in water
    \item \textbf{Timing}: Before bed (in addition to 5\,g morning dose for 10\,g total daily)
    \item \textbf{Mechanism}: Simple sugar that's a direct building block of ATP molecule; replenishes cellular ATP stores
    \item \textbf{Timeline}: Some people notice effect within days; assess at 2 weeks
    \item \textbf{Benefit}: Gives cells raw material to maintain ATP production overnight
\end{itemize}

\paragraph{3. Slow-Release Carbohydrate Before Bed (Optional).}

Extends glucose availability into sleep:
\begin{itemize}
    \item \textbf{Options}:
    \begin{itemize}
        \item Small portion oatmeal (1/2 cup)
        \item 1--2 rice cakes with nut butter
        \item Small banana
        \item Greek yogurt + berries (protein slows carb absorption)
    \end{itemize}
    \item \textbf{Rationale}: Provides slow glucose release overnight without spiking blood sugar
    \item \textbf{Caution}: Not a substitute for MCT oil or D-Ribose; use as adjunct if needed
\end{itemize}

\paragraph{4. Magnesium Glycinate at Bedtime (Already Implemented).}

Helps muscles relax despite suboptimal ATP:
\begin{itemize}
    \item \textbf{Dose}: 300--400\,mg magnesium glycinate
    \item \textbf{Mechanism}: Magnesium is the ``off switch'' for muscle contraction; helps muscles work with less ATP
    \item \textbf{Already in protocol}: Continue taking as documented
\end{itemize}

\paragraph{Long-Term Solution}

\paragraph{Acetyl-L-Carnitine (Root Cause Repair).}

Gradually opens the carnitine shuttle over 4--6 weeks:
\begin{itemize}
    \item \textbf{Starting 2026-01-21}: 1000\,mg daily
    \item \textbf{Mechanism}: Repairs the blocked carnitine shuttle, allowing long-chain fat oxidation overnight
    \item \textbf{Timeline}: 4--6 weeks for initial effect; 3--6 months for maximum benefit
    \item \textbf{Outcome}: Eventually enables normal fat burning during sleep, reducing reliance on bedtime interventions
    \item \textbf{Expectation}: This is the actual fix; MCT oil and D-Ribose are temporary supports while repair happens
\end{itemize}

\paragraph{Complete Bedtime Protocol}

\paragraph{Immediate Implementation (Start Tonight).}
\begin{enumerate}
    \item \textbf{30--60 minutes before bed}: 1 teaspoon MCT oil
    \item \textbf{Before bed}: Magnesium glycinate 300--400\,mg (already doing)
    \item \textbf{Optional}: Small slow-carb snack if still experiencing severe cramps
\end{enumerate}

\paragraph{Add This Week.}
\begin{enumerate}
    \item \textbf{Get D-Ribose powder}
    \item \textbf{Protocol}: 5\,g in morning, 5\,g before bed (10\,g total daily)
    \item \textbf{Expected timeline}: Assess at 2 weeks for nocturnal cramp reduction
\end{enumerate}

\paragraph{Expected Timeline.}
\begin{itemize}
    \item \textbf{Days 1--7}: MCT oil + D-Ribose provide immediate overnight ATP support; may reduce cramp frequency/severity
    \item \textbf{Weeks 2--4}: Continue bedtime protocol; assess improvement in morning energy and nighttime cramps
    \item \textbf{Weeks 4--6}: Acetyl-L-Carnitine begins opening carnitine shuttle; gradual improvement in natural fat oxidation overnight
    \item \textbf{Month 3+}: Reduced reliance on bedtime interventions as fat-burning pathway restores
\end{itemize}

\paragraph{Monitoring Checklist}

Track the following to assess effectiveness:
\begin{itemize}
    \item Nocturnal cramp frequency (number per night)
    \item Nocturnal cramp locations (throat, neck, legs, other)
    \item Morning exhaustion severity (0--10 scale)
    \item ``How tired am I after 8 hours sleep compared to before bed?''
    \item Time to feel ``functional'' after waking (even with stimulants)
\end{itemize}

%%%%%%%%%%%%%%%%%%%%%%%%%%%%%%%%%%%%%%%%%%%%%%%%%%%%%%%%%%%%%%%%%%%%%%%%%%%%%%%
% ANTIHISTAMINE TRIAL TRACKING
%%%%%%%%%%%%%%%%%%%%%%%%%%%%%%%%%%%%%%%%%%%%%%%%%%%%%%%%%%%%%%%%%%%%%%%%%%%%%%%

\paragraph{Antihistamine/MCAS Trial Tracking}
\label{sec:antihistamine-trial}

This section provides a structured template for tracking empirical antihistamine trials for suspected mast cell activation. See Section~\ref{sec:mcas-mild-moderate} for full protocol details and Chapter~\ref{ch:immune-dysfunction}, Section~\ref{sec:mcas} for pathophysiology.

\paragraph{Trial Protocol Summary}

\paragraph{Indication for Trial}
Check if ANY of the following apply:
\begin{itemize}
    \item[$\Box$] Food sensitivities/intolerances (especially new-onset or progressive)
    \item[$\Box$] Documented allergies (elevated IgE to foods, pollens, environmental allergens)
    \item[$\Box$] Flushing, hives, itching
    \item[$\Box$] Reactive to fragrances, chemicals, smoke
    \item[$\Box$] symptômes digestifs (post-meal nausea, bloating, diarrhea)
    \item[$\Box$] Unexplained anxiety or panic-like episodes
    \item[$\Box$] Fluctuating brain fog (worse after eating or exposure to triggers)
    \item[$\Box$] Orthostatic intolerance with documented MCAS features
\end{itemize}

\paragraph{Selected Protocol}
Choose antihistamine regimen:
\begin{itemize}
    \item[$\Box$] \textbf{Option 1 (Standard)}: Loratadine 10 mg OR fexofenadine 180 mg + famotidine 20 mg BID
    \item[$\Box$] \textbf{Option 2 (Superior)}: Rupatadine 10--20 mg + famotidine 20 mg BID
    \item[$\Box$] \textbf{Option 3 (Natural)}: Quercetin 500--1000 mg + famotidine 20 mg BID
    \item[$\Box$] \textbf{Combination}: Rupatadine + famotidine + quercetin
\end{itemize}

\paragraph{Low-Histamine Diet}
\begin{itemize}
    \item[$\Box$] Yes, implementing strict low-histamine diet
    \item[$\Box$] No, antihistamines only
\end{itemize}

\paragraph{Baseline Assessment (Pre-Trial)}

\paragraph{Date Started:} \rule{4cm}{0.4pt}

\paragraph{Baseline Symptoms} (rate 0--10 before starting trial):
\begin{table}[htbp]
\centering
\begin{tabular}{lc}
\toprule
\textbf{Symptom} & \textbf{Baseline Severity (0--10)} \\
\midrule
Brain fog / cognitive clarity & \rule{1cm}{0.4pt} \\
Energy level & \rule{1cm}{0.4pt} \\
Post-meal fatigue & \rule{1cm}{0.4pt} \\
symptômes digestifs (nausea, bloating, diarrhea) & \rule{1cm}{0.4pt} \\
Flushing / skin reactions & \rule{1cm}{0.4pt} \\
Anxiety / panic-like episodes & \rule{1cm}{0.4pt} \\
Orthostatic tolerance (standing ability) & \rule{1cm}{0.4pt} \\
Allergic symptoms (sneezing, itching) & \rule{1cm}{0.4pt} \\
\bottomrule
\end{tabular}
\end{table}

\paragraph{Weekly Progress Tracking}

\paragraph{Week 1}
\begin{itemize}
    \item \textbf{Dates}: \rule{3cm}{0.4pt} to \rule{3cm}{0.4pt}
    \item \textbf{Medications taken}: \rule{8cm}{0.4pt}
    \item \textbf{Adherence}: \rule{2cm}{0.4pt} \% (days taken / 7 days)
    \item \textbf{Side effects}: \rule{10cm}{0.4pt}
    \item \textbf{Symptom changes}:
    \begin{table}[htbp]
    \centering
    \begin{tabular}{lcc}
    \toprule
    \textbf{Symptom} & \textbf{Week 1 (0--10)} & \textbf{Change from Baseline} \\
    \midrule
    Brain fog & \rule{1cm}{0.4pt} & \rule{2cm}{0.4pt} \\
    Energy & \rule{1cm}{0.4pt} & \rule{2cm}{0.4pt} \\
    Post-meal fatigue & \rule{1cm}{0.4pt} & \rule{2cm}{0.4pt} \\
    symptômes digestifs & \rule{1cm}{0.4pt} & \rule{2cm}{0.4pt} \\
    Flushing & \rule{1cm}{0.4pt} & \rule{2cm}{0.4pt} \\
    Anxiety & \rule{1cm}{0.4pt} & \rule{2cm}{0.4pt} \\
    Orthostatic tolerance & \rule{1cm}{0.4pt} & \rule{2cm}{0.4pt} \\
    Allergic symptoms & \rule{1cm}{0.4pt} & \rule{2cm}{0.4pt} \\
    \bottomrule
    \end{tabular}
    \end{table}
    \item \textbf{Notes}: \rule{10cm}{0.4pt}
\end{itemize}

\paragraph{Week 2}
\begin{itemize}
    \item \textbf{Dates}: \rule{3cm}{0.4pt} to \rule{3cm}{0.4pt}
    \item \textbf{Medications taken}: \rule{8cm}{0.4pt}
    \item \textbf{Adherence}: \rule{2cm}{0.4pt} \%
    \item \textbf{Side effects}: \rule{10cm}{0.4pt}
    \item \textbf{Symptom changes}:
    \begin{table}[htbp]
    \centering
    \begin{tabular}{lcc}
    \toprule
    \textbf{Symptom} & \textbf{Week 2 (0--10)} & \textbf{Change from Baseline} \\
    \midrule
    Brain fog & \rule{1cm}{0.4pt} & \rule{2cm}{0.4pt} \\
    Energy & \rule{1cm}{0.4pt} & \rule{2cm}{0.4pt} \\
    Post-meal fatigue & \rule{1cm}{0.4pt} & \rule{2cm}{0.4pt} \\
    symptômes digestifs & \rule{1cm}{0.4pt} & \rule{2cm}{0.4pt} \\
    Flushing & \rule{1cm}{0.4pt} & \rule{2cm}{0.4pt} \\
    Anxiety & \rule{1cm}{0.4pt} & \rule{2cm}{0.4pt} \\
    Orthostatic tolerance & \rule{1cm}{0.4pt} & \rule{2cm}{0.4pt} \\
    Allergic symptoms & \rule{1cm}{0.4pt} & \rule{2cm}{0.4pt} \\
    \bottomrule
    \end{tabular}
    \end{table}
    \item \textbf{Notes}: \rule{10cm}{0.4pt}
\end{itemize}

\paragraph{Week 3}
\begin{itemize}
    \item \textbf{Dates}: \rule{3cm}{0.4pt} to \rule{3cm}{0.4pt}
    \item \textbf{Medications taken}: \rule{8cm}{0.4pt}
    \item \textbf{Adherence}: \rule{2cm}{0.4pt} \%
    \item \textbf{Symptom changes}: Brain fog \rule{1cm}{0.4pt}, Energy \rule{1cm}{0.4pt}, Digestif\rule{1cm}{0.4pt}, Flushing \rule{1cm}{0.4pt}
    \item \textbf{Notes}: \rule{10cm}{0.4pt}
\end{itemize}

\paragraph{Week 4}
\begin{itemize}
    \item \textbf{Dates}: \rule{3cm}{0.4pt} to \rule{3cm}{0.4pt}
    \item \textbf{Medications taken}: \rule{8cm}{0.4pt}
    \item \textbf{Adherence}: \rule{2cm}{0.4pt} \%
    \item \textbf{Symptom changes}: Brain fog \rule{1cm}{0.4pt}, Energy \rule{1cm}{0.4pt}, Digestif\rule{1cm}{0.4pt}, Flushing \rule{1cm}{0.4pt}
    \item \textbf{Notes}: \rule{10cm}{0.4pt}
\end{itemize}

\paragraph{Discontinuation Test (Week 4)}

\paragraph{Purpose}
To confirm whether antihistamines are providing benefit. Stop medications for 2--3 days and monitor for symptom worsening.

\paragraph{Discontinuation Period}
\begin{itemize}
    \item \textbf{Stopped medications on}: \rule{4cm}{0.4pt}
    \item \textbf{Duration off medications}: \rule{1cm}{0.4pt} days
    \item \textbf{Symptom changes during discontinuation}:
    \begin{itemize}
        \item[$\Box$] Symptoms worsened significantly (confirms benefit)
        \item[$\Box$] Symptoms unchanged (no MCAS component)
        \item[$\Box$] Symptoms improved (paradoxical response)
    \end{itemize}
    \item \textbf{Specific symptoms that worsened}: \rule{8cm}{0.4pt}
    \item \textbf{Resumed medications on}: \rule{4cm}{0.4pt}
    \item \textbf{Symptoms after resuming}:
    \begin{itemize}
        \item[$\Box$] Rapid improvement (confirms treatment effect)
        \item[$\Box$] No change
    \end{itemize}
\end{itemize}

\paragraph{Final Assessment}

\paragraph{Overall Response}
\begin{itemize}
    \item[$\Box$] \textbf{Clear benefit} --- Continue antihistamine therapy long-term
    \item[$\Box$] \textbf{Partial benefit} --- Consider optimizing dose or adding quercetin
    \item[$\Box$] \textbf{No benefit} --- Discontinue (symptoms not MCAS-driven)
    \item[$\Box$] \textbf{Adverse effects} --- Discontinue and try alternative H1 blocker
\end{itemize}

\paragraph{Percent Improvement} (overall symptom burden): \rule{2cm}{0.4pt} \%

\paragraph{Most Improved Symptoms}:
\begin{enumerate}
    \item \rule{6cm}{0.4pt}
    \item \rule{6cm}{0.4pt}
    \item \rule{6cm}{0.4pt}
\end{enumerate}

\paragraph{Symptoms That Did NOT Improve}:
\begin{enumerate}
    \item \rule{6cm}{0.4pt}
    \item \rule{6cm}{0.4pt}
\end{enumerate}

\paragraph{Long-Term Plan}
\begin{itemize}
    \item[$\Box$] Continue current regimen indefinitely
    \item[$\Box$] Increase dose (specify): \rule{6cm}{0.4pt}
    \item[$\Box$] Add quercetin or other mast cell stabilizer
    \item[$\Box$] Switch to rupatadine for superior PAF antagonism
    \item[$\Box$] Discontinue antihistamines
    \item[$\Box$] Other: \rule{8cm}{0.4pt}
\end{itemize}

\paragraph{Clinical Notes}:
\begin{itemize}
    \item \rule{14cm}{0.4pt}
    \item \rule{14cm}{0.4pt}
    \item \rule{14cm}{0.4pt}
\end{itemize}

%%%%%%%%%%%%%%%%%%%%%%%%%%%%%%%%%%%%%%%%%%%%%%%%%%%%%%%%%%%%%%%%%%%%%%%%%%%%%%%
% DAILY SYMPTOM JOURNAL
%%%%%%%%%%%%%%%%%%%%%%%%%%%%%%%%%%%%%%%%%%%%%%%%%%%%%%%%%%%%%%%%%%%%%%%%%%%%%%%

\paragraph{Daily Symptom Journal}
\label{sec:personal-journal}

This section serves as a longitudinal record of symptoms, medications, and disease evolution. Regular documentation enables pattern recognition, supports clinical consultations, and provides evidence for treatment adjustments.

\paragraph{Journal Entry Template}
\label{subsec:journal-template}

Each entry should capture:
\begin{itemize}
    \item \textbf{Date and time}
    \item \textbf{Overall energy level} (0--10 scale)
    \item \textbf{Sleep quality} (hours, refreshing or not)
    \item \textbf{Primary symptoms} and severity
    \item \textbf{Medications taken} (with doses and timing)
    \item \textbf{Activities} (type and duration)
    \item \textbf{Triggers identified}
    \item \textbf{Notable observations}
\end{itemize}

\paragraph{Severity Rating Scale}
\label{subsec:medical-severity-scale}

\begin{table}[htbp]
\centering
\caption{Symptom Severity Scale}
\label{tab:medical-severity-scale}
\begin{tabular}{cl}
\toprule
\textbf{Score} & \textbf{Description} \\
\midrule
0 & Absent \\
1--2 & Mild: noticeable but not limiting \\
3--4 & Moderate: affects function, manageable \\
5--6 & Significant: substantially limits activity \\
7--8 & Severe: minimal function possible \\
9--10 & Extreme: incapacitating \\
\bottomrule
\end{tabular}
\end{table}

%------------------------------------------------------------------------------
% JOURNAL ENTRIES BEGIN HERE
%------------------------------------------------------------------------------

\paragraph{January 2026}
\label{subsec:journal-2026-01}

\paragraph{2026-01-20.}
\begin{description}
    \item[Energy:] /10
    \item[Sleep:] hours, refreshing: Yes/No
    \item[Symptoms:]
    \begin{itemize}
        \item Fatigue: /10
        \item Brain fog: /10
        \item Air hunger: /10
        \item Leg exhaustion: /10
        \item Joint pain (knees/shoulders/wrists): /10
        \item Muscle cramps: /10
        \item Migraine: Yes/No
    \end{itemize}
    \item[Medications:]
    \begin{itemize}
        \item Usual medication: Yes
        \item Usual supplements: Yes
    \end{itemize}
    \item[Activities:]
    \item[Fréquence cardiaque data:] FC max: , time above threshold:
    \item[Observations:] Took 250\,mL water + 10\,mL grenadine + salt/sugar mixture (oral rehydration solution).
\end{description}

\paragraph{2026-01-21.}
\begin{description}
    \item[Energy:] /10
    \item[Sleep:] hours, refreshing: Yes/No
    \item[Symptoms:]
    \begin{itemize}
        \item Fatigue: /10 (physically tired)
        \item Brain fog: /10 (mentally ``present'')
        \item Air hunger: /10
        \item Leg exhaustion: /10
        \item Joint pain (knees/shoulders/wrists): /10
        \item Muscle cramps: /10
        \item Migraine: Yes/No
    \end{itemize}
    \item[Medications:]
    \begin{itemize}
        \item Usual medication: Yes
        \item Usual supplements: Yes
        \item CoQ10: Yes
    \end{itemize}
    \item[Activities:] Sitting at computer (tiring)
    \item[Fréquence cardiaque data:] FC max: , time above threshold:
    \item[Observations:] Morning assessment: mentally ``present'' but still physically tired. Sitting at computer is tiring. Took same as yesterday (250\,mL water + 10\,mL grenadine + salt/sugar mixture) plus CoQ10.
\end{description}

\paragraph{2026-01-22 --- Day 2 of Electrolyte Protocol: SEVERE CRASH.}
\begin{description}
    \item[Energy:] 2--3/10 (severe crash 1200--1430)
    \item[Sleep:] Forced sleep during crash window (1200--1430)
    \item[Symptoms:]
    \begin{itemize}
        \item Fatigue: 8/10 (severe during crash; manageable outside)
        \item Brain fog: Moderate
        \item Air hunger: Not noted
        \item Leg exhaustion: Not specifically noted
        \item Joint pain (knees/shoulders): \textbf{9/10 --- rapid onset leading to severe crash}
        \begin{itemize}
            \item \textbf{Timeline}: Felt OK at wake (06:30) $\rightarrow$ joint pain onset by 08:30 $\rightarrow$ severe crash at noon (12:00)
            \item \textbf{Onset pattern}: 2-hour window from first symptoms to full crash
            \item Patient description: \emph{``joints were really painful, the kind where you just want it gone in any possible way''}
            \item Pain rapidly intensified throughout morning; peak severity during crash window
            \item Knees, shoulders primarily affected
        \end{itemize}
        \item Muscle cramps: Not specifically noted
        \item Migraine: No
    \end{itemize}
    \item[Medications:]
    \begin{itemize}
        \item \textbf{LDN}: 4\,mg (morning dose)
        \item Morning: Provigil 100\,mg
        \item Magnesium glycinate initiated this day (first dose)
        \item Electrolyte solution: 500\,mL (250\,mL $\times$ 2 doses) --- day 2 of protocol
    \end{itemize}
    \item[Activities:] Morning childcare; both children home Wednesday afternoon
    \begin{itemize}
        \item \textbf{No extraordinary exertion identified}
        \item Normal baseline activities (morning childcare routine, after-school care)
        \item No unusual cognitive or physical tasks reported
        \item Suggests very low PEM threshold or cumulative effect from preceding days
    \end{itemize}
    \item[Données fréquence cardiaque~:] Non suivi
    \item[Crash characteristics:]
    \begin{itemize}
        \item \textbf{Timing}: 1200--1430 (afternoon window)
        \item \textbf{Duration}: 2.5 hours forced sleep
        \item \textbf{Onset pattern}: Felt OK at wake (06:30) $\rightarrow$ joint pain by 08:30 $\rightarrow$ crash at 12:00
        \item \textbf{Warning window}: 3.5 hours from symptom onset to crash (2 hours early warning before crash)
        \item \textbf{Severity}: Unable to remain awake; overwhelming exhaustion
        \item \textbf{Joint pain as crash prodrome}: Rapid onset joint pain preceded crash by 3.5 hours, suggesting inflammatory/cytokine cascade as early warning sign
    \end{itemize}
    \item[Observations:]
    \begin{itemize}
        \item \textbf{PEM without identifiable trigger}: No obvious exertion to explain severity
        \item \textbf{Afternoon crash window}: Consistent with previous observations of afternoon vulnerability
        \item \textbf{Joint pain as crash indicator}: Inflammatory component prominent during PEM
        \item \textbf{Magnesium initiated}: First dose taken this day (evening likely); effect to be assessed next day
    \end{itemize}
\end{description}

\paragraph{2026-01-23 --- Day 3 of Electrolyte Protocol: MARKED IMPROVEMENT.}
\begin{description}
    \item[Energy:] 5--6/10 (substantially improved from day 2)
    \item[Sleep:] Not specifically documented
    \item[Symptoms:]
    \begin{itemize}
        \item Fatigue: 4/10 (afternoon: more tired, but ``currently OK'')
        \item Brain fog: \textbf{2/10 --- significant improvement}
        \begin{itemize}
            \item Able to focus without methylphenidate
            \item Only modafinil 100\,mg morning dose taken
            \item Describes ability to focus and engage cognitively
        \end{itemize}
        \item Air hunger: Not noted
        \item Leg exhaustion: Not noted
        \item Joint pain: \textbf{1/10 --- mostly resolved}
        \begin{itemize}
            \item Dramatic improvement from day 2 (9/10 $\rightarrow$ 1/10)
            \item Patient notes: \emph{``most joint pain is gone''}
            \item Knees, shoulders no longer significantly symptomatic
        \end{itemize}
        \item Muscle cramps: Not noted
        \item Migraine: No
    \end{itemize}
    \item[Medications:]
    \begin{itemize}
        \item \textbf{LDN}: 4\,mg (morning dose)
        \item Morning: Provigil 100\,mg only (no methylphenidate)
        \item Magnesium glycinate: Continued (second day)
        \item Acetyl-L-carnitine, riboflavin, standard supplement stack
        \item Electrolyte solution: 500\,mL (250\,mL $\times$ 2 doses) --- day 3 of protocol
    \end{itemize}
    \item[Activities:] Morning childcare, after-school care (normal baseline activities)
    \item[Données fréquence cardiaque~:] Non suivi
    \item[Afternoon pattern:]
    \begin{itemize}
        \item Patient notes: \emph{``afternoon: more tired, but currently OK''}
        \item Fatigue present but not disabling (contrast to day 2 severe crash)
        \item No forced sleep episode
        \item Sitting/rest preferred but functional
    \end{itemize}
    \item[Orthostatic status:]
    \begin{itemize}
        \item Patient notes: \emph{``orthostatic was always +- acceptable, at least I mostly don't feel dizzy when standing up''}
        \item No orthostatic problems throughout 3-day trial
        \item Some tiredness when standing (prefers to sit) but no dizziness
        \item Suggests primary benefit of electrolytes is not blood pressure/orthostatic but rather cellular/metabolic
    \end{itemize}
    \item[PEM assessment:]
    \begin{itemize}
        \item Patient explicitly notes: \emph{``PEM: not tested yet, I don't dare''}
        \item Appropriately cautious approach given day 2 crash
        \item Wisely establishing baseline stability before testing exertion limits
    \end{itemize}
    \item[Observations --- CRITICAL FINDINGS:]
    \begin{itemize}
        \item \textbf{Rapid electrolyte response (3 days)}: Cognitive improvement noticeable
        \item \textbf{Magnesium rapid effect (24--48 hrs)}: Joint pain resolved dramatically
        \item \textbf{Reduced stimulant requirement}: Maintained focus without methylphenidate
        \item \textbf{Orthostatic tolerance preserved}: Suggests electrolyte benefit is metabolic/cellular rather than purely cardiovascular
        \item \textbf{Afternoon vulnerability persists but manageable}: Crash pattern timing consistent but severity reduced
        \item \textbf{Appropriate pacing awareness}: Patient correctly avoiding PEM testing during early intervention phase
    \end{itemize}
\end{description}

% Continue journal entries below

\paragraph{2026-01-24 --- Day 4 of Electrolyte Protocol: Continued Improvement Despite Sleep Deficit.}
\begin{description}
    \item[Energy:] 6/10 (feeling rather good, clear head)
    \item[Sleep:] 4--5 hours (bedtime 02:30--03:00)
    \item[Symptoms:]
    \begin{itemize}
        \item Fatigue: 5/10 (tired, anticipating need for nap)
        \item Brain fog: \textbf{2/10 --- clear head this morning}
        \item Muscle stiffness: Ongoing (cramp-like, similar to past days)
        \item Joint pain (knees/shoulders/wrists): \textbf{Improved from Thursday (2026-01-22)}
        \item Overall: Tired but cognitively clear
    \end{itemize}
    \item[Medications:]
    \begin{itemize}
        \item \textbf{LDN}: 4\,mg (morning dose)
        \item \textbf{Supplements}: All protocol supplements taken
        \item \textbf{Ritalin}: None yet
        \item \textbf{Provigil}: None yet
    \end{itemize}
    \item[Notable observations:]
    \begin{itemize}
        \item Cognitive clarity maintained despite minimal sleep
        \item Joint pain significantly reduced from severe Thursday crash
        \item Muscle stiffness ongoing but distinct from joint pain
        \item Pattern suggests electrolyte protocol supporting cognitive function even under sleep stress
    \end{itemize}
\end{description}

%------------------------------------------------------------------------------

\paragraph{February 2026}
\label{subsec:journal-2026-02}

\paragraph{2026-02-03 to 2026-02-05 --- RilatineMR 30mg Trial.}
\begin{description}
    \item[Medication:] RilatineMR (methylphenidate modified-release) 30\,mg
    \item[Trial dates:] 2026-02-04 and 2026-02-05 (consecutive days)
    \item[Subjective response:] \textbf{Felt good, not really tired}
    \item[Key observation:] Notable positive response with improved wakefulness and reduced subjective fatigue
    \item[Critical question raised by patient:]
    \begin{itemize}
        \item Does methylphenidate represent \textbf{actual increased energy production}?
        \item Or is it \textbf{masking fatigue while consuming more energy than being produced}?
        \item This distinction is critical for safety and pacing strategy
    \end{itemize}
    \item[Clinical interpretation:]
    \begin{itemize}
        \item Methylphenidate is a \textbf{stimulant that masks true energy levels} (see Section~\ref{subsec:medications-under-consideration}, False Energy Risk warning)
        \item It allows ``borrowing'' energy from depleted reserves without increasing actual ATP production
        \item The positive subjective feeling does NOT indicate increased cellular energy production
        \item \textbf{Risk}: Operating beyond true metabolic capacity can trigger PEM/crash
        \item \textbf{Critical safeguard}: Fréquence cardiaque monitoring essential---trust objective measurements over subjective feelings
    \end{itemize}
    \item[Recommended monitoring:]
    \begin{itemize}
        \item Track heart rate continuously during methylphenidate use
        \item Compare activity levels on methylphenidate days vs. baseline
        \item Monitor for delayed PEM 24--48 hours after use
        \item Document any crashes following periods of methylphenidate-enhanced activity
        \item Assess whether ``feeling good'' correlates with actual increased functional capacity or just masked fatigue
    \end{itemize}
    \item[Next steps:]
    \begin{itemize}
        \item Continue trial with strict heart rate monitoring
        \item Document objective activity metrics (steps, duration, exertion level)
        \item Track PEM episodes in relation to methylphenidate use
        \item Evaluate whether this medication enables sustainable activity increase or leads to boom-bust cycles
        \item Consider trial period of 2--4 weeks to assess pattern
    \end{itemize}
\end{description}

%%%%%%%%%%%%%%%%%%%%%%%%%%%%%%%%%%%%%%%%%%%%%%%%%%%%%%%%%%%%%%%%%%%%%%%%%%%%%%%
