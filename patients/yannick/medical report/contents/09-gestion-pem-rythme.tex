\section{Gestion du PEM et rythme}

\subsection{Seuils d'activité actuels (déterminés empiriquement)}

Basé sur données du 25 janvier - 13 février 2026:

\begin{longtable}{p{4cm}p{3cm}p{6.5cm}}
\toprule
\textbf{Activité} & \textbf{Durée sûre} & \textbf{Preuves} \\
\midrule
Travail debout (repassage, cuisine) & <30 min sans pause & 12 fév: 30 min a déclenché crash \\
\midrule
Marche (courses) & <60 min & 11 fév: 1h20 a déclenché crash d'après-midi \\
\midrule
Travail cognitif assis & Variable & Surveiller avec acouphènes comme indicateur fatigue \\
\midrule
Conduite & Restreindre jusqu'à évaluation & 11 fév: événement autonome pendant conduite \\
\bottomrule
\end{longtable}

\subsection{Protocole de rythme}

\begin{enumerate}
\item \textbf{Surveillance fréquence cardiaque}: Rester sous 97 bpm (seuil anaérobie: (220-44) × 0,55)
\item \textbf{Acouphènes comme signal arrêt}: Quand acouphènes apparaissent, réduire immédiatement niveau d'activité
\item \textbf{Ratio repos-activité 3:1}: Pour chaque période d'effort, repos pour 3× la durée
\item \textbf{Évaluation pré-activité}: Évaluer fragilité matinale avant planifier activités debout
\item \textbf{Fractionner tâches}: Diviser activités en segments de 15 minutes avec 15 minutes repos assis entre
\item \textbf{Alternatives assises}: Repasser assis; utiliser tabouret pour travail cuisine; livraison courses en ligne
\item \textbf{Protection post-stimulant}: Jours après utilisation Ritalin, planifier repos strict (vulnérabilité rebond)
\end{enumerate}

\subsection{Signaux d'avertissement PEM}

Cesser toute activité immédiatement si:
\begin{itemize}
\item Fréquence cardiaque dépasse 97 bpm
\item Apparition acouphènes
\item Faiblesse ou ``jambes en gelée''
\item Pouls élevé palpable
\item Sensation pseudo-hypoglycémique (tremblements, transpiration, faiblesse)
\item Traitement cognitif notablement ralenti
\end{itemize}
