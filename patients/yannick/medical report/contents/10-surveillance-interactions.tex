\subsection{Suivi quotidien (auto-rapport patient)}

\begin{longtable}{p{3.5cm}p{4.5cm}p{5.5cm}}
\toprule
\textbf{Paramètre} & \textbf{Comment mesurer} & \textbf{Cible} \\
\midrule
Niveau d'énergie & Échelle 0-10, matin et soir & Stabilité tendance, éviter <3/10 \\
\midrule
Fonction cognitive & Échelle 0-10 & Stabilité tendance \\
\midrule
Acouphènes & Présent/absent + intensité 0-10 & Utiliser comme biomarqueur fatigue \\
\midrule
Douleur & Échelle 0-10 + localisation & Identifier corrélations activité-douleur \\
\midrule
Fréquence cardiaque & Moniteur continu, enregistrer max & Rester sous 97 bpm \\
\midrule
Sommeil & Heures, qualité, perturbations & Améliorer continuité \\
\midrule
Temps debout & Minutes cumulées & Rester dans enveloppe \\
\midrule
Médicaments pris & Doses exactes et timing & Assurer cohérence \\
\bottomrule
\end{longtable}

\subsection{Évaluation hebdomadaire}

\begin{longtable}{p{6cm}p{7.5cm}}
\toprule
\textbf{Paramètre} & \textbf{Objectif} \\
\midrule
Épisodes PEM (compte, sévérité, déclencheurs) & Calibration seuil activité \\
\midrule
Fréquence migraines & Efficacité traitement \\
\midrule
Événements autonomes (faiblesse, tremblements, pouls élevé) & Identification schéma \\
\midrule
Tendance capacité fonctionnelle globale & Trajectoire maladie \\
\bottomrule
\end{longtable}

\subsection{Si nouveaux médicaments démarrés}

\begin{longtable}{p{3.5cm}p{5cm}p{4.5cm}}
\toprule
\textbf{Médicament} & \textbf{Surveillance clé} & \textbf{Fréquence} \\
\midrule
Ivabradine & FC repos, symptômes bradycardie & Quotidien 2 semaines, puis hebdo \\
\midrule
Propranolol & FC, TA, niveau fatigue, tolérance exercice & Quotidien 2 semaines \\
\midrule
Midodrine & TA en décubitus (avant s'allonger), picotements cuir chevelu & Chaque dose 1 semaine \\
\midrule
Fludrocortisone & TA, poids, niveaux potassium & TA quotidien; analyses à 2 et 6 semaines \\
\midrule
Pyridostigmine & Symptômes GI, crampes musculaires, FC & Quotidien 1 semaine \\
\bottomrule
\end{longtable}

\subsection{Critères de succès pour essais médicamenteux}

\begin{longtable}{p{4cm}p{9.5cm}}
\toprule
\textbf{Critère} & \textbf{Définition} \\
\midrule
\textbf{Succès} & $\geq$ 20\% réduction événements autonomes ET/OU $\geq$ 2 points amélioration énergie quotidienne moyenne \\
\midrule
\textbf{Succès partiel} & Amélioration symptômes sans gains énergie OU amélioration énergie avec nouveaux effets secondaires \\
\midrule
\textbf{Échec} & Pas d'amélioration après durée essai adéquate OU effets secondaires intolérables \\
\midrule
\textbf{Durée essai} & Minimum 4 semaines pour chaque médicament avant évaluation (6-8 semaines pour LDN) \\
\bottomrule
\end{longtable}

\section{QUESTIONS POUR DISCUSSION}

\subsection{Pour médecin généraliste / soins primaires}

\begin{enumerate}
\item Vu les événements récurrents de dysrégulation autonome (10-13 fév), une référence urgente en cardiologie ou médecine autonome est-elle justifiée?
\item Devrions-nous restreindre la conduite jusqu'à ce que les tests autonomes formels soient complétés?
\item Le schéma actuel d'utilisation intermittente de stimulant (Ritalin certains jours, pas d'autres) contribue-t-il aux événements autonomes de rebond? Une utilisation quotidienne cohérente à faible dose serait-elle plus sûre?
\item Pouvons-nous obtenir mesure glucose sanguin pendant le prochain épisode pseudo-hypoglycémique pour exclure vraie hypoglycémie?
\item Panel métabolique de base et niveaux cortisol devraient-ils être vérifiés vu l'instabilité autonome?
\end{enumerate}

\subsection{Pour cardiologie / spécialiste autonome}

\begin{enumerate}
\item Basé sur le schéma symptômes (pouls élevé en station debout, faiblesse, tremblements pendant transitions sommeil-éveil, préservation cognitive), test d'inclinaison formel est-il indiqué?
\item Vu hypersensibilité vasovagale documentée (pré-2018) et dysfonction autonome post-commotion, quelle est la caractérisation la plus appropriée du syndrome autonome de ce patient?
\item L'ivabradine est-elle appropriée comme agent contrôle fréquence cardiaque première ligne, vu ses effets neutres sur pression artérielle et la préoccupation du patient sur aggravation fatigue avec bêta-bloquants?
\item Surveillance Holter devrait-elle être effectuée spécifiquement pour capturer le schéma transition sommeil-éveil (phases de 25 minutes de faiblesse suivies de tremblements)?
\item La combinaison d'utilisation stimulant (méthylphénidate, qui augmente FC/TA) avec instabilité autonome crée-t-elle un schéma dangereux qui devrait être abordé pharmacologiquement?
\end{enumerate}

\subsection{Pour médecine du sommeil}

\begin{enumerate}
\item Polysomnographie avec surveillance autonome (FC, TA, HRV continus) est-elle indiquée pour évaluer dysrégulation autonome dépendante du stade de sommeil?
\item Vu la voie fluorure-pinéale hypothétique, les niveaux de mélatonine salivaire chronométrés aideraient-ils à guider la supplémentation en mélatonine?
\item Le patient a hypersomnie idiopathique prédatant diagnostic EM/SFC. Une réévaluation de ce diagnostic est-elle justifiée vu le tableau autonome plus large?
\item Les événements autonomes post-sieste (faiblesse, tremblements au réveil) sont-ils cohérents avec un trouble de transition de sommeil connu?
\end{enumerate}

\subsection{Pour neurologie}

\begin{enumerate}
\item Vu l'historique de commotion (juin 2018, amnésie post-traumatique 5h) et détérioration autonome subséquente, imagerie neurologique (IRM cérébrale avec focus sur tronc cérébral/centres autonomes) est-elle indiquée?
\item Le tremblement des mains (présent depuis 16 ans, s'aggravant) avec tremblements autonomes récents -- sont-ils les mêmes ou différents phénomènes?
\item Caractérisation formelle tremblements devrait-elle être effectuée pour distinguer tremblement essentiel, tremblement autonome et tremblement potentiel post-TCC?
\end{enumerate}

\section{CONSIDÉRATIONS D'INTERACTIONS MÉDICAMENTEUSES}

\subsection{Médicaments actuels et ajouts potentiels nouveaux}

{\tiny
\begin{longtable}{p{2cm}p{1.6cm}p{1.8cm}p{1.6cm}p{2cm}p{1.8cm}}
\toprule
\textbf{Méd.\ actuel} & \textbf{Iva\-bra\-dine} & \textbf{Pro\-pra\-no\-lol} & \textbf{Mido\-drine} & \textbf{Fludro\-corti\-sone} & \textbf{Pyrido\-stig\-mine} \\
\midrule
LDN 3-4mg & Pas d'in\-ter\-action & Pas d'in\-ter\-action & Pas d'in\-ter\-action & Pas d'in\-ter\-action & Pas d'in\-ter\-action \\
\midrule
Céti\-rizine & Pas d'in\-ter\-action & Pas d'in\-ter\-action & Pas d'in\-ter\-action & Pas d'in\-ter\-action & Pas d'in\-ter\-action \\
\midrule
Ritalin MR 30mg & Sur\-veiller FC & \textbf{ATTEN\-TION}: effets FC op\-posés & Sur\-veiller TA & Pas d'in\-ter\-action & Pas d'in\-ter\-action \\
\midrule
Moda\-finil & Sur\-veiller FC & Pré\-occu\-pation lé\-gère & Sur\-veiller TA & Pas d'in\-ter\-action & Pas d'in\-ter\-action \\
\midrule
Gly\-cinate mag\-nésium & Pas d'in\-ter\-action & Pas d'in\-ter\-action & Pas d'in\-ter\-action & \textbf{Sur\-veiller K+} & Pas d'in\-ter\-action \\
\bottomrule
\end{longtable}
}

\textbf{Interactions clés à surveiller:}
\begin{enumerate}
\item \textbf{Ritalin + bêta-bloquant}: Effets cardiovasculaires opposés. Méthylphénidate augmente FC/TA; propranolol diminue FC/TA. Peut partiellement annuler effets thérapeutiques de chacun, ou peut causer réponses autonomes imprévisibles. Utiliser doses efficaces les plus faibles des deux.

\item \textbf{Ritalin + ivabradine}: Les deux affectent fréquence cardiaque par différents mécanismes. Méthyl\-phéni\-date augmente FC (sym\-patho\-mi\-mé\-tique); ivabradine diminue FC (blocage canal If). Cette combinaison peut en fait fournir contrôle équilibré -- l'ivabradine peut prévenir tachy\-cardie induite par stimulant tout en préservant bénéfices cognitifs stimulant. Surveiller FC étroitement.

\item \textbf{Fludrocortisone + électrolytes}: Les deux affectent équilibre hydrique/électrolytes. Surveiller niveaux potassium étroitement lors combinaison minéralocorticoïde avec solutions électrolytes contenant potassium.
\end{enumerate}
