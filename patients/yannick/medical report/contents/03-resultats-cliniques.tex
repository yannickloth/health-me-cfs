\section{Résultats Cliniques et Tests Diagnostiques}

\subsection{Résultats de Laboratoire (2025)}

\subsubsection{Hématologie et Statut Ferrique}

\begin{table}[htbp]
\centering
\caption{Statut Ferrique et Hématologie (2025)}
\label{tab:statut-ferrique}
\begin{tabular}{lccl}
\toprule
\textbf{Paramètre} & \textbf{Résultat} & \textbf{Référence} & \textbf{Note Clinique} \\
\midrule
Hémoglobine & 15.6 g/dL & 13.5--17.6 & Normal \\
Ferritine & 40--55 $\mu$g/L & 20--300 & \textbf{Sous-optimal pour EM/SFC} \\
Fer & 107 $\mu$g/dL & 65--175 & Normal \\
Transferrine & 3.12 g/L & 1.74--3.64 & Normal \\
Saturation transferrine & 25\% & 15--50 & Normal \\
Vitamine B12 & 383--424 ng/L & 187--883 & Normal \\
Folate & 2.8--4.2 $\mu$g/L & 2.3--17.6 & Normal-bas \\
\bottomrule
\end{tabular}
\end{table}

\paragraph{Interprétation de la ferritine.}
Bien que la ferritine 40--55 $\mu$g/L soit dans la plage de référence standard, un somnologue consultant a spécifiquement noté: \emph{``Un taux supérieur à 70--75 $\mu$g/L est recommandé''} dans le contexte des mouvements périodiques des membres durant le sommeil. Cette cible est également recommandée pour les patients EM/SFC étant donné le rôle du fer dans:
\begin{itemize}
    \item Synthèse de la dopamine (cofacteur de la tyrosine hydroxylase)
    \item Chaîne de transport d'électrons mitochondriale (cytochromes)
    \item Gestion du syndrome des jambes sans repos
\end{itemize}

\subsubsection{Marqueurs Immunitaires et Inflammatoires}

\begin{table}[htbp]
\centering
\caption{Marqueurs Immunitaires (Octobre--Novembre 2025)}
\label{tab:marqueurs-immunitaires}
\begin{tabular}{lccl}
\toprule
\textbf{Paramètre} & \textbf{Résultat} & \textbf{Référence} & \textbf{Note Clinique} \\
\midrule
\multicolumn{4}{l}{\textit{Marqueurs rhumatoïdes}} \\
Facteur rhumatoïde & 119--176 IU/mL & $<$14--20 & \textbf{Fortement positif} \\
Anti-CCP & $<$0.8 U/mL & $<$7 & Négatif \\
ANA & Négatif & $<$1/80 & Normal \\
\midrule
\multicolumn{4}{l}{\textit{Inflammation}} \\
CRP & 1.6--3.6 mg/L & $<$5--8.5 & Normal \\
\midrule
\multicolumn{4}{l}{\textit{Complément}} \\
C3 & 1.39--1.49 g/L & 0.82--1.85 & Normal \\
C4 & 0.39--0.42 g/L & 0.10--0.53 & Normal supérieur \\
\midrule
\multicolumn{4}{l}{\textit{Immunoglobulines}} \\
IgG & 14.4 g/L & 5.40--18.22 & Normal \\
IgA & 2.80 g/L & 0.63--4.84 & Normal \\
IgM & 0.95 g/L & 0.22--2.40 & Normal \\
\bottomrule
\end{tabular}
\end{table}

\paragraph{Interprétation du facteur rhumatoïde.}
Le FR fortement élevé (119--176 IU/mL) avec Anti-CCP \textbf{négatif} exclut effectivement la polyarthrite rhumatoïde. Un FR élevé sans Anti-CCP se produit dans:
\begin{itemize}
    \item Infections chroniques (incluant états post-viraux)
    \item Autres conditions auto-immunes
    \item EM/SFC (activation immunitaire non spécifique)
    \item Individus sains (faux positif, surtout adultes âgés)
\end{itemize}
L'ANA négatif plaide davantage contre une maladie auto-immune systémique.

\subsubsection{Sérologie Virale}

\begin{table}[htbp]
\centering
\caption{Sérologie Virale (Octobre 2025)}
\label{tab:serologie-virale}
\begin{tabular}{lccl}
\toprule
\textbf{Virus} & \textbf{IgG} & \textbf{IgM} & \textbf{Interprétation} \\
\midrule
EBV (VCA) & $>$750 U/mL & Négatif & Infection passée, titre très élevé \\
Parvovirus B19 & 61.0 U/mL & Négatif & Infection passée \\
CMV & 0.9 U/mL & Négatif & Pas d'exposition \\
Hépatite B & Négatif & --- & Pas d'infection/immunité \\
Hépatite C & Négatif & --- & Pas d'infection \\
Toxoplasmose & $<$0.5 UI/mL & Négatif & Pas d'exposition \\
Borrelia (Lyme) & 6.7 U/mL & Négatif & Pas d'infection \\
Bartonella & 1/64 & Négatif & Au seuil de détection \\
\bottomrule
\end{tabular}
\end{table}

\paragraph{Interprétation EBV.}
Le VCA IgG EBV très élevé ($>$750 U/mL) indique une infection EBV passée avec réponse anticorps robuste. L'EBV est l'un des déclencheurs les plus courants de l'EM/SFC. Le titre élevé suggère soit:
\begin{itemize}
    \item Forte réponse immunitaire initiale à l'infection passée
    \item Possible réactivation virale continue à bas niveau
    \item Stimulation immunitaire persistante par antigènes EBV
\end{itemize}
Cette découverte supporte le modèle étiologique post-infectieux de l'EM/SFC.

\subsection{Polysomnographie avec MSLT (Décembre 2018)}

Polysomnographie complète avec Test de Latences Multiples du Sommeil (MSLT) effectuée au CHA Libramont, Laboratoire du Sommeil, 07--08 décembre 2018.

\subsubsection{Caractéristiques du Patient au Moment de l'Étude}

\begin{itemize}
    \item Âge: 37 ans
    \item Poids: 72 kg; Taille: 175 cm; IMC: 23.5
    \item Plainte principale: \emph{``Fatigue présente depuis l'adolescence''}
    \item Pas de caféine, pas de tabac, pas d'alcool
    \item Activité physique: Natation 4$\times$/semaine
    \item Chronotype: Type vespéral
    \item Besoin de sommeil: 8 heures + sieste de 1.5 heure
    \item Récemment arrêté Concerta (juillet 2018), pris 4 kg en 3 mois
\end{itemize}

\subsubsection{Scores des Questionnaires}

\begin{table}[htbp]
\centering
\caption{Résultats des Questionnaires du Sommeil (2018 et 2021)}
\label{tab:questionnaires-sommeil}
\begin{tabular}{lccc}
\toprule
\textbf{Échelle} & \textbf{2018} & \textbf{2021} & \textbf{Interprétation} \\
\midrule
Échelle de Somnolence d'Epworth & 16/24 & 14/24 & Pathologique ($>$10) \\
Score de Sévérité de la Fatigue & 4.5 & --- & Fatigue anormale \\
Dépression de Pichot & --- & 10/13 & Trouble de l'humeur suggéré \\
Anxiété de Goldberg & --- & 6/7 & Trouble anxieux suggéré \\
Index de Sévérité de l'Insomnie & --- & 18/28 & Modéré (16 pts diurne) \\
\bottomrule
\end{tabular}
\end{table}

\subsubsection{Résultats de la Polysomnographie Nocturne}

\begin{table}[htbp]
\centering
\caption{Paramètres de la Polysomnographie (Décembre 2018)}
\label{tab:resultats-psg}
\begin{tabular}{lccc}
\toprule
\textbf{Paramètre} & \textbf{Résultat} & \textbf{Normal} & \textbf{Évaluation} \\
\midrule
\multicolumn{4}{l}{\textit{Durée du Sommeil}} \\
Temps au lit & 518 min & --- & --- \\
Temps de sommeil total (TST) & 429 min & --- & Normal \\
Période de sommeil & 515 min & --- & --- \\
\midrule
\multicolumn{4}{l}{\textit{Indices de Qualité du Sommeil}} \\
Efficacité du sommeil (TST/TRS) & 82.8\% & $>$86\% & \textbf{Réduite} \\
Continuité du sommeil (TST/TPS) & 83.3\% & $>$95\% & \textbf{Insuffisante} \\
Index de qualité (SWS+REM/TST) & 54.9\% & $>$35\% & Bon \\
\midrule
\multicolumn{4}{l}{\textit{Architecture du Sommeil}} \\
N1 (sommeil léger) & 2 min (0.5\%) & 2--5\% & Bas \\
N2 (intermédiaire) & 191 min (44.6\%) & 45--55\% & Normal \\
N3 (profond/SWS) & 141 min (32.8\%) & 15--33\% & Normal-élevé \\
Sommeil REM & 95 min (22.1\%) & 20--25\% & Normal \\
\midrule
\multicolumn{4}{l}{\textit{Fragmentation du Sommeil}} \\
Changements de stade & 131 & --- & \textbf{Élevé} \\
WASO (éveil après début sommeil) & 86 min & $<$30 min & \textbf{Excessif} \\
Nombre de réveils & 25/nuit & --- & Élevé \\
Index de micro-éveils & 6.1/h & $<$10/h & Normal \\
\midrule
\multicolumn{4}{l}{\textit{Latences du Sommeil}} \\
Latence d'endormissement & 13 min & $<$30 min & Normal \\
Latence REM & 72 min & 70--120 min & Normal \\
\bottomrule
\end{tabular}
\end{table}

\subsubsection{Mouvements Périodiques des Membres}

\begin{table}[htbp]
\centering
\caption{Analyse des Mouvements Périodiques des Membres}
\label{tab:analyse-mpm}
\begin{tabular}{lcc}
\toprule
\textbf{Paramètre} & \textbf{Résultat} & \textbf{Normal} \\
\midrule
Index MPM (durant le sommeil) & 13.3/h & $<$5/h \\
Index MPM (durant N1) & 30.0/h & --- \\
Index MPM (durant N2) & 10.7/h & --- \\
Index MPM (durant N3) & 11.9/h & --- \\
Durée MPM (moyenne) & 10.2 sec & --- \\
\bottomrule
\end{tabular}
\end{table}

\paragraph{Interprétation des MPM.}
L'index MPM de 13.3/h est élevé (normal $<$5/h) et contribue à la fragmentation du sommeil. Le somnologue consultant a spécifiquement noté qu'une ferritine $>$70--75 $\mu$g/L est recommandée pour les patients avec mouvements périodiques des membres.

\subsubsection{Événements Respiratoires}

\begin{table}[htbp]
\centering
\caption{Analyse Respiratoire}
\label{tab:analyse-respiratoire}
\begin{tabular}{lcc}
\toprule
\textbf{Paramètre} & \textbf{Résultat} & \textbf{Interprétation} \\
\midrule
Index Apnée-Hypopnée (IAH) & 3.8/h & Normal ($<$5/h) \\
IAH en REM & 9.5/h & Léger \\
IAH en supination & 7.7/h & Léger positionnel \\
Apnées centrales & 4 événements & Minimal \\
Apnées obstructives & 3 événements & Minimal \\
Hypopnées obstructives & 24 événements & Type prédominant \\
SpO$_2$ moyenne & 95.9\% & Normal \\
Temps SpO$_2$ $<$90\% & 0 min & Normal \\
\bottomrule
\end{tabular}
\end{table}

\paragraph{Interprétation respiratoire.}
L'IAH global est dans les limites normales. L'étude a conclu: \emph{``L'analyse de la respiration ne met pas en évidence d'apnées, d'hypopnées ou de désaturation.''} Les événements respiratoires ne sont pas la cause principale de la perturbation du sommeil.

\subsubsection{Test de Latences Multiples du Sommeil (MSLT)}

\begin{table}[htbp]
\centering
\caption{Résultats MSLT (Décembre 2018)}
\label{tab:resultats-mslt}
\begin{tabular}{lcccl}
\toprule
\textbf{Heure Sieste} & \textbf{Latence} & \textbf{Stades Atteints} & \textbf{SOREMP} & \textbf{Note} \\
\midrule
09:00 & 0.5 min & N1, N2, N3 & Non & Extrêmement rapide \\
11:00 & 3.0 min & N1, N2, N3 & Non & Rapide \\
13:00 & 12.0 min & N1, N2 & Non & Normal \\
15:00 & Pas de sommeil & --- & Non & Ne s'est pas endormi \\
\midrule
\textbf{Latence moyenne} & \textbf{9.0 min} & --- & \textbf{0/4} & \textbf{Pathologique} \\
\bottomrule
\end{tabular}
\end{table}

\paragraph{Interprétation du MSLT.}
\begin{itemize}
    \item Latence moyenne de sommeil de 9 minutes est pathologique ($<$10 min indique somnolence diurne excessive)
    \item Absence de périodes de sommeil REM au début (SOREMPs) exclut la narcolepsie
    \item Le pattern montre une \textbf{somnolence prédominante matinale}---s'est endormi en 30 secondes à 9h, 3 minutes à 11h
    \item Amélioration l'après-midi (12 min à 13h, pas de sommeil à 15h)
\end{itemize}

Conclusion du rapport: \emph{``Présence de somnolence pathologique essentiellement en matinée (endormissement rapide et présence de sommeil lent profond).''}

\subsubsection{Diagnostic Officiel (Étude du Sommeil 2018)}

\begin{tcolorbox}[colback=gray!5!white,colframe=gray!75!black,title=Diagnostic de la Polysomnographie]
\textbf{Dyssomnie} caractérisée par:
\begin{itemize}
    \item Fragmentation du sommeil
    \item Nombre élevé de changements de stade (131)
    \item Mouvements périodiques des membres durant le sommeil (index 13.3/h)
    \item Pas d'événements respiratoires significatifs
\end{itemize}

\textbf{Somnolence diurne excessive} (Epworth 16/24) avec:
\begin{itemize}
    \item Risque de s'endormir en conduisant
    \item MSLT pathologique (latence moyenne 9 min)
    \item Pattern prédominant matinal
    \item Pas de caractéristiques de narcolepsie (pas de SOREMPs)
\end{itemize}

\textbf{Plainte de fatigue anormale} (Score de Sévérité de la Fatigue 4.5)
\end{tcolorbox}

\subsection{Consultation en Somnologie (Novembre 2021)}

Consultation en pathologie du sommeil à la Clinique Saint-Luc Bouge, novembre 2021.

\subsubsection{Observations Cliniques Clés}

\begin{itemize}
    \item \textbf{Apparition de la fatigue}: Âge 15--16 ans (adolescence)
    \item \textbf{Pattern de fatigue}: Fluctuant, avec phases de 6--10 jours de fatigue physique et mentale extrême, céphalées, brouillard mental, irritabilité
    \item \textbf{Burnout}: Fin 2017
    \item \textbf{Antécédents familiaux}: Mère et deux sœurs diagnostiquées avec TDAH
    \item \textbf{Cognitif}: QI $>$135, a sauté la 6e primaire, excellentes facilités académiques
    \item \textbf{Poids}: 74 kg pour 173 cm (IMC 24.7)---gain de 5--6 kg sur 3 ans
\end{itemize}

\subsubsection{Conclusion Clinique}

\begin{quote}
\emph{``Votre patient présente un tableau complexe de fatigue chronique d'étiologie indéterminée. Le bilan du sommeil réalisé au CHA n'a pas été décisif quant à un trouble du sommeil spécifique. L'hypersomnie idiopathique suspectée est un trouble se caractérisant par un allongement anormal du temps de sommeil avec persistance de fatigue/somnolence durant les phases d'éveil.''}

---Somnologue consultant
\end{quote}

\subsubsection{Recommandations Cliniques}

\begin{enumerate}
    \item Cible de ferritine: $>$70--75 $\mu$g/L pour gestion des MPM
    \item Considérer réévaluation complète d'hypersomnie (actigraphie + PSG + MSLT + repos au lit)
    \item Évaluation TDAH/HP suggérée (Dr.\ Linsmeaux, clinique TDAH)
    \item Traitement Provigil continué (100 mg $\times$3/jour)
\end{enumerate}
