% FILE: Systematic sleep-activity-rebound tracking protocol
\section{Sleep-Activity-Rebound Tracking Protocol}
\label{app:activity-tracking}

This protocol provides systematic guidance for tracking sleep patterns, activity levels, and post-exertional crashes to identify correlations and optimize pacing strategies.

\subsection{Protocol Overview}
\label{subsec:protocol-overview}

\paragraph{Objective}.

To systematically document:

\begin{itemize}
    \item Daily sleep architecture and quality
    \item Activity levels and cognitive exertion
    \item Subsequent crashes (delayed post-exertional malaise)
    \item Correlation patterns between activity dose and crash severity
    \item Environmental and physiological factors (light exposure, medication timing, menstrual cycle, illness)
\end{itemize}

\paragraph{Duration}.

Minimum 8 weeks continuous tracking (supports statistical analysis for patterns). Ongoing tracking recommended for long-term management.

\paragraph{Tools Required}.

\begin{itemize}
    \item Sleep tracking device (actiwatch, smartwatch with sleep tracking, or manual log)
    \item Activity quantification system (MET-hours, activity scoring, or simple categorical system)
    \item Symptom severity scale (10-point scale for fatigue, pain, cognitive function)
    \item Crash/rebound tracking form (see template below)
\end{itemize}

\subsection{Sleep Tracking Component}
\label{subsec:sleep-tracking}

Track daily at bedtime and upon waking.

\subsubsection{Daily Sleep Log Template}

\textbf{Date}: \_\_\_\_\_\_\_\_\_\_\_

\textbf{Sleep Metrics (previous night)}:

\begin{itemize}
    \item Bedtime (lights off): \_\_\_\_ AM/PM
    \item Sleep onset latency (time to fall asleep): \_\_\_\_ minutes
    \item Total awakenings during night: \_\_\_\_ (count)
    \item Total time awake after sleep onset: \_\_\_\_ minutes
    \item Wake time (final wake-up): \_\_\_\_ AM/PM
    \item Total sleep duration: \_\_\_\_ hours \_\_\_\_ minutes
    \item Sleep efficiency (if trackable): \_\_\_\_\%
\end{itemize}

\textbf{Sleep Quality Assessment}:

\begin{itemize}
    \item Overall sleep quality (1--10 scale; 1=terrible, 10=excellent): \_\_\_\_
    \item Restfulness upon waking (1--10; 1=no rest, 10=fully rested): \_\_\_\_
    \item Dream recall (yes/no): \_\_\_\_
    \item Nightmare/disturbing dreams (yes/no): \_\_\_\_
    \item Night sweats (yes/no): \_\_\_\_
    \item Morning autonomic symptoms:
    \begin{itemize}
        \item Orthostatic dizziness upon standing (none/mild/moderate/severe): \_\_\_\_
        \item Morning HR (if available): \_\_\_\_ bpm
        \item Morning blood pressure (if available): \_\_\_\_/\_\_\_\_ mmHg
    \end{itemize}
\end{itemize}

\textbf{Overnight Symptoms}:

\begin{itemize}
    \item Muscle cramps (yes/no; location): \_\_\_\_
    \item Breathing difficulties/air hunger (yes/no): \_\_\_\_
    \item Sleep-wake transition events (yes/no; describe): \_\_\_\_
    \item Tremor or involuntary movements (yes/no): \_\_\_\_
\end{itemize}

\textbf{Medications/Supplements Taken Before Sleep}:

\begin{itemize}
    \item LDN 4mg (yes/no): \_\_\_\_
    \item Melatonin (dose): \_\_\_\_ mg or (none)
    \item Other sleep aids (list): \_\_\_\_
    \item Evening Ritalin dose (if applicable): \_\_\_\_ mg
    \item Other medications (list): \_\_\_\_
\end{itemize}

\textbf{Environmental Factors}:

\begin{itemize}
    \item Sleep environment temperature: \_\_\_\_ °C
    \item Light exposure (hours before bed): \_\_\_\_ (dim/normal/bright)
    \item Screen/blue light exposure (hours before bed): yes/no
    \item Caffeine intake (time, amount): \_\_\_\_
    \item Physical activity yesterday (see activity section below)
    \item Cognitive exertion yesterday (see activity section below)
    \item Stress level yesterday (1--10): \_\_\_\_
\end{itemize}

\subsection{Activity and Exertion Tracking Component}
\label{subsec:activity-tracking}

Track throughout the day. Simple categorical system is preferred over complex quantification to reduce cognitive burden.

\subsubsection{Activity Quantification System}

\textbf{Option A: Categorical (Recommended for ME/CFS)}

\begin{itemize}
    \item \textbf{Sedentary (S)}: Lying in bed, sitting, minimal movement. Examples: reading, watching videos, light computer work.

    \item \textbf{Light (L)}: Slow walking, light household tasks, self-care, light cognitive work. Examples: shower, prepare light meal, email, document notes.

    \item \textbf{Moderate (M)}: Sustained walking, household chores, moderate cognitive engagement. Examples: grocery shopping, cleaning, prolonged conversation, focused writing project.

    \item \textbf{Exertional (E)}: Sustained physical activity, high cognitive demand, or combined exertion. Examples: exercise, sustained standing, complex problem-solving, social events, driving long distances.
\end{itemize}

\textbf{Option B: METs-Based (If wearable device available)}

\begin{itemize}
    \item Record actual METs (metabolic equivalent hours) from activity tracker
    \item Normal daily activity: 1.5--3 METs
    \item ME/CFS baseline often: $<$2 METs for safe activity envelope
\end{itemize}

\textbf{Option C: Subjective Energy Expenditure}

\begin{itemize}
    \item Estimate percentage of daily energy envelope used: \_\_\_\%
    \item Rate exertion level of primary activity (1--10): \_\_\_\_
\end{itemize}

\subsubsection{Daily Activity Log Template}

\textbf{Date}: \_\_\_\_\_\_\_\_\_\_\_

\textbf{Morning Activity (0700--1200 hours)}:

\begin{itemize}
    \item Primary activity and category (S/L/M/E): \_\_\_\_
    \item Duration: \_\_\_\_ minutes
    \item Secondary activities and durations: \_\_\_\_
    \item Cognitive exertion level (1--10): \_\_\_\_
    \item Physical exertion level (1--10): \_\_\_\_
    \item Energy level throughout morning (1--10 scale): \_\_\_\_
    \item Symptoms during activity (fatigue, pain, brain fog, etc.): \_\_\_\_
\end{itemize}

\textbf{Midday/Afternoon Activity (1200--1800 hours)}:

\begin{itemize}
    \item Primary activity and category (S/L/M/E): \_\_\_\_
    \item Duration: \_\_\_\_ minutes
    \item Secondary activities and durations: \_\_\_\_
    \item Cognitive exertion level (1--10): \_\_\_\_
    \item Physical exertion level (1--10): \_\_\_\_
    \item Energy level throughout afternoon (1--10 scale): \_\_\_\_
    \item Symptoms during activity: \_\_\_\_
\end{itemize}

\textbf{Evening Activity (1800--2300 hours)}:

\begin{itemize}
    \item Primary activity and category (S/L/M/E): \_\_\_\_
    \item Duration: \_\_\_\_ minutes
    \item Secondary activities and durations: \_\_\_\_
    \item Cognitive exertion level (1--10): \_\_\_\_
    \item Physical exertion level (1--10): \_\_\_\_
    \item Energy level throughout evening (1--10 scale): \_\_\_\_
    \item Symptoms during activity: \_\_\_\_
\end{itemize}

\textbf{Daily Summary Metrics}:

\begin{itemize}
    \item Total time in Sedentary activity: \_\_\_\_ hours
    \item Total time in Light activity: \_\_\_\_ hours
    \item Total time in Moderate activity: \_\_\_\_ hours
    \item Total time in Exertional activity: \_\_\_\_ hours
    \item Total METs (if available): \_\_\_\_
    \item Overall activity level impression: (low/normal/excessively high)
    \item Overall energy expenditure relative to safe envelope: \_\_\_\_\%
\end{itemize}

\textbf{Medication Timing and Doses}:

\begin{itemize}
    \item Ritalin MR morning (yes/no; dose): \_\_\_\_
    \item Ritalin immediate-release, if taken (dose, time): \_\_\_\_
    \item Other medications (list with times): \_\_\_\_
\end{itemize}

\textbf{Symptom Snapshot at End of Day}:

\begin{itemize}
    \item Fatigue level (1--10): \_\_\_\_
    \item Pain level (1--10): \_\_\_\_
    \item Cognitive function (1--10): \_\_\_\_
    \item Mood/emotional status (1--10): \_\_\_\_
    \item Overall symptom burden (1--10): \_\_\_\_
\end{itemize}

\subsection{Crash/Post-Exertional Malaise (PEM) Tracking Component}
\label{subsec:crash-tracking}

Track within 24 hours of crash onset and daily until resolution.

\subsubsection{Crash Event Log Template}

\textbf{Crash Date}: \_\_\_\_\_\_\_\_\_\_\_

\textbf{Preceding Activity (24--48 hours before crash)}.

\begin{itemize}
    \item Primary exertional activity: \_\_\_\_
    \item Activity category (S/L/M/E): \_\_\_\_
    \item Activity duration: \_\_\_\_ minutes/hours
    \item Exertion level (1--10): \_\_\_\_
    \item Cognitive exertion (1--10): \_\_\_\_
    \item Physical exertion (1--10): \_\_\_\_
    \item Whether activity was in safe energy envelope: (yes/no; borderline)
    \item Other precipitating factors (illness, infection, emotional stress, poor sleep): \_\_\_\_
\end{itemize}

\textbf{Crash Onset}:

\begin{itemize}
    \item Time of crash onset: \_\_\_\_
    \item Hours after exertion: \_\_\_\_ (24--72 hours typical for ME/CFS PEM)
    \item Did crash feel typical or different: \_\_\_\_
\end{itemize}

\textbf{Crash Severity and Symptoms}:

\begin{itemize}
    \item Overall PEM severity (1--10; 1=mild, 10=most severe ever): \_\_\_\_
    \item Fatigue severity (1--10): \_\_\_\_
    \item Pain severity (1--10): \_\_\_\_
    \item Cognitive dysfunction severity (1--10): \_\_\_\_
    \item Autonomic symptoms present (orthostasis, palpitations, tremor, etc.): \_\_\_\_
    \item Functional impact: (minimal/mild/moderate/severe/bedbound)
    \item Able to self-care: (yes/no; partial)
    \item Able to do light cognitive tasks: (yes/no)
\end{itemize}

\textbf{Crash Duration and Recovery Pattern}:

\begin{itemize}
    \item Crash duration: \_\_\_\_ hours/days
    \item Recovery pattern: (gradual/fluctuating/stepwise improvement)
    \item Peak severity timing: \_\_\_\_ hours after onset
    \item Return to baseline: \_\_\_\_ days after onset
    \item Residual symptoms after apparent recovery: (yes/no; describe)
\end{itemize}

\textbf{Crash Recovery Interventions}:

\begin{itemize}
    \item Rest enforced (bed rest, minimal activity)
    \item Medication adjustments (stimulants held, pain meds taken, etc.): \_\_\_\_
    \item Other interventions (hydration, elevation, heating pad, etc.): \_\_\_\_
    \item Effectiveness: (not effective/partially effective/highly effective)
\end{itemize}

\subsection{Correlation Analysis Guidance}
\label{subsec:correlation-analysis}

After 8 weeks of tracking, analyze patterns to identify triggers and optimal pacing.

\subsubsection{Activity-Crash Correlation}

\begin{itemize}
    \item \textbf{Activity dose threshold}: What level of activity (S/L/M/E or METs) reliably triggers crashes?
    \begin{itemize}
        \item Example: ``E-level activities always trigger PEM. M-level activities cause PEM $\sim$50\% of time. L-level activities tolerated well.''
    \end{itemize}

    \item \textbf{Temporal lag}: How many hours/days after exertion does crash typically occur?
    \begin{itemize}
        \item Example: ``PEM onset 12--24 hours after exertion; resolution 3--5 days.''
    \end{itemize}

    \item \textbf{Cumulative effect}: Does exertion on successive days increase crash risk?
    \begin{itemize}
        \item Example: ``Single L-activity day tolerated. Two L-activity days in a row triggers PEM.''
    \end{itemize}

    \item \textbf{Activity type sensitivity}: Are certain activity types more triggering than others?
    \begin{itemize}
        \item Example: ``Cognitive exertion triggers cognitive PEM (brain fog). Physical exertion triggers muscle pain/fatigue.''
    \end{itemize}
\end{itemize}

\subsubsection{Sleep-Activity-Crash Cycle}

\begin{itemize}
    \item \textbf{Sleep quality and activity tolerance}: Does poor sleep predict next-day activity intolerance?
    \begin{itemize}
        \item Example: ``Sleep $<$6 hours → $>$50\% chance of next-day PEM with even L-activity.''
    \end{itemize}

    \item \textbf{Activity-sleep disruption}: Does activity cause subsequent sleep disturbance?
    \begin{itemize}
        \item Example: ``M-level activity on day $n$ → fragmented sleep night $n$ → PEM day $n+1$.''
    \end{itemize}

    \item \textbf{Sleep as recovery factor}: Can increased sleep duration support higher activity tolerance?
    \begin{itemize}
        \item Example: ``With 7--8 hours sleep, can tolerate M-activity without PEM. With $<$6 hours, cannot.''
    \end{itemize}
\end{itemize}

\subsubsection{Medication Effects on Activity Tolerance and Sleep}

\begin{itemize}
    \item \textbf{Stimulant (Ritalin) effects}:
    \begin{itemize}
        \item Does Ritalin enable higher activity tolerance? (Improved function but increased risk of overactivity?)
        \item Does Ritalin dose affect sleep quality?
        \item Is there next-day rebound when Ritalin not taken?
        \item Example: ``On Ritalin, can do M-activity comfortably. Off Ritalin, crash risk from same activity is 80\%.''
    \end{itemize}

    \item \textbf{LDN effects}: Does 4mg LDN evening dose improve sleep? Does it affect next-day activity tolerance?
\end{itemize}

\subsubsection{Environmental and Physiological Factors}

\begin{itemize}
    \item \textbf{Light exposure}: Does increased light exposure improve sleep and activity tolerance?
    \item \textbf{Temperature}: Are symptoms (especially muscle cramps) temperature-sensitive?
    \item \textbf{Infection/illness}: Does acute illness change activity-crash thresholds?
    \item \textbf{Menstrual cycle} (if applicable): Does cycle phase affect symptoms or activity tolerance?
\end{itemize}

\subsection{Safe Activity Envelope Calculation}
\label{subsec:safe-envelope}

Based on tracking data, calculate individualized safe activity envelope.

\paragraph{Method}.

\begin{enumerate}
    \item Identify activity level that does NOT trigger PEM (baseline tolerance)
    \item Identify activity level that triggers PEM $>$50\% of the time (threshold)
    \item Safe envelope = activity level significantly below threshold
    \item Apply 30--50\% safety buffer
\end{enumerate}

\paragraph{Example Calculation}.

\begin{itemize}
    \item Finding: M-level activities trigger PEM on 6 of 10 attempts (60\%)
    \item L-level activities tolerated on 9 of 10 attempts (90\%)
    \item Safe threshold: L-level activity
    \item Safe envelope: Stay consistently at L-level, occasional short M-activity acceptable only on high-energy days
\end{itemize}

\subsection{Monthly Analysis and Reporting}
\label{subsec:monthly-analysis}

Monthly summary to identify trends:

\begin{itemize}
    \item \textbf{Trend}: Improving, stable, or worsening disease?
    \item \textbf{Sleep quality}: Average nightly sleep, fragmentation trends, quality trends
    \item \textbf{Activity tolerance}: Safe activity level; crash threshold
    \item \textbf{Crash frequency}: How many crashes per month? Average duration?
    \item \textbf{Medication effects}: Changes in stimulant dosing/timing, effects observed
    \item \textbf{Symptom patterns}: Are certain symptoms (cognitive vs.\ pain vs.\ autonomic) more prominent during crashes?
    \item \textbf{Modifications}: Any activity pacing changes, medication adjustments, or interventions trialed?
    \item \textbf{Next month goals}: Objectives for next month's tracking/management
\end{itemize}

\subsection{Integration with Clinical Management}
\label{subsec:integration-management}

\paragraph{Use for Physician Communication}.

Share monthly tracking summaries with treating physicians to:

\begin{itemize}
    \item Document disease trajectory objectively
    \item Demonstrate medication or intervention effects
    \item Identify patterns requiring investigation (e.g., sleep-autonomic cycle)
    \item Support or refute hypotheses (e.g., fluoride-pineal effect on sleep)
\end{itemize}

\paragraph{Use for Hypothesis Testing}.

\begin{itemize}
    \item \textbf{Fluoride-pineal hypothesis}: Does melatonin supplementation (if trialed) improve sleep architecture? Do sleep improvements correlate with reduced autonomic dysregulation?
    \item \textbf{Carnitine shuttle}: Does Acetyl-L-Carnitine supplementation increase activity tolerance or reduce crash severity?
    \item \textbf{Stimulant strategy}: Does modified Ritalin dosing (split vs.\ single, different timing) improve sleep without reducing cognitive benefit?
\end{itemize}

\paragraph{Emergency Indicators}.

While tracking, watch for red flags requiring medical evaluation:

\begin{itemize}
    \item Persistent worsening (activity tolerance declining despite stable pacing)
    \item New symptom onset (especially autonomic events)
    \item Crash severity increasing (becoming bedbound when previously functional)
    \item Sleep deterioration (insomnia, severe fragmentation, polysomnography-documented abnormalities)
    \item Repeated autonomic dysregulation events
\end{itemize}

