% FILE: Diagnostic test protocol for Yannick
% Tier 1 tests with timing, logistics, and clinical reasoning
\section{Diagnostic Test Protocol: Tier 1 Recommended Tests}
\label{app:diagnostic-protocol}

This protocol outlines recommended diagnostic tests for Yannick's case, prioritized by clinical urgency, logistical feasibility, and diagnostic value. All tests are preliminary recommendations requiring physician review and clinical judgment.

\subsection{Priority Rationale}
\label{subsec:protocol-rationale}

Tests are prioritized based on:

\begin{enumerate}
    \item \textbf{Recent critical event}: Autonomic dysregulation during sleep-wake transition (Feb 11, 2026)
    \item \textbf{Diagnostic uncertainty}: ADHD vs.\ secondary attention deficit, specific autonomic syndrome
    \item \textbf{Mitochondrial dysfunction}: Presumed but not formally tested
    \item \textbf{Progressive sensory loss}: Vision and hearing require characterization
    \item \textbf{Feasibility}: Tests available in Belgium, covered by insurance where possible
\end{enumerate}

\subsection{Tier 1 Tests: Autonomic and Sleep Function}
\label{subsec:tier1-autonomic}

These tests address the recent critical event and possible sleep-related autonomic dysregulation.

\subsubsection{Polysomnography (Sleep Study)}
\label{subsubsec:protocol-psg}

\paragraph{Clinical Indication.}
Recent autonomic dysregulation during sleep-wake transition suggests sleep architecture abnormalities. Standard polysomnography is essential for:

\begin{itemize}
    \item Characterizing sleep stages and transitions
    \item Identifying REM sleep behavior abnormalities
    \item Detecting periodic breathing or sleep apnea
    \item Documenting autonomic events during sleep
    \item Timing of symptoms relative to sleep stages
\end{itemize}

\paragraph{Test Specifications.}

\begin{itemize}
    \item \textbf{Duration}: Full night study (7--8 hours)
    \item \textbf{Parameters to record}:
    \begin{itemize}
        \item EEG (electroencephalography): Standard sleep staging electrodes (C3-A2, C4-A1, O1-A2, O2-A1, Fz, Cz, Pz)
        \item EOG (electro-oculography): Eye movements for REM sleep identification
        \item EMG (electromyography): Chin EMG for sleep/wake detection; leg EMG for periodic leg movements
        \item Nasal airflow and respiratory effort (thoracic and abdominal)
        \item Oxygen saturation (continuous pulse oximetry)
        \item Heart rate and rhythm (ECG or cardiac monitoring)
        \item Body position (positional sensor)
        \item \textbf{Autonomic monitoring}: If available, include HRV (heart rate variability) analysis during transitions
    \end{itemize}
    \item \textbf{Timing}: Early evening start (20:00--21:00) to capture full sleep cycle
    \item \textbf{Location}: Sleep medicine lab (likely Vivalia or similar facility in Belgium)
\end{itemize}

\paragraph{Logistics.}

\begin{itemize}
    \item \textbf{Scheduling}: Contact sleep medicine clinic at Vivalia Arlon or referral to specialized sleep center
    \item \textbf{Preparation}: Avoid stimulants day of study (verify with physician); maintain normal sleep schedule 1 week prior
    \item \textbf{Cost}: Usually covered by insurance in Belgium (mutuelle/sécurité sociale)
    \item \textbf{Duration}: Single night study; results available within 1--2 weeks
\end{itemize}

\paragraph{Diagnostic Value for Yannick.}

\begin{itemize}
    \item Certainty of sleep architecture abnormality assessment: 0.85
    \item Ability to characterize autonomic dysregulation triggers: 0.70
    \item Probability of finding sleep-wake transition abnormality: 0.60--0.75
\end{itemize}

\subsubsection{Actigraphy (Continuous Activity and Sleep Tracking)}
\label{subsubsec:protocol-actigraphy}

\paragraph{Clinical Indication.}
Actigraphy provides objective, long-term tracking of sleep-wake patterns and circadian rhythm disruption, complementing polysomnography.

\paragraph{Test Specifications.}

\begin{itemize}
    \item \textbf{Device}: Wrist-worn actigraph (e.g., ActiWatch, Actiwatch 2, or equivalent)
    \item \textbf{Duration}: Minimum 7 days; optimal 14 days
    \item \textbf{Recording}: Continuous light exposure, activity, and rest-activity patterns
    \item \textbf{Analysis}: Sleep efficiency, sleep onset latency, total sleep time, fragmentation
\end{itemize}

\paragraph{Logistics.}

\begin{itemize}
    \item \textbf{Wear schedule}: Continuous wear (including sleep and showering if device is water-resistant)
    \item \textbf{Patient burden}: Minimal; non-invasive wrist watch
    \item \textbf{Cost}: Variable; may be arranged through sleep medicine clinic
    \item \textbf{Duration}: 1--2 weeks recording + 1 week analysis
\end{itemize}

\paragraph{Specific Value for Fluoride-Pineal Hypothesis.}

Light exposure tracking (actigraphy's built-in light sensor) directly relates to circadian disruption caused by pineal dysfunction. If melatonin production is impaired (fluoride hypothesis), expected findings:

\begin{itemize}
    \item Circadian phase desynchronization
    \item Irregular sleep-wake timing
    \item Increased fragmentation despite attempting regular sleep schedule
    \item Abnormal light-entrainment response
\end{itemize}

\subsubsection{Timed Melatonin Levels}
\label{subsubsec:protocol-melatonin}

\paragraph{Clinical Indication.}
Direct assessment of melatonin production to test the fluoride-pineal hypothesis (see Section~\ref{subsec:yannick-fluoride-hypothesis}).

\paragraph{Test Specifications.}

\begin{itemize}
    \item \textbf{Sample type}: Salivary melatonin (preferred) or plasma melatonin
    \item \textbf{Timing}: Critically important for interpretation
    \begin{itemize}
        \item \textbf{Evening sample} (22:00--23:00): Assess peak melatonin production
        \item \textbf{Night sample} (02:00--03:00): Assess sustained nocturnal levels
        \item \textbf{Morning sample} (06:00--07:00): Assess melatonin clearance
    \end{itemize}
    \item \textbf{Reference ranges}:
    \begin{itemize}
        \item Evening normal: 5--50 pg/mL (saliva)
        \item Night normal: 5--30 pg/mL (saliva)
        \item Morning normal: $<$3 pg/mL (saliva)
    \end{itemize}
\end{itemize}

\paragraph{Logistics.}

\begin{itemize}
    \item \textbf{Scheduling}: Arrange with sleep medicine clinic or endocrinology
    \item \textbf{Patient requirements}:
    \begin{itemize}
        \item Dim room lighting from 1 hour before evening sample
        \item No caffeine or stimulants after 18:00
        \item Regular sleep schedule maintained for 1 week prior
        \item Saliva samples collected in darkness or dim red light
    \end{itemize}
    \item \textbf{Cost}: May not be covered by insurance; typically €50--100 for panel
    \item \textbf{Duration}: Results in 2--3 weeks
\end{itemize}

\paragraph{Expected Findings if Fluoride-Pineal Hypothesis Supported.}

\begin{itemize}
    \item Low evening melatonin ($<$5 pg/mL)
    \item Blunted nocturnal surge (night levels not significantly elevated)
    \item Delayed peak (if present, peaks later than 22:00--23:00)
    \item Early morning clearance failure (morning levels $>$3 pg/mL)
\end{itemize}

\subsubsection{Autonomic Function Testing (Tilt Table Test)}
\label{subsubsec:protocol-tilt}

\paragraph{Clinical Indication.}
Given recent autonomic dysregulation and suspected POTS or orthostatic intolerance, formal autonomic assessment is indicated.

\paragraph{Test Specifications.}

\begin{itemize}
    \item \textbf{Duration}: 20--30 minutes
    \item \textbf{Protocol}:
    \begin{itemize}
        \item Supine baseline (5 minutes)
        \item Passive tilt (60--80 degrees) for 10 minutes
        \item Return to supine (5 minutes)
    \end{itemize}
    \item \textbf{Measurements}:
    \begin{itemize}
        \item Heart rate (continuous)
        \item Blood pressure (continuous or frequent)
        \item Symptoms (dizziness, chest pain, weakness, tremor)
        \item ECG (if indicated)
    \end{itemize}
\end{itemize}

\paragraph{Diagnostic Criteria.}

\begin{itemize}
    \item \textbf{POTS}: Heart rate increase $\geq$30 bpm or absolute HR $>$120 bpm within 10 minutes of upright posture
    \item \textbf{Orthostatic hypotension}: Systolic BP decrease $\geq$20 mmHg or diastolic BP decrease $\geq$10 mmHg
    \item \textbf{Dysautonomia}: Disproportionate HR/BP response to position change
\end{itemize}

\paragraph{Logistics.}

\begin{itemize}
    \item \textbf{Location}: Cardiology or autonomic physiology lab (Vivalia)
    \item \textbf{Preparation}:
    \begin{itemize}
        \item No stimulants 48 hours prior (important for Yannick given Ritalin use)
        \item Well-hydrated baseline
        \item Normal sleep night before
    \end{itemize}
    \item \textbf{Cost}: Usually covered by insurance
    \item \textbf{Duration}: ~45 minutes total (15 min prep + 20 min test + 10 min recovery)
    \item \textbf{Risk}: Low; procedure well-tolerated; abort if severe symptoms occur
\end{itemize}

\paragraph{Expected Findings if Sleep-Autonomic Link Present.}

\begin{itemize}
    \item Exaggerated HR response to tilt
    \item Sluggish blood pressure response
    \item Symptoms of fatigue/weakness during tilt (especially if not on Ritalin)
    \item Slow recovery to baseline after tilt completion
\end{itemize}

\subsection{Tier 1 Tests: Metabolic and Mitochondrial Function}
\label{subsec:tier1-metabolic}

\subsubsection{Basic Metabolic Panel with Lactate}
\label{subsubsec:protocol-lactate}

\paragraph{Clinical Indication.}
Direct assessment of lactate levels and lactate clearance to evaluate mitochondrial function capacity.

\paragraph{Test Specifications.}

\begin{itemize}
    \item \textbf{Test components}:
    \begin{itemize}
        \item Fasting venous lactate (baseline)
        \item 10 minutes of controlled mild exertion (e.g., stair climbing)
        \item Post-exertion lactate at 5, 15, 30, and 60 minutes
    \end{itemize}
    \item \textbf{Normal values}:
    \begin{itemize}
        \item Resting lactate: 0.5--1.5 mmol/L
        \item Post-exertion should return to baseline within 10--30 minutes
    \end{itemize}
    \item \textbf{ME/CFS pattern}: Elevated resting lactate or markedly delayed clearance (6--11x slower than normal)
\end{itemize}

\paragraph{Logistics.}

\begin{itemize}
    \item \textbf{Location}: General lab with capability for timed venous sampling
    \item \textbf{Timing}: Morning, fasting (no breakfast; allowed water)
    \item \textbf{Patient preparation}:
    \begin{itemize}
        \item No stimulants 24 hours prior (Ritalin not taken morning of test)
        \item Rest baseline 15 minutes before first sample
        \item Stair climbing or walking at constant pace for ~10 minutes (moderate intensity)
    \end{itemize}
    \item \textbf{Cost}: Moderate; usually €50--150 depending on lab
    \item \textbf{Duration}: 1.5 hours (baseline + exertion + timed samples)
\end{itemize}

\paragraph{Diagnostic Value.}

\begin{itemize}
    \item Sensitivity for mitochondrial dysfunction: 0.70
    \item Specificity for ME/CFS: 0.60
    \item Certainty for this patient: 0.75 (lactate responsiveness documented historically)
\end{itemize}

\subsubsection{Carnitine Panel}
\label{subsubsec:protocol-carnitine}

\paragraph{Clinical Indication.}
Assessment of total carnitine, free carnitine, and acyl-carnitine levels. Low carnitine (especially free carnitine) suggests carnitine shuttle dysfunction (documented in ME/CFS).

\paragraph{Test Specifications.}

\begin{itemize}
    \item \textbf{Plasma levels measured}:
    \begin{itemize}
        \item Total carnitine (normal: 26--67 $\mu$mol/L)
        \item Free carnitine (normal: 15--42 $\mu$mol/L)
        \item Acyl-carnitine ratio (normal: $<$0.4)
    \end{itemize}
    \item \textbf{Interpretation}:
    \begin{itemize}
        \item Low free carnitine suggests insufficient substrate for fatty acid oxidation
        \item Elevated acyl-carnitine ratio suggests impaired fat metabolism
    \end{itemize}
\end{itemize}

\paragraph{Logistics.}

\begin{itemize}
    \item \textbf{Location}: Specialized biochemistry lab (may require referral to university hospital)
    \item \textbf{Timing}: Morning, fasting
    \item \textbf{Cost}: €80--150; may not be covered by insurance
    \item \textbf{Duration}: Results in 1--2 weeks
\end{itemize}

\paragraph{Treatment Implications.}

If low free carnitine found:

\begin{itemize}
    \item Acetyl-L-Carnitine supplementation (currently started for Yannick) is justified
    \item Consider L-Carnitine as adjunct
    \item Repeat after 3--6 months of supplementation to assess response
\end{itemize}

\subsubsection{CoQ10 Levels}
\label{subsubsec:protocol-coq10}

\paragraph{Clinical Indication.}
CoQ10 is essential cofactor in mitochondrial electron transport chain. Low levels common in ME/CFS.

\paragraph{Test Specifications.}

\begin{itemize}
    \item \textbf{Measured level}: Plasma ubiquinol-10 (the reduced, active form)
    \item \textbf{Normal range}: 0.6--2.5 $\mu$mol/L
    \item \textbf{ME/CFS pattern}: Often $<$0.6 $\mu$mol/L
\end{itemize}

\paragraph{Logistics.}

\begin{itemize}
    \item \textbf{Location}: Specialized biochemistry lab
    \item \textbf{Cost}: €50--100; usually not covered
    \item \textbf{Duration}: 1--2 weeks
\end{itemize}

\paragraph{Treatment Implications.}

If low, initiate or optimize CoQ10 supplementation:

\begin{itemize}
    \item Dose: 100--300 mg daily (ubiquinone form; ubiquinol preferred for bioavailability)
    \item Expect 8--12 weeks for steady-state levels
    \item Retest after 3--6 months
\end{itemize}

\subsection{Tier 1 Tests: Immune Characterization}
\label{subsec:tier1-immune}

\subsubsection{Cytokine Panel (Inflammatory Markers)}
\label{subsubsec:protocol-cytokines}

\paragraph{Clinical Indication.}
ME/CFS typically shows elevated pro-inflammatory cytokines (IL-6, TNF-$\alpha$, IL-1$\beta$). Post-viral state (after Jan 2026 URI) provides opportunity to assess immune response pattern.

\paragraph{Test Specifications.}

\begin{itemize}
    \item \textbf{Measured cytokines}:
    \begin{itemize}
        \item IL-6 (interleukin-6): Normal $<$4 pg/mL
        \item TNF-$\alpha$ (tumor necrosis factor-alpha): Normal $<$2 pg/mL
        \item IL-1$\beta$ (interleukin-1 beta): Normal $<$2 pg/mL
        \item IL-8 (interleukin-8): Normal $<$10 pg/mL
        \item IL-10 (interleukin-10): Normal $<$5 pg/mL (anti-inflammatory)
    \end{itemize}
    \item \textbf{ME/CFS pattern}: Elevated IL-6 and TNF-$\alpha$; IL-10 response may be blunted
\end{itemize}

\paragraph{Logistics.}

\begin{itemize}
    \item \textbf{Location}: Immunology lab (university hospital or specialized lab)
    \item \textbf{Timing}: Morning, fasting (cytokines show diurnal variation)
    \item \textbf{Cost}: €150--300; usually not covered
    \item \textbf{Duration}: 2--3 weeks (specialized testing)
\end{itemize}

\paragraph{Clinical Value.}

\begin{itemize}
    \item Confirms inflammatory component of disease
    \item Baseline for assessing anti-inflammatory interventions (LDN, others)
    \item Prognostic significance (higher IL-6 correlates with severity)
\end{itemize}

\subsubsection{Blood Gas Analysis (Venous)}
\label{subsubsec:protocol-bloodgas}

\paragraph{Clinical Indication.}
Assessment of oxygen delivery and acid-base status, particularly relevant to ``air hunger'' symptom and respiratory dysfunction.

\paragraph{Test Specifications.}

\begin{itemize}
    \item \textbf{Measured parameters}:
    \begin{itemize}
        \item pO2 (venous oxygen pressure): Normal 35--50 mmHg
        \item pCO2 (venous carbon dioxide pressure): Normal 40--50 mmHg
        \item pH: Normal 7.35--7.40
        \item HCO3 (bicarbonate): Normal 22--26 mEq/L
        \item Base excess: Normal -2 to +2 mEq/L
    \end{itemize}
    \item \textbf{Pathological pattern in ME/CFS}:
    \begin{itemize}
        \item High venous pO2 (if tissue oxygenation is impaired)
        \item Low or low-normal pCO2 (hyperventilation response)
        \item Possible lactic acidosis (low pH, negative base excess)
    \end{itemize}
\end{itemize}

\paragraph{Logistics.}

\begin{itemize}
    \item \textbf{Location}: Any general lab or hospital blood gas analyzer
    \item \textbf{Timing}: Any time (no prep required)
    \item \textbf{Cost}: €20--40; usually covered
    \item \textbf{Duration}: 15 minutes (immediate results)
\end{itemize}

\paragraph{Special Consideration for Yannick.}

Obtain samples in:

\begin{itemize}
    \item Supine position (resting)
    \item Immediately after standing for 1--2 minutes (if tolerated)
    \item 5 minutes post-standing (if available)

This may reveal orthostatic-dependent changes in oxygenation.
\end{itemize}

\subsection{Tier 1 Tests: Cognitive and Psychiatric}
\label{subsec:tier1-cognitive}

\subsubsection{Formal Neuropsychological Testing}
\label{subsubsec:protocol-neuro}

\paragraph{Clinical Indication.}
Document objective cognitive deficits and clarify ADHD vs.\ ME/CFS-related dysfunction.

\paragraph{Test Battery.}

\begin{itemize}
    \item \textbf{Attention/Executive function}:
    \begin{itemize}
        \item Continuous Performance Test (CPT)
        \item Wisconsin Card Sorting Test (WCST)
        \item Trail-Making Test A \& B
    \end{itemize}
    \item \textbf{Memory}:
    \begin{itemize}
        \item California Verbal Learning Test (CVLT)
        \item Digit span (forward and backward)
        \item Logical memory subtest
    \end{itemize}
    \item \textbf{Processing speed}:
    \begin{itemize}
        \item Digit symbol coding
        \item Symbol digit modalities
    \end{itemize}
    \item \textbf{Mood/behavior}:
    \begin{itemize}
        \item Beck Depression Inventory (BDI-II)
        \item Beck Anxiety Inventory (BAI)
        \item Fatigue Severity Scale (FSS)
    \end{itemize}
\end{itemize}

\paragraph{Special Considerations.}

\begin{itemize}
    \item \textbf{Timing}: Test on day when on Ritalin (to establish baseline with treatment)
    \item \textbf{Duration}: 2--3 hours
    \item \textbf{Interpretation}: ME/CFS often shows scattered deficits rather than global impairment (unlike dementia or primary psychiatric illness)
    \item \textbf{Fatigue effect}: May be limited by fatigue; psychologist should note and account for
\end{itemize}

\paragraph{Logistics.}

\begin{itemize}
    \item \textbf{Location}: Neuropsychology clinic (university hospital, private practice)
    \item \textbf{Cost}: €400--800; usually not covered by insurance
    \item \textbf{Duration}: Scheduling + testing 2--3 weeks
\end{itemize}

\subsection{Summary: Tier 1 Test Schedule}
\label{subsec:tier1-schedule}

\textbf{Immediate priority (next 2 weeks)}:

\begin{itemize}
    \item Sleep medicine referral and polysomnography scheduling
    \item Tilt table test scheduling
    \item Blood draw (lactate challenge, carnitine, CoQ10, cytokines, blood gas)
\end{itemize}

\textbf{Short-term (2--4 weeks)}:

\begin{itemize}
    \item Actigraphy start (2 weeks continuous)
    \item Melatonin level sampling (timed, multiple points)
    \item Neuropsychological testing scheduling
\end{itemize}

\textbf{Concurrent/ongoing}:

\begin{itemize}
    \item Sleep-activity tracking protocol (see Section~\ref{app:activity-tracking})
    \item Continue current mitochondrial support supplements
    \item Sleep medicine specialist consultation (see Section~\ref{app:sleep-specialist})
\end{itemize}

