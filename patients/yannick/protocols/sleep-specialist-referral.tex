% FILE: Sleep medicine specialist referral and consultation questions
\section{Sleep Medicine Specialist Referral: Clinical Questions}
\label{app:sleep-specialist}

This document outlines specific clinical questions for sleep medicine evaluation of Yannick's recent autonomic dysregulation during sleep-wake transitions and chronic sleep dysfunction.

\subsection{Referral Summary}
\label{subsec:referral-summary}

\paragraph{Patient}: Yannick, 44 years old

\paragraph{Chief Complaint}: Autonomic dysregulation during sleep-wake transitions; chronic sleep dysfunction in ME/CFS; recent acute dysregulation event (Feb 11, 2026)

\paragraph{Relevant Medical History}:

\begin{itemize}
    \item ME/CFS diagnosis (severe, post-2018 progression)
    \item ADHD with stimulant treatment (Ritalin MR 30mg, recently resumed)
    \item Previous modafinil trial (now discontinued)
    \item Recent upper respiratory infection (Jan 25, 2026) with severe autonomic exacerbation
    \item Post-viral fatigue and weakness (Feb 2--3, 2026)
    \item Activity-triggered post-exertional malaise (Feb 8--10, 2026)
    \item \textbf{Critical event}: Acute autonomic dysregulation during sleep-wake transition (Feb 11, 2026)
\end{itemize}

\subsection{Sleep Architecture and Function Questions}
\label{subsec:sleep-questions}

\subsubsection{Sleep Quality and Continuity}

\begin{enumerate}
    \item \textbf{Sleep architecture characterization}: What is the patient's sleep stage distribution (N1, N2, N3, REM) and does the distribution differ from healthy controls? Is there:
    \begin{itemize}
        \item Reduced deep sleep (N3)?
        \item Abnormal REM proportion or timing?
        \item Fragmentation or frequent arousals?
    \end{itemize}

    \item \textbf{Sleep continuity}: How many arousals per hour occur? Are there identifiable triggers for arousals (periodic breathing, apneic events, movements)?

    \item \textbf{Sleep efficiency}: What percentage of time in bed is actually spent asleep? (Normal: 85--90\%; target: 80\%+)

    \item \textbf{REM sleep abnormalities}: Is there:
    \begin{itemize}
        \item REM behavior disorder (incomplete atonia, movements during REM)?
        \item Sleep paralysis episodes?
        \item Vivid or disturbing dreams?
        \item Delayed REM latency?
    \end{itemize}
\end{enumerate}

\subsubsection{Breathing During Sleep}

\begin{enumerate}
    \item \textbf{Sleep apnea screening}: Apnea-hypopnea index (AHI)---is there evidence of:
    \begin{itemize}
        \item Obstructive sleep apnea (OSA)?
        \item Central sleep apnea (CSA)?
        \item Periodic breathing (Cheyne-Stokes pattern)?
        \item Hypoventilation during sleep?
    \end{itemize}

    \item \textbf{Oxygen desaturation}: Are there documented drops in oxygen saturation during sleep? If present:
    \begin{itemize}
        \item Magnitude (how low does SpO2 drop)?
        \item Frequency (how many desaturation events)?
        \item Duration (how long does SpO2 remain low)?
    \end{itemize}

    \item \textbf{Dysfunctional breathing pattern}: Is there evidence of:
    \begin{itemize}
        \item Reduced tidal volume?
        \item Loss of diaphragmatic drive?
        \item Accessory muscle usage?
        \item Paradoxical breathing (chest-abdomen asynchrony)?
    \end{itemize}

    \item \textbf{Connection to ``air hunger''}: Does the breathing dysfunction documented in polysomnography correlate with daytime air hunger symptom?
\end{enumerate}

\subsection{Autonomic Function During Sleep}
\label{subsec:autonomic-sleep-questions}

These questions directly address the recent critical event (autonomic dysregulation during sleep-wake transition).

\subsubsection{Heart Rate and Blood Pressure Patterns}

\begin{enumerate}
    \item \textbf{Sleep-stage-dependent autonomic variation}: Is there abnormal heart rate or blood pressure variation correlated with sleep stages?
    \begin{itemize}
        \item Expected: HR decreases during sleep, increases during REM
        \item Abnormal: Exaggerated swings, failure to decrease, inappropriate tachycardia
    \end{itemize}

    \item \textbf{Sleep-wake transition autonomic dysregulation}: Specifically during:
    \begin{itemize}
        \item Wake-to-sleep transition: Does HR or BP change abnormally?
        \item Sleep-to-wake transition: Is there exaggerated HR surge upon awakening?
        \item REM-to-NREM transitions: Are autonomic changes dysregulated?
    \end{itemize}

    \item \textbf{Sleep-stage-specific findings}:
    \begin{itemize}
        \item \textbf{N1-N2 transitions}: Stabilized HR/BP or abnormal surges?
        \item \textbf{N2-N3 transitions}: Deep sleep normally associated with HR decrease; is this present?
        \item \textbf{REM sleep}: Is there appropriate phasic HR variability or blunted response?
    \end{itemize}

    \item \textbf{Circadian patterning}: Is there normal overnight decline in HR/BP or are patterns flattened/inverted?
\end{enumerate}

\subsubsection{Sympathetic/Parasympathetic Balance}

\begin{enumerate}
    \item \textbf{Heart rate variability (HRV) analysis}:
    \begin{itemize}
        \item Is HRV normal for age?
        \item Is there appropriate parasympathetic activation during sleep (high HRV)?
        \item Is there excessive sympathetic activity (low HRV, elevated HR)?
    \end{itemize}

    \item \textbf{Autonomic balance by sleep stage}:
    \begin{itemize}
        \item NREM: Should show parasympathetic dominance (low HR, high HRV)
        \item REM: Variable but generally lower HRV due to phasic sympathetic surges
        \item Wake: Sympathetic dominance (higher HR, lower HRV)
    \end{itemize}

    \item \textbf{Abnormal sympathetic activation}: Evidence of:
    \begin{itemize}
        \item Inappropriate tachycardia during sleep?
        \item Sudden HR surges during transitions?
        \item Phasic sweating captured on ECG/autonomic monitors?
    \end{itemize}

    \item \textbf{Vagal tone assessment}: Is there evidence of parasympathetic impairment (blunted HR response, absent expected bradycardia in deep sleep)?
\end{enumerate}

\subsection{Movement-Related and Periodic Phenomena}
\label{subsec:movement-questions}

\begin{enumerate}
    \item \textbf{Periodic leg movements (PLM)}: Is there evidence of:
    \begin{itemize}
        \item Periodic leg movement index (PLMI) $>$5--15 per hour?
        \item Arousal association with leg movements?
        \item Relationship to sleep stage (typically NREM, particularly N1-N2)?
    \end{itemize}

    \item \textbf{Muscle tone and EMG activity}:
    \begin{itemize}
        \item Is there appropriate REM atonia (loss of tone during REM)?
        \item Evidence of muscle contractures or cramping during sleep?
        \item Tremor or involuntary movements captured on EMG?
    \end{itemize}

    \item \textbf{Sleep-related movement disorders}: Any evidence of:
    \begin{itemize}
        \item Restless legs syndrome (RLS)?
        \item REM sleep behavior disorder (RBD)?
        \item Sleep-related bruxism (teeth grinding)?
        \item Other periodic or stereotyped movements?
    \end{itemize}
\end{enumerate}

\subsection{Circadian Rhythm and Melatonin Function Questions}
\label{subsec:circadian-questions}

These questions specifically address the fluoride-pineal-sleep hypothesis (see Section~\ref{subsec:yannick-fluoride-hypothesis}).

\begin{enumerate}
    \item \textbf{Circadian phase}: Is the patient's circadian rhythm phase-aligned with clock time or is there evidence of:
    \begin{itemize}
        \item Delayed sleep phase (difficulty falling asleep at normal bedtime)?
        \item Advanced sleep phase (early morning awakening)?
        \item Irregular sleep-wake pattern (non-24-hour rhythm)?
    \end{itemize}

    \item \textbf{Melatonin-related questions}:
    \begin{itemize}
        \item Is there evidence of melatonin dysfunction (suggested by sleep-wake dysregulation)?
        \item Does sleep architecture show features consistent with melatonin insufficiency (REM abnormalities, reduced deep sleep)?
        \item Is there abnormal light sensitivity or light entrainment?
    \end{itemize}

    \item \textbf{Pineal calcification assessment}: Given the fluoride hypothesis:
    \begin{itemize}
        \item Is there clinical or imaging evidence of pineal abnormality?
        \item Can sleep dysfunction be explained by documented pineal pathology?
        \item Would brain imaging (MRI) with pineal imaging be appropriate?
    \end{itemize}

    \item \textbf{Melatonin supplementation trial recommendations}: If melatonin dysfunction is confirmed:
    \begin{itemize}
        \item What dose and timing would be appropriate for this patient?
        \item Should exogenous melatonin replacement trial be considered?
        \item What response parameters should be monitored?
    \end{itemize}

    \item \textbf{Light therapy assessment}:
    \begin{itemize}
        \item Would light therapy (bright light exposure) help reset circadian rhythm?
        \item Timing: Should light exposure be morning (phase advance) or evening (phase delay)?
        \item Intensity and duration recommendations?
    \end{itemize}
\end{enumerate}

\subsection{Drug-Sleep Interaction Questions}
\label{subsec:drug-interaction-questions}

Critical given Yannick's medication history and recent Ritalin resumption.

\begin{enumerate}
    \item \textbf{Ritalin MR (methylphenidate) and sleep}:
    \begin{itemize}
        \item Does stimulant dose or timing cause sleep disruption?
        \item Is current morning dosing (30mg MR) associated with documented sleep fragmentation?
        \item Is there evidence of stimulant rebound insomnia (worse sleep on off-days)?
        \item Recommendation: Split dosing vs. morning-only vs. alternative timing?
    \end{itemize}

    \item \textbf{Modafinil (Provigil) legacy effects}:
    \begin{itemize}
        \item Extended half-life of modafinil (~15 hours) may persist for weeks after discontinuation
        \item Could prior modafinil use be contributing to current sleep dysfunction?
        \item Timeline: When was modafinil discontinued? Is sufficient washout period passed (3--4 weeks)?
    \end{itemize}

    \item \textbf{LDN (Low-Dose Naltrexone) interaction}:
    \begin{itemize}
        \item Typical dosing: 4mg at bedtime; should enhance sleep (opioid antagonism)
        \item Any evidence of sleep disruption despite LDN?
        \item Should dosing timing or amount be adjusted?
    \end{itemize}

    \item \textbf{Supplement-drug interactions}:
    \begin{itemize}
        \item Acetyl-L-Carnitine timing: Morning or evening? Does timing affect sleep?
        \item CoQ10, Riboflavin, other supplements: Any sleep effects documented?
        \item Recommendations for optimal supplement timing relative to stimulants and sleep?
    \end{itemize}
\end{enumerate}

\subsection{Clinical Event Characterization: Feb 11 Autonomic Dysregulation}
\label{subsec:feb11-event}

\paragraph{Event Description}.

Feb 11, 2026: Acute autonomic dysregulation during sleep-wake transition.

\paragraph{Sleep Specialist Questions Regarding This Event}.

\begin{enumerate}
    \item \textbf{Was this a sleep-related hypoventilation event?}
    \begin{itemize}
        \item Sudden breathing cessation or severe reduction during transition?
        \item Associated oxygen desaturation?
        \item Arousal response?
    \end{itemize}

    \item \textbf{Was this a sleep apnea arousal?}
    \begin{itemize}
        \item Apneic event (cessation of breathing) during sleep?
        \item Associated autonomic response (sympathetic surge)?
        \item Arousal from sleep?
    \end{itemize}

    \item \textbf{Was this REM-related autonomic dysregulation?}
    \begin{itemize}
        \item Occurred during REM sleep or REM-to-NREM transition?
        \item REM-associated muscle atonia but preserved consciousness (sleep paralysis variant)?
        \item Fragmented REM episodes?
    \end{itemize}

    \item \textbf{Was this a postural/orthostatic response?}
    \begin{itemize}
        \item Patient attempted to sit up or change position?
        \item Orthostatic intolerance triggered by position change?
        \item Inadequate blood pressure regulation during transition from supine to upright?
    \end{itemize}

    \item \textbf{Can polysomnography or actigraphy characterization predict future events?}
    \begin{itemize}
        \item What sleep stage or physiological pattern precedes dysregulation?
        \item What early warning signs should patient monitor?
        \item Preventive interventions?
    \end{itemize}
\end{enumerate}

\subsection{ME/CFS-Specific Sleep Dysfunction Questions}
\label{subsec:mecfs-sleep-questions}

\begin{enumerate}
    \item \textbf{Is the sleep dysfunction consistent with typical ME/CFS pattern?}
    \begin{itemize}
        \item Documented findings in ME/CFS: Abnormal sleep architecture, reduced deep sleep, REM abnormalities, sleep fragmentation
        \item Does this patient's polysomnography match ME/CFS phenotype?
    \end{itemize}

    \item \textbf{Orthostatic-sleep connection}: Given documented orthostatic intolerance:
    \begin{itemize}
        \item Does sleep dysfunction exacerbate daytime orthostatic symptoms?
        \item Is there reciprocal relationship (poor sleep $\to$ worse POTS; worse POTS $\to$ poor sleep)?
        \item Interventions to break this cycle?
    \end{itemize}

    \item \textbf{Activity-sleep-crash cycle}: Given activity-triggered PEM:
    \begin{itemize}
        \item Does activity on day $n$ cause sleep disruption on night $n$?
        \item Is sleep disruption followed by next-day autonomic dysregulation?
        \item Can improved sleep prevent or reduce PEM severity?
    \end{itemize}

    \item \textbf{Cognitive function and sleep}:
    \begin{itemize}
        \item Is cognitive dysfunction (brain fog, attention deficit) sleep-dependent?
        \item Does improved sleep quality improve daytime cognition?
        \item Sleep architecture alterations that could explain cognitive symptoms?
    \end{itemize}

    \item \textbf{Recommended sleep medicine interventions for ME/CFS}:
    \begin{itemize}
        \item Is cognitive behavioral therapy for insomnia (CBT-I) appropriate (modified for ME/CFS, no increase in activity)?
        \item Pharmacological options safe in ME/CFS (avoid those that worsen autonomic dysfunction)?
        \item Sleep hygiene recommendations specific to this patient's constraints (activity limitations, autonomic issues)?
    \end{itemize}
\end{enumerate}

\subsection{Consultation Outcomes and Next Steps}
\label{subsec:consultation-outcomes}

\paragraph{Requested Action Items for Sleep Medicine Specialist}.

\begin{enumerate}
    \item Full polysomnography with autonomic monitoring (HRV, ECG, blood pressure trending)
    \item Actigraphy for 14 days (light exposure, sleep-wake cycles, circadian characterization)
    \item Timed melatonin levels (if not performed elsewhere) to assess pineal function
    \item Assessment of medication-sleep interactions, particularly stimulant timing
    \item Written summary addressing specific questions (above)
    \item Recommendations for:
    \begin{itemize}
        \item Ritalin timing optimization (if affecting sleep)
        \item Melatonin supplementation trial (dose, timing, duration for N-of-1 assessment)
        \item Sleep-wake-autonomic management strategies
        \item Red flags for emergency evaluation (severe apnea, prolonged desaturation, severe dysautonomia)
    \end{itemize}
\end{enumerate}

\paragraph{Expected Physician Review Timeline}.

\begin{itemize}
    \item Initial polysomnography: Schedule within 2--4 weeks
    \item Results analysis: 1--2 weeks post-study
    \item Sleep specialist consultation: 1 week after results available
    \item Trial implementation: 2--4 weeks
    \item Follow-up assessment: 6--8 weeks after intervention start
\end{itemize}

\paragraph{Integration with Other Protocols}.

Sleep medicine findings should be integrated with:

\begin{itemize}
    \item Diagnostic test protocol (Section~\ref{app:diagnostic-protocol}) - coordinate timing of studies
    \item Activity-sleep-rebound tracking protocol (Section~\ref{app:activity-tracking}) - establish baseline before interventions
    \item Fluoride-pineal hypothesis investigation (Section~\ref{subsec:yannick-fluoride-hypothesis}) - melatonin results inform hypothesis
\end{itemize}

