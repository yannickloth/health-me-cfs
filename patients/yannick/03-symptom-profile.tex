% FILE: Detailed symptom profile for Yannick
% Migrated from appendix-i-personal-symptoms.tex with updated cross-references
\section{Detailed Symptom Profile}
\label{app:personal-symptoms}

This section documents a detailed personal symptom profile for use in clinical reasoning, treatment planning, and understanding symptom interconnections. The symptoms described here illustrate how ME/CFS manifests in this individual case, with pathophysiological explanations based on current research.

For additional information, see:
\begin{itemize}
    \item Section~\ref{app:medical-management}: Current medical management, protocols, and interventions
    \item Section~\ref{app:clinical-findings}: Clinical findings, laboratory results, and medical history
    \item Section~\ref{app:case-analysis}: Case analysis, diagnostic reasoning, and treatment plans
\end{itemize}

\subsection{Primary Symptoms}
\label{sec:personal-primary}

\subsubsection{Constant Fatigue and Exertion Intolerance}
\label{subsec:personal-fatigue}

The dominant symptom is a persistent sensation of \textbf{running on empty}---a profound energy deficit that is not relieved by rest. This differs qualitatively from normal tiredness:

\begin{itemize}
    \item Constant feeling of exhaustion regardless of activity level
    \item Sensation of ``emptiness'' or depleted reserves
    \item Inability to sustain even minor physical or cognitive efforts
    \item No recuperation from sleep or rest periods
\end{itemize}

\paragraph{Pathophysiological Basis.}
According to the 2024 NIH deep phenotyping study, the brain's temporoparietal junction (TPJ) shows decreased activity in ME/CFS patients. This region is responsible for effort-based decision-making. The ``empty'' feeling represents a physiological signal from a brain that has detected inadequate energy reserves, not a psychological state.

The underlying metabolic dysfunction involves:
\begin{enumerate}
    \item \textbf{Carnitine shuttle failure}: Long-chain fatty acids cannot be transported into mitochondria efficiently, effectively ``locking'' fuel outside the cellular engines.
    \item \textbf{Pyruvate dehydrogenase (PDH) dysfunction}: Creates a ``backup'' in the TCA cycle, preventing efficient processing of both fats and sugars.
    \item \textbf{Compensatory glycolysis}: The body over-relies on anaerobic sugar metabolism, producing minimal ATP and excessive lactic acid.
\end{enumerate}

\subsubsection{Cognitive Impairment: Complex Presentation}
\label{subsec:personal-cognitive}

The cognitive dysfunction has \textbf{multiple overlapping components} with diagnostic uncertainty regarding primary versus secondary etiologies:

\paragraph{Attention Deficit (ADHD-Like Symptoms of Uncertain Etiology).}
\label{subsubsec:personal-adhd}

\textbf{Clinical History.}
Severe attention and focus difficulties present since \textbf{childhood through adolescence and university years}:
\begin{itemize}
    \item Could read a page multiple times without processing or retaining content
    \item Did not understand what ``being focused'' meant until experiencing it on methylphenidate
    \item Reading comprehension failure despite adequate intelligence and effort
    \item Profound difficulty with sustained attention
\end{itemize}

\textbf{Response to Methylphenidate.}
Treatment with Ritalin (methylphenidate) during university studies was \textbf{transformative} for understanding cognition:
\begin{itemize}
    \item First experience of what ``focus'' actually feels like
    \item Ability to understand what the author of scientific and IT books wants the reader to learn
    \item Learning what kind of mental effort is \textit{supposed} to be required
    \item Realization of what it means to genuinely focus and comprehend material
    \item Made studying easier, though energy and motivation remained limiting factors
    \item Completed two degrees with honours, but recognized this was far below true capacity with adequate energy
    \item This experiential learning helped improve function even beyond medication effects
    \item \textbf{Dramatic dose-response relationship}:
    \begin{itemize}
        \item No medication: Severe cognitive impairment, chronic fatigue
        \item 1 tablet: Moderate improvement but still energy-limited
        \item 2 tablets: Fully mentally engaged, even excited/impatient---``day and night'' difference
        \item Suggests stimulant is compensating for profound underlying energy deficit
    \end{itemize}
\end{itemize}

\textbf{Response to Modafinil (Provigil).}
Modafinil has been used as a daily baseline medication, currently being phased out in favor of methylphenidate monotherapy:
\begin{itemize}
    \item Effective at reducing the subjective feeling of being ``too tired''
    \item Does not guarantee mental clarity or cognitive improvement
    \item \textbf{Comparison with methylphenidate}: Ritalin is superior because it also addresses tiredness while additionally providing mental clarity and stronger motivational drive
    \item \textbf{Cost considerations}: Both medications are expensive; practical decision to maintain only one medication given superior efficacy of methylphenidate
    \item \textbf{Physical symptoms persist}: Objective physical fatigue and air hunger remain regardless of either stimulant medication
    \item \textbf{Clinical significance}: Demonstrates dissociation between:
    \begin{itemize}
        \item Subjective tiredness (partially responsive to stimulants)
        \item Objective physical fatigue and metabolic dysfunction (unresponsive to stimulants)
    \end{itemize}
\end{itemize}

\textbf{Diagnostic Uncertainty: Primary ADHD vs.\ Secondary Attention Deficit.}
The etiology of these attention deficits remains uncertain despite evaluation:
\begin{itemize}
    \item \textbf{ADHD testing}: Multiple evaluations, all negative
    \item \textbf{Family history}: Mother and 2 sisters with positive ADHD diagnoses (suggests genetic predisposition)
    \item \textbf{Dose-response pattern}: The dramatic dose-response relationship (0 vs.\ 1 vs.\ 2 tablets producing stepwise ``day and night'' differences) suggests the stimulant is primarily compensating for energy deficit rather than correcting a dopamine signaling disorder
    \item \textbf{Competing hypothesis}: Energy deficits cause secondary attention impairment
    \begin{itemize}
        \item Energy-deprived brains prioritize survival functions over executive functions
        \item Sustained attention requires significant metabolic resources
        \item When ATP is scarce, the brain ``turns off'' non-essential cognitive processes
        \item Anyone with chronic energy insufficiency will exhibit ADHD-like symptoms
        \item Stimulants increase catecholamine availability, providing compensatory ``metabolic drive''
    \end{itemize}
    \item \textbf{Diagnostic dilemma}: Lifelong energy deficits mean no ``normal energy baseline'' exists
    \begin{itemize}
        \item Cannot test whether attention normalizes with adequate energy (never had adequate energy to test this)
        \item Family history suggests genetic vulnerability, but negative testing argues against primary ADHD
        \item Stimulant response doesn't prove ADHD (stimulants improve attention in many energy-deficit states)
        \item The subjective feeling of chronic tiredness argues for energy deficit as primary mechanism
    \end{itemize}
\end{itemize}

\textbf{Clinical Implication.}
Regardless of whether this represents primary ADHD or secondary attention deficit from metabolic dysfunction, methylphenidate remains \textbf{essential for baseline cognitive function}. The distinction matters for:
\begin{itemize}
    \item \textbf{Prognosis}: If secondary to energy deficit, addressing mitochondrial dysfunction might reduce stimulant dependence over time
    \item \textbf{Treatment strategy}: Primary ADHD requires lifelong stimulants; secondary attention deficits might respond to metabolic interventions (Acetyl-L-Carnitine, CoQ10, etc.)
    \item \textbf{Interpretation}: Stimulant need reflects either neurodevelopmental disorder or compensatory mechanism for metabolic insufficiency (or both)
\end{itemize}

\paragraph{Progressive Brain Fog (ME/CFS Pattern).}
\label{subsubsec:personal-brainfog}

\textbf{Clinical History.}
In addition to the attention deficit, a separate pattern of \textbf{energy-dependent cognitive fatigue} has been present since teenage years (age $\sim$13--15), with \textbf{progressive worsening over 30+ years}:
\begin{itemize}
    \item Episodes of mental fog that occur and worsen throughout the day
    \item Cognitive fatigue that worsens with exertion (cognitive PEM)
    \item Progressive increase in frequency and severity over decades
    \item Not fully responsive to stimulant medication alone
\end{itemize}

This pattern suggests slow-onset metabolic or mitochondrial disorder beginning in adolescence, though it may overlap with or explain the attention deficits described above.

\textbf{Current Presentation.}
The combined cognitive dysfunction manifests as:
\begin{itemize}
    \item Difficulty with concentration and sustained attention (lifelong baseline)
    \item Slowed mental processing (progressive energy-dependent)
    \item Word-finding difficulties (progressive energy-dependent)
    \item Short-term memory impairment (both baseline and exertion-sensitive)
    \item Difficulty with complex or multi-step reasoning (both baseline and exertion-sensitive)
    \item Worsening with physical or cognitive exertion (progressive PEM pattern)
\end{itemize}

Distinguishing which symptoms represent primary attention deficit versus secondary energy-dependent dysfunction is not clinically possible given lifelong energy insufficiency.

\textbf{Pathophysiological Basis.}
The brain consumes approximately 20\% of the body's total energy. When mitochondrial function is impaired, the brain ``dims the lights'' to conserve power---a state researchers term \textbf{neuro-exhaustion}. The 2024 NIH study found abnormally low levels of catecholamines (norepinephrine, dopamine) in cerebrospinal fluid, which are essential for cognitive function and motor control.

Acetyl-L-carnitine specifically addresses brain fog because the acetyl group crosses the blood-brain barrier, providing fuel directly to neurons.

\paragraph{Social Interaction as Painful Exertion.}
\label{subsubsec:personal-social-pain}

\textbf{Clinical History.}
For at least \textbf{2 decades} (since approximately early adulthood), social interaction has been experienced as painful and exhausting rather than enjoyable:

\begin{itemize}
    \item Socializing at work, discussing with colleagues, or engaging in conversation felt painful
    \item The subjective experience was identical to avoiding pain or being forced to do something painful while exhausted
    \item Approach to social interaction: ``I must do it, but keep the pain minimal''
    \item In most cases, there was no enjoyment or fun in social engagement
    \item This was a constant baseline experience, not limited to periods of worsening
    \item Others noticed and commented that the patient was ``not obviously happy''---the absence of visible enjoyment or positive affect was externally observable
\end{itemize}

\textbf{Pathophysiological Basis.}
Social interaction is a high-energy cognitive and emotional task requiring:

\begin{enumerate}
    \item \textbf{Sustained attention and cognitive processing}: Following conversation, processing language, formulating responses, maintaining context---all require significant prefrontal cortex activity and sustained ATP production.

    \item \textbf{Emotional regulation and affect generation}: Smiling, making appropriate facial expressions, modulating tone, and generating emotional responses are metabolically demanding processes requiring coordination between limbic system and motor control.

    \item \textbf{Executive function load}: Social interaction requires continuous monitoring of social cues, adjusting behavior in real-time, suppressing irrelevant responses, and maintaining socially appropriate conduct---high executive function demands.

    \item \textbf{Sensory processing burden}: Processing faces, voices, body language, and environmental context simultaneously creates high sensory load.

    \item \textbf{Motivation and reward system engagement}: Normal social interaction activates dopamine reward pathways. When dopamine and energy are chronically insufficient (as documented in ME/CFS and suggested by excellent stimulant response), social interaction loses rewarding properties and becomes purely effortful.
\end{enumerate}

When baseline metabolic capacity is insufficient, the brain experiences social demands as it would physical exertion beyond capacity: as painful, something to avoid, something to minimize. The ``pain avoidance'' framing is an accurate perception of the brain's energy crisis during cognitively demanding social tasks.

\textbf{Observable Impact: Flat Affect and Absence of Positive Expression.}
The external observation that the patient was ``not obviously happy'' reflects the metabolic cost of generating and displaying positive affect:

\begin{itemize}
    \item \textbf{Affect requires energy}: Smiling, animated facial expressions, vocal prosody, and body language signaling enjoyment all require muscular activation and sustained motor control---metabolically expensive processes.

    \item \textbf{Energy conservation prioritization}: When ATP is scarce, the brain conserves energy by reducing ``non-essential'' outputs, including expressive affect. The result is flat or reduced emotional expression even when some degree of internal positive feeling may be present.

    \item \textbf{Dopamine and reward visibility}: Low dopamine levels impair both the experience of reward and the motivation to express it. Others perceive this as absence of happiness because the neurological substrate for expressing enjoyment is impaired.

    \item \textbf{Not masking or suppression}: This is distinct from consciously hiding emotions. The absence of visible happiness reflects genuine inability to generate the energetic and neurochemical processes required for positive emotional expression.
\end{itemize}

This observable lack of positive affect, combined with the internal experience of social interaction as painful, demonstrates the profound impact of energy deficit on emotional and social functioning. It also confirms that this is not purely subjective---the metabolic impairment manifests visibly to others.

\textbf{Interpersonal Consequences: Misinterpretation as Contempt.}
The flat affect and absence of visible enjoyment created significant interpersonal difficulties:

\begin{itemize}
    \item \textbf{Others' emotional response}: People interacting with the patient became unhappy themselves, unable to understand why the patient appeared unengaged or unhappy

    \item \textbf{Misattribution to contempt}: The lack of positive emotional expression was often interpreted as \textbf{contempt}---as if the patient looked down on others or found them unworthy of engagement

    \item \textbf{Reality versus perception}: The patient was not feeling contempt but rather experiencing profound exhaustion and pain. However, to observers lacking this context, flat affect combined with apparent disengagement reads as disdain or superiority

    \item \textbf{Damage to relationships}: This misinterpretation created barriers in professional and personal relationships. Colleagues and acquaintances felt rejected or judged when the actual issue was metabolic incapacity to generate appropriate social signals

    \item \textbf{Inability to explain}: Without understanding the physiological basis, the patient could not effectively communicate ``I'm not contemptuous, I'm exhausted and in pain''---especially when the exhaustion itself impairs the cognitive and emotional resources needed for such explanations

    \item \textbf{Vicious cycle}: Others' negative reactions (hurt, defensiveness, withdrawal) made social interactions even more stressful and energy-draining, further reducing the patient's capacity to engage
\end{itemize}

\textbf{Clinical Note:} This pattern---flat affect due to energy conservation being misinterpreted as contempt, coldness, or disinterest---is likely common in ME/CFS but rarely documented. It represents a significant source of social disability beyond the direct metabolic symptoms. Patients are blamed for ``attitude problems'' when the actual issue is neurometabolic failure to generate expected social signals.

\subsubsection{Progressive Vision Impairment}
\label{subsec:personal-vision}

\paragraph{Formal Diagnosis.}
Progressive presbyopia with baseline hypermetropia (farsightedness).

\paragraph{Prescription History.}
Formal eye examination on 10 August 2022:
\begin{itemize}
    \item \textbf{Left eye}: +0.75 SPH (distance), +1.5 ADD (near)
    \item \textbf{Right eye}: +1.0 SPH (distance), +1.75 ADD (near)
    \item \textbf{Lens type}: Progressive/multifocal lenses
\end{itemize}

\paragraph{Clinical History and Progression.}
Rapid onset of presbyopia-like vision changes beginning around 2021:
\begin{itemize}
    \item Age at onset: Mid-30s to early 40s (approximately age 40; younger than typical presbyopia onset at 45+)
    \item Progressive near-vision blur requiring reading glasses
    \item \textbf{Current status (2026)}: Prescription likely outdated due to rapid progression
    \begin{itemize}
        \item Patient estimates current need at $\sim$1.5 diopters left, $\sim$1.75 right (may be higher)
        \item Continually needs to hold reading material further away
        \item Rapid worsening over past 5 years suggests metabolic rather than purely age-related cause
    \end{itemize}
    \item \textbf{Energy-dependent variation}: Vision quality fluctuates with energy levels
    \begin{itemize}
        \item Better focus and clarity on higher-energy days
        \item Blurrier, more difficult accommodation on low-energy days
        \item Motivation to focus depends on energy level
        \item Suggests metabolic/energy-dependent component rather than purely structural
    \end{itemize}
    \item One small diffuse floater in right eye (intermittent; possibly benign, but warrants monitoring)
\end{itemize}

\paragraph{Pathophysiological Hypothesis.}
The energy-dependent variation in vision suggests ciliary muscle dysfunction related to metabolic impairment:
\begin{itemize}
    \item \textbf{Ciliary muscle fatigue}: The ciliary muscles control lens accommodation (focusing). Like other muscles, they require ATP for contraction and relaxation.
    \item \textbf{Mitochondrial dysfunction}: When systemic ATP production is impaired, small muscles like the ciliary body may be unable to sustain focus, particularly for near vision (which requires sustained contraction).
    \item \textbf{Day-to-day variation}: Vision quality tracking with energy levels supports metabolic hypothesis rather than fixed structural changes alone.
\end{itemize}

\paragraph{Clinical Significance.}
Rapid progression of presbyopia at a relatively young age (onset $\sim$40 years old with significant worsening by age 45) suggests a metabolic or mitochondrial basis rather than normal aging. This finding adds to the evidence of widespread metabolic dysfunction affecting even small muscle groups. If mitochondrial support improves, vision accommodation may partially improve, though structural presbyopic changes (if present) would not reverse.

\subsubsection{Progressive Hearing Loss}
\label{subsec:personal-hearingloss}

\paragraph{Formal Diagnosis.}
\textbf{Hypoacousie neurosensorielle bilatérale} (Bilateral sensorineural hearing loss), diagnosed 29 August 2024 at Vivalia Arlon.

\paragraph{Audiogram Results.}
\begin{itemize}
    \item \textbf{Right ear}: Normal hearing up to 1000~Hz, then progressive high-frequency loss (drops to $-70$~dB at 8000~Hz)
    \item \textbf{Left ear}: Mild loss starting at 500~Hz ($\sim$20--30~dB), worsening in high frequencies ($-70$~dB at 8000~Hz)
    \item \textbf{Pattern}: High-frequency sensorineural hearing loss, bilateral
\end{itemize}

\paragraph{Clinical Examination.}
Physical examination was normal: tympan bilateral, oropharynx, vocal cords, and rhinopharynx showed no abnormalities.

\paragraph{Recommended Treatment.}
\begin{itemize}
    \item Audioprothèse (hearing aid) consultation
    \item Vocal audiogram in noise
    \item \textbf{Status}: No remediation applied yet (as of January 2026)
\end{itemize}

\paragraph{Clinical Significance for ME/CFS.}
Sensorineural hearing loss is common in ME/CFS patients and likely shares mitochondrial and oxidative stress mechanisms with the progressive vision problems documented above. The inner ear cochlear hair cells are among the most energy-demanding cells in the body, with mitochondrial density second only to brain tissue. These specialized sensory cells require exceptionally high ATP production to maintain the electrochemical gradients necessary for sound transduction.

Progressive high-frequency loss is consistent with mitochondrial dysfunction affecting these ATP-dependent sensory cells. The bilateral, progressive nature of the hearing loss, combined with the energy-dependent variability observed in vision, strongly suggests systemic mitochondrial dysfunction as a unifying mechanism affecting multiple high-energy-demand sensory systems.

\paragraph{Therapeutic Implications.}
\begin{itemize}
    \item Mitochondrial support (CoQ10, riboflavin, Acetyl-L-Carnitine) may slow progression
    \item Antioxidants (taurine, N-acetylcysteine) may protect remaining cochlear hair cells from oxidative damage
    \item Monitor progression as a biomarker for treatment efficacy
    \item Consider hearing protection strategies to prevent further damage
\end{itemize}

\subsubsection{Migraines}
\label{subsec:personal-migraines}

Recurring migraines with the following characteristics:
\begin{itemize}
    \item Frequently triggered after periods of exertion
    \item Associated with the oxidative stress from lactic acid surges
    \item May be exacerbated by medications causing vasoconstriction (e.g., methylphenidate, modafinil)
\end{itemize}

\paragraph{Pathophysiological Basis.}
Migraines in ME/CFS are frequently triggered by a ``metabolic threshold'' event. When the brain's energy demand exceeds supply, it triggers a wave of neurological inflammation. The neuroinflammation caused by lactic acid surges creates conditions favorable for migraine initiation.

Riboflavin (vitamin B2) at 400\,mg/day is particularly relevant because it is a precursor to FAD (flavin adenine dinucleotide), a vital electron carrier in the mitochondrial energy chain. It typically requires 4--12 weeks of consistent use to reduce migraine frequency.

\subsubsection{Post-Exertional Malaise (PEM)}
\label{subsec:personal-pem}

\paragraph{Clinical History.}
Post-exertional malaise has been present for \textbf{decades}, though its severity and characteristics have evolved over time. This is not a recent symptom that appeared after 2017 burnout---it has been a lifelong pattern that has progressively worsened.

\paragraph{Early Manifestations (Working Years).}
\begin{itemize}
    \item Required full-day recovery sleep (Saturday mornings + afternoons) to function for evening activities
    \item Mid-exertion energy collapse during table tennis matches leading to performance deterioration
    \item Extreme compensatory strategies to maintain employment (weekend crash-and-recover cycles)
\end{itemize}

\paragraph{Exercise Intolerance Progression.}
The loss of exercise tolerance demonstrates disease progression:
\begin{itemize}
    \item \textbf{Historical (date uncertain):} Could swim 1\,km daily
    \begin{itemize}
        \item Physical fitness improved (better table tennis performance)
        \item Mental fog and daytime sleepiness persisted (not cured by exercise)
        \item Still required weekend crash-recovery cycles
        \item Exercise provided \textbf{some benefit} despite underlying metabolic dysfunction
    \end{itemize}
    \item \textbf{Recent (2025/2026):} Attempted same swimming regimen for 4--5 months
    \begin{itemize}
        \item Result: \textbf{Constant mental fog} (cognitive PEM worsened)
        \item Functional consequence: Work underperformance leading to job loss
        \item Demonstrates transition from ``exercise provides net benefit despite symptoms'' to ``exercise causes disabling cognitive dysfunction that eliminates function''
    \end{itemize}
\end{itemize}

\paragraph{Current Pattern.}
\begin{itemize}
    \item PEM remains present and activity-limiting
    \item Crashes can be physical (muscle fatigue, cramps) or cognitive (brain fog, processing impairment)
    \item Delayed onset: crashes may occur hours to days after exertion
    \item Recovery unpredictable, ranging from days to weeks
\end{itemize}

\paragraph{Pathophysiological Basis.}
PEM represents the body's inability to meet energy demands beyond minimal baseline. When mitochondrial ATP production is impaired, any activity that exceeds this ceiling triggers a systemic energy crisis. The delayed nature of crashes reflects the time it takes for cellular energy deficits to accumulate and trigger inflammatory responses.

% Continue with musculoskeletal and other symptoms from the original file...
