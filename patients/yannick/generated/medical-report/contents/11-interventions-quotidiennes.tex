\section{Interventions quotidiennes}

\subsection{Cascade PEM: Points d'intervention temporels}

Basé sur modèle événementiel de malaise post-effort (EPC PEM Cascade Model, certitude 0.7), avec corrélation aux événements patient récents:

\subsubsection{Fenêtres temporelles et opportunités d'intervention}

\begin{enumerate}
\item \textbf{E1 → E2: Activité → Décalage métabolique (30min--4h)}
    \begin{itemize}
    \item \textbf{Patient Feb 12 11:15--11:45}: 30min repassage debout → faiblesse, pouls élevé (activation E1→E2)
    \item \textbf{Prévention primaire}: Surveillance FC <97 bpm (0,55 × [220-âge]); pacing basé FC
    \item \textbf{Biomarqueurs}: Lactate >2,0 mmol/L, marqueurs ROS élevés (95\% probabilité chez patients EM/SFC)
    \item \textbf{Intervention}: ARRÊT IMMÉDIAT activité si FC dépasse seuil; repos horizontal obligatoire
    \end{itemize}

\item \textbf{E2 → E3: Décalage métabolique → Activation immunitaire (4--24h)}
    \begin{itemize}
    \item \textbf{Patient Feb 12 après-midi/soir}: Sieste 1h20 non réparatrice → probablement transition E2→E3
    \item \textbf{Fenêtre critique anti-inflammatoire}: 4--24h post-activité
    \item \textbf{Biomarqueurs}: Cytokines pro-inflammatoires (IL-1$\alpha$, IL-8, IFN-$\gamma$, CXCL1)
    \item \textbf{Interventions possibles}:
        \begin{itemize}
        \item Quercétine 1000mg (stabilisateur mastocytes, anti-inflammatoire naturel)
        \item Famotidine 20mg BID (bloqueur H2, effets anti-inflammatoires)
        \item LDN dose timing optimisé (modulation immunitaire)
        \item Repos strict horizontal (prévenir progression cascade)
        \end{itemize}
    \item \textbf{Probabilité activation}: 87\% chez patients <3 ans maladie; réduite >3 ans
    \end{itemize}

\item \textbf{E3 → E4: Activation immunitaire → Pic symptomatique (12--48h)}
    \begin{itemize}
    \item \textbf{Patient Feb 13 midi}: Faiblesse après préparation déjeuner → confirmation E4 (Jour 2 post-crash)
    \item \textbf{Durée médiane jusqu'à pic}: 48h post-activité déclenchante
    \item \textbf{Manifestation symptômes}: 100\% probabilité une fois activation immunitaire établie
    \item \textbf{Gestion symptômes}:
        \begin{itemize}
        \item Repos horizontal strict (position assise NON réparatrice pour ce patient)
        \item Hydratation + électrolytes (expansion volume sanguin)
        \item Aucune activité debout (seuil <30min déjà dépassé)
        \end{itemize}
    \end{itemize}

\item \textbf{E4 → E5a/E5b: Pic → Récupération vs Chronification (7--21 jours)}
    \begin{itemize}
    \item \textbf{CRITIQUE - Patient actuellement à ce stade (Feb 13)}
    \item \textbf{Récupération complète (E5a)}: 40\% probabilité SI repos $\geq$7 jours ininterrompu
    \item \textbf{Activation chronique (E5b)}: 60\% probabilité SI repos <7j OU nouveaux déclencheurs
    \item \textbf{Impact chronicité}: Réduction baseline 5--10\% fonction; ATP baseline -5\%
    \item \textbf{RECOMMANDATION URGENTE}:
        \begin{itemize}
        \item \textbf{Repos $\geq$14 jours recommandé} (dépasse minimum 7j, augmente probabilité E5a >60\%)
        \item AUCUNE activité debout >10min
        \item Reprise activité graduelle SEULEMENT après normalisation symptômes
        \item Éviter absolument nouveaux déclencheurs pendant fenêtre récupération
        \end{itemize}
    \end{itemize}
\end{enumerate}

\subsubsection{Boucle rétroaction chronique (BR1)}

\textbf{Pattern préoccupant identifié}: Patient montre épisodes PEM récurrents (11 fév, 12 fév, 13 fév) suggérant entrée possible boucle chronique immune-métabolique.

\textbf{Caractéristiques boucle}:
\begin{itemize}
\item Chaque cycle: ATP baseline × 0,95 (perte permanente 5\%)
\item Chaque cycle: Difficulté récupération × 1,1 (10\% plus difficile récupérer)
\item Convergence: ATP baseline → minimum critique (déclin progressif)
\item \textbf{Probabilité alimentation boucle}: 60\% si repos insuffisant
\end{itemize}

\textbf{Conditions rupture boucle}:
\begin{enumerate}
\item \textbf{Repos >14 jours ininterrompu} (permet réparation complète) - PRIORITÉ ABSOLUE
\item \textbf{Intervention anti-inflammatoire} (brise étape activation immunitaire) - protocole SAMA
\item \textbf{Éducation pacing} (prévenir re-déclenchement) - surveillance FC strict
\item \textbf{Résolution spontanée} (<10\% probabilité, mécanisme peu clair)
\end{enumerate}

\subsection{Support métabolisme énergétique}

\begin{enumerate}
\item \textbf{Acetyl-L-Carnitine 1000mg (matin)}
    \begin{itemize}
    \item \textbf{Fonction}: Ouvre ``navette carnitine'' pour transport graisses à longue chaîne dans mitochondries
    \item \textbf{Justification}: Aborde cause racine dysfonction métabolisme graisse (``running on empty'')
    \item \textbf{Timeline}: 4--6 semaines effet initial; 3--6 mois bénéfice maximum
    \item \textbf{Forme acétyl}: Traverse barrière hémato-encéphalique pour support cognitif
    \item \textbf{Preuves}: Correction racine vs bypass temporaire MCT oil
    \end{itemize}

\item \textbf{CoQ10 Ubiquinol 100--200mg (avec graisse alimentaire)}
    \begin{itemize}
    \item \textbf{Fonction}: ``Bougie d'allumage'' chaîne transport électrons; cofacteur essentiel synthèse ATP
    \item \textbf{Justification}: Support machinerie production énergie mitochondriale
    \item \textbf{CRITIQUE}: \textcolor{red}{Fat-soluble - DOIT prendre avec graisse alimentaire sinon absorption <10\%}
    \item \textbf{Forme ubiquinol}: Active, réduite (meilleure absorption qu'ubiquinone)
    \end{itemize}

\item \textbf{Riboflavin (B2) 400mg (dîner avec graisse)}
    \begin{itemize}
    \item \textbf{Fonction triple}:
        \begin{itemize}
        \item Précurseur FAD (flavine adénine dinucléotide) - essentiel bêta-oxydation (combustion graisses)
        \item Cofacteur critique chaîne transport électrons
        \item Prévention migraines (prouvé à 400mg/jour)
        \end{itemize}
    \item \textbf{Justification}: Support métabolisme graisses (synergie acetyl-L-carnitine) + prévention migraines déclenchées vasoconstriction stimulant
    \item \textbf{Timeline}: 4--12 semaines pour prévention migraines
    \item \textbf{CRITIQUE}: \textcolor{red}{Fat-soluble - prendre dîner contenant graisse}
    \end{itemize}

\item \textbf{MCT Oil 1 càs (matin) + 1 càc (coucher)}
    \begin{itemize}
    \item \textbf{Fonction}: Triglycérides chaîne moyenne (C8-C10) contournent navette carnitine cassée
    \item \textbf{Justification URGENCE}: \textbf{BYPASS ÉNERGÉTIQUE IMMÉDIAT} pendant réparation acetyl-L-carnitine
    \item \textbf{Mécanisme}: Va direct au foie pour production énergie; NE NÉ\-CES\-SITE PAS navette carnitine
    \item \textbf{Support absorption}: Aide absorption vitamines fat-soluble (D3, CoQ10, B2)
    \item \textbf{Timing}: 1 càc avant coucher pour support ATP nocturne (prévention crampes)
    \item \textbf{CRITIQUE}: \textcolor{red}{Commencer 1 càc, augmenter lentement sur 1--2 semaines (éviter diarrhée)}
    \item \textbf{Note}: Ceci est \textbf{PAS huile coco} - huile MCT est pure C8/C10 concentrée uniquement
    \end{itemize}

\item \textbf{D-Ribose 5g (coucher + matin pour 10g/jour total)}
    \begin{itemize}
    \item \textbf{Fonction}: Sucre simple qui est brique construction directe molécule ATP
    \item \textbf{Justification}: Reconstitue réserves ATP cellulaires rapidement; contourne voies métaboliques complexes
    \item \textbf{Ciblage}: Déplétion ATP nocturne (pendant jeûne nuit, corps devrait brûler graisse - navette bloquée → ATP s'épuise)
    \item \textbf{Effet}: ATP faible cause crampes nocturnes et sommeil non réparateur
    \item \textbf{Timeline}: Certains notent effet en jours; évaluer à 2 semaines pour réduction crampes
    \end{itemize}
\end{enumerate}

\subsection{Support malabsorption graisses (déficience chronique vitamine D suggère ceci)}

\begin{enumerate}
\item \textbf{MetaDigest TOTAL (Metagenics) - avant repas}
    \begin{itemize}
    \item \textbf{Formule enzyme complète}: lipase (décompose graisses), protéase (protéines), amylase (glucides), cellulase (fibres), lactase (laitier)
    \item \textbf{Justification}: Pancréas nécessite énergie pour produire enzymes; dysfonction mitochondriale réduit production enzyme → maldigestion/malabsorption
    \item \textbf{Évidence}: Déficience chronique vitamine D malgré supplémentation suggère fortement malabsorption graisses
    \item \textbf{Timing}: Prendre immédiatement avant ou avec première bouchée repas contenant vitamines fat-soluble
    \item \textbf{Synergy avec MCT oil}: MCT + enzymes assurent vitamines fat-soluble absorbent réellement
    \end{itemize}
\end{enumerate}

\subsection{Protocole électrolytes (pour support autonome)}

\begin{enumerate}
\item \textbf{Solution électrolyte custom 250mL, 2×/jour}
    \begin{itemize}
    \item \textbf{Sodium}: Expanse volume sanguin (effet ``éponge'' tirant eau dans circulation)
    \item \textbf{Potassium}: Permet relaxation musculaire; maintient charge électrique cellulaire
    \item \textbf{Glucose}: Améliore absorption sodium via transporteur SGLT1; fournit énergie rapide quand combustion graisses altérée
    \item \textbf{Justification EM/SFC}: Implique typiquement faible volume sanguin et intolérance orthostatique
    \item \textbf{Dose après-midi}: Nettoie acide lactique accumulé depuis activités matinales
    \item \textbf{Formule}: 7g mélange sec (sucre + sel Jozo faible sodium + sel table) dans 250mL eau
    \item \textbf{Alternative}: 4,3g par dose (version faible sucre)
    \end{itemize}
\end{enumerate}

\subsection{Optimisation timing magnésium}

\begin{enumerate}
\item \textbf{Magnésium Glycinate 300--400mg (coucher)}
    \begin{itemize}
    \item \textbf{Fonction double}:
        \begin{itemize}
        \item ``Interrupteur off'' pour contraction musculaire - permet relaxation
        \item Cofacteur critique pour 300+ réactions enzymatiques incluant synthèse ATP
        \end{itemize}
    \item \textbf{Timing coucher}: Cible crampes nocturnes quand ATP est au plus bas
    \item \textbf{Forme glycinate}: Effet pH minimal (safe coucher, 6--8h après stimulants)
    \item \textbf{CRITIQUE}: \textcolor{red}{Jamais utiliser magnésium carbonate/oxide - cause dose dumping méthylphénidate}
    \end{itemize}
\end{enumerate}

\label{app:daily-journal}

Cette annexe sert d'enregistrement longitudinal des symptômes, médicaments et évolution de la maladie. La documentation régulière permet la reconnaissance de patterns, soutient les consultations cliniques, et fournit des preuves pour les ajustements de traitement.

\subsection{Modèle d'entrée de journal}
\label{sec:journal-template}

Chaque entrée quotidienne devrait systématiquement capturer les symptômes, médicaments et observations pour permettre la reconnaissance de patterns au fil du temps. Utilisez l'échelle de sévérité du Tableau~\ref{tab:severity-scale} pour toutes les évaluations de symptômes.

\subsubsection{Éléments quotidiens requis}

\paragraph{Sommeil et énergie.}
\begin{itemize}
    \item \textbf{Sommeil}: Heures dormies, qualité du sommeil (réparateur/non réparateur), interruptions
    \item \textbf{Niveau d'énergie global}: Échelle 0--10 (évaluation subjective)
    \item \textbf{État matinal}: Comment vous vous sentiez au réveil
\end{itemize}

\paragraph{Symptômes principaux (Évaluer 0--10).}
\begin{itemize}
    \item \textbf{Fatigue}: Niveau d'épuisement physique
    \item \textbf{Brouillard cérébral}: Clarté mentale/fonction cognitive (score plus bas = pensée plus claire)
    \item \textbf{Céphalée/Migraine}: Sévérité (0 si absent, noter localisation/type si présent)
    \item \textbf{Faim d'air}: Inconfort respiratoire/dyspnée
    \item \textbf{Épuisement des jambes}: Fatigue/lourdeur des membres inférieurs
    \item \textbf{Douleur articulaire}: Spécifier localisations (genoux/épaules/poignets/chevilles) et sévérité
    \item \textbf{Crampes musculaires}: Fréquence et sévérité
    \item \textbf{Autres symptômes}: Tous symptômes additionnels (nausée, vertiges, problèmes sensoriels, etc.)
\end{itemize}

\paragraph{Médicaments et suppléments (Liste quotidienne).}
\begin{itemize}
    \item \textbf{LDN}: Dose et heure de prise
    \item \textbf{Stimulants}: Doses et timing Rilatine/Provigil (noter nombre total de pilules)
    \item \textbf{Support mitochondrial}: Urolithin A, CoQ10, Riboflavine B2
    \item \textbf{Vitamines}: Vitamine D (si jour de dose hebdomadaire), Vitamine C, B-complexe
    \item \textbf{Minéraux}: Glycinate de magnésium, fer
    \item \textbf{Électrolytes}: Solution personnalisée (nombre de portions)
    \item \textbf{Support digestif}: MetaDigest (quand démarré), huile MCT (quand démarrée)
    \item \textbf{Autres}: Tous suppléments ou médicaments additionnels
\end{itemize}

\paragraph{Activités et effort.}
\begin{itemize}
    \item \textbf{Activités physiques}: Type, durée, difficulté perçue
    \item \textbf{Activités cognitives}: Travail mental, temps d'écran, demandes de concentration
    \item \textbf{Données fréquence cardiaque}: FC maximale, temps passé au-dessus du seuil, FC au repos
    \item \textbf{Respect du pacing}: Êtes-vous resté dans les limites sécuritaires?
\end{itemize}

\paragraph{Effets perçus et observations.}
\begin{itemize}
    \item \textbf{Effets des suppléments}: Changements notables après prise de nouveaux suppléments (positifs ou négatifs)
    \item \textbf{Effets L-Carnitine} (quand démarrée): Changements d'énergie, clarté cognitive, symptômes musculaires, effets digestifs
    \item \textbf{Fonction sensorielle}: Clarté de vision aujourd'hui (0--10), clarté d'audition (si changements notés)
    \item \textbf{Corrélation sensorielle-énergie}: La vision/audition semble-t-elle pire les jours de faible énergie?
    \item \textbf{Déclencheurs identifiés}: Activités, aliments, stresseurs qui ont aggravé les symptômes
    \item \textbf{Interventions utiles}: Ce qui a apporté un soulagement (repos, hydratation, suppléments spécifiques)
    \item \textbf{Patterns notables}: Connexions entre symptômes, timing, ou interventions
    \item \textbf{Questions pour le médecin}: Observations à discuter au prochain rendez-vous
\end{itemize}

\subsubsection{Échelle d'évaluation de sévérité}
\label{subsec:severity-scale}

\begin{table}[htbp]
\centering
\caption{Échelle de sévérité des symptômes}
\label{tab:severity-scale}
\begin{tabular}{cl}
\toprule
\textbf{Score} & \textbf{Description} \\
\midrule
0 & Absent \\
1--2 & Léger: notable mais ne limite pas \\
3--4 & Modéré: affecte la fonction, gérable \\
5--6 & Significatif: limite substantiellement l'activité \\
7--8 & Sévère: fonction minimale possible \\
9--10 & Extrême: incapacitant \\
\bottomrule
\end{tabular}
\end{table}

%------------------------------------------------------------------------------
% LES ENTRÉES DE JOURNAL COMMENCENT ICI
%------------------------------------------------------------------------------

\subsection{Janvier 2026}
\label{sec:journal-2026-01}

\subsubsection{2026-01-20}

\begin{description}
    \item[Énergie:] /10
    \item[Sommeil:] heures, réparateur: Oui/Non
    \item[Symptômes:]
    \begin{itemize}
        \item Fatigue: /10
        \item Brouillard cérébral: /10
        \item Faim d'air: /10
        \item Épuisement des jambes: /10
        \item Douleur articulaire (genoux/épaules/poignets): /10
        \item Crampes musculaires: /10
        \item Migraine: Oui/Non
    \end{itemize}
    \item[Médicaments:]
    \begin{itemize}
        \item Médicaments habituels: Oui
        \item Suppléments habituels: Oui
    \end{itemize}
    \item[Activités:]
    \item[Données fréquence cardiaque:] FC max: , temps au-dessus du seuil:
    \item[Observations:] Pris 250\,mL eau + 10\,mL grenadine + mélange sel/sucre (solution de réhydratation orale).
\end{description}

\subsubsection{2026-01-21}

\begin{description}
    \item[Sommeil et énergie:]
    \begin{itemize}
        \item Sommeil: \underline{\hspace{2cm}} heures, qualité: \underline{\hspace{3cm}} (réparateur/non réparateur)
        \item Énergie globale: \underline{\hspace{1cm}}/10
        \item État matinal: \underline{\hspace{6cm}}
    \end{itemize}

    \item[Symptômes (échelle 0--10):]
    \begin{itemize}
        \item Fatigue: \underline{\hspace{1cm}}/10
        \item Brouillard cérébral: \underline{\hspace{1cm}}/10
        \item Céphalée/Migraine: \underline{\hspace{1cm}}/10 (localisation/type: \underline{\hspace{3cm}})
        \item Faim d'air: \underline{\hspace{1cm}}/10
        \item Épuisement des jambes: \underline{\hspace{1cm}}/10
        \item Douleur articulaire: \underline{\hspace{1cm}}/10 (localisations: \underline{\hspace{4cm}})
        \item Crampes musculaires: \underline{\hspace{1cm}}/10
        \item Autres: \underline{\hspace{8cm}}
    \end{itemize}

    \item[Médicaments et suppléments:]
    \begin{itemize}
        \item LDN 3\,mg: $\square$ (heure: \underline{\hspace{2cm}})
        \item Rilatine MR 30\,mg: $\square$ $\square$ (heures: \underline{\hspace{3cm}})
        \item Provigil 100\,mg: $\square$ $\square$ (heures: \underline{\hspace{3cm}})
        \item Total pilules stimulant aujourd'hui: \underline{\hspace{1cm}}/3 max
        \item Urolithin A + NAD+: $\square$ (2 capsules)
        \item CoQ10 ubiquinol: $\square$ (1--2 capsules)
        \item \textbf{NOUVEAU: Riboflavine B2 400\,mg}: $\boxtimes$ \textbf{(DÉMARRÉ AUJOURD'HUI)}
        \item Vitamine C 500\,mg: $\square$
        \item B-complexe (BEFACT FORTE): $\square$
        \item \textbf{NOUVEAU: Glycinate de magnésium (Metagenics)}: $\boxtimes$ \textbf{(DÉMARRÉ AUJOURD'HUI - remplace Magnecaps)}
        \item Fer (FerroDyn FORTE): $\square$
        \item Vitamine D 25000\,U.I.: $\square$ (hebdomadaire - si applicable)
        \item Solution électrolytes: \underline{\hspace{1cm}} portions
        \item Autres: \underline{\hspace{6cm}}
    \end{itemize}

    \item[Activités et effort:]
    \begin{itemize}
        \item Physique: \underline{\hspace{8cm}}
        \item Cognitif: \underline{\hspace{8cm}}
        \item Fréquence cardiaque: Max \underline{\hspace{2cm}} bpm, temps au-dessus du seuil: \underline{\hspace{2cm}}
        \item Respect du pacing: $\square$ Bon $\square$ Limites dépassées
    \end{itemize}

    \item[Effets perçus et observations:]
    \begin{itemize}
        \item Effets nouveaux suppléments (Riboflavine/Mg): \underline{\hspace{6cm}}
        \item Déclencheurs identifiés: \underline{\hspace{6cm}}
        \item Interventions utiles: \underline{\hspace{6cm}}
        \item Patterns notables: \underline{\hspace{6cm}}
        \item Questions pour le médecin: \underline{\hspace{6cm}}
    \end{itemize}
\end{description}

%------------------------------------------------------------------------------
% MODÈLE VIERGE POUR JOURS FUTURS
%------------------------------------------------------------------------------

\subsubsection{AAAA-MM-JJ} % Copier ce modèle pour nouvelles entrées

\begin{description}
    \item[Sommeil et énergie:]
    \begin{itemize}
        \item Sommeil: \underline{\hspace{2cm}} heures, qualité: \underline{\hspace{3cm}}
        \item Énergie globale: \underline{\hspace{1cm}}/10
        \item État matinal: \underline{\hspace{6cm}}
    \end{itemize}

    \item[Symptômes (0--10):]
    \begin{itemize}
        \item Fatigue: \underline{\hspace{1cm}}/10
        \item Brouillard cérébral: \underline{\hspace{1cm}}/10
        \item Céphalée/Migraine: \underline{\hspace{1cm}}/10 (localisation: \underline{\hspace{3cm}})
        \item Faim d'air: \underline{\hspace{1cm}}/10
        \item Épuisement des jambes: \underline{\hspace{1cm}}/10
        \item Douleur articulaire: \underline{\hspace{1cm}}/10 (localisations: \underline{\hspace{4cm}})
        \item Crampes musculaires: \underline{\hspace{1cm}}/10
        \item Autres: \underline{\hspace{8cm}}
    \end{itemize}

    \item[Médicaments/Suppléments:]
    \begin{itemize}
        \item LDN 3\,mg: $\square$ | Rilatine: $\square$ $\square$ | Provigil: $\square$ $\square$ (total: \underline{\hspace{1cm}}/3)
        \item Urolithin A: $\square$ | CoQ10: $\square$ | Riboflavine B2: $\square$
        \item Vit C: $\square$ | B-complexe: $\square$ | Mg glycinate: $\square$ | Fer: $\square$ | Vit D: $\square$
        \item Électrolytes: \underline{\hspace{1cm}}$\times$ | MetaDigest: $\square$ | Huile MCT: $\square$
        \item Autres: \underline{\hspace{6cm}}
    \end{itemize}

    \item[Activités:] \underline{\hspace{8cm}}

    \item[Fréquence cardiaque:] Max \underline{\hspace{2cm}} bpm, temps seuil: \underline{\hspace{2cm}}

    \item[Observations:] \underline{\hspace{10cm}}
\end{description}

% Ajouter nouveaux mois comme sections:
% \subsection{Février 2026}
% \label{sec:journal-2026-02}
% ...
