\section{RÉSUMÉ EXÉCUTIF}

\subsection{Préoccupations principales nécessitant une attention urgente}

\begin{enumerate}
\item \textbf{Épisodes récurrents de dysrégulation autonome} (10-13 février 2026): Multiples épisodes de faiblesse généralisée, tremblements ressemblant à une hypoglycémie, pouls élevé et intolérance posturale survenant lors des transitions sommeil-éveil et après une activité debout minimale (30 minutes).

\item \textbf{Seuil d'activité sévèrement réduit}: Les activités debout aussi brèves que 30 minutes (repassage, cuisine, courses) déclenchent des crashes autonomes et un malaise post-effort (PEM), représentant une détérioration fonctionnelle significative.

\item \textbf{Épisode autonome pendant conduite}: Un épisode de dysrégulation autonome de 50 minutes s'est produit pendant que le patient conduisait le 11 février 2026, impliquant une faiblesse progressive suivie de tremblements. \textbf{Note patient:} Faiblesse remarquée mais pas de risque d'évanouissement ou d'endormissement; conduite tolérée même sur trajets longs.

\item \textbf{Sommeil non réparateur}: Les siestes de l'après-midi de 1 à 3 heures ne parviennent pas à restaurer l'énergie; le sommeil nocturne est fragmenté.

\item \textbf{Dissociation cognitive-physique}: Tout au long de ces épisodes, la fonction cognitive est relativement préservée (``la tête va bien'') tandis que les symptômes physiques sont sévères, suggérant une défaillance principalement autonome/périphérique plutôt qu'une défaillance du système nerveux central.
\end{enumerate}

\subsection{Schéma clinique clé}

Le patient démontre un schéma cohérent sur plusieurs jours:
\begin{itemize}
\item Ligne de base cognitive matinale bonne se détériorant rapidement avec toute activité debout
\item Instabilité autonome se manifestant par un pouls élevé, faiblesse, tremblements et symptômes pseudo-hypoglycémiques
\item Le repos n'est pas réparateur (les siestes ne reconstituent pas les réserves d'énergie)
\item Fonction cognitive relativement préservée même pendant les épisodes physiques sévères
\item Seuil d'activité sévèrement réduit à environ 30 minutes en position debout
\end{itemize}

\subsection{Recommandations immédiates (résumé)}

\begin{enumerate}
\item \textbf{Urgent}: Tests formels de fonction autonome (test d'inclinaison, moniteur Holter, signes vitaux orthostatiques)
\item \textbf{À considérer}: Support autonome pharmacologique (ivabradine, propranolol faible dose, midodrine ou fludrocortisone)
\item \textbf{Optimiser}: Dosage actuel de LDN (stabiliser à 3mg ou 4mg plutôt qu'alterner)
\item \textbf{Mettre en œuvre}: Rythme d'activité strict avec surveillance de la fréquence cardiaque (limite FC cible: 97 bpm basé sur (220-44) × 0,55)
\item \textbf{Sécurité}: Prudence recommandée lors de conduite pendant épisodes de faiblesse; patient rapporte tolérance conduite sans risque évanouissement/endormissement
\end{enumerate}
