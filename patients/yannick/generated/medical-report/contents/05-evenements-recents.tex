
\section{Événements récents}

\subsection{Chronologie jour par jour}

\subsubsection{8 février (samedi) -- Activité malgré la douleur}
\begin{itemize}
\item Énergie: 4/10, Douleur: 6/10 (articulations et hanches)
\item Activité: Travaux ménagers poursuivis malgré la douleur
\item Résultat: Enveloppe énergétique sûre dépassée
\end{itemize}

\subsubsection{9 février (dimanche) -- Crash PEM sévère}
\begin{itemize}
\item Énergie: 1/10, Sévérité PEM: 8/10
\item Durée: $\sim$7 heures
\item Déclencheur: Travaux ménagers du samedi
\item Résolution: À un état acceptable en après-midi/soirée
\end{itemize}

\subsubsection{10 février (lundi/mardi) -- Pattern d'utilisation Ritalin et rebond}
\begin{itemize}
\item \textbf{Lundi}: Ritalin MR 30mg pris → excellente réponse (énergie 6/10, cognitif 8/10)
\item \textbf{Mardi}: Pas de Ritalin → rebond sévère: sommeil excessif (4-4,5h diurne), faiblesse, tremblements similaires à hypoglycémie, énergie 2/10, cognitif 3/10
\end{itemize}

\subsubsection{11 février (mercredi) -- ÉVÉNEMENT AUTONOME CRITIQUE}
\begin{itemize}
\item \textbf{Matin}: 1h20 courses → fatigué, douleur aux jambes
\item \textbf{14:50-15:00}: Réveil de sieste d'après-midi
\item \textbf{15:00-15:25} (Phase 1): Faiblesse généralisée pendant CONDUITE
\item \textbf{15:25-15:50} (Phase 2): Tremblements/secousses pendant CONDUITE
\item \textbf{15:50+} (Phase 3): Résolution, fonction cognitive OK, fatigue persiste
\item \textbf{Schéma}: Phases organisées de 25 minutes; préservation cognitive; spécifique autonome
\end{itemize}

\textbf{PROBLÈME DE SÉCURITÉ POTENTIEL}: 30 minutes de déficience autonome lors de l'utilisation d'un véhicule.

\subsubsection{12 février (jeudi) -- Crash déclenché par activité}
\begin{itemize}
\item \textbf{09:45}: Bon état cognitif, corps ``fragile''
\item \textbf{11:15-11:45}: 30 min debout/repassage → faiblesse, pouls élevé, sensation hypoglycémique
\item \textbf{Après-midi}: Sieste 1h20 → récupération incomplète
\item \textbf{Fin après-midi}: Deuxième 30 min repassage → à la limite de mal de tête et crash
\item LDN réduit à 3mg (de 4mg typique); Cétirizine ajoutée
\end{itemize}

\subsubsection{13 février (vendredi) -- Jour post-crash avec PEM confirmé}
\begin{itemize}
\item \textbf{Nuit}: Mauvais sommeil: réveil 04:30, impossible de se rendormir jusqu'à 05:30, réveil forcé 06:30
\item \textbf{Matin}: Fatigue généralisée depuis le réveil; cognitif: ``La tête va bien'' (préservée malgré fatigue physique)
\item \textbf{Matin}: Sieste $\sim$1h
\item \textbf{Midi}: Faiblesse après préparation déjeuner + manger avec enfant
\item \textbf{Après-midi}: Travail assis à l'ordinateur → fatigue; douleur auriculaire légère (otalgie); somnolence extrême (``je pourrais dormir pour l'éternité'')
\item \textbf{Pattern critique}: Faiblesse déclenchée par activité légère (préparation repas) MALGRÉ sieste matinale → confirme PEM actif, pas simple dette sommeil
\item \textbf{Progression symptômes}: Douleur auriculaire + somnolence extrême suggèrent PIC SYMPTOMATIQUE (E4 dans cascade PEM) ~28h post-déclencheur (12 fév 11:45 → 13 fév après-midi)
\item LDN retourné à 4mg; Cétirizine continuée; Ritalin non pris
\end{itemize}

\textbf{CONFIRMATION PEM AU PIC (E4)}:
\begin{itemize}
\item Faiblesse post-déjeuner + fatigue travail assis confirme PEM actif (Jour 2 post-crash du 12 février)
\item Douleur auriculaire peut indiquer activation immunitaire (cytokines IL-1$\beta$, TNF-$\alpha$ affectant trompe d'Eustache) OU réponse SAMA/histamine OU dysfonction autonome
\item Somnolence extrême caractéristique pic symptomatique: épuisement métabolique profond + cytokines somnogènes (IL-1$\beta$)
\item Timeline E1→E4: 28h (dans plage documentée 24-72h, médiane 48h)
\item \textbf{FENÊTRE CRITIQUE}: Prochains 7-14 jours déterminent récupération (E5a, 40-60\% probabilité si repos $\geq$14j) vs détérioration chronique (E5b, 60\% probabilité si repos <7j, réduction baseline permanente 5-10\%)
\end{itemize}

\subsubsection{14 février (samedi) -- Amélioration apparente trompeuse}
\begin{itemize}
\item \textbf{Sommeil}: 6,5h, qualité OK (amélioration vs 13 fév)
\item \textbf{Énergie subjective}: ``Généralement bonne journée'', ``OK'', ``pas de forte sensation de fatigue''
\item \textbf{Cognitif}: Préservé - pas de mal de tête, pas de brouillard mental
\item \textbf{Activité SUBSTANTIELLE}: >2h debout (courses + cuisine déjeuner >1h + coupe cheveux 1h), plusieurs sessions travail cognitif
\item \textbf{Symptômes}: Douleur articulaire genou droit (côté médial, intra-articulaire - distinct de douleur musculaire habituelle)
\item \textbf{Médicaments}: LDN 4mg, Cétirizine; Ritalin MR non pris
\item \textbf{Suppléments}: LCAR 1000mg, CoQ10 100mg, B2 400mg, NAD+ 2 caps, BEFACT FORTE, FerroDyn FORTE + Vit C
\item \textbf{Évaluation fin de journée}: Se sentait OK, aucun symptôme fort
\end{itemize}

\textbf{INTERPRÉTATION CRITIQUE -- FAUSSE IMPRESSION DE RÉCUPÉRATION}:
\begin{itemize}
\item Jour 3 post-crash (12 fév): Capacité d'activité substantielle + sensation OK $\neq$ récupération réelle
\item Pattern EM/SFC classique: Se sentir capable → dépasser enveloppe → crash retardé 24-48h
\item Charge activité 14 fév (>2h debout) DÉPASSE enveloppe sûre pendant fenêtre récupération
\item Activité pendant récupération crash primaire → risque crash secondaire
\end{itemize}

\subsubsection{15 février (dimanche) -- CRASH PEM RETARDÉ + Symptômes sinusaux}
\begin{itemize}
\item \textbf{Mal de tête}: SÉVÈRE, ``énorme mal de tête'', toute la journée
\item \textbf{Énergie}: Sévèrement réduite, ``tout était dur à faire''
\item \textbf{Évaluation subjective}: ``C'était du PEM'' (confirmé par patient)
\item \textbf{Symptômes sinusaux/auriculaires}:
    \begin{itemize}
    \item Nez bouché
    \item Douleur sinusale (``douleur et mauvaise sensation autour de cette zone'')
    \item Douleur oreille gauche diffuse
    \item Atteinte trompe d'Eustache (côté gauche)
    \end{itemize}
\item \textbf{Timeline}: 24h post-activités 14 fév → onset crash retardé CLASSIQUE
\item \textbf{Médicaments}: LDN 3mg, Cétirizine (mêmes suppléments que 14 fév)
\end{itemize}

\textbf{PEM RETARDÉ CONFIRMÉ (E4 - Pic secondaire)}:
\begin{itemize}
\item Timeline: 14 fév activités → 15 fév crash (délai 24h) = pattern EM/SFC classique
\item Preuve: Patient se sentait ``OK'' 14 fév soir → ``énorme mal de tête'' + ``tout était dur'' 15 fév
\item Composante double: PEM (``c'était du PEM'') + inflammation sinusale/auriculaire
\item Hypothèses mécanisme:
    \begin{itemize}
    \item Sinusite/infection respiratoire haute (nez bouché, douleur sinusale, trompe d'Eustache)
    \item OU inflammation activée par crash (cytokines IL-1$\beta$, TNF-$\alpha$)
    \item OU vulnérabilité immunitaire (crash primaire → système immunitaire affaibli → infection)
    \item PLUS PROBABLE: Combinaison - PEM + inflammation sinusale/infection
    \end{itemize}
\item \textbf{Pattern douleur auriculaire récurrente}: 13 fév otalgie (résolu 14 fév) → 15 fév douleur oreille gauche/trompe d'Eustache (récurrence)
\item \textbf{CRASH COMPOSÉ}: Deux cycles PEM superposés (12 fév→13 fév primaire + 14 fév→15 fév secondaire)
\end{itemize}

\subsubsection{16 février (dimanche) -- Continuation PEM (Jour 2 crash secondaire)}
\begin{itemize}
\item \textbf{Matin}: Fatigué ``dès le matin'', ``vraiment fatigué'', ``forte fatigue''
\item \textbf{Cognitif}: 6-7/10 (modérément altéré - en dessous de baseline)
\item \textbf{Progression}: Matin fatigué → ``Je sens que je devrais dormir''
\item \textbf{Soir}: ``Extrêmement fatigué'', a dû se reposer, onset acouphènes (deux oreilles)
\item \textbf{Symptômes sinusaux/auriculaires - EN AMÉLIORATION}:
    \begin{itemize}
    \item Nez bouché: Amélioré (``pas complètement libre, mais sans la douleur d'hier'')
    \item Douleur sinusale: RÉSOLU
    \item Douleur oreille gauche: RÉSOLU
    \item NOUVEAU: Acouphènes (deux oreilles, onset soir)
    \end{itemize}
\item \textbf{Médicaments}: LDN 3mg, Cétirizine (mêmes suppléments)
\end{itemize}

\textbf{CONTINUATION PEM - Jour 2 post-crash secondaire}:
\begin{itemize}
\item Fatigue forte matin → ``extrêmement fatigué'' soir = aggravation progressive malgré activité minimale
\item Altération cognitive 6-7/10 (vs 13 fév ``tête va bien'') = atteinte CNS (cohérent mal de tête 15 fév)
\item Pattern différent crash primaire (12→13 fév physique/autonome) vs secondaire (14→15→16 fév cognitif/CNS)
\item \textbf{Amélioration inflammation}: Douleur sinusale/auriculaire résolu, congestion s'améliore
\item \textbf{Acouphènes}: Nouveau symptôme, peut indiquer:
    \begin{itemize}
    \item Récupération dysfonction trompe Eustache (normalisation pression)
    \item Symptôme fatigue sévère EM/SFC (onset quand ``extrêmement fatigué'')
    \item Inflammation résiduelle oreille moyenne
    \end{itemize}
\item \textbf{Pronostic}: Inflammation sinusale/infection s'améliore; reste principalement symptômes PEM (fatigue forte, altération cognitive)
\end{itemize}

\textbf{LEÇON CLINIQUE CRITIQUE -- ``SE SENTIR OK'' N'EST PAS FIABLE}:
\begin{itemize}
\item Défi central EM/SFC démontré: 14 fév capacité + sensation OK → 15 fév crash sévère 24h plus tard
\item Ne peut pas se fier à tolérance même-jour pour juger sécurité activité
\item DOIT surveiller réponse 48h post-activité pour crash retardé
\item Charge activité 14 fév semblait tolérée → preuve 15-16 fév qu'elle DÉPASSAIT enveloppe vraie
\item \textbf{Recalibrage enveloppe énergétique nécessaire}: Baseline sûre BEAUCOUP plus basse qu'activité 14 fév
\item \textbf{Fenêtre critique étendue}: Crash composé augmente risque réduction baseline chronique (E5b)
\item \textbf{Repos strict minimum 14 jours} requis (16 fév onward) pour maximiser probabilité récupération baseline (E5a)
\end{itemize}

\subsubsection{17 février (mardi) -- Premier bon jour post-crash composé}
\begin{itemize}
\item \textbf{Énergie}: Bonne, ``globalement pas fatigué''
\item \textbf{Cognitif}: Bon, non altéré (vs 6-7/10 la veille) -- récupération complète de l'altération cognitive
\item \textbf{Faiblesse}: Absente
\item \textbf{Douleur}: Absente -- pas de mal de tête, pas de douleur corporelle
\item \textbf{Activité}: Sédentaire/passive uniquement: ordinateur, canapé, conduite -- pas d'effort physique testé
\item \textbf{Sommeil}: Sieste $\sim$1h30 le matin
\item \textbf{Médicaments}: LDN 3mg
\item \textbf{Suppléments}: Magnésium glycinate, MetaDigest, MCT, D-ribose + protocole habituel complet
\item \textbf{Question patient}: Se demande s'il pourrait tolérer un effort, et à quel niveau
\end{itemize}

\textbf{OBSERVATION PATIENT -- EFFET POSSIBLE LDN/L-CARNITINE}:
\begin{itemize}
\item Patient se demande si LDN et L-carnitine commencent à avoir des effets visibles
\item LDN maintenu à 3mg depuis 6 jours (réduit de 4mg le 12 fév); L-carnitine dans le protocole suppléments
\item Hypothèse plausible: LDN -- modulation neuroimmune, inhibition microgliale; L-carnitine -- support mitochondrial, transport acides gras, synthèse ATP
\item Facteurs confondants: trajectoire naturelle récupération PEM (Jour 3 post-pic cohérent avec résolution typique), activité sédentaire limitant la demande métabolique, rôle possible des autres suppléments (magnésium glycinate, D-ribose, MCT)
\item \textit{À surveiller}: corrélation avec les jours suivants pour distinguer effet traitement de récupération naturelle
\end{itemize}

\textbf{PREMIER JOUR POSITIF -- RÉCUPÉRATION EN COURS (Jour 6 post-crash 12 fév)}:
\begin{itemize}
\item Trajectoire: 12 fév crash → 13 fév pic primaire → 14 fév suractivité → 15 fév pic secondaire → 16 fév continuation PEM sévère → \textbf{17 fév: premier bon jour}
\item \textbf{Récupération cognitive confirmée}: Altération 6-7/10 le 16 fév → non altéré le 17 fév; cohérent avec résolution de la composante CNS du crash secondaire
\item \textbf{Activité sédentaire uniquement}: L'absence de faiblesse est notée dans ce contexte -- aucun effort physique n'a été réellement testé
\item \textbf{Sentiment subjectif de capacité croissant}: Le patient se demande s'il pourrait tolérer un effort -- signal positif de retour de la vitalité subjective; doit être interprété avec prudence (pattern 14 fév)
\item \textbf{Fenêtre de surveillance 48h obligatoire}: Les symptômes du 18-19 fév détermineront si le niveau d'activité du 17 fév était dans l'enveloppe sûre
\item Si les 18-19 fév restent asymptomatiques: une reprise progressive et graduelle des efforts peut être envisagée avec surveillance 48h à chaque palier
\end{itemize}

\subsubsection{18 février (mardi) -- Journée apparemment bonne avec activité excessive}
\begin{itemize}
\item \textbf{Énergie subjective}: Très bonne journée, aucune douleur
\item \textbf{Cognitif}: Esprit clair, non altéré
\item \textbf{Activité EXCESSIVE}: Conduite à Auchan + >2 heures debout (marche, courses, shopping)
\item \textbf{Médicaments}: 1 Provigil, Rupatall 10mg, Montelukast
\item \textbf{Suppléments}: L-Carnitine, Urolithin + NAD+, CoQ10
\item \textbf{Évaluation subjective}: Se sentait bien toute la journée, aucun symptôme d'alarme
\end{itemize}

\textbf{RÉPÉTITION EXACTE DU PATTERN 14 FÉVRIER -- PROVIGIL MASQUANT LES SIGNAUX}:
\begin{itemize}
\item Patient se sent ``très bien'' → dépasse largement l'enveloppe énergétique (>2h debout)
\item \textbf{FACTEUR CONFONDANT CRITIQUE}: Provigil pris le matin -- masque les signaux de fatigue, permet sentiment de capacité illusoire
\item Activité 18 fév (>2h Auchan) DÉPASSE largement seuil sûr établi (1,5h max selon recommandations 18 fév)
\item Pattern identique: 14 fév (>2h activité, se sentait OK) → 15 fév (crash sévère); maintenant 18 fév (>2h activité, se sentait bien) → 19 fév (crash attendu)
\item \textbf{INSIGHT PATIENT POST-HOC}: ``C'était trop'' -- reconnaissance rétrospective, mais pas en temps réel pendant l'activité
\end{itemize}

\subsubsection{19 février (mercredi) -- CRASH PEM SÉVÈRE (3e crash en 13 jours)}
\begin{itemize}
\item \textbf{Fatigue}: Sévère, ``totalement en train de crasher'', ``pas beaucoup d'énergie'', ``très fatigué''
\item \textbf{Mal de tête}: Fort (``strong headache'')
\item \textbf{Besoin de dormir}: Marqué, nécessité de repos horizontal
\item \textbf{Douleurs}: Aucune autre douleur rapportée (vs crashes précédents avec douleurs articulaires)
\item \textbf{Timeline}: 24h post-activité Auchan → onset crash CLASSIQUE
\item \textbf{Médicaments matin}: Rupatall 10mg, Montelukast, L-Carnitine, Urolithin + NAD+, CoQ10
\end{itemize}

\textbf{CRASH PEM CONFIRMÉ (E4) -- TROISIÈME OCCURRENCE EN 13 JOURS}:
\begin{itemize}
\item \textbf{Timeline}: 18 fév activités >2h → 19 fév crash (délai 24h) = pattern EM/SFC classique répété
\item \textbf{Pattern récurrent identique}: 9 fév crash, 15 fév crash, maintenant 19 fév crash -- tous précédés d'une journée ``bonne'' avec suractivité
\item \textbf{Séquence invariante}: Se sentir bien → dépasser enveloppe → crash 24h plus tard
\item \textbf{Provigil comme facteur de risque}: Les deux crashes récents (15 fév, 19 fév) suivent des jours où patient se sentait ``très bien'' -- probable masquage des signaux de fatigue permettant suractivité
\item \textbf{Insight patient présent mais inefficace}: Reconnaît ``c'était trop'' rétrospectivement, mais incapable de limiter l'activité en temps réel quand il se sent bien
\end{itemize}

\textbf{PATTERN CRITIQUE -- ÉCHEC RÉPÉTÉ DE PACING LORS DES JOURS ``BONS''}:
\begin{itemize}
\item \textbf{Défi central démontré 3 fois}: Sentiment de capacité $\neq$ capacité réelle
\item \textbf{Provigil aggrave le problème}: Améliore sensation subjective mais ne change pas l'enveloppe énergétique réelle → encourage suractivité → crash inévitable 24h plus tard
\item \textbf{Baseline réelle BEAUCOUP plus basse}: Activité >2h dépasse systématiquement l'enveloppe sûre
\item \textbf{Recommandations objectives ignorées}: Limite 1,5h établie le 18 fév après analyse crash 15 fév → dépassée massivement le jour même
\item \textbf{Nécessité surveillance objective}: Patient ne peut PAS se fier à ses sensations subjectives pour juger sécurité activité
\item \textbf{Fenêtre critique immédiate}: Repos strict 3-7 jours minimum requis pour éviter détérioration chronique baseline
\end{itemize}

\textbf{IMPLICATIONS TRAITEMENT PROVIGIL}:
\begin{itemize}
\item Provigil améliore fonction cognitive et énergie subjective (effet positif)
\item MAIS masque signaux de fatigue → permet suractivité → déclenche crashes (effet négatif net)
\item Pattern: Pas de Provigil = activité limitée naturellement par fatigue; Provigil = suractivité suivie de crash
\item \textbf{Recommandation}: Réévaluation stratégie Provigil avec médecin -- possiblement contre-productif pour gestion PEM
\item Alternatives: (1) Réduire dose; (2) Utiliser seulement pour activités essentielles planifiées avec repos prévu; (3) Combiner avec limites objectives strictes (timer, podomètre)
\end{itemize}

\subsection{Schémas cliniques identifiés clés}

\begin{enumerate}
\item \textbf{Échec de transition d'état autonome}: Les épisodes surviennent immédiatement au réveil du sommeil, avec progression de phase organisée (faiblesse → tremblements → résolution). La fonction cognitive est préservée tout au long, indiquant une défaillance principalement autonome plutôt que métabolique.

\item \textbf{Effondrement du seuil d'activité}: 30 minutes d'activité debout dépassent maintenant l'enveloppe énergétique, même les jours avec bonne ligne de base matinale. Ceci représente une détérioration fonctionnelle significative.

\item \textbf{Vulnérabilité au rebond stimulant}: Les jours sans Ritalin MR suivant les jours avec Ritalin montrent des symptômes exagérés (tremblements, faiblesse, sommeil excessif), suggérant une dynamique d'upregulation/downregulation du SNC.

\item \textbf{Repos non réparateur}: Les siestes d'après-midi (1-3 heures) échouent constamment à restaurer l'énergie. Ceci est caractéristique de la dysfonction du sommeil dans l'EM/SFC.

\item \textbf{Dissociation cognitive-physique}: La fonction cognitive est relativement préservée (``la tête va bien'') même pendant les épisodes physiques sévères, suggérant que la dysfonction primaire est autonome/périphérique plutôt qu'une défaillance métabolique centrale.

\item \textbf{NOUVEAU -- Provigil comme facteur de risque PEM}: L'utilisation de Provigil masque les signaux de fatigue, permet sentiment illusoire de capacité, encourage suractivité, et précipite crashes PEM sévères 24h plus tard. Pattern démontré deux fois (14→15 fév, 18→19 fév). Le patient ne peut pas se fier à ses sensations subjectives lors des jours sous Provigil pour juger sécurité de l'activité.
\end{enumerate}
