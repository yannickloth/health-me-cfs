\section{Recommandations thérapeutiques}

\subsection{Gestion de la dysrégulation autonome}

\subsubsection{Non pharmacologique (première ligne)}

\begin{enumerate}
\item \textbf{Augmenter apport hydrique à 2-3L/jour} avec électrolytes adéquats
\begin{itemize}
\item Preuves: US ME/CFS Clinician Coalition (Bateman et al. 2021) recommande hydratation agressive comme première ligne pour intolérance orthostatique
\item Patient utilise actuellement solution électrolytes 2×/jour; envisager augmentation à 3×/jour
\end{itemize}

\item \textbf{Augmenter apport sodium alimentaire} (si pression artérielle le permet)
\begin{itemize}
\item Cible: 5-10g sodium/jour (sous supervision médicale)
\item Surveiller pression artérielle; contre-indiqué en hypertension
\item Preuves: Stock et al. (2022) recommandent augmentations modestes avec surveillance TA
\end{itemize}

\item \textbf{Vêtements de compression}
\begin{itemize}
\item Bas de compression montant jusqu'à la taille (30-40 mmHg) plutôt que mi-bas (aux genoux)
\item Les liants abdominaux fournissent support retour veineux additionnel
\item Preuves: Recommandé par US ME/CFS Clinician Coalition (2021)
\end{itemize}

\item \textbf{Gestion posturale}
\begin{itemize}
\item Éviter station debout prolongée (seuil actuel: <30 minutes)
\item S'asseoir ou s'allonger quand possible pendant activités
\item Se lever lentement des positions allongée/assise
\item Élever tête de lit 10-15 degrés (peut améliorer tolérance orthostatique matinale)
\end{itemize}
\end{enumerate}

\subsection{Prévention et gestion du PEM}

\subsubsection{Identification de l'enveloppe d'activité}

Basé sur données récentes (8-13 février 2026), l'enveloppe d'activité sûre actuelle du patient est:

\begin{longtable}{p{4cm}p{3.5cm}p{6cm}}
\toprule
\textbf{Type d'activité} & \textbf{Durée maximale} & \textbf{Notes} \\
\midrule
Travail debout/vertical & <30 minutes & Repassage, cuisine, courses ont tous déclenché crashes à 30 min \\
\midrule
Travail cognitif assis & \textbf{FATIGANT} & Position assise fatigante, pas de récupération possible, envie constante de s'allonger; PEM même en position assise \\
\midrule
Marche (courses) & <60 minutes & 1h20 marche a déclenché crash d'après-midi le 11 fév \\
\midrule
Conduite & Toléré avec prudence & Faiblesse notée 11 fév mais pas de risque évanouissement/endormissement; toléré même trajets longs \\
\bottomrule
\end{longtable}

\subsubsection{Rythme basé sur fréquence cardiaque}

\begin{itemize}
\item \textbf{Limite FC cible:} 97 bpm ((220 - 44) × 0,55)
\item \textbf{Justification:} Rester sous seuil anaérobie prévient accumulation acide lactique et déclencheurs PEM
\item \textbf{Mise en œuvre:} Moniteur fréquence cardiaque continu pendant toutes activités
\item \textbf{Preuves:} Protocole de rythme Workwell Foundation; étude de faisabilité Davenport et al. (2025) sur surveillance FC pour prévention PEM
\end{itemize}

\subsubsection{Protocole de gestion PEM}

Quand les symptômes PEM se développent:
\begin{enumerate}
\item Cesser immédiatement toute activité non essentielle
\item S'allonger (position horizontale réduit stress autonome)
\item S'hydrater avec électrolytes
\item Ne pas tenter de ``pousser à travers''
\item Permettre minimum 24-48 heures de repos avant réévaluer capacité d'activité
\item Surveiller aggravation sur 24-72 heures (apparition PEM souvent retardée)
\end{enumerate}

\subsection{Optimisation du sommeil}

Problèmes de sommeil actuels:
\begin{itemize}
\item Sommeil nocturne fragmenté (réveil à 04:30, incapable de se rendormir)
\item Siestes diurnes non réparatrices (1-3 heures)
\item Douleur nocturne perturbant le sommeil
\end{itemize}

\textbf{Recommandations:}
\begin{enumerate}
\item Référence médecine du sommeil pour polysomnographie avec surveillance autonome
\item Évaluer dysrégulation autonome dépendante du stade de sommeil
\item Envisager essai supplémentation mélatonine (1-3mg, 30-60 min avant heure cible sommeil) -- aborde dysfonction pinéale hypothétique
\item Maintenir horaire sommeil-éveil cohérent quand possible
\item Aborder douleur nocturne (actuellement fesse droite; envisager évaluation musculo-squelettique)
\end{enumerate}

\subsection{Optimisation médicamenteuse}

\textbf{Problèmes actuels:}
\begin{enumerate}
\item \textbf{Incohérence dose LDN}: Alternance 3mg et 4mg empêche pharmacocinétique état stable
\begin{itemize}
\item Recommandation: Choisir dose cohérente; si 4mg cause effets secondaires, stabiliser à 3mg
\end{itemize}

\item \textbf{Schéma rebond stimulant}: Utilisation intermittente Ritalin cause jours rebond sévères
\begin{itemize}
\item Recommandation: Discuter avec médecin si utilisation quotidienne faible dose cohérente serait préférable à utilisation intermittente forte dose
\item Alternative: Planifier ``jours rebond'' avec repos strict et pas de conduite
\end{itemize}

\item \textbf{Protocole SAMA incomplet}: Patient prend actuellement SEULEMENT cétirizine (H1 basique). Le protocole médicamenteux de référence liste rupatadine + famotidine + quercétine, mais le patient confirme ne PAS les prendre actuellement.
\begin{itemize}
\item Recommandation: Envisager ajout protocole SAMA complet (voir section suivante pour détails)
\end{itemize}
\end{enumerate}

\subsection{Recommandations protocole SAMA (Syndrome activation mastocytes)}

\textbf{Contexte:} Le SAMA est de plus en plus reconnu comme comorbidité dans l'EM/SFC, avec médiateurs dérivés mastocytes contribuant à fatigue, brouillard mental et dysfonction autonome. Patient prend actuellement SEULEMENT cétirizine (H1 basique).

\textbf{Protocole SAMA recommandé complet:}

\paragraph{Quercétine -- 500-1000mg par jour}
\begin{itemize}
\item \textbf{Classification:} Stabilisateur mastocytes naturel (flavonoïde)
\item \textbf{Mécanisme:} Inhibe libération histamine et médiateurs inflammatoires des mastocytes
\item \textbf{Dosage:} 500-1000mg matin avec repas
\item \textbf{Preuves:} Études in vitro et animales montrent inhibition dégranulation mastocytes
\item \textbf{Sécurité:} Bien toléré; peut interférer avec certains médicaments (vérifier interactions)
\end{itemize}

\paragraph{Rupatadine -- 10-20mg par jour}
\begin{itemize}
\item \textbf{Classification:} Antihistaminique H1 + antagoniste PAF + stabilisateur mastocytes (triple action)
\item \textbf{Mécanisme:} Supérieur à cétirizine: bloque H1 + PAF (facteur activation plaquettes) + stabilise mastocytes
\item \textbf{Dosage:} 10-20mg matin
\item \textbf{Avantage vs. cétirizine:} Triple mécanisme vs. simple H1; stabilisation mastocytes documentée
\item \textbf{Preuves:} Études cliniques SAMA montrent efficacité supérieure aux H1 simples
\end{itemize}

\paragraph{Famotidine -- 20mg deux fois par jour}
\begin{itemize}
\item \textbf{Classification:} Bloqueur H2 (antagoniste récepteurs histamine-2)
\item \textbf{Mécanisme:} Complémente blocage H1 (rupatadine/cétirizine); bloque voie H2 distincte
\item \textbf{Dosage:} 20mg matin + 20mg soir
\item \textbf{Justification:} Protocole SAMA complet nécessite blocage H1 + H2
\item \textbf{Preuves:} Combinaison H1+H2 plus efficace que H1 seul pour SAMA
\end{itemize}

\textbf{Recommandation globale:}
\begin{enumerate}
\item \textbf{Ajouter famotidine 20mg 2×/jour} (bloqueur H2 manquant) -- priorité ÉLEVÉE
\item \textbf{Envisager substitution cétirizine → rupatadine 10-20mg} (triple action supérieure)
\item \textbf{Ajouter quercétine 500-1000mg} (stabilisateur mastocytes naturel)
\end{enumerate}

\textbf{Justification pour ce patient:} Dysfonction autonome et symptômes pseudo-hy\-po\-gly\-cé\-miques peuvent être partiellement médiés par activation mastocytes. Protocole SAMA complet pourrait réduire fréquence événements autonomes.

\section{AJOUTS MÉDICAMENTEUX POTENTIELS}

\subsection{Ivabradine (Procoralan/Corlanor)}

\textbf{Indication:} Contrôle fréquence cardiaque dans intolérance orthostatique / symptômes type POTS\\
\textbf{Dose initiale proposée:} 2,5mg deux fois par jour, titrer à 5-7,5mg deux fois par jour

\textbf{Mécanisme:} Inhibiteur sélectif du canal If (funny) dans le nœud sinusal. Réduit fréquence cardiaque sans abaisser pression artérielle. N'affecte pas contractilité cardiaque.

\textbf{Preuves:}
\begin{itemize}
\item \textbf{Essai randomisé (Taub et al. 2021, JACC):} Ivabradine supérieur au placebo pour réduire fréquence cardiaque et améliorer qualité de vie dans POTS hyperadrénergique (changement FC debout-couché: 13,1 bpm vs. 17,0 bpm placebo, p=0,001).
\item \textbf{Revue systématique (Frontiers in Neurology, 2024):} Ivabradine et midodrine démontrèrent taux le plus élevé d'amélioration symptomatique parmi médicaments POTS.
\item \textbf{Résultats rapportés patients (2025):} Chez patients EM/SFC et COVID long, ivabradine (66,8\%) eut impact positif significativement plus élevé que bêta-bloquants.
\item \textbf{Essais en cours:} Étude COVIVA (ivabradine pour POTS COVID-long); RECOVER-AUTONOMIC (achèvement prévu nov 2026).
\end{itemize}

\textbf{Avantages pour ce patient:}
\begin{itemize}
\item N'abaisse PAS pression artérielle (important pour patients avec hypotension orthostatique potentielle)
\item Contourne dysfonction niveau récepteur en inhibant directement courant If (pertinent si anticorps anti-GPCR présents)
\item Peut aborder pouls élevé observé pendant activités debout
\item Mieux toléré que bêta-bloquants chez beaucoup patients EM/SFC
\end{itemize}

\textbf{Risques:}
\begin{itemize}
\item Bradycardie (dose-dépendante)
\item Phosphènes (perturbations visuelles, typiquement transitoires)
\item Fibrillation auriculaire (rare, < 1\%)
\item Pas extensivement étudié dans EM/SFC spécifiquement
\end{itemize}

\textbf{Qualité preuves:} Moyenne-Élevée pour POTS; Moyenne pour extrapolation EM/SFC.

\textbf{Évaluation risque/bénéfice:} FAVORABLE -- aborde le pouls élevé documenté du patient pendant station debout avec effets minimaux sur pression artérielle. La préservation cognitive du patient pendant événements autonomes suggère que contrôle fréquence cardiaque seul peut être suffisant.

\subsection{Propranolol faible dose (Bêta-bloquant non sélectif)}

\textbf{Indication:} Contrôle fréquence cardiaque, réduction tremblements\\
\textbf{Dose initiale proposée:} 10mg une fois par jour, titrer à 10-20mg deux fois par jour

\textbf{Mécanisme:} Antagoniste bêta-adrénergique non sélectif. Réduit fréquence cardiaque, débit cardiaque et tremblements périphériques. Réduit aussi suractivité sympathique.

\textbf{Preuves:}
\begin{itemize}
\item \textbf{Raj et al. (2009, Circulation):} Propranolol faible dose (20mg) réduisit significativement tachycardie et améliora symptômes dans POTS. Constatation clé: FAIBLES doses fonctionnent mieux; doses plus élevées peuvent paradoxalement aggraver symptômes.
\item \textbf{Arnold et al. (2013, PMC):} Propranolol faible dose améliora VO2max chez patients POTS, suggérant bénéfices capacité exercice.
\item \textbf{Revue systématique (2025):} Bêta-bloquants montrèrent plus grande réduction variabilité fréquence cardiaque parmi traitements POTS.
\end{itemize}

\textbf{Avantages pour ce patient:}
\begin{itemize}
\item Peut aborder directement symptômes tremblements (proéminents dans événements récents)
\item Propriétés anti-migraine (pertinent vu historique migraines)
\item Profil sécurité bien caractérisé
\item Peu coûteux
\item Peut réduire suractivité sympathique contribuant à instabilité autonome
\end{itemize}

\textbf{Risques:}
\begin{itemize}
\item Peut abaisser pression artérielle (problématique si hypotension orthostatique présente)
\item Peut aggraver fatigue (effet secondaire commun pertinent pour EM/SFC)
\item Peut masquer symptômes hypoglycémie (pertinent vu épisodes pseudo-hy\-po\-gly\-cé\-miques)
\item Risque bronchospasme (patient a historique asthme enfance, bien que résolu)
\item Peut réduire davantage tolérance exercice
\end{itemize}

\textbf{Qualité preuves:} Moyenne-Élevée pour POTS; Faible pour EM/SFC spécifiquement.

\textbf{Évaluation risque/bénéfice:} MODÉRÉ -- contrôle tremblements et prévention migraine sont attrayants, mais aggravation fatigue est préoccupation significative. Le principe ``moins c'est plus'' s'applique: commencer très faible (10mg). Surveiller exacerbation fatigue.

\textbf{IMPORTANT:} Propranolol faible dose (10-20mg) recommandé sur doses standard. Doses plus élevées peuvent aggraver symptômes dans POTS/EM/SFC.

\subsection{Midodrine (Agoniste alpha-1 adrénergique)}

\textbf{Indication:} Intolérance orthostatique, hypotension orthostatique symptomatique\\
\textbf{Dose initiale proposée:} 2,5mg deux fois par jour (matin et midi), titrer à 5-10mg trois fois par jour

\textbf{Mécanisme:} Prodrogue convertie en desglymidodrine, agoniste alpha-1 adrénergique sélectif. Cause vasoconstriction périphérique, augmentant retour veineux et pression artérielle.

\textbf{Preuves:}
\begin{itemize}
\item \textbf{Revue systématique (Frontiers in Neurology, 2024):} Midodrine démontra parmi taux les plus élevés d'amélioration symptomatique pour POTS.
\item \textbf{Données renouvellement ordonnances:} 33,91\% taux succès traitement pour midodrine dans POTS.
\item \textbf{US ME/CFS Clinician Coalition (2021):} Listé parmi options pharmacologiques première ligne pour intolérance orthostatique dans EM/SFC.
\item \textbf{Guidance traitement CDC ME/CFS:} Midodrine recommandé pour hypotension orthostatique et POTS.
\end{itemize}

\textbf{Avantages pour ce patient:}
\begin{itemize}
\item Aborde intolérance orthostatique directement
\item Peut réduire événements autonomes déclenchés par changement postural
\item Bien caractérisé; approuvé FDA pour hypotension orthostatique
\item Ne cause pas dépression SNC
\end{itemize}

\textbf{Risques:}
\begin{itemize}
\item Hypertension en position couchée (ne pas prendre avant s'allonger; dernière dose >4h avant coucher)
\item Rétention urinaire
\item Piloérection (``chair de poule'')
\item Picotements cuir chevelu
\item Maux de tête (pertinent vu historique migraines)
\end{itemize}

\textbf{Qualité preuves:} Moyenne pour EM/SFC; Élevée pour hypotension orthostatique.

\textbf{Évaluation risque/bénéfice:} FAVORABLE si hypotension orthostatique confirmée par test inclinaison. Moins approprié si constatation primaire est tachycardie sans hypotension (auquel cas ivabradine ou bêta-bloquant faible dose préféré).

\textbf{TIMING CRITIQUE:} Dernière dose doit être prise au moins 4 heures avant s'allonger pour éviter hypertension en position couchée.

\subsection{Fludrocortisone (Minéralocorticoïde synthétique)}

\textbf{Indication:} Expansion volume sanguin pour intolérance orthostatique\\
\textbf{Dose initiale proposée:} 0,05mg par jour, titrer à 0,1-0,2mg par jour

\textbf{Mécanisme:} Minéralocorticoïde synthétique qui augmente réabsorption sodium et eau dans reins, expansant volume plasmatique. Aborde déficit volume sanguin documenté dans EM/SFC (Hurwitz et al.: 93,8\% patientes et 50\% patients masculins EM/SFC ont masse globules rouges réduite).

\textbf{Preuves:}
\begin{itemize}
\item \textbf{Freitas et al. (2000):} Combinaison bêta-bloquant (bisoprolol) + fludrocortisone montra amélioration clinique dans intolérance orthostatique. Combinaison plus efficace que monothérapie.
\item \textbf{Raj et al. (2005, Circulation):} Déficits volume sanguin marqués documentés chez patients POTS avec niveaux aldostérone paradoxalement normaux à bas.
\item \textbf{Données renouvellement ordonnances:} 42,78\% taux succès traitement pour fludrocortisone dans POTS (le plus élevé parmi médicaments POTS communs).
\item \textbf{US ME/CFS Clinician Coalition (2021):} Listé parmi options première ligne pour intolérance orthostatique dans EM/SFC.
\end{itemize}

\textbf{Avantages pour ce patient:}
\begin{itemize}
\item Aborde déficit volume sanguin probable (93,8\% patientes, 50\% patients masculins EM/SFC affectés)
\item Peut réduire fréquence événements orthostatiques
\item Dosage une fois par jour (simple)
\item Peut être combiné avec autres agents (midodrine, bêta-bloquants)
\end{itemize}

\textbf{Risques:}
\begin{itemize}
\item Hypokaliémie (surveiller niveaux potassium)
\item Rétention liquidienne / œdème
\item Hypertension (surveiller pression artérielle)
\item Maux de tête
\item Gain de poids
\item Long terme: suppression surrénale potentielle à doses plus élevées
\end{itemize}

\textbf{Qualité preuves:} Moyenne pour intolérance orthostatique EM/SFC; Moyenne-Élevée pour POTS.

\textbf{Évaluation risque/bénéfice:} FAVORABLE comme thérapie adjuvante. Particulièrement approprié si déficit volume sanguin documenté. Nécessite surveillance électrolytes (potassium).

\subsection{Pyridostigmine (Mestinon)}

\textbf{Indication:} Dysfonction autonome, intolérance à l'exercice\\
\textbf{Dose initiale proposée:} 30mg deux fois par jour, titrer à 60mg trois fois par jour

\textbf{Mécanisme:} Inhibiteur acétylcholinestérase qui améliore tonus parasympathique (vagal) en prévenant dégradation acétylcholine. Améliore équilibre autonome.

\textbf{Preuves:}
\begin{itemize}
\item \textbf{Étude croisée randomisée:} 30mg pyridostigmine fournit soulagement symptômes dans 4 heures et réduisit fréquences cardiaques debout chez patients POTS.
\item \textbf{Étude rétrospective (n=300 patients POTS):} Environ 50\% expérimentèrent amélioration symptômes orthostatiques.
\item \textbf{Enquête rapportée patients:} $\sim$70\% patients rapportèrent au moins quelque efficacité pour POTS.
\item \textbf{Life Improvement Trial (OMF, 2024-en cours):} Étudie effets synergiques pyridostigmine + LDN dans EM/SFC.
\item \textbf{Revue systématique (2025):} Études uniques impliquant effets hémodynamiques bénéfiques dans POTS.
\end{itemize}

\textbf{Avantages pour ce patient:}
\begin{itemize}
\item Peut aborder échec transition état autonome (hypothèse primaire pour événement 11 fév)
\item Améliore tonus vagal, qui peut stabiliser transitions autonomes pendant cycles sommeil-éveil
\item Effets secondaires cardiovasculaires minimaux
\item Peut être combiné avec autres agents autonomes
\item Potentiellement synergique avec LDN (étudié dans Life Improvement Trial)
\end{itemize}

\textbf{Risques:}
\begin{itemize}
\item Malaise gastro-intestinal (plus commun; nausée, diarrhée, crampes)
\item Salivation accrue
\item Crampes musculaires (patient a déjà crampes chroniques -- surveiller attentivement)
\item Fréquence urinaire
\item Fasciculations
\end{itemize}

\textbf{ATTENTION pour ce patient:} Vu hypersensibilité vagale documentée et historique syncope vasovagale, pyridostigmine (qui AMÉLIORE tonus vagal) devrait être utilisé avec extrême prudence. Commencer à dose la plus faible avec surveillance étroite est essentiel.

\textbf{Qualité preuves:} Moyenne pour POTS; Faible-Moyenne pour EM/SFC.

\textbf{Évaluation risque/bénéfice:} INCERTAIN -- justification est forte (modulation autonome), mais hypersensibilité vagale documentée du patient crée risque spécifique. Discuter attentivement avec spécialiste.

\subsection{Cimétidinée pour modulation immunitaire EBV}

\textbf{Classification:} Antagoniste H2 avec effets immunomodulateurs\\
\textbf{Dosage suggéré:} 200~mg deux fois par jour (BID)

\textbf{Rationale spécifique à ce patient:}\\
Le VCA IgG EBV est très élevé ($>$750~U/mL, résultat octobre 2025), ce qui est cohérent avec une stimulation virale chronique. Ce niveau élevé suggère une réponse immunitaire persistante contre l'EBV, possiblement associée à une réactivation latente. La cimétidinée a un mécanisme immunomodulateur spécifiquement pertinent dans ce contexte.

\textbf{Mécanisme d'action:}
\begin{enumerate}
\item Les récepteurs H2 sur les lymphocytes T-suppresseurs inhibent l'immunité cellulaire
\item La cimétidinée débloque ces récepteurs~$\rightarrow$ levée de la suppression immunitaire
\item Activité T-cellulaire et NK augmentée~$\rightarrow$ meilleur contrôle des cellules infectées par EBV
\item Réduction de la charge virale latente~$\rightarrow$ réduction de la stimulation immunitaire chronique
\end{enumerate}

\textbf{Base de preuves:}
\begin{itemize}
\item Cas Ursula (cas documenté EM/SFC viral-immunitaire): cimétidinée 200~mg BID a produit une amélioration dramatique de l'énergie (``sortie du lit'') chez une patiente avec EBV IgG très élevé, dépletion lymphocytes B et NK bas. Les titres EBV ont diminué sous traitement (EBV IgG: 596~$\rightarrow$~514~E/mL~; EBNA: 213~$\rightarrow$~156~E/mL).
\item Mécanisme publié: la cimétidinée lève l'immunosuppression médiée par les récepteurs H2 sur les T-suppresseurs, renforçant la cytotoxicité T/NK contre les virus herpétiques (dont EBV) (Cohen et al., Puri et al., littérature sur les analogues H2).
\item Note importante: La famotidine (autre bloqueur H2) est connue pour provoquer des effets centraux sévères chez certains patients EM/SFC; la cimétidinée a un profil pharmacocinétique différent (distribution tissulaire distincte).
\end{itemize}

\textbf{Bilan complémentaire recommandé avant/pendant essai:}
\begin{itemize}
\item EBV Early Antigen IgG (EA-IgG) -- détecte réactivation active
\item EBV VCA IgM -- confirme réactivation récente
\item EBV PCR sanguin -- quantifie charge virale active
\item Numération lymphocytaire: CD19+ (lymphocytes B), NK (CD56+) -- évalue le déficit d'immunité cellulaire
\end{itemize}

\textbf{Qualité preuves:} Faible-Moyenne~-- mécanisme bien établi; données essentiellement observationnelles et cas cliniques; pas de RCT cimétidinée dans l'EM/SFC.

\textbf{Évaluation risque/bénéfice:} FAVORABLE -- profil de sécurité bien établi (médicament largement utilisé), coût faible, mécanisme plausible et cohérent avec les résultats EBV du patient. Recommandé comme essai thérapeutique diagnostique après bilan immunologique.

\subsection{Tableau comparatif: Ajouts médicamenteux potentiels}

{\scriptsize
\begin{longtable}{p{1.4cm}p{1.2cm}p{0.9cm}p{0.9cm}p{1cm}p{1cm}p{0.8cm}}
\toprule
\textbf{Méd.} & \textbf{Cible} & \textbf{TA} & \textbf{FC} & \textbf{Fatig.} & \textbf{Preuv.} & \textbf{Pri.} \\
\midrule
Ivabr. & Fréq.C & Neutre & $\downarrow$ & Faible & Moy. & \textbf{ÉL.} \\
\midrule
Pr.f & FC+tr & $\downarrow$ & $\downarrow$ & Modéré & Faible & Moy. \\
\midrule
Midodr. & Press.art & $\uparrow$ & Neutre & Faible & Moy. & Moy. \\
\midrule
Fludro. & Vol.sang & $\uparrow$ & Neutre & Faible & Moy. & Moy. \\
\midrule
Pyrid. & Éq.aut. & Neutre & $\downarrow$ & Faible & Fb.-M. & \textbf{Att.} \\
\midrule
Ciméti. & Immun.EBV & Neutre & Neutre & Modéré & Fb.-M. & \textbf{ÉL.} \\
\bottomrule
\end{longtable}
}

\textbf{Ordre priorité recommandé (pour présentation spécifique de ce patient):}
\begin{enumerate}
\item \textbf{Ivabradine} -- meilleures preuves pour contrôle FC sans effets TA; aborde plainte autonome centrale
\item \textbf{Cimétidinée} -- rationale fort lié aux résultats EBV ($>$750 U/mL); essai diagnostique et thérapeutique; profil de sécurité favorable
\item \textbf{Fludrocortisone} -- aborde déficit volume sanguin probable; dosage simple
\item \textbf{Midodrine} -- si hypotension orthostatique confirmée
\item \textbf{Propranolol faible dose} -- si tremblements restent problématiques; attention avec fatigue
\item \textbf{Pyridostigmine} -- différer jusqu'à hypersensibilité vagale mieux caractérisée
\end{enumerate}
