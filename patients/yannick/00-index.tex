% FILE: Patient case index and overview - Yannick (age 44)
\chapter{Patient Case: Yannick}
\label{app:case-yannick}

This section documents the case study of Yannick, a 44-year-old patient with severe ME/CFS and a complex medical history spanning decades. This patient case is presented as a detailed clinical study illustrating ME/CFS manifestations, diagnostic uncertainty, and treatment challenges in a real-world scenario.

\section{Patient Demographics}
\label{sec:yannick-demographics}

\begin{itemize}
    \item \textbf{Name}: Yannick
    \item \textbf{Date of Birth}: March 22, 1981
    \item \textbf{Age at Documentation}: 44 years (as of February 2026)
    \item \textbf{Nationality}: Belgian
    \item \textbf{Language}: French (fluent)
    \item \textbf{Sex}: Male
\end{itemize}

\section{Clinical Summary}
\label{sec:yannick-clinical-summary}

Yannick presents with severe ME/CFS characterized by:

\begin{enumerate}
    \item \textbf{Profound metabolic dysfunction} affecting energy production (mitochondrial dysfunction)
    \item \textbf{Lifelong cognitive deficits} with unclear etiology (ADHD vs.\ secondary to energy deficit)
    \item \textbf{Autonomic dysregulation} with recent acute episodes (Feb 2026)
    \item \textbf{Multi-system progressive deterioration} spanning 30+ years
    \item \textbf{Complex medication history} including stimulants (Ritalin, Concerta, Provigil)
\end{enumerate}

\section{Disease Timeline}
\label{sec:yannick-timeline}

\textbf{Early manifestations (childhood-adolescence):}
\begin{itemize}
    \item Age $\sim$13--15: Progressive brain fog and cognitive fatigue (ME/CFS pattern)
    \item Age 16 (circa 1997): Hand tremor noticed by others
    \item Age $\sim$20 (circa 2001): Muscle cramps onset (25-year history)
    \item Age 20+: Severe attention deficits; started methylphenidate (Ritalin) treatment
\end{itemize}

\textbf{Working years (age 20--45):}
\begin{itemize}
    \item Age 20--45: Maintained employment with extreme compensatory strategies
    \item Ongoing ADHD/stimulant medication management
    \item Progressive worsening of ME/CFS-pattern symptoms (fatigue, brain fog, cognitive decline)
    \item Trial of modafinil (Provigil) in addition to methylphenidate
\end{itemize}

\textbf{Disease escalation (2018):}
\begin{itemize}
    \item June 2018: Head trauma (burnout-related incident); functional capacity collapsed
    \item Post-2018: Transition to severe/very severe ME/CFS, unable to maintain employment
\end{itemize}

\textbf{Recent developments (2025--2026):}
\begin{itemize}
    \item Late 2025: Trial of increased swimming/exercise for 4--5 months (failed, worsened cognitive PEM)
    \item January 25, 2026: Upper respiratory infection with severe autonomic exacerbation
    \item February 2--3, 2026: Post-viral weakness and fatigue
    \item February 8--10, 2026: Activity-triggered PEM and Ritalin MR resumption trial
    \item February 11, 2026: Autonomic dysregulation during sleep-wake transition (critical event)
\end{itemize}

\section{Key Clinical Features}
\label{sec:yannick-features}

\subsection{Core ME/CFS Symptoms}

\begin{enumerate}
    \item \textbf{Post-exertional malaise} (decades-long pattern, progressive worsening)
    \item \textbf{Profound fatigue} unrelieved by rest; sensation of ``running on empty''
    \item \textbf{Cognitive impairment} (multiple overlapping components)
    \item \textbf{Exertion intolerance} (physical and cognitive)
    \item \textbf{Autonomic dysfunction} with recent acute dysregulation episodes
\end{enumerate}

\subsection{Musculoskeletal Manifestations}

\begin{enumerate}
    \item Muscle cramps (25-year history, age of onset ~20)
    \item Finger contractures (reverse extension pattern)
    \item Neck muscle cramps and contractures
    \item Early tremor (present since age 16)
    \item Joint pain (knuckles, knees, shoulders, wrists)
    \item Chronic leg exhaustion
\end{enumerate}

\subsection{Neurological and Cognitive}

\begin{enumerate}
    \item Attention deficit (severe, lifelong; excellent response to methylphenidate)
    \item Brain fog (progressive, energy-dependent)
    \item Social interaction as painful exertion (20+ year history)
    \item Word-finding difficulties
    \item Short-term memory impairment
    \item Progressive vision decline (presbyopia onset age 40)
    \item Bilateral sensorineural hearing loss (diagnosed August 2024)
\end{enumerate}

\subsection{Autonomic and Cardiovascular}

\begin{enumerate}
    \item Orthostatic intolerance (worse with infection)
    \item Progressive air hunger
    \item Possible POTS or dysautonomia components
    \item Recent autonomic dysregulation during sleep-wake transition
\end{enumerate}

\subsection{Immune and Allergic}

\begin{enumerate}
    \item Increased food allergies (new onset, nut allergies confirmed)
    \item Oral allergy syndrome pattern
    \item Elevated soy IgG
    \item Childhood asthma (resolved)
\end{enumerate}

\section{Diagnostic and Investigational Status}
\label{sec:yannick-diagnostic-status}

\subsection{Confirmed Diagnoses}

\begin{itemize}
    \item \textbf{ME/CFS}: Clinical diagnosis based on symptom constellation and response patterns
    \item \textbf{Bilateral sensorineural hearing loss}: Confirmed August 2024 (high-frequency loss)
    \item \textbf{Nut allergies}: Laboratory confirmed (FX1 panel)
    \item \textbf{Pollen allergies}: Documented (TX5, TX6)
    \item \textbf{Presbyopia}: Progressive, with hypermetropia baseline
\end{itemize}

\subsection{Diagnostic Uncertainty}

\begin{itemize}
    \item \textbf{ADHD vs.\ secondary attention deficit}: Severe attention deficits respond dramatically to methylphenidate, but formal ADHD testing negative. Unclear if primary neurodevelopmental disorder or energy-deficit-induced secondary cognitive impairment.
    \item \textbf{Specific autonomic syndrome}: Orthostatic symptoms documented, but formal dysautonomia testing status unclear
    \item \textbf{Mitochondrial dysfunction specifics}: Presumed but not formally tested with mitochondrial function assays
\end{itemize}

\section{Current Medical Management}
\label{sec:yannick-current-management}

See Section~\ref{app:medical-management} for detailed current medications and protocols.

\textbf{Key medications:}

\begin{itemize}
    \item \textbf{Ritalin MR 30mg}: Methylphenidate extended-release (recently resumed)
    \item \textbf{Low-Dose Naltrexone (LDN) 4mg}: For immune modulation and pain
    \item \textbf{Acetyl-L-Carnitine}: For mitochondrial support
    \item Supporting supplements: CoQ10, Riboflavin, others (see detailed management)
\end{itemize}

\section{Recent Critical Event: Autonomic Dysregulation (February 11, 2026)}
\label{sec:yannick-critical-event}

\textbf{Event}: Acute autonomic dysregulation during sleep-wake transition.

\textbf{Significance}: Represents acute worsening or novel manifestation requiring urgent investigation. May indicate need for sleep medicine evaluation and autonomic function testing.

See Section~\ref{subsec:personal-pem-ritalin-feb2026} for detailed documentation of the preceding activity-PEM-medication sequence.

\section{Research Hypotheses and Testable Predictions}
\label{sec:yannick-hypotheses}

See Section~\ref{app:research-hypotheses} for detailed research hypotheses with certainty assessments.

\textbf{Key hypothesis under investigation}:

\begin{itemize}
    \item \textbf{Fluoride-pineal-sleep hypothesis}: Fluoride accumulation may contribute to pineal dysfunction → melatonin dysregulation → autonomic dysregulation exacerbation (see Section~\ref{subsec:yannick-fluoride-hypothesis})
\end{itemize}

\section{Suggested Diagnostic Work-Up}
\label{sec:yannick-diagnostic-workup}

See the following sections for detailed diagnostic protocols:

\begin{enumerate}
    \item Section~\ref{app:diagnostic-protocol}: Tier 1 diagnostic tests with timing and logistics
    \item Section~\ref{app:sleep-specialist}: Sleep medicine specialist referral questions
    \item Section~\ref{app:activity-tracking}: Systematic sleep-activity-rebound tracking protocol
\end{enumerate}

\section{Organization of This Case Study}
\label{sec:yannick-organization}

This patient case is organized as follows:

\begin{itemize}
    \item \textbf{This file (00-index.tex)}: Overview and navigation
    \item \textbf{01-demographics-and-history.tex}: Detailed medical and personal history
    \item \textbf{02-current-status.tex}: Current functional status and medications
    \item \textbf{03-symptom-profile.tex}: Detailed symptom documentation (from appendix-i-personal-symptoms)
    \item \textbf{04-medical-management.tex}: Treatment protocols and medication management (from appendix-i-a-medical-management)
    \item \textbf{05-clinical-findings.tex}: Laboratory results and clinical assessments (from appendix-i-b-clinical-findings)
    \item \textbf{06-case-analysis.tex}: Diagnostic reasoning and case interpretation (from appendix-i-c-case-analysis)
    \item \textbf{07-research-hypotheses.tex}: Testable hypotheses with certainty assessments
    \item \textbf{protocols/diagnostic-test-protocol.tex}: Recommended diagnostic tests
    \item \textbf{protocols/sleep-specialist-referral.tex}: Sleep medicine evaluation questions
    \item \textbf{protocols/sleep-tracking-protocol.tex}: Systematic sleep-activity tracking
    \item \textbf{protocols/activity-tracking-protocol.tex}: Activity and rebound tracking
\end{itemize}

\section{Medical Disclaimer}
\label{sec:yannick-disclaimer}

\begin{tcolorbox}[colback=red!5!white,colframe=red!75!black,title=Medical Disclaimer]
This patient case documentation is for educational and clinical reasoning purposes. All recommendations, hypotheses, and diagnostic suggestions are preliminary and require review and approval by qualified physicians before clinical implementation. This documentation does not constitute medical advice and should not be used as a substitute for professional medical consultation.

The patient presented in this case has authorized this documentation for educational use in understanding ME/CFS manifestations and clinical reasoning in complex cases.
\end{tcolorbox}

